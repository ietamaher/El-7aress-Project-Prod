% ============================================================================
% LESSON 4 - MOTION MODES & SURVEILLANCE (UPDATED FOR QT6)
% ============================================================================
\lesson{4: MOTION MODES \& SURVEILLANCE}

\noindent
\begin{tabular}{@{}ll@{}}
\textbf{Duration:} & 3 hours \\
\textbf{Type:} & Classroom + Practical \\
\textbf{References:} & Zone management guide, surveillance procedures \\
\end{tabular}

\section{Introduction}

\textbf{Lesson Purpose}: This lesson teaches motion mode selection, automated surveillance patterns, TRP/scan zone selection using joystick buttons, and zone management for safe and effective operations.

\textbf{Learning Objectives}:
\begin{itemize}
    \item Explain the purpose of each motion mode
    \item Switch between motion modes safely using Buttons 11/13
    \item Operate automatic sector scan mode
    \item Utilize Target Reference Point (TRP) scan mode
    \item Select TRP pages and scan zones using Buttons 14/16
    \item Define and manage no-fire zones
    \item Define and manage no-traverse zones
    \item Save and load zone configurations
\end{itemize}

\section{Motion Modes Overview}

\subsection{What Are Motion Modes?}

Motion modes control \textbf{how the gimbal moves} during operations:

\begin{itemize}
    \item \textbf{Manual Mode}: You control gimbal directly with joystick
    \item \textbf{Auto Sector Scan}: System automatically scans between two points
    \item \textbf{TRP Scan}: System sequentially visits pre-defined Target Reference Points
\end{itemize}

\textbf{Purpose}: Different missions require different surveillance patterns. Motion modes let you switch between direct control and automated surveillance.

\begin{notebox}[FUTURE CAPABILITY]
The system architecture supports radar integration (radar slew/tracking modes) for future upgrades, but radar is NOT currently installed on standard El 7arress RCWS systems.
\end{notebox}

\subsection{Mode Selection}

\textbf{How to Change Modes}:
\begin{itemize}
    \item Press joystick \button{Button 11} or \button{Button 13} (either button cycles modes)
    \item Modes cycle in sequence:
\end{itemize}

\begin{lstlisting}
Manual → AutoSectorScan → TRP Scan → Manual (repeats)
\end{lstlisting}

\textbf{Current Mode Display}: OSD bottom center shows:
\begin{itemize}
    \item \osd{MODE: Manual}
    \item \osd{MODE: Sector Scan}
    \item \osd{MODE: TRP}
\end{itemize}

\begin{cautionbox}[RESTRICTION]
Cannot change modes during tracking acquisition. Must abort tracking first (Button 4 double-click).
\end{cautionbox}

\section{Manual Mode}

\textbf{Description}: Direct operator control via joystick (default mode)

\textbf{When to Use}:
\begin{itemize}
    \item Direct target engagement
    \item Precise aiming
    \item Immediate threat response
    \item Search operations requiring operator judgment
\end{itemize}

\textbf{Operation}:
\begin{itemize}
    \item Joystick LEFT/RIGHT → Azimuth control
    \item Joystick UP/DOWN → Elevation control
    \item Stick deflection = gimbal speed (modified by Speed Select switch)
    \item Center stick = gimbal stops
\end{itemize}

\begin{notebox}
Already Covered: See Lesson 2, Section 2.4 for detailed joystick control techniques
\end{notebox}

\section{Auto Sector Scan Mode}

\subsection{What Is Sector Scan?}

\textbf{Definition}: Automated gimbal movement that continuously scans between two pre-defined points (left and right azimuth limits)

\textbf{Purpose}:
\begin{itemize}
    \item Perimeter surveillance without constant operator input
    \item Monitoring a defined sector for extended periods
    \item Frees operator to monitor other systems, communications, or threats outside scan area
    \item Reduces operator fatigue during long surveillance missions
\end{itemize}

\textbf{Visual Representation}:
\begin{lstlisting}
    Left Limit              Right Limit
        ↓                       ↓
        ●←←←←←←←scan←←←←←←←●
        ●→→→→→→→scan→→→→→→→●
             (repeats)
\end{lstlisting}

Gimbal continuously scans left-to-right at defined elevation, then reverses right-to-left, and repeats indefinitely until mode change.

\subsection{Activating Sector Scan}

\textbf{Prerequisites}:
\begin{enumerate}
    \item At least one sector scan zone must be defined (see Section 4.7)
    \item Sector scan zone must be set to Active
    \item System must be in operational state (not emergency stop or fault)
\end{enumerate}

\begin{procedurebox}[SECTOR SCAN ACTIVATION]
\begin{enumerate}
    \item \textbf{Cycle to Sector Scan Mode}:
    \begin{itemize}
        \item Press \button{Button 11} or \button{Button 13} until OSD displays: \osd{MODE: Sector Scan}
    \end{itemize}

    \item \textbf{System Automatic Behavior}:
    \begin{itemize}
        \item Gimbal automatically slews to left limit of active sector zone
        \item Begins scanning to right limit (smooth, constant speed movement)
        \item Reaches right limit, reverses direction
        \item Scans back to left limit
        \item Repeats continuously until mode change or operator interruption
    \end{itemize}

    \item \textbf{Scan Parameters}:
    \begin{itemize}
        \item \textbf{Scan Speed}: Default 5°/second (configurable per zone)
        \item \textbf{Elevation}: Fixed elevation defined in zone (typically 0° horizon)
        \item \textbf{Azimuth Range}: Between left and right limits defined in zone
    \end{itemize}
\end{enumerate}
\end{procedurebox}

\subsection{While Sector Scanning}

\textbf{Operator Actions - CAN DO}:
\begin{itemize}
    \item Monitor video feed as gimbal scans sector
    \item Switch cameras (day/thermal - camera toggle button)
    \item Zoom in/out (\button{Button 6/8}) to examine areas of interest
    \item Activate LRF (\button{Button 1}) to range targets during scan
    \item Change thermal LUT (\button{Button 7/9}) for better target contrast
    \item Initiate tracking (\button{Button 4}) on detected threat:
    \begin{itemize}
        \item Aborts sector scan
        \item Switches to Manual mode
        \item Enters tracking acquisition
    \end{itemize}
    \item Switch to different scan zone (\button{Button 14/16}) - see Section 4.5
\end{itemize}

\textbf{Operator Actions - CANNOT DO}:
\begin{itemize}
    \item Joystick stick axes do NOT control gimbal (inputs ignored during sector scan)
    \item Cannot manually slew gimbal while in AutoSectorScan mode
    \item Must exit mode first to regain manual gimbal control
\end{itemize}

\textbf{To Override / Exit Sector Scan}:
\begin{itemize}
    \item \textbf{Method 1}: Press \button{Button 11 or 13} to cycle to next mode (typically TRP Scan or Manual)
    \item \textbf{Method 2}: Press \button{Button 4} (tracking button) to initiate tracking
    \begin{itemize}
        \item System automatically switches to Manual mode
        \item Enters tracking acquisition
        \item Gimbal control returns to operator
    \end{itemize}
\end{itemize}

\subsection{Multiple Sector Zones}

\textbf{If Multiple Sectors Defined}:
\begin{itemize}
    \item System scans currently active sector zone
    \item Only ONE sector can be active at a time
    \item Use \button{Button 14/16} to switch between defined sectors (see Section 4.5)
\end{itemize}

\section{TRP Scan Mode}

\subsection{What Is TRP Scan?}

\textbf{TRP} = \textbf{Target Reference Point} (pre-defined location of interest)

\textbf{Definition}: System sequentially slews to each TRP in a defined list, dwells (pauses) for observation at each location, then automatically moves to next TRP. After completing the list, system loops back to first TRP and repeats.

\textbf{Purpose}:
\begin{itemize}
    \item \textbf{Checkpoint verification}: Monitor multiple checkpoints (e.g., Gate 1, Gate 2, Gate 3) systematically
    \item \textbf{Known threat areas}: Periodically check sniper hide sites, ambush points, IED locations
    \item \textbf{Key terrain monitoring}: Bridges, hilltops, road intersections, dead space areas
    \item \textbf{More efficient than manual}: Eliminates operator fatigue from repetitive manual searching
\end{itemize}

\textbf{Visual Representation}:
\begin{lstlisting}
TRP 1 (dwell 30s) → TRP 2 (dwell 30s) → TRP 3 (dwell 30s) → TRP 1 (loop)
\end{lstlisting}

\subsection{Activating TRP Scan}

\textbf{Prerequisites}:
\begin{enumerate}
    \item At least one TRP must be defined in active TRP page (see Section 4.8)
    \item TRP page must be selected and active
\end{enumerate}

\begin{procedurebox}[TRP SCAN ACTIVATION]
\begin{enumerate}
    \item \textbf{Cycle to TRP Scan Mode}:
    \begin{itemize}
        \item Press \button{Button 11} or \button{Button 13} until OSD displays: \osd{MODE: TRP}
    \end{itemize}

    \item \textbf{System Automatic Behavior}:
    \begin{itemize}
        \item Gimbal slews rapidly to first TRP in list
        \item Arrives at TRP position (azimuth + elevation)
        \item Dwells (stationary) for configured time (default: 30 seconds)
        \item OSD displays: \osd{TRP: [Name]} and \osd{DWELL: XXs}
        \item After dwell completes, slews to next TRP
        \item Dwells again at second TRP
        \item Continues through entire TRP list
        \item Loops back to first TRP and repeats
    \end{itemize}

    \item \textbf{TRP Parameters}:
    \begin{itemize}
        \item \textbf{Dwell Time}: Configurable per TRP (5-120 seconds)
        \item \textbf{Slew Speed}: Fast slew between TRPs (typically 30-60°/sec)
        \item \textbf{Position Accuracy}: ±2° tolerance on arrival
    \end{itemize}
\end{enumerate}
\end{procedurebox}

\subsection{While TRP Scanning}

\textbf{During Dwell} (gimbal stationary at TRP):
\begin{itemize}
    \item Observe video feed carefully (TRP is point of interest)
    \item Switch cameras (day/thermal) for better target identification
    \item Zoom in for detailed observation (\button{Button 6})
    \item Fire LRF to range objects (\button{Button 1})
    \item Initiate tracking if threat detected (\button{Button 4})
    \item OSD displays remaining dwell time: \osd{DWELL: 15s}
\end{itemize}

\textbf{During Slew} (gimbal moving between TRPs):
\begin{itemize}
    \item Gimbal is in rapid motion
    \item Video may be blurred during slew (depending on slew speed)
    \item Wait for next TRP arrival and dwell to observe
    \item OSD displays: \osd{SLEWING TO: [TRP Name]}
\end{itemize}

\textbf{Switching TRP Pages}:
\begin{itemize}
    \item If multiple TRP pages defined, use \button{Button 14/16} to switch (see Section 4.5)
\end{itemize}

\subsection{TRP Scan Best Practices}

\begin{enumerate}
    \item \textbf{Define TRPs During Mission Planning}:
    \begin{itemize}
        \item Pre-mission: Identify key locations on map
        \item Convert to azimuth/elevation coordinates
        \item Enter TRPs via menu during mission prep
        \item Test TRP scan before mission start (validate positions)
    \end{itemize}

    \item \textbf{Prioritize TRPs}: Place highest-threat areas first in list
    \begin{itemize}
        \item Most likely threat areas visited more frequently (earlier in loop)
        \item Less critical areas placed later in list
    \end{itemize}

    \item \textbf{Appropriate Dwell Times}:
    \begin{itemize}
        \item \textbf{Short dwell (10-15s)}: Quick status checks, friendly checkpoints
        \item \textbf{Medium dwell (30-45s)}: Normal threat areas, standard surveillance
        \item \textbf{Long dwell (60-120s)}: High-threat areas, detailed observation required
    \end{itemize}

    \item \textbf{Combine with Manual}: Use TRP scan for routine surveillance, switch to Manual mode when threat detected
\end{enumerate}

% ============================================================================
\section{TRP \& Scan Zone Selection (Buttons 14/16)}
% ============================================================================

\subsection{Purpose}

When multiple TRP pages or sector scan zones are defined, operators can quickly cycle through them using dedicated joystick buttons \textbf{without accessing the menu system}.

\textbf{Benefits}:
\begin{itemize}
    \item Rapid zone/page switching during operations
    \item Hands remain on joystick (no menu navigation required)
    \item Eyes remain on video feed (situational awareness maintained)
    \item Faster reaction to threats in different sectors
\end{itemize}

\subsection{Button Functions}

\begin{itemize}
    \item \button{Button 14}: Select Next TRP page or scan zone
    \item \button{Button 16}: Select Previous TRP page or scan zone
\end{itemize}

\textbf{Active Modes}: AutoSectorScan, TRP Scan

\textbf{Inactive in Manual Mode}: Buttons 14/16 have no effect in Manual mode.

\subsection{Sector Scan Zone Selection (AutoSectorScan Mode)}

\textbf{Prerequisites}:
\begin{itemize}
    \item System in AutoSectorScan mode (\osd{MODE: Sector Scan})
    \item Multiple sector scan zones defined
\end{itemize}

\begin{procedurebox}[CYCLE THROUGH SCAN ZONES]
\begin{enumerate}
    \item System is actively scanning current sector zone (e.g., "Sector 1: Front Perimeter")

    \item \textbf{Press \button{Button 14} (Next)}:
    \begin{itemize}
        \item System immediately aborts current scan
        \item Loads next defined sector zone (e.g., "Sector 2: East Side")
        \item Gimbal slews to left limit of new zone
        \item OSD displays: \osd{SECTOR: [Zone Name]}
        \item Scanning resumes in new zone
    \end{itemize}

    \item \textbf{Press \button{Button 16} (Previous)}:
    \begin{itemize}
        \item Returns to previous sector zone in list
        \item Same behavior as Button 14 but reverse direction through zone list
    \end{itemize}

    \item \textbf{Zone List Wrapping}:
    \begin{itemize}
        \item At last zone: Button 14 wraps to first zone
        \item At first zone: Button 16 wraps to last zone
    \end{itemize}
\end{enumerate}
\end{procedurebox}

\textbf{Example Scenario}:
\begin{lstlisting}
Zones defined: "Front Gate", "East Perimeter", "Rear Area"

Currently scanning: "Front Gate"
Press Button 14 → Switch to "East Perimeter"
Press Button 14 → Switch to "Rear Area"
Press Button 14 → Wrap to "Front Gate"
\end{lstlisting}

\subsection{TRP Page Selection (TRP Scan Mode)}

\textbf{Prerequisites}:
\begin{itemize}
    \item System in TRP Scan mode (\osd{MODE: TRP})
    \item Multiple TRP pages defined
\end{itemize}

\textbf{TRP Page Concept}:
\begin{itemize}
    \item TRPs are organized into "pages" (groups of TRPs)
    \item Each page can hold multiple TRPs (typically 3-10 per page)
    \item Pages allow organization by mission phase, threat area, or sector
    \item Example pages: "Checkpoint Scan", "Hilltop Overwatch", "Route Monitoring"
\end{itemize}

\begin{procedurebox}[CYCLE THROUGH TRP PAGES]
\begin{enumerate}
    \item System is visiting TRPs in current page (e.g., "Page 1: Checkpoints")

    \item \textbf{Press \button{Button 14} (Next)}:
    \begin{itemize}
        \item System immediately aborts current TRP dwell
        \item Loads next TRP page (e.g., "Page 2: Hilltops")
        \item Gimbal slews to first TRP in new page
        \item OSD displays: \osd{TRP PAGE: [Name]} and \osd{TRP 1: [TRP Name]}
        \item Scanning resumes with TRPs from new page
    \end{itemize}

    \item \textbf{Press \button{Button 16} (Previous)}:
    \begin{itemize}
        \item Returns to previous TRP page in list
        \item Begins scanning at first TRP of that page
    \end{itemize}

    \item \textbf{Page List Wrapping}:
    \begin{itemize}
        \item At last page: Button 14 wraps to first page
        \item At first page: Button 16 wraps to last page
    \end{itemize}
\end{enumerate}
\end{procedurebox}

\textbf{Example Scenario}:
\begin{lstlisting}
Pages defined:
  Page 1 "Checkpoints": Gate 1, Gate 2, Gate 3
  Page 2 "Overwatch": Hill 203, Bunker, Tower
  Page 3 "Route": Bridge, Intersection, Curve

Currently on Page 1, visiting "Gate 2"
Press Button 14 → Switch to Page 2, visit "Hill 203"
Press Button 14 → Switch to Page 3, visit "Bridge"
Press Button 14 → Wrap to Page 1, visit "Gate 1"
\end{lstlisting}

\subsection{If Only One Zone/Page Defined}

\textbf{Behavior}:
\begin{itemize}
    \item Buttons 14/16 have no effect (no alternative zone/page to switch to)
    \item OSD continues displaying current zone/page name
    \item No error message displayed
\end{itemize}

\subsection{Operational Use Cases}

\textbf{Use Case 1: Threat Direction Change}
\begin{enumerate}
    \item AutoSectorScan mode monitoring front perimeter
    \item Threat report from east side
    \item Press \button{Button 14} to switch to "East Perimeter" scan zone
    \item Immediately begin monitoring east without menu access
\end{enumerate}

\textbf{Use Case 2: Mission Phase Transition}
\begin{enumerate}
    \item TRP Scan mode checking "Route Monitoring" page (convoy departure)
    \item Convoy departed, switch to overwatch phase
    \item Press \button{Button 14} to load "Hilltop Overwatch" page
    \item System now monitors overwatch TRPs relevant to current phase
\end{enumerate}

\textbf{Use Case 3: Shared Sectors (Multi-Operator)}
\begin{enumerate}
    \item Multiple operators on vehicle, each assigned sector
    \item Operator takes over from another shift
    \item Quickly press \button{Button 14/16} to find assigned sector
    \item Resume surveillance without supervisor assistance
\end{enumerate}

\section{Motion Mode Quick Reference}

\small
\begin{longtable}{|L{0.20\textwidth}|L{0.25\textwidth}|L{0.20\textwidth}|L{0.28\textwidth}|}
\hline
\rowcolor{milblue!20}
\textbf{Mode} & \textbf{Use Case} & \textbf{Gimbal Control} & \textbf{Exit to Manual} \\
\hline
Manual & Direct engagement & Joystick & N/A (already manual) \\
\hline
AutoSectorScan & Perimeter surveillance & Automatic & Press Button 11/13 or Button 4 (tracking) \\
\hline
TRP Scan & Checkpoint monitoring & Automatic & Press Button 11/13 or Button 4 (tracking) \\
\hline
\end{longtable}

\textbf{Emergency Return to Manual}: Press Button 11/13 repeatedly until \osd{MODE: Manual} displays

\section{Zone Management - Sector Scans}

\subsection{Defining Sector Scan Zones}

Sector scan zones set the \textbf{left and right azimuth limits} and \textbf{elevation} for AutoSectorScan mode.

\textbf{Access}: Main Menu → Zone Definitions → Sector Scans

\subsubsection{Sector Scan Zone Parameters}

\small
\begin{longtable}{|L{0.25\textwidth}|L{0.35\textwidth}|L{0.33\textwidth}|}
\hline
\rowcolor{milblue!20}
\textbf{Parameter} & \textbf{Description} & \textbf{Typical Value} \\
\hline
Name & Zone identifier & "Front Gate", "Perimeter East" \\
\hline
Left Limit (Az) & Starting azimuth & 045° \\
\hline
Right Limit (Az) & Ending azimuth & 135° \\
\hline
Elevation & Scan elevation angle & 0° (horizon) \\
\hline
Scan Speed & Degrees per second & 5°/sec \\
\hline
Active & Enable/disable zone & ON / OFF \\
\hline
\end{longtable}

\subsubsection{Creating a Sector Scan Zone}

\textbf{Method 1: Manual Entry} (via menu)

\begin{procedurebox}[MANUAL SECTOR CREATION]
\begin{enumerate}
    \item \textbf{Access Menu}: \button{MENU ✓} → Zone Definitions → Sector Scans → Add New Sector
    \item \textbf{Enter Name}: Use ▲/▼ to enter name characters, \button{MENU ✓} to confirm
    \item \textbf{Set Left Limit}:
    \begin{itemize}
        \item Method A: Slew gimbal to desired left position, press \button{MENU ✓} to "Capture Current Position"
        \item Method B: Manually enter azimuth value using ▲/▼ (000-359°)
    \end{itemize}
    \item \textbf{Set Right Limit}: Same as left limit
    \item \textbf{Set Elevation}: Enter elevation angle (typically 0°)
    \item \textbf{Set Scan Speed}: Enter degrees/second (5°/sec recommended)
    \item \textbf{Enable Zone}: Set Active = ON
    \item \textbf{Save}: \button{MENU ✓} → "Save Sector Scan"
\end{enumerate}
\end{procedurebox}

\textbf{Method 2: Quick Capture} (using joystick during manual mode)

\begin{procedurebox}[QUICK SECTOR CAPTURE]
\begin{enumerate}
    \item \textbf{Position Gimbal at Left Limit}:
    \begin{itemize}
        \item Use Manual mode to slew gimbal to desired left limit of scan area
        \item Press and hold joystick function button for 2 seconds
        \item OSD displays: \osd{LEFT LIMIT CAPTURED}
    \end{itemize}

    \item \textbf{Position Gimbal at Right Limit}:
    \begin{itemize}
        \item Slew gimbal to desired right limit of scan area
        \item Press and hold function button for 2 seconds
        \item OSD displays: \osd{RIGHT LIMIT CAPTURED - SECTOR ZONE CREATED}
    \end{itemize}

    \item \textbf{System Auto-Creates Zone}:
    \begin{itemize}
        \item Default name: "Sector X" (where X = next available number)
        \item Default elevation: Current gimbal elevation at time of right limit capture
        \item Default scan speed: 5°/sec
    \end{itemize}

    \item \textbf{Edit if Needed}: Access menu to rename zone or adjust parameters
\end{enumerate}
\end{procedurebox}

\subsection{Activating / Deactivating Sector Zones}

\textbf{Multiple Zones}:
\begin{itemize}
    \item You can define multiple sector scan zones
    \item All zones remain in memory
    \item Use \button{Button 14/16} to switch between zones during AutoSectorScan
\end{itemize}

\textbf{To Activate a Specific Zone via Menu}:
\begin{enumerate}
    \item \button{MENU ✓} → Zone Definitions → Sector Scans
    \item Use ▲/▼ to select desired zone
    \item \button{MENU ✓} → "Set as Active"
    \item OSD displays: \osd{SECTOR ZONE: [Name] ACTIVE}
\end{enumerate}

\textbf{To Deactivate All Sectors}:
\begin{enumerate}
    \item \button{MENU ✓} → Zone Definitions → Sector Scans
    \item Select active zone
    \item \button{MENU ✓} → "Deactivate"
    \item AutoSectorScan mode will not function until a zone is re-activated
\end{enumerate}

\section{Zone Management - Target Reference Points}

\subsection{Defining TRPs}

TRPs are \textbf{fixed azimuth/elevation positions} the system can automatically slew to during TRP Scan mode.

\textbf{Access}: Main Menu → Zone Definitions → TRPs

\subsubsection{TRP Parameters}

\small
\begin{longtable}{|L{0.25\textwidth}|L{0.35\textwidth}|L{0.33\textwidth}|}
\hline
\rowcolor{milblue!20}
\textbf{Parameter} & \textbf{Description} & \textbf{Typical Value} \\
\hline
Name & TRP identifier & "Gate 1", "Bunker", "Hill 203" \\
\hline
Azimuth & Direction to TRP & 090° \\
\hline
Elevation & Angle to TRP & +5° \\
\hline
Dwell Time & Observation time & 30 seconds \\
\hline
TRP Page & Which page/group & "Checkpoints", "Overwatch" \\
\hline
Active & Enable/disable & ON / OFF \\
\hline
\end{longtable}

\subsubsection{Creating a TRP}

\textbf{Method 1: Capture Current Position}

\begin{procedurebox}[TRP CAPTURE]
\begin{enumerate}
    \item \textbf{Manual Mode}: Slew gimbal to desired TRP location
    \item \textbf{Zoom/Focus}: Zoom in on exact point of interest, center in reticle
    \item \textbf{Access Menu}: \button{MENU ✓} → Zone Definitions → TRPs → Add TRP
    \item \textbf{Capture Position}: Select "Capture Current Position", \button{MENU ✓} to confirm
    \item \textbf{Enter Name}: Use ▲/▼ to enter TRP name (e.g., "Gate 2"), \button{MENU ✓} to confirm
    \item \textbf{Assign to Page}: Select TRP page (or create new page)
    \item \textbf{Set Dwell Time}: Enter seconds (5-120), Default: 30 seconds
    \item \textbf{Enable}: Set Active = ON
    \item \textbf{Save}: \button{MENU ✓} → "Save TRP"
\end{enumerate}
\end{procedurebox}

\textbf{Method 2: Manual Coordinate Entry}

\begin{procedurebox}[TRP MANUAL ENTRY]
\begin{enumerate}
    \item \textbf{Access Menu}: \button{MENU ✓} → Zone Definitions → TRPs → Add TRP
    \item \textbf{Enter Azimuth}: Use ▲/▼ to enter degrees (000-359)
    \item \textbf{Enter Elevation}: Use ▲/▼ to enter degrees (-20 to +60)
    \item \textbf{Continue}: Enter name, assign page, set dwell time, enable, and save
\end{enumerate}
\end{procedurebox}

\subsection{Managing TRP Pages}

\textbf{TRP Page Concept}:
\begin{itemize}
    \item TRPs are organized into pages (groups)
    \item Each page represents a set of related TRPs
    \item Switch between pages using \button{Button 14/16} during TRP Scan mode
\end{itemize}

\textbf{Creating TRP Pages}:
\begin{enumerate}
    \item \button{MENU ✓} → Zone Definitions → TRPs → TRP Pages → Add Page
    \item Enter page name (e.g., "Checkpoint Scan")
    \item Save page
    \item Assign TRPs to this page when creating/editing TRPs
\end{enumerate}

\textbf{TRP Sequence Within Page}:
\begin{itemize}
    \item TRPs are visited in the order they appear in page list
    \item To reorder TRPs within a page:
    \begin{enumerate}
        \item \button{MENU ✓} → Zone Definitions → TRPs → [Select Page]
        \item Select TRP from list
        \item \button{MENU ✓} → "Move Up" or "Move Down"
        \item Save changes
    \end{enumerate}
\end{itemize}

\textbf{Editing TRPs}:
\begin{itemize}
    \item Select TRP from list
    \item \button{MENU ✓} → "Edit TRP"
    \item Modify parameters (position, dwell time, page assignment, etc.)
    \item Save changes
\end{itemize}

\textbf{Deleting TRPs}:
\begin{itemize}
    \item Select TRP from list
    \item \button{MENU ✓} → "Delete TRP"
    \item Confirm deletion (warning: cannot be undone)
\end{itemize}

\section{Zone Management - No-Fire \& No-Traverse}

\subsection{Viewing No-Fire Zones}

\textbf{Access}: Main Menu → Zone Definitions → No-Fire Zones

\textbf{Display}:
\begin{itemize}
    \item List of all defined no-fire zones
    \item Each zone shows:
    \begin{itemize}
        \item Name (e.g., "Friendly FOB", "Civilian Area 1", "Hospital Zone")
        \item Boundary type (Polygon, Circle, Arc)
        \item Coordinates/dimensions
        \item Active status (ON/OFF)
    \end{itemize}
\end{itemize}

\textbf{Operator Permission}:
\begin{itemize}
    \item \textbf{CAN}: View zones, see boundaries on map overlay (if available)
    \item \textbf{CANNOT}: Modify boundaries, delete zones, override zones
\end{itemize}

\begin{notebox}
No-fire zone modification usually requires commander/supervisor authorization. Operators can view but not edit.
\end{notebox}

\subsection{Viewing No-Traverse Zones}

\textbf{Access}: Main Menu → Zone Definitions → No-Traverse Zones

\textbf{Display}:
\begin{itemize}
    \item List of all defined no-traverse zones
    \item Each zone shows:
    \begin{itemize}
        \item Name (e.g., "Rear 90°", "Antenna Area", "Vehicle Structure")
        \item Azimuth limits (e.g., 270° to 360° and 0° to 90°)
        \item Active status
    \end{itemize}
\end{itemize}

\textbf{Purpose Reminder}:
\begin{itemize}
    \item No-traverse zones prevent gimbal movement into restricted areas
    \item Protects vehicle structure, antennas, equipment, personnel on vehicle
    \item System will NOT allow gimbal to enter no-traverse zone (hard stop at boundary)
\end{itemize}

\section{Saving \& Loading Zone Configurations}

\subsection{Saving Zone Configuration}

\textbf{Purpose}: Save all zones (sector scans, TRP pages, no-fire, no-traverse) to file for later use

\begin{procedurebox}[SAVE CONFIGURATION]
\begin{enumerate}
    \item \textbf{Access Menu}: \button{MENU ✓} → Zone Definitions → Save/Load → Save Configuration
    \item \textbf{Enter Filename}: Use ▲/▼ to enter filename (e.g., "MISSION\_20250206\_PERIMETER")
    \item \textbf{Confirm Save}: \button{MENU ✓} → "Save"
    \item OSD displays: \osd{ZONE CONFIG SAVED}
\end{enumerate}
\end{procedurebox}

\textbf{File Location}: Saved to internal non-volatile storage (typically /configs/zones/)

\textbf{What Is Saved}:
\begin{itemize}
    \item All sector scan zones (with limits, speeds, names)
    \item All TRP pages and TRPs (positions, dwell times)
    \item No-fire zones (if authorized to save)
    \item No-traverse zones
    \item Active zone selections
\end{itemize}

\subsection{Loading Zone Configuration}

\textbf{Purpose}: Load previously saved zone configuration for a specific mission

\begin{procedurebox}[LOAD CONFIGURATION]
\begin{enumerate}
    \item \textbf{Access Menu}: \button{MENU ✓} → Zone Definitions → Save/Load → Load Configuration
    \item \textbf{Select File}: Use ▲/▼ to browse saved configurations (filenames displayed)
    \item \textbf{Preview}: Select file, press \button{MENU ✓} → "Preview" to see contents before loading
    \item \textbf{Confirm Load}: \button{MENU ✓} → "Load"
    \item OSD displays: \osd{ZONE CONFIG LOADED: [Filename]}
\end{enumerate}
\end{procedurebox}

\begin{warningbox}
Loading a configuration \textbf{OVERWRITES} current zones in memory. Save current zones first if you need to preserve them.
\end{warningbox}

\subsection{Default Zone Configuration}

\textbf{Default Zones}:
\begin{itemize}
    \item System ships with default no-traverse zones (vehicle-specific, protect vehicle structure)
    \item Default no-fire zones may be empty (mission-dependent, set by command)
    \item No default sector scan zones or TRPs (mission-specific)
\end{itemize}

\textbf{Restoring Defaults}:
\begin{enumerate}
    \item \button{MENU ✓} → Zone Definitions → Save/Load → Restore Defaults
    \item Confirmation prompt: "RESTORE DEFAULT ZONES? Current zones will be lost."
    \item \button{MENU ✓} → "YES, Restore"
    \item OSD displays: \osd{DEFAULTS RESTORED}
\end{enumerate}

\section{Surveillance Best Practices}

\subsection{Choosing the Right Mode}

\small
\begin{longtable}{|L{0.30\textwidth}|L{0.25\textwidth}|L{0.38\textwidth}|}
\hline
\rowcolor{milblue!20}
\textbf{Situation} & \textbf{Recommended Mode} & \textbf{Rationale} \\
\hline
Direct threat engagement & Manual & Full control, immediate precise response \\
\hline
Perimeter watch (quiet period) & AutoSectorScan & Automated, frees operator attention for other tasks \\
\hline
Checkpoint routine monitoring & TRP Scan & Efficient for fixed location checks \\
\hline
High-threat area scan & Manual & Requires operator judgment and rapid reaction \\
\hline
Long duration surveillance & AutoSectorScan or TRP & Reduces operator fatigue \\
\hline
\end{longtable}

\subsection{Combining Modes with Tracking}

\textbf{Workflow Example - Threat Detection During AutoSectorScan}:
\begin{enumerate}
    \item Start in AutoSectorScan mode (perimeter surveillance)
    \item System scanning "Front Perimeter" sector (left-right continuously)
    \item Threat detected during scan (e.g., vehicle approaching)
    \item Press \button{Button 4} (tracking button):
    \begin{itemize}
        \item System switches to Manual mode
        \item Enters tracking acquisition
        \item Yellow acquisition gate appears
    \end{itemize}
    \item Size gate around threat using D-Pad
    \item Press \button{Button 4} again (double-click) to lock onto threat
    \item System tracks threat automatically (Active Lock)
    \item Engage or monitor as threat is tracked
    \item After engagement or threat passes:
    \begin{itemize}
        \item Double-click \button{Button 4} to abort tracking
        \item Press \button{Button 11/13} to return to AutoSectorScan mode
        \item Resume perimeter surveillance
    \end{itemize}
\end{enumerate}

\subsection{Zone Discipline}

\textbf{Before Mission}:
\begin{checklist}
    \item Load appropriate zone configuration for mission area
    \item Verify no-fire zones match current ROE (Rules of Engagement)
    \item Test sector scans and TRP pages (validate all positions)
    \item Brief all operators on zone boundaries and restrictions
    \item Confirm \button{Button 14/16} zone selection works correctly
\end{checklist}

\textbf{During Mission}:
\begin{checklist}
    \item Respect all zone warnings displayed on OSD
    \item Never attempt to override no-fire zones without proper authorization
    \item Report zone boundary errors or outdated zones to command immediately
    \item Update TRPs as mission evolves (if authorized and situation allows)
    \item Use \button{Button 14/16} to adapt surveillance to changing threat directions
\end{checklist}

\textbf{After Mission}:
\begin{checklist}
    \item Save zone configuration if modified during mission
    \item Debrief on zone effectiveness (were zones adequate? too restrictive?)
    \item Recommend adjustments for future missions in same area
    \item Note any false alarms or zone violations in operator log
\end{checklist}

\section{Student Review Questions}

\begin{enumerate}
    \item What are the three motion modes available on the RCWS? (Note: Radar not currently installed)
    \item What joystick buttons (by number) are used to cycle between motion modes?
    \item What is the purpose of Auto Sector Scan mode?
    \item What does TRP stand for?
    \item What is the default dwell time at a TRP?
    \item What joystick buttons (by number) are used to select next/previous TRP page or scan zone?
    \item During AutoSectorScan, how do you switch to a different scan zone without using the menu?
    \item How do you create a sector scan zone using the quick capture method?
    \item Can operators modify no-fire zone boundaries?
    \item What happens when you load a zone configuration file?
    \item Which motion mode is recommended for direct threat engagement?
    \item How do you reorder TRPs within a TRP page?
    \item What happens when you press Button 4 (tracking) while in AutoSectorScan mode?
    \item If only one TRP page is defined, what happens when you press Button 14?
    \item Why are TRPs organized into pages instead of a single list?
\end{enumerate}
