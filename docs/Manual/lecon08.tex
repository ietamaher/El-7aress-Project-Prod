% ============================================================================
% LESSON 8 - EMERGENCY PROCEDURES
% ============================================================================
\lesson{8 - EMERGENCY PROCEDURES}

\begin{tabular}{@{}ll@{}}
\textbf{Duration:} & 3 hours (Classroom 1h + Practical 2h) \\
\textbf{Type:} & Classroom + Practical \\
\textbf{References:} & Emergency procedures manual, safety guidelines \\
\end{tabular}

\section{LEARNING OBJECTIVES}

Upon completion of this lesson, operators will be able to:
\begin{itemize}
    \item Execute Emergency Stop (E-Stop)
    \item Perform emergency tracking abort
    \item Respond to weapon malfunctions
    \item Handle runaway gimbal
    \item Execute emergency system shutdown
\end{itemize}

% ============================================================================
\section{EMERGENCY STOP (E-STOP)}

\subsection{E-STOP BUTTON}

\subsubsection{Location}
\textbf{Red mushroom button on Control Panel}

\subsubsection{When to Use}

Use E-Stop \textbf{IMMEDIATELY} when:
\begin{itemize}
    \item ❌ Personnel in line of fire
    \item ❌ Gimbal moving toward restricted area
    \item ❌ Runaway/uncontrolled motion
    \item ❌ Any immediate danger
\end{itemize}

\subsection{E-STOP ACTIVATION}

\begin{procedurebox}[EMERGENCY STOP ACTIVATION]
\textbf{Action:} \textbf{STRIKE E-STOP BUTTON}

\textbf{Effect (IMMEDIATE):}
\begin{enumerate}
    \item All gimbal motion halted
    \item Weapon fire control disabled
    \item Tracking stopped
    \item System functions locked out
    \item OSD: \textcolor{milred}{\osd{EMERGENCY STOP ACTIVE}} (RED)
\end{enumerate}
\end{procedurebox}

\begin{warningbox}[E-STOP IS IRREVERSIBLE]
Once activated, E-Stop cannot be overridden. System remains locked until manually reset.
\end{warningbox}

\subsubsection{E-Stop System Behavior}

\textbf{What Stops:}
\begin{itemize}
    \item All servo motor power
    \item Weapon arming circuits
    \item Tracking algorithms
    \item Automatic motion modes
\end{itemize}

\textbf{What Continues:}
\begin{itemize}
    \item Power to cameras (video feed maintained)
    \item Power to control panels
    \item System monitoring (can view System Status)
    \item Communication systems
\end{itemize}

% ----------------------------------------------------------------------------
\subsection{E-STOP RESET}

\subsubsection{Pre-Reset Safety Verification}

Only reset after verifying \textbf{ALL} items:

\begin{checklist}
    \item Immediate danger cleared
    \item Personnel clear
    \item No equipment damage
    \item Cause identified
    \item Safe to resume
\end{checklist}

\begin{procedurebox}[E-STOP RESET PROCEDURE]
\begin{enumerate}
    \item Twist E-STOP button clockwise
    \item Button pops out
    \item System resets (takes 5-10 seconds)
    \item Check System Status for faults
    \item Perform health check before resuming ops
\end{enumerate}
\end{procedurebox}

\begin{warningbox}[DO NOT RESET IF]
\textbf{DO NOT RESET E-STOP IF:}
\begin{itemize}
    \item Cause unknown
    \item Damage visible
    \item Personnel in danger
    \item Maintenance required
\end{itemize}
\end{warningbox}

\subsubsection{Post-Reset Verification}

After E-Stop reset, verify:
\begin{checklist}
    \item All devices show "Connected" in System Status
    \item No fault flags present
    \item Gimbal responds to joystick inputs (test in manual mode)
    \item Tracking system functional (test acquisition)
    \item Safety systems operational (test Master Arm/Disarm)
\end{checklist}

% ============================================================================
\section{EMERGENCY TRACKING ABORT}

\subsection{WHEN TO ABORT TRACKING}

Abort tracking \textbf{immediately} when:

\begin{itemize}
    \item ❌ Tracking wrong target
    \item ❌ Target enters restricted zone
    \item ❌ Friendlies near target
    \item ❌ Lost positive ID
    \item ❌ System erratic
    \item ❌ Civilian identified
\end{itemize}

\subsection{TRACKING ABORT PROCEDURE}

\begin{procedurebox}[EMERGENCY TRACKING ABORT]
\textbf{Procedure:} \textbf{DOUBLE-CLICK \button{Button 4}} (<500ms)

\textbf{Effect:}
\begin{itemize}
    \item Tracking stops immediately
    \item Tracking gate disappears
    \item Returns to Manual mode
    \item Weapon fire inhibited
\end{itemize}
\end{procedurebox}

\begin{notebox}
No Dead Man Switch required for abort. Works from any phase.
\end{notebox}

\subsubsection{Tracking Abort vs. E-Stop}

\textbf{Use Tracking Abort When:}
\begin{itemize}
    \item Wrong target being tracked
    \item Need to quickly disengage tracking
    \item System otherwise functioning normally
\end{itemize}

\textbf{Use E-Stop When:}
\begin{itemize}
    \item Immediate danger to personnel
    \item System malfunction (runaway gimbal)
    \item Need to halt \textbf{all} system motion
\end{itemize}

\subsection{Post-Abort Actions}

After aborting tracking:

\begin{enumerate}
    \item Verify gimbal in Manual mode
    \item Assess situation (why was abort necessary?)
    \item Re-acquire correct target (if applicable)
    \item Report incident (if target misidentification occurred)
\end{enumerate}

% ============================================================================
\section{WEAPON EMERGENCY PROCEDURES}

\subsection{ACCIDENTAL DISCHARGE}

\subsubsection{Recognition}

\textbf{If weapon fires unintentionally:}
\begin{itemize}
    \item Fire button not pressed, but weapon firing
    \item Single shot when burst intended (or vice versa)
    \item Weapon fires on Master Arm engagement
\end{itemize}

\subsubsection{Immediate Actions}

\begin{procedurebox}[ACCIDENTAL DISCHARGE RESPONSE]
\textbf{Immediate Actions (in order):}
\begin{enumerate}
    \item \textbf{Release \button{Button 5}} (Fire) - cease fire
    \item \textbf{Release \button{Button 0}} (Master Arm) - disarm
    \item \textbf{Point gimbal to safe direction} (skyward)
    \item \textbf{E-STOP} if motion continues
    \item \textbf{Turn Station power OFF}
\end{enumerate}
\end{procedurebox}

\begin{warningbox}[CRITICAL SAFETY VIOLATION]
Accidental discharge is a \textbf{CRITICAL SAFETY EVENT}. All operations must cease until investigation complete.
\end{warningbox}

\subsubsection{Post-Incident Actions}

\begin{checklist}
    \item Ensure weapon safe
    \item Check for damage/casualties
    \item Notify chain of command immediately
    \item Preserve system logs
    \item Do NOT resume operations until investigation complete
\end{checklist}

\textbf{Investigation Requirements:}
\begin{itemize}
    \item Maintenance personnel must inspect fire control circuits
    \item Software logs must be reviewed
    \item Trigger mechanism must be tested
    \item Safety interlocks must be verified
    \item Commander's authorization required before resuming operations
\end{itemize}

% ----------------------------------------------------------------------------
\subsection{WEAPON JAM / MALFUNCTION}

\subsubsection{Symptoms}

\begin{itemize}
    \item Weapon fires then stops mid-burst
    \item Unusual sounds (click, no firing)
    \item Fire button pressed, no response
    \item Partial firing (rounds not feeding properly)
\end{itemize}

\subsubsection{Immediate Actions}

\begin{procedurebox}[WEAPON JAM RESPONSE]
\begin{enumerate}
    \item \textbf{Release Fire button} (cease trigger)
    \item \textbf{Release Master Arm} (disarm)
    \item \textbf{Turn Gun Arm switch to SAFE}
    \item \textbf{Point gimbal to safe direction}
    \item \textbf{Notify command} - weapon malfunction
\end{enumerate}
\end{procedurebox}

\begin{warningbox}[DO NOT ATTEMPT TO CLEAR JAM]
\textbf{DO NOT:}
\begin{itemize}
    \item ❌ Attempt to clear jam from operator station
    \item ❌ Continue firing attempts
    \item ❌ Inspect weapon without proper clearance
\end{itemize}

\textbf{Action:} Weapon maintenance personnel required.
\end{warningbox}

\subsubsection{Operator Limitations}

\textbf{Operators are NOT authorized to:}
\begin{itemize}
    \item Clear weapon jams
    \item Perform weapon maintenance
    \item Inspect weapon mechanisms
    \item Override weapon safety interlocks
\end{itemize}

\textbf{Operators ARE authorized to:}
\begin{itemize}
    \item Safe the weapon (Gun Arm to SAFE)
    \item Report malfunction symptoms
    \item Secure the area
    \item Await maintenance personnel
\end{itemize}

% ============================================================================
\section{RUNAWAY GIMBAL}

\subsection{SYMPTOMS}

\textbf{Runaway gimbal indicators:}
\begin{itemize}
    \item Gimbal moves without joystick input
    \item Cannot stop gimbal with joystick
    \item Gimbal moving toward no-traverse zone
    \item Erratic, unpredictable motion
    \item Gimbal oscillating rapidly
\end{itemize}

\subsection{IMMEDIATE ACTIONS}

\begin{procedurebox}[RUNAWAY GIMBAL RESPONSE]
\textbf{Immediate Actions (in order):}
\begin{enumerate}
    \item \textbf{E-STOP} (strike immediately)
    \item \textbf{Verify E-Stop engaged} (button latched, motion stopped)
    \item \textbf{Turn Station power OFF}
    \item \textbf{Notify maintenance} - servo malfunction
\end{enumerate}
\end{procedurebox}

\begin{warningbox}[DO NOT RESET]
Do NOT reset E-Stop until:
\begin{itemize}
    \item Maintenance personnel inspect system
    \item Cause identified and resolved
    \item Safety verified
\end{itemize}
\end{warningbox}

\subsection{Runaway Gimbal Causes}

\textbf{Common Causes:}
\begin{itemize}
    \item Servo driver malfunction
    \item Software control loop error
    \item Encoder feedback failure
    \item Electrical short circuit
    \item Control signal interference
\end{itemize}

\textbf{All require maintenance intervention.}

\subsection{Prevention}

\textbf{Reduce risk of runaway gimbal:}
\begin{itemize}
    \item Perform daily pre-operation checks
    \item Monitor servo torque and temperature
    \item Report unusual gimbal behavior immediately
    \item Keep E-Stop button clear and accessible
    \item Practice E-Stop activation during training
\end{itemize}

% ============================================================================
\section{LOST COMMUNICATION}

\subsection{OPERATOR TO COMMAND}

\subsubsection{Symptoms}
\begin{itemize}
    \item Radio/intercom dead
    \item No response to radio calls
    \item Static or interference on all channels
\end{itemize}

\subsubsection{Actions}

\begin{procedurebox}[LOST COMMUNICATION WITH COMMAND]
\begin{checklist}
    \item Switch to backup radio
    \item Use hand signals (if line of sight)
    \item Continue mission per SOP (if applicable)
    \item Return to safe position (if no comms recovery)
\end{checklist}
\end{procedurebox}

\textbf{Standing Orders:}
\begin{itemize}
    \item If communications lost during active engagement: Complete engagement, then return to safe position
    \item If communications lost during surveillance: Continue surveillance, attempt recovery every 5 minutes
    \item If communications not recovered within 30 minutes: Return to base/rally point
\end{itemize}

% ----------------------------------------------------------------------------
\subsection{SYSTEM TO DEVICES}

\subsubsection{Symptoms}
\begin{itemize}
    \item Device disconnected (LRF, Camera, Servo)
    \item "Device Offline" message in System Status
    \item Loss of specific functionality (ranging, video, motion)
\end{itemize}

\subsubsection{Actions}

\begin{procedurebox}[LOST DEVICE COMMUNICATION]
\begin{enumerate}
    \item Check System Status (identify device)
    \item Power cycle system (soft reboot)
    \item If device does not reconnect: Operate without device (degraded mode)
    \item Notify maintenance
\end{enumerate}
\end{procedurebox}

\subsubsection{Mission Impact}

\spectable{
\begin{tabular}{|L{0.25\textwidth}|L{0.65\textwidth}|}
\hline
\rowcolor{milblue!20}
\textbf{Device Offline} & \textbf{Mission Impact} \\
\hline
\textbf{LRF offline} & Manual range estimation only, no LAC. Reduced accuracy at long range. \\
\hline
\textbf{Camera offline} & Switch to alternate camera. If both offline: mission abort. \\
\hline
\textbf{Servo offline} & Mission abort - cannot control gimbal. \\
\hline
\textbf{IMU offline} & No stabilization. Manual mode only. Degraded tracking. \\
\hline
\textbf{Joystick offline} & Use DCU backup controls (limited functionality). \\
\hline
\end{tabular}
}

% ============================================================================
\section{EMERGENCY SYSTEM SHUTDOWN}

\subsection{WHEN TO PERFORM FULL SHUTDOWN}

Perform emergency shutdown when:
\begin{itemize}
    \item Fire/smoke from system
    \item Multiple critical faults
    \item Safety concern requiring immediate powerdown
    \item Ordered by command
    \item Electrical burning smell
    \item Unusual arcing or sparking
\end{itemize}

\subsection{EMERGENCY SHUTDOWN PROCEDURE}

\begin{procedurebox}[EMERGENCY SYSTEM SHUTDOWN]
\begin{enumerate}
    \item \textbf{E-STOP} (if not already engaged)
    \item \textbf{Turn Gun Arm to SAFE}
    \item \textbf{Turn Station Enable switch to OFF}
    \item \textbf{Turn main power switch to OFF}
    \item \textbf{Disconnect external power} (if directed)
\end{enumerate}
\end{procedurebox}

\subsection{Post-Shutdown Actions}

\begin{checklist}
    \item Ensure all motion stopped
    \item Ensure weapon safe
    \item Notify command/maintenance
    \item Secure area (prevent unauthorized restart)
    \item Document circumstances leading to shutdown
\end{checklist}

\begin{warningbox}[DO NOT RESTART]
Do NOT attempt to restart system until:
\begin{itemize}
    \item Cause of emergency identified
    \item Maintenance personnel inspect system
    \item Commander authorizes restart
\end{itemize}
\end{warningbox}

\subsection{Fire Response}

\textbf{If fire or smoke observed:}
\begin{enumerate}
    \item Perform emergency shutdown (above procedure)
    \item Evacuate immediate area
    \item Alert fire response team
    \item Use fire extinguisher if:
    \begin{itemize}
        \item Fire is small and contained
        \item Safe to approach
        \item Proper extinguisher type available (electrical fires: CO2 or dry chemical)
    \end{itemize}
    \item Do NOT use water on electrical fires
\end{enumerate}

% ============================================================================
\section{EMERGENCY QUICK REFERENCE}

\begin{quickref}
\begin{longtable}{|L{0.22\textwidth}|L{0.35\textwidth}|L{0.30\textwidth}|}
\hline
\rowcolor{milred!20}
\textbf{Emergency} & \textbf{Action} & \textbf{Button/Switch} \\
\hline
\endfirsthead

\hline
\rowcolor{milred!20}
\textbf{Emergency} & \textbf{Action} & \textbf{Button/Switch} \\
\hline
\endhead

\textbf{Immediate Danger} & Emergency Stop & E-STOP (red mushroom) \\
\hline
\textbf{Wrong Target} & Tracking Abort & \button{Button 4} (double-click) \\
\hline
\textbf{Accidental Fire} & Cease Fire, Disarm & Release \button{Button 5}, Release \button{Button 0} \\
\hline
\textbf{Weapon Jam} & Disarm, Safe Weapon & Release \button{Button 0}, Gun Arm SAFE \\
\hline
\textbf{Runaway Gimbal} & Emergency Stop, Power Off & E-STOP, Station Power OFF \\
\hline
\textbf{Fire/Smoke} & Full Shutdown & E-STOP → Station OFF → Main Power OFF \\
\hline
\end{longtable}
\end{quickref}

% ============================================================================
\section{EMERGENCY RESPONSE TRAINING}

\subsection{Required Emergency Drills}

All operators must practice the following emergency responses:

\subsubsection{Drill 1: E-Stop Activation}
\begin{itemize}
    \item Practice striking E-Stop button from operating position
    \item Time to E-Stop: <1 second from recognition
    \item Verify system lockout
    \item Practice reset procedure
\end{itemize}

\subsubsection{Drill 2: Tracking Abort}
\begin{itemize}
    \item Practice double-click abort during tracking exercise
    \item Time to abort: <500ms
    \item Verify tracking stops immediately
    \item Practice re-acquisition after abort
\end{itemize}

\subsubsection{Drill 3: Accidental Discharge Response}
\begin{itemize}
    \item Simulated accidental discharge scenario
    \item Practice immediate cease-fire and disarm sequence
    \item Practice safe gimbal positioning
    \item Practice emergency shutdown
\end{itemize}

\subsubsection{Drill 4: Runaway Gimbal Response}
\begin{itemize}
    \item Simulated runaway gimbal (instructor-initiated)
    \item Practice immediate E-Stop
    \item Practice emergency shutdown
    \item Verify no attempt to reset without authorization
\end{itemize}

\subsection{Emergency Response Evaluation Criteria}

\textbf{Proficiency Standards:}
\begin{itemize}
    \item E-Stop activation: <1 second recognition-to-action time
    \item Tracking abort: <500ms double-click execution
    \item Correct procedure selection: 100\% (no procedural errors)
    \item Post-emergency verification: All steps completed
\end{itemize}

\begin{warningbox}[TRAINING REQUIREMENT]
Operators must demonstrate proficiency in all emergency procedures before operational qualification.
\end{warningbox}

% ============================================================================
\section{STRESS MANAGEMENT IN EMERGENCIES}

\subsection{Staying Calm Under Pressure}

\textbf{Emergency Response Psychology:}
\begin{itemize}
    \item Recognize stress response (tunnel vision, rapid heartbeat, loss of fine motor control)
    \item Focus on immediate action (don't overthink)
    \item Trust your training (muscle memory)
    \item Breathe (deliberate breathing reduces stress)
\end{itemize}

\subsection{Decision Making in Emergencies}

\textbf{Priority Framework:}
\begin{enumerate}
    \item \textbf{Life Safety:} Prevent injury to personnel (highest priority)
    \item \textbf{Equipment Protection:} Prevent damage to system (secondary)
    \item \textbf{Mission Continuity:} Resume operations (lowest priority)
\end{enumerate}

\textbf{When in Doubt:}
\begin{itemize}
    \item Default to most conservative action (E-Stop)
    \item Request guidance from command
    \item Do NOT resume operations until safe
\end{itemize}

% ============================================================================
% END OF LESSON 8
% ============================================================================