% ============================================================================
% LESSON 3 - MENU SYSTEM & SETTINGS
% ============================================================================
\lesson{3: MENU SYSTEM \& SETTINGS}

\noindent
\begin{tabular}{@{}ll@{}}
\textbf{Duration:} & 2 hours \\
\textbf{Type:} & Practical \\
\textbf{References:} & Menu navigation guide, configuration manual \\
\end{tabular}

\section{Introduction}

\textbf{Lesson Purpose}: This lesson teaches complete menu navigation, display configuration, and system settings management.

\textbf{Learning Objectives}:
\begin{itemize}
    \item Navigate all menu structures using DCU controls
    \item Configure reticle types and display colors
    \item Access system status information
    \item Modify operational settings safely
\end{itemize}

\section{Menu Navigation Basics}

\subsection{Accessing the Main Menu}

\textbf{Method}: Press \button{MENU ✓} button on DCU

\textbf{Result}:
\begin{itemize}
    \item Video feed dims (still visible in background)
    \item Menu window appears in center of screen
    \item Current selection is highlighted
    \item Menu title displayed at top
\end{itemize}

\subsection{Menu Controls}

\small
\begin{longtable}{|L{0.25\textwidth}|L{0.68\textwidth}|}
\hline
\rowcolor{milblue!20}
\textbf{Button} & \textbf{Function} \\
\hline
\button{MENU ✓} & Open menu / Confirm selection / Exit menu \\
\hline
\button{MENU ▲} & Move selection up / Increase value \\
\hline
\button{MENU ▼} & Move selection down / Decrease value \\
\hline
\end{longtable}

\textbf{Alternative}: Joystick HAT switch can also be used for menu navigation

\subsection{Navigation Workflow}

\textbf{5-Step Process}:

\begin{enumerate}
    \item \textbf{Open Menu}: Press \button{MENU ✓} - Main menu appears
    \item \textbf{Navigate}: Press ▲ or ▼ to highlight desired option
    \item \textbf{Select}: Press \button{MENU ✓} to enter submenu or activate option
    \item \textbf{Return}: Navigate to "Return..." option, press \button{MENU ✓}
    \item \textbf{Exit}: Continue pressing "Return..." until back at live video
\end{enumerate}

\begin{notebox}[TIP]
If you get lost in menus, keep selecting "Return..." until you exit completely.
\end{notebox}

\section{Main Menu Structure}

\subsection{Complete Menu Tree}

\begin{lstlisting}
MAIN MENU
├── --- RETICLE & DISPLAY ---
│   ├── Personalize Reticle
│   ├── Personalize Colors
│   └── Display Brightness
│
├── --- BALLISTICS ---
│   ├── Zeroing (Lesson 6)
│   ├── Clear Active Zero
│   ├── Windage (Lesson 6)
│   ├── Clear Active Windage
│   ├── Environmental Settings (Lesson 6)
│   └── Clear Environmental Settings
│
├── --- CALIBRATION ---
│   └── Preset Home Position
│
├── --- SYSTEM ---
│   ├── Zone Definitions (Lesson 4)
│   ├── Detection: ENABLED/DISABLED
│   └── Shutdown System
│
├── --- INFO ---
│   ├── Help/About
│   └── Return ...
\end{lstlisting}

\begin{notebox}[MENU UPDATE]
The menu structure includes options for Display Brightness, Calibration section with Preset Home Position, Detection toggle, and separate Clear functions for ballistic settings.
\end{notebox}

\begin{notebox}
Ballistics and Zone Management are covered in detail in later lessons. This lesson focuses on display settings and basic navigation.
\end{notebox}

\section{Reticle \& Display Settings}

\subsection{Personalize Reticle}

\textbf{Access}: Main Menu → "Personalize Reticle"

\textbf{Purpose}: Select reticle type (crosshair style) for aiming

\subsubsection{Available Reticle Types}

\textbf{1. Box Crosshair} (Default)
\begin{lstlisting}
        │
    ┌───┼───┐
────┤   +   ├────
    └───┼───┘
        │
\end{lstlisting}
\begin{itemize}
    \item Cross with surrounding box
    \item Good for general purpose and tracking
    \item Recommended for most operations
\end{itemize}

\textbf{2. Brackets Reticle}
\begin{lstlisting}
    ┌─     ─┐
        +
    └─     ─┘
\end{lstlisting}
\begin{itemize}
    \item Corner brackets with center crosshair
    \item Enhanced visibility
    \item Good for low-contrast targets
\end{itemize}

\textbf{3. Duplex Crosshair}
\begin{lstlisting}
  ████│████
──────+──────
  ████│████
\end{lstlisting}
\begin{itemize}
    \item Thick outer lines, thin center
    \item Sniper/precision style
    \item Good for long-range
\end{itemize}

\textbf{4. Fine Crosshair}
\begin{lstlisting}
      │
    ──+──
      │
\end{lstlisting}
\begin{itemize}
    \item Thin precision crosshair
    \item Minimal obstruction
    \item Best for extreme precision
\end{itemize}

\textbf{5. Chevron Reticle}
\begin{lstlisting}
      ˅
    ──+──
\end{lstlisting}
\begin{itemize}
    \item Downward pointing chevron
    \item CQB (Close Quarters Battle) style
    \item Good for rapid engagement
\end{itemize}

\subsubsection{How to Change Reticle}

\begin{procedurebox}[CHANGING RETICLE]
\begin{enumerate}
    \item Press \button{MENU ✓}
    \item Navigate to "Personalize Reticle" (▼)
    \item Press \button{MENU ✓} to enter
    \item Use ▲/▼ to highlight desired reticle
    \item Press \button{MENU ✓} to select
    \item Reticle changes immediately on screen
    \item Navigate to "Return..." (▼)
    \item Press \button{MENU ✓} to return to Main Menu
\end{enumerate}
\end{procedurebox}

\textbf{Recommendation}: Use \textbf{Box Crosshair} for combat, \textbf{Fine Crosshair} for long-range surveillance.

\subsection{Personalize Colors}

\textbf{Access}: Main Menu → "Personalize Colors"

\textbf{Purpose}: Change OSD color scheme for visibility

\subsubsection{Available Color Themes}

\small
\begin{longtable}{|L{0.18\textwidth}|L{0.25\textwidth}|L{0.50\textwidth}|}
\hline
\rowcolor{milblue!20}
\textbf{Theme} & \textbf{Primary Color} & \textbf{Use Case} \\
\hline
Green & Bright green & Default - good for day and night \\
\hline
Red & Red & Night vision compatible \\
\hline
Yellow & Yellow & High contrast, bright conditions \\
\hline
Cyan & Cyan & Alternative for user preference \\
\hline
White & White & Maximum contrast \\
\hline
\end{longtable}

\textbf{What Changes}:
\begin{itemize}
    \item Reticle color
    \item OSD text color
    \item Menu text color
    \item Tracking box color (when not status-coded)
\end{itemize}

\textbf{What Does NOT Change}:
\begin{itemize}
    \item Warning messages (always red or yellow)
    \item Tracking status colors (yellow/red/green based on state)
    \item CCIP pipper color
\end{itemize}

\subsubsection{How to Change Color}

\begin{procedurebox}[CHANGING COLOR SCHEME]
\begin{enumerate}
    \item Press \button{MENU ✓}
    \item Navigate to "Personalize Colors" (▼)
    \item Press \button{MENU ✓} to enter
    \item Use ▲/▼ to highlight desired color
    \item Press \button{MENU ✓} to select
    \item OSD changes immediately
    \item Evaluate visibility
    \item Change again if needed, or select "Return..."
\end{enumerate}
\end{procedurebox}

\textbf{Best Practices}:
\begin{itemize}
    \item \textbf{Green}: Good all-around choice (default)
    \item \textbf{Red}: Use with night vision equipment
    \item \textbf{Yellow}: Use in bright sunlight or snow
    \item \textbf{White}: Maximum contrast on dark backgrounds
\end{itemize}

\subsection{Display Brightness}

\textbf{Access}: Main Menu → "Display Brightness"

\textbf{Purpose}: Adjust DCU display brightness for optimal visibility in varying ambient light conditions.

\subsubsection{Brightness Control Parameters}

\small
\begin{longtable}{|L{0.25\textwidth}|L{0.68\textwidth}|}
\hline
\rowcolor{milblue!20}
\textbf{Parameter} & \textbf{Value/Details} \\
\hline
\textbf{Range} & 10\% to 100\% \\
\hline
\textbf{Step Size} & 10\% per button press \\
\hline
\textbf{Default} & System default (typically 70\%) \\
\hline
\textbf{Preview} & Real-time preview as you adjust \\
\hline
\textbf{Cancel} & BACK restores original brightness \\
\hline
\end{longtable}

\subsubsection{How to Adjust Brightness}

\begin{procedurebox}[ADJUSTING DISPLAY BRIGHTNESS]
\begin{enumerate}
    \item Press \button{MENU ✓}
    \item Navigate to "Display Brightness" (▼)
    \item Press \button{MENU ✓} to enter brightness adjustment
    \item \textbf{Screen displays}: "Adjust display brightness."
    \item Use \button{MENU ▲} to \textbf{increase} brightness (+10\%)
    \item Use \button{MENU ▼} to \textbf{decrease} brightness (-10\%)
    \item \textbf{Preview is immediate} - observe screen as you adjust
    \item Press \button{MENU ✓} to \textbf{APPLY} the new brightness
    \item \textbf{Confirmation}: "Brightness Applied" message appears
    \item Press \button{MENU ✓} to return to Main Menu
\end{enumerate}
\end{procedurebox}

\begin{notebox}[CANCEL OPTION]
Press \button{BACK} at any time during adjustment to \textbf{cancel} and restore the previous brightness level. This is useful if you accidentally set it too dark or bright.
\end{notebox}

\textbf{Brightness Recommendations}:
\begin{itemize}
    \item \textbf{100\%}: Bright daylight, direct sunlight
    \item \textbf{70-80\%}: Normal daylight, overcast conditions (default)
    \item \textbf{40-60\%}: Dawn/dusk, shaded areas
    \item \textbf{20-30\%}: Night operations (preserves night vision adaptation)
    \item \textbf{10\%}: Extreme low-light (minimum visible)
\end{itemize}

\begin{warningbox}[NIGHT OPERATIONS]
High display brightness at night will illuminate your position and impair your night vision adaptation. Reduce brightness to 30\% or lower for night operations.
\end{warningbox}

\section{Calibration Menu Options}

\subsection{Preset Home Position}

\textbf{Access}: Main Menu → "Preset Home Position"

\textbf{Purpose}: Define the HOME reference position for the gimbal. This is where the gimbal returns when the HOME function is activated.

\subsubsection{Why Set a Custom Home Position?}

The default home position may not be optimal for your vehicle configuration or operational needs:
\begin{itemize}
    \item Orient weapon toward expected threat direction
    \item Align with vehicle front for stowage
    \item Position for vehicle travel mode
    \item Set up for specific defensive sectors
\end{itemize}

\subsubsection{Preset Home Position Procedure}

\begin{procedurebox}[SETTING PRESET HOME POSITION]
\textbf{Step 1: Position Gimbal}
\begin{enumerate}
    \item Press \button{MENU ✓}
    \item Navigate to "Preset Home Position"
    \item Press \button{MENU ✓} to enter procedure
    \item \textbf{Screen displays}: "POSITION GIMBAL TO HOME LOCATION"
    \item Use \textbf{JOYSTICK} to slew the gimbal to your desired home position
    \item Current Az/El position is displayed on screen in real-time
    \item Align with your visual/mechanical reference
    \item Press \button{MENU ✓} when positioned correctly
\end{enumerate}

\textbf{Step 2: Confirm Home Position}
\begin{enumerate}
    \setcounter{enumi}{8}
    \item \textbf{Screen displays}: "CONFIRM HOME POSITION"
    \item Displayed position: "AZ: XX.XX deg | EL: XX.XX deg"
    \item Review the position carefully
    \item Press \button{MENU ✓} to \textbf{CONFIRM} and save
\end{enumerate}

\textbf{Step 3: Completion}
\begin{enumerate}
    \setcounter{enumi}{12}
    \item \textbf{Screen displays}: "PRESET HOME POSITION SET!"
    \item Motor controller stores the position as zero reference
    \item Press \button{MENU ✓} to return to Main Menu
\end{enumerate}
\end{procedurebox}

\begin{notebox}[MOTOR CONTROLLER]
The preset home position command (HR10) is sent to the motor controller (PLC42) which stores it in non-volatile memory. This position will persist across power cycles.
\end{notebox}

\begin{cautionbox}[CANCEL]
Press \button{BACK} at any step to cancel the procedure and return to the Main Menu without changing the home position.
\end{cautionbox}

\textbf{Verifying Home Position}:
\begin{enumerate}
    \item After setting, press \button{Home} (joystick Button 4)
    \item Gimbal should slew to the position you just set
    \item If incorrect, repeat the Preset Home Position procedure
\end{enumerate}

\section{System Menu Options}

\subsection{Zone Definitions}

\textbf{Access}: Main Menu → "Zone Definitions"

\textbf{Purpose}: Manage no-fire zones, no-traverse zones, sector scans, and TRPs

\textbf{Detailed Coverage}: See Lesson 4 (Motion Modes \& Surveillance)

\textbf{Quick Access Functions}:
\begin{itemize}
    \item View active zones
    \item Enable/disable zones
    \item Navigate to zone editor (supervisor/commander function)
\end{itemize}

\begin{notebox}[OPERATOR NOTE]
Zone modification usually requires supervisor authorization. Operators can VIEW zones but typically cannot CHANGE them.
\end{notebox}

\subsection{Detection Toggle}

\textbf{Access}: Main Menu → "Detection: ENABLED/DISABLED"

\textbf{Purpose}: Enable or disable AI-based automatic target detection for the Day camera.

\subsubsection{Detection Availability}

\small
\begin{longtable}{|L{0.25\textwidth}|L{0.68\textwidth}|}
\hline
\rowcolor{milblue!20}
\textbf{Condition} & \textbf{Detection Status} \\
\hline
Day Camera Active & \textbf{Available} - Can be enabled/disabled \\
\hline
Night Camera Active & \textbf{Unavailable} - Menu shows "(Night - Unavailable)" \\
\hline
\end{longtable}

\textbf{Menu Display States}:
\begin{itemize}
    \item \texttt{"Detection: ENABLED"} - Detection is ON (Day camera)
    \item \texttt{"Detection: DISABLED"} - Detection is OFF (Day camera)
    \item \texttt{"Detection (Night - Unavailable)"} - Night camera active, cannot toggle
\end{itemize}

\subsubsection{How to Toggle Detection}

\begin{procedurebox}[ENABLING/DISABLING DETECTION]
\begin{enumerate}
    \item Ensure \textbf{Day camera} is active (switch camera if needed)
    \item Press \button{MENU ✓}
    \item Navigate to "Detection: ENABLED" or "Detection: DISABLED"
    \item Press \button{MENU ✓} to toggle the state
    \item Detection state changes immediately
    \item Menu closes automatically
\end{enumerate}
\end{procedurebox}

\textbf{Detection Indicators}:
\begin{itemize}
    \item When ENABLED: Detection boxes appear around identified targets
    \item When DISABLED: No automatic target highlighting
    \item OSD shows detection status in system status area
\end{itemize}

\begin{warningbox}[NIGHT CAMERA LIMITATION]
Detection is \textbf{ONLY} available on the Day camera. If you switch to Night camera while detection is enabled, detection will automatically suspend and resume when you switch back to Day camera.
\end{warningbox}

\begin{notebox}[PERFORMANCE NOTE]
Detection processing uses significant computing resources. Disable detection when not needed to improve system responsiveness.
\end{notebox}

\subsection{System Status}

\textbf{Access}: Main Menu → "System Status"

\textbf{Purpose}: View detailed system health and diagnostics

\textbf{Detailed Coverage}: See Lesson 7 (System Status \& Monitoring)

\textbf{Quick Preview}:

Displays status of all subsystems:
\begin{twocol}
\begin{itemize}
    \item Cameras (Day/Night)
    \item Servos (Azimuth/Elevation)
    \item Laser Rangefinder
    \item Joystick
    \item Stabilization System
    \item Tracking System
    \item Weapon Actuator
\end{itemize}
\end{twocol}

\textbf{When to Check}:
\begin{itemize}
    \item At startup (verify all green)
    \item When Fault light illuminates
    \item Before critical operations
    \item During troubleshooting
\end{itemize}

\subsection{Shutdown System}

\textbf{Access}: Main Menu → "Shutdown System"

\textbf{Purpose}: Perform orderly software shutdown before powering off

\textbf{Why Use Menu Shutdown}:
\begin{itemize}
    \item Saves configuration settings
    \item Saves zone data
    \item Closes log files properly
    \item Prevents data corruption
    \item Powers down subsystems in correct sequence
\end{itemize}

\subsubsection{Shutdown Procedure}

\begin{procedurebox}[MENU SHUTDOWN]
\begin{enumerate}
    \item Press \button{MENU ✓}
    \item Navigate to "Shutdown System" (▼ multiple times)
    \item Press \button{MENU ✓}
    \item Confirmation prompt appears: "SHUTDOWN SYSTEM?"
    \item Select "YES, Shutdown"
    \item Press \button{MENU ✓} to confirm
    \item System shuts down:
    \begin{itemize}
        \item "SHUTTING DOWN..." message appears
        \item Progress indicator shows shutdown steps
        \item Cameras power off
        \item Motors de-energize
        \item "SHUTDOWN COMPLETE - Safe to power off" message
    \end{itemize}
    \item Disable \button{Station Enable} switch on DCU
    \item Cut vehicle power (if end of shift)
\end{enumerate}
\end{procedurebox}

\begin{warningbox}[IMPORTANT]
Wait for "SHUTDOWN COMPLETE" message before cutting power. Interrupting shutdown can corrupt configuration files.
\end{warningbox}

\subsection{About / Info}

\textbf{Access}: Main Menu → "About"

\textbf{Purpose}: Display system information for troubleshooting and support

\textbf{Information Displayed}:
\begin{itemize}
    \item System name: "El 7arress RCWS"
    \item Software version (e.g., "v4.5.2")
    \item Build date
    \item Serial number (if configured)
    \item Uptime (hours since power-on)
    \item Operator name (if logged in)
\end{itemize}

\textbf{Use Case}: Provide this information when reporting issues to maintenance.

\section{Ballistics Menu (Overview)}

\textbf{Access}: Main Menu → "--- BALLISTICS ---" section

The ballistics menu provides access to fire control settings. These are covered in detail in \textbf{Lesson 6} but are introduced here for awareness.

\subsection{Ballistics Submenu Options}

\subsubsection{1. Zeroing}
\begin{itemize}
    \item Align weapon point of impact with camera crosshair
    \item Adjust azimuth and elevation offsets
    \item Save/load zeroing profiles
    \item \textbf{Detailed in Lesson 6.1}
\end{itemize}

\subsubsection{2. Environmental Parameters}
\begin{itemize}
    \item Set temperature (\degree C)
    \item Set altitude (meters above sea level)
    \item Set crosswind speed and direction
    \item Apply environmental corrections to ballistics
    \item \textbf{Detailed in Lesson 6.2}
\end{itemize}

\subsubsection{3. Lead Angle Compensation (View Only)}
\begin{itemize}
    \item \textbf{View} lead angle status (Off/On/Lag/ZoomOut)
    \item Displays current LAC system state
    \item \textbf{NOTE: LAC cannot be toggled from menu!}
    \item \textbf{LAC is activated ONLY via joystick Button 2}
    \item See Lesson 6.3 for LAC activation procedure
\end{itemize}

\begin{warningbox}[IMPORTANT]
Lead Angle Compensation (LAC) can \textbf{ONLY} be toggled using joystick \textbf{Button 2} while holding the Dead Man Switch (Button 3). There is NO menu option to enable/disable LAC.
\end{warningbox}

\begin{cautionbox}[OPERATOR NOTE]
Do not modify ballistics settings unless trained. Incorrect settings can cause missed shots or dangerous ricochets. Zeroing and environmental settings are usually performed by designated personnel.
\end{cautionbox}

\section{Menu Quick Reference}

\subsection{Common Menu Tasks}

\subsubsection{Task 1: Change Reticle}

\begin{quickref}
\texttt{MENU ✓ → "Personalize Reticle" → MENU ✓}\\
\texttt{→ Select reticle (▲/▼) → MENU ✓}\\
\texttt{→ "Return..." → MENU ✓}\\
\textbf{Time}: ~10 seconds
\end{quickref}

\subsubsection{Task 2: Change Color Scheme}

\begin{quickref}
\texttt{MENU ✓ → "Personalize Colors" → MENU ✓}\\
\texttt{→ Select color (▲/▼) → MENU ✓}\\
\texttt{→ "Return..." → MENU ✓}\\
\textbf{Time}: ~10 seconds
\end{quickref}

\subsubsection{Task 3: Check System Status}

\begin{quickref}
\texttt{MENU ✓ → "System Status" → MENU ✓}\\
\texttt{→ Review status → "Return..." → MENU ✓}\\
\textbf{Time}: ~15 seconds (plus review time)
\end{quickref}

\subsubsection{Task 4: Shutdown via Menu}

\begin{quickref}
\texttt{MENU ✓ → "Shutdown System" → MENU ✓}\\
\texttt{→ "YES, Shutdown" → MENU ✓}\\
\texttt{→ Wait for "SHUTDOWN COMPLETE"}\\
\texttt{→ Disable Station Enable → Cut power}\\
\textbf{Time}: ~45 seconds
\end{quickref}

\subsection{Menu Navigation Tips}

\begin{enumerate}
    \item \textbf{Muscle Memory}: Practice menu navigation until you can do it without looking at button labels
    \item \textbf{HAT Switch Alternative}: Use joystick HAT switch if your hands are already on the joystick
    \item \textbf{Quick Exit}: If lost in menus, repeatedly press \button{MENU ✓} on section headers to back out quickly
    \item \textbf{Video Still Visible}: Menu is semi-transparent - you can still monitor situation while in menu
    \item \textbf{Menu Timeout}: Some menus auto-exit after 60 seconds of inactivity
    \item \textbf{Combat Discipline}: Minimize menu time during operations
\end{enumerate}

\subsection{Menu Troubleshooting}

\small
\begin{longtable}{|L{0.25\textwidth}|L{0.30\textwidth}|L{0.38\textwidth}|}
\hline
\rowcolor{milblue!20}
\textbf{Problem} & \textbf{Possible Cause} & \textbf{Solution} \\
\hline
Menu won't open & Button stuck or fault & Try joystick HAT switch, restart system \\
\hline
Can't select option & On section header & Use ▲/▼ to move to selectable item \\
\hline
Menu frozen & Software hang & Press Emergency Stop, restart system \\
\hline
Settings don't save & Shutdown without menu & Always use "Shutdown System" menu \\
\hline
Menu text unreadable & Color scheme issue & Change to White or Yellow theme \\
\hline
\end{longtable}

\section{Menu Best Practices}

\subsection{When to Use Menus}

\textbf{DO use menus for}:
\begin{itemize}
    \item Changing display preferences (reticle, color)
    \item Checking system status
    \item Reviewing zone definitions
    \item Configuring ballistics (when trained)
    \item Orderly system shutdown
\end{itemize}

\textbf{DO NOT use menus during}:
\begin{itemize}
    \item Active engagement
    \item Emergency situations
    \item When gimbal must be controlled continuously
    \item Under time pressure
\end{itemize}

\begin{notebox}[RULE OF THUMB]
Menus are for setup and configuration, not combat operations.
\end{notebox}

\subsection{Settings That Persist}

\textbf{Saved Between Power Cycles}:
\begin{itemize}
    \item Reticle type selection
    \item Color scheme
    \item Zeroing offsets (if saved)
    \item Environmental parameters (if saved)
    \item Zone definitions
\end{itemize}

\textbf{NOT Saved} (reset on power-up):
\begin{itemize}
    \item Gimbal position (returns to home)
    \item Active tracking (aborted)
    \item Temporary warnings
    \item Menu navigation position
\end{itemize}

\subsection{Operator vs. Supervisor Functions}

\textbf{Operator Can}:
\begin{itemize}
    \item Change reticle and colors
    \item View system status
    \item View zones
    \item Access ballistics menus (view)
    \item Shutdown system
\end{itemize}

\textbf{Operator Usually CANNOT} (requires authorization):
\begin{itemize}
    \item Modify zone boundaries
    \item Override no-fire zones
    \item Change ballistics profiles
    \item Access maintenance menus
    \item Modify system configuration files
\end{itemize}

\textbf{Consult your unit SOP for specific authorization levels.}

\section{Student Review Questions}

\begin{enumerate}
    \item What button is used to access the main menu?
    \item Name three available reticle types.
    \item Which color scheme is recommended for night vision operations?
    \item What menu option would you use to view system diagnostics?
    \item Why is it important to use the menu shutdown before cutting power?
    \item Can operators modify no-fire zone boundaries?
    \item What settings are saved between power cycles?
    \item What is the recommended reticle for general combat operations?
    \item How do you exit from a submenu?
    \item What should you do if the menu becomes frozen?
\end{enumerate}