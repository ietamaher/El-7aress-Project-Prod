% ============================================================================
% LESSON 5 - TARGET ENGAGEMENT PROCESS
% ============================================================================
\lesson{5 - TARGET ENGAGEMENT PROCESS}

\begin{tabular}{@{}ll@{}}
\textbf{Duration:} & 4 hours (Classroom 1h + Simulator 3h) \\
\textbf{Type:} & Classroom + Practical \\
\textbf{References:} & Operator manual, tracking system documentation \\
\end{tabular}

\section{LEARNING OBJECTIVES}

Upon completion of this lesson, operators will be able to:
\begin{itemize}
    \item Execute complete target engagement sequence
    \item Operate tracking system through all phases
    \item Adjust acquisition gate for target selection
    \item Perform emergency tracking abort
    \item Execute simulated weapons engagement
\end{itemize}

% ============================================================================
\section{TARGET ENGAGEMENT SEQUENCE}

\subsection{THE SIX-PHASE ENGAGEMENT CYCLE}

The complete engagement process follows six distinct phases:

\begin{center}
\texttt{DETECT → IDENTIFY → ACQUIRE → TRACK → ENGAGE → ASSESS}
\end{center}

\spectable{
\begin{tabular}{|L{0.15\textwidth}|L{0.45\textwidth}|L{0.25\textwidth}|}
\hline
\rowcolor{milblue!20}
\textbf{Phase} & \textbf{Operator Actions} & \textbf{Expected Duration} \\
\hline
\textbf{1. DETECT} & Scan area (manual/auto modes), visual detection & Continuous \\
\hline
\textbf{2. IDENTIFY} & Slew to target, zoom in, PID (Positive ID), verify ROE & 5-15 seconds \\
\hline
\textbf{3. ACQUIRE} & Enter tracking acquisition, position gate, size gate & 5-10 seconds \\
\hline
\textbf{4. TRACK} & Lock-on, monitor track quality & Continuous until engagement \\
\hline
\textbf{5. ENGAGE} & Range, LAC (if moving), Master Arm, Fire, Observe & 2-30 seconds \\
\hline
\textbf{6. ASSESS} & BDA (Battle Damage Assessment), re-engage or cease & 5-10 seconds \\
\hline
\end{tabular}
}

\subsection{DETECT \& IDENTIFY (PHASES 1-2)}

\subsubsection{Detection Methods}
\begin{itemize}
    \item Manual scan (joystick control)
    \item Auto Sector Scan (Lesson 4)
    \item TRP Scan (Lesson 4)
    \item Radar cues (if available)
\end{itemize}

\subsubsection{Identification Requirements}
\begin{itemize}
    \item \textbf{PID (Positive Identification)} mandatory before engagement
    \item Use zoom to magnify target (\button{Button 6/8})
    \item Thermal camera may aid identification (night, obscurants)
    \item Verify target meets Rules of Engagement (ROE)
    \item Confirm NOT friendly forces
\end{itemize}

\begin{warningbox}
Failure to achieve PID before engagement violates ROE and may result in fratricide.
\end{warningbox}

% ============================================================================
\section{TRACKING SYSTEM OVERVIEW}

\subsection{TRACKING PHASE STATE MACHINE}

The tracking system operates through discrete phases:

\begin{center}
\begin{tikzpicture}[
    node distance=1.5cm,
    state/.style={rectangle, rounded corners, draw=milblue, fill=milblue!10, thick, minimum width=3cm, minimum height=0.8cm, align=center},
    arrow/.style={->, thick, >=stealth}
]
    \node[state] (off) {OFF};
    \node[state, below of=off] (acq) {ACQUISITION};
    \node[state, below of=acq] (pend) {LOCK PENDING};
    \node[state, below of=pend] (lock) {ACTIVE LOCK};
    \node[state, right of=lock, node distance=4cm] (coast) {COAST};
    \node[state, below of=lock] (fire) {FIRING};
    
    \draw[arrow] (off) -- node[right, font=\tiny] {Button 4} (acq);
    \draw[arrow] (acq) -- node[right, font=\tiny] {Size gate + Button 4} (pend);
    \draw[arrow] (pend) -- node[right, font=\tiny] {Lock success} (lock);
    \draw[arrow] (lock) -- node[above, font=\tiny] {Target lost} (coast);
    \draw[arrow] (coast) -- node[below, font=\tiny] {Re-acquired} (lock);
    \draw[arrow] (lock) -- (fire);
    \draw[arrow] (fire) -- node[left, font=\tiny] {Cease fire} (lock);
    \draw[arrow, dashed] (fire.east) -- ++(2,0) |- node[right, font=\tiny] {Abort} (off.east);
\end{tikzpicture}
\end{center}

\subsection{TRACKING PHASE REFERENCE}

\begin{longtable}{|L{0.13\textwidth}|L{0.13\textwidth}|L{0.10\textwidth}|L{0.10\textwidth}|L{0.11\textwidth}|L{0.18\textwidth}|}
\hline
\rowcolor{milblue!20}
\textbf{Phase} & \textbf{OSD Status} & \textbf{Box Color} & \textbf{Box Style} & \textbf{Duration} & \textbf{Gimbal Control} \\
\hline
\endfirsthead

\hline
\rowcolor{milblue!20}
\textbf{Phase} & \textbf{OSD Status} & \textbf{Box Color} & \textbf{Box Style} & \textbf{Duration} & \textbf{Gimbal Control} \\
\hline
\endhead

\textbf{Off} & MODE: Manual & None & N/A & Until acquisition & Operator (joystick) \\
\hline
\textbf{Acquisition} & ACQUISITION & Yellow & Solid & 5-10 sec & Operator (joystick) \\
\hline
\textbf{Lock Pending} & LOCK PENDING & Cyan & Solid/Flash & 0.5-2 sec & Operator (hold steady) \\
\hline
\textbf{Active Lock} & TRACKING & Green & Dashed & Continuous & Automatic (tracker) \\
\hline
\textbf{Coast} & TRACKING (COAST) & Yellow & Dashed & 1-5 sec & Automatic (predict) \\
\hline
\textbf{Firing} & TRACKING (FIRING) & Green & Solid & While firing & Automatic (hold) \\
\hline
\end{longtable}

\subsection{TRACKING CONTROLS}

\spectable{
\begin{tabular}{|L{0.20\textwidth}|L{0.40\textwidth}|L{0.30\textwidth}|}
\hline
\rowcolor{milblue!20}
\textbf{Control} & \textbf{Function} & \textbf{Active Phase(s)} \\
\hline
\button{Button 4} (single) & Start Acquisition / Request Lock-On & Off → Acq, Acq → Lock Pending \\
\hline
\button{Button 4} (double <1 sec) & \textbf{EMERGENCY ABORT} & Any phase → Off \\
\hline
\button{D-Pad ▲} & Decrease gate height (-4 px) & Acquisition only \\
\hline
\button{D-Pad ▼} & Increase gate height (+4 px) & Acquisition only \\
\hline
\button{D-Pad ◄} & Decrease gate width (-4 px) & Acquisition only \\
\hline
\button{D-Pad ►} & Increase gate width (+4 px) & Acquisition only \\
\hline
\end{tabular}
}

% ============================================================================
\section{ACQUISITION PHASE}

\subsection{ENTERING ACQUISITION MODE}

\subsubsection{Prerequisites}
\begin{checklist}
    \item Motion mode: Manual
    \item Target on-screen and identified
    \item Target approximately centered in reticle
\end{checklist}

\begin{procedurebox}[ENTERING ACQUISITION]
\textbf{1. Press \button{Button 4} (Track Select)}

\textbf{Result:}
\begin{itemize}
    \item System enters \textbf{ACQUISITION} phase
    \item Yellow acquisition gate appears (centered on reticle)
    \item OSD displays: \osd{MODE: ACQUISITION}
    \item Joystick gimbal control remains active
    \item D-Pad now controls gate size
\end{itemize}
\end{procedurebox}

\subsection{SIZING THE ACQUISITION GATE}

\textbf{Objective:} Frame target with 10-30\% margin on all sides

\subsubsection{D-Pad Controls}
\begin{itemize}
    \item \button{UP ▲}: Decrease height
    \item \button{DOWN ▼}: Increase height
    \item \button{LEFT ◄}: Decrease width
    \item \button{RIGHT ►}: Increase width
    \item \textbf{Step size:} 4 pixels per press
\end{itemize}

\subsubsection{Gate Sizing Guidelines}

\spectable{
\begin{tabular}{|L{0.22\textwidth}|L{0.35\textwidth}|L{0.23\textwidth}|}
\hline
\rowcolor{milblue!20}
\textbf{Target Framing} & \textbf{Effect on Tracking} & \textbf{Recommendation} \\
\hline
Too Tight (<10\% margin) & Tracker may lose target if target rotates/expands & ❌ Avoid \\
\hline
Optimal (10-30\% margin) & Best tracking performance & ✅ Ideal \\
\hline
Too Loose (>50\% margin) & Background clutter may confuse tracker & ❌ Avoid \\
\hline
\end{tabular}
}

\subsubsection{Best Practices}
\begin{itemize}
    \item \textbf{Vehicles:} Frame entire hull, exclude ground
    \item \textbf{Personnel:} Include torso/legs, minimize background
    \item \textbf{Moving targets:} Size slightly larger (anticipate motion)
    \item \textbf{Default size:} 100×100 pixels (adjust as needed)
\end{itemize}

\subsection{REQUESTING LOCK-ON}

\subsubsection{When Ready}
\begin{checklist}
    \item Target fully visible and framed in gate
    \item Target has good contrast against background
    \item Target not moving erratically
\end{checklist}

\textbf{Action:} Press \button{Button 4} (second press)

\textbf{Result:}
\begin{itemize}
    \item System → \textbf{LOCK PENDING} phase
    \item Gate color: Yellow → Cyan (or flashing)
    \item OSD: \osd{MODE: LOCK PENDING}
    \item Tracker initialization begins (0.5-2 seconds)
\end{itemize}

\begin{cautionbox}
Do NOT make rapid gimbal movements during Lock Pending. Hold gimbal steady.
\end{cautionbox}

% ============================================================================
\section{LOCK PENDING → ACTIVE LOCK}

\subsection{LOCK PENDING PHASE}

\textbf{Purpose:} Tracking system initializes on target

\textbf{Duration:} 0.5 to 2 seconds (typically ~1 second)

\textbf{Operator Action:} \textbf{WAIT} - maintain steady gimbal

\subsubsection{What Happens}
\begin{enumerate}
    \item System captures reference image of target
    \item Initializes tracker algorithm
    \item Calculates target features
    \item Begins tracking target in video stream
\end{enumerate}

\subsubsection{Transition Outcomes}

\textbf{✅ SUCCESS:}
\begin{itemize}
    \item System → \textbf{ACTIVE LOCK} phase
    \item Gate color: Green
    \item Gate style: Dashed outline
    \item OSD: \osd{MODE: TRACKING} or \osd{ACTIVE LOCK}
    \item Gimbal control switches to automatic
\end{itemize}

\textbf{❌ FAILURE (rare):}
\begin{itemize}
    \item System → \textbf{ACQUISITION} phase (retry)
    \item Possible causes: Poor contrast, target too small, target motion
    \item Operator: Adjust gate size or abort and restart
\end{itemize}

\subsection{ACTIVE LOCK PHASE}

\subsubsection{System Behavior}
\begin{itemize}
    \item Tracker follows target at 30 Hz (30 times/second)
    \item Gimbal automatically moves to keep target centered
    \item Reticle stays on target center
    \item \textbf{Joystick axis inputs IGNORED} (tracking in control)
\end{itemize}

\subsubsection{Visual Indicators}
\begin{itemize}
    \item \textbf{Tracking Gate:} Green dashed outline around target
    \item \textbf{OSD:} \osd{MODE: TRACKING} or \osd{ACTIVE LOCK}
    \item \textbf{Control Panel:} TRACKING light ON (green)
    \item \textbf{Track Confidence:} >70\% (good), 50-70\% (marginal), <50\% (poor)
\end{itemize}

\subsection{OPERATOR ROLE DURING ACTIVE LOCK}

\textbf{Primary Role:} \textbf{MONITOR} - system is tracking automatically

\subsubsection{Monitor For}

\textbf{1. Track Quality:}
\begin{itemize}
    \item ✅ Green gate = Good track
    \item ⚠️ Yellow gate = Marginal (may lose soon)
    \item Track confidence >70\% (good), <50\% (prepare for coast)
\end{itemize}

\textbf{2. Target Status:}
\begin{itemize}
    \item Target correctly identified (didn't jump to wrong object)
    \item Target still valid (meets engagement criteria)
    \item Target not obscured or about to be obscured
\end{itemize}

\textbf{3. Gimbal Position:}
\begin{itemize}
    \item Staying within operational limits
    \item Not approaching no-traverse zones
    \item Elevation within -20\degree\ to +60\degree
\end{itemize}

\subsubsection{Controls Still Active}

\spectable{
\begin{tabular}{|L{0.30\textwidth}|L{0.60\textwidth}|}
\hline
\rowcolor{milgreen!20}
\multicolumn{2}{|c|}{\textbf{ACTIVE CONTROLS}} \\
\hline
\button{Button 0} & Master Arm \\
\hline
\button{Button 2} & LAC toggle \\
\hline
\button{Button 3} & Dead Man Switch \\
\hline
\button{Button 4} & Double-click abort \\
\hline
\button{Button 5} & Fire \\
\hline
\button{Button 6/8} & Zoom (use cautiously, may affect track) \\
\hline
\hline
\rowcolor{milred!20}
\multicolumn{2}{|c|}{\textbf{BLOCKED CONTROLS}} \\
\hline
\button{Button 11/13} & Mode cycle - BLOCKED during tracking \\
\hline
Joystick axes (X/Y) & IGNORED during tracking \\
\hline
\end{tabular}
}

% ============================================================================
\section{COAST MODE}

\subsection{WHEN COAST ACTIVATES}

\textbf{Triggers:}
\begin{itemize}
    \item Target temporarily obscured (passes behind object)
    \item Target leaves field of view briefly
    \item Tracker loses visual lock
    \item Dust, smoke, or other obscurants
\end{itemize}

\subsection{System Behavior}
\begin{itemize}
    \item Tracker \textbf{predicts} target position based on last known velocity
    \item Gimbal continues to predicted position
    \item System attempts to re-acquire target
    \item Tracking gate: Green → Yellow (dashed)
\end{itemize}

\subsection{Display Changes}
\begin{itemize}
    \item \textbf{OSD:} \osd{MODE: COAST} or \osd{TRACKING (COAST)}
    \item \textbf{Gate:} Yellow/amber dashed outline
    \item \textbf{Warning:} "COASTING - TARGET LOST" may display
\end{itemize}

\subsection{COAST OUTCOMES}

\textbf{Typical Duration:} 1-5 seconds

\subsubsection{Outcome 1: Target Re-Acquired (✅ Success)}
\begin{itemize}
    \item Target reappears in field of view
    \item Tracker re-locks on target
    \item System → \textbf{ACTIVE LOCK} phase
    \item Gate: Yellow → Green
    \item Tracking continues normally
\end{itemize}

\subsubsection{Outcome 2: Coast Timeout (❌ Failure)}
\begin{itemize}
    \item Target not re-acquired within timeout (~5 seconds)
    \item System gives up
    \item System → \textbf{OFF} phase
    \item Tracking stops
    \item Operator must restart tracking if desired
\end{itemize}

\subsection{Operator Action During Coast}

\begin{itemize}
    \item ✅ Wait patiently (system attempting re-acquisition)
    \item ✅ Be ready for track to resume
    \item ❌ Do NOT abort prematurely (give system time)
    \item ❌ Do NOT make manual gimbal movements (joystick ignored)
\end{itemize}

\subsubsection{Abort Coast If:}
\begin{itemize}
    \item Target definitely not coming back (destroyed, permanently obscured)
    \item Tracker coasting in wrong direction (confused)
    \item Mission changed
\end{itemize}

% ============================================================================
\section{TRACKING ABORT (EMERGENCY)}

\subsection{WHEN TO ABORT TRACKING}

Abort tracking \textbf{IMMEDIATELY} if:
\begin{itemize}
    \item ❌ Tracking wrong target (friendly, civilian, incorrect target)
    \item ❌ Target no longer valid (fails ROE)
    \item ❌ Safety concern (entering no-fire zone, gimbal obstruction)
    \item ❌ Mission change (new priority, orders to cease)
    \item ❌ Tracking erratic (unexpected gimbal behavior)
\end{itemize}

\subsection{ABORT PROCEDURE}

\begin{procedurebox}[EMERGENCY ABORT]
\textbf{Action:} \textbf{DOUBLE-CLICK \button{Button 4}} (<1 second between presses)

\textbf{Effect (IMMEDIATE):}
\begin{enumerate}
    \item Tracking \textbf{STOPS}
    \item System → \textbf{OFF} phase
    \item Gimbal holds current position
    \item Tracking gate disappears
    \item Weapon fire \textbf{INHIBITED} (even if Master Arm engaged)
    \item OSD: \osd{MODE: MANUAL}
\end{enumerate}
\end{procedurebox}

\subsubsection{Timing}
\begin{itemize}
    \item Press 1 → Press 2 within \textbf{1000 milliseconds} (one second)
    \item \textbf{Too slow} (>1 second): System interprets as two single presses (may restart tracking)
\end{itemize}

\begin{warningbox}[CRITICAL]
Practice double-click timing during training until muscle memory established.
\end{warningbox}

\subsection{AFTER ABORT}

\textbf{System State:}
\begin{itemize}
    \item Manual mode active
    \item Joystick gimbal control restored
    \item No tracking active
\end{itemize}

\textbf{Next Actions (mission-dependent):}
\begin{itemize}
    \item Re-acquire correct target and restart tracking
    \item Return to surveillance mode
    \item Engage different target
    \item Follow commander's orders
\end{itemize}

% ============================================================================
\section{WEAPON CHARGING (COCKING ACTUATOR)}

\subsection{COCKING ACTUATOR OVERVIEW}

The cocking actuator is a servo-controlled mechanism that cycles the weapon's bolt to chamber a round. This must be performed before the weapon can fire.

\subsubsection{Weapon-Specific Charging Cycles}

\small
\begin{longtable}{|L{0.18\textwidth}|L{0.18\textwidth}|L{0.55\textwidth}|}
\hline
\rowcolor{milblue!20}
\textbf{Weapon Type} & \textbf{Cycles Req.} & \textbf{Notes} \\
\hline
\textbf{M2HB (.50 cal)} & \textbf{2 cycles} & Closed bolt weapon - Cycle 1: pulls bolt, picks up round; Cycle 2: chambers round, weapon ready \\
\hline
M240B & 1 cycle & Open bolt weapon - single cycle chambers round \\
\hline
M249 SAW & 1 cycle & Open bolt weapon \\
\hline
MK19 (40mm) & 1 cycle & Grenade launcher \\
\hline
\end{longtable}

\begin{warningbox}[M2HB CRITICAL]
The M2HB requires \textbf{TWO} complete charging cycles before the weapon is ready to fire. A single cycle will NOT chamber a round.
\end{warningbox}

\subsection{CHARGING MODES}

The charging system supports two operational modes:

\subsubsection{Mode 1: Short Press (Automatic Cycle)}

\begin{procedurebox}[SHORT PRESS CHARGING]
\begin{enumerate}
    \item Press \button{Charge} button (joystick Button 9)
    \item Release button immediately (short press)
    \item System \textbf{automatically} cycles: Extend → Retract
    \item For M2HB: System automatically performs second cycle
    \item \textbf{Charging Complete} when retraction finishes
    \item 4-second lockout begins (prevents immediate re-charge)
\end{enumerate}
\end{procedurebox}

\textbf{Advantages}: Faster, automatic multi-cycle for M2HB
\textbf{Best For}: Standard charging operations

\subsubsection{Mode 2: Continuous Hold (Manual Control)}

\begin{procedurebox}[CONTINUOUS HOLD CHARGING]
\begin{enumerate}
    \item Press and \textbf{HOLD} \button{Charge} button
    \item Actuator extends (pulls bolt back)
    \item \textbf{Hold position} while button is held
    \item \textbf{Release} button to initiate retraction
    \item Actuator retracts (releases bolt)
    \item Repeat for additional cycles (M2HB)
    \item 4-second lockout begins after final cycle
\end{enumerate}
\end{procedurebox}

\textbf{Advantages}: Operator controls timing, can hold bolt back
\textbf{Best For}: Clearing jams, inspection, controlled operations

\subsection{CHARGING STATE MACHINE}

\begin{center}
\begin{tikzpicture}[
    node distance=1.3cm,
    state/.style={rectangle, rounded corners, draw=milblue, fill=milblue!10, thick, minimum width=2.5cm, minimum height=0.7cm, align=center, font=\small},
    arrow/.style={->, thick, >=stealth}
]
    \node[state] (idle) {IDLE};
    \node[state, below of=idle] (ext) {EXTENDING};
    \node[state, below of=ext] (exd) {EXTENDED (HOLD)};
    \node[state, below of=exd] (ret) {RETRACTING};
    \node[state, right of=ret, node distance=4cm] (lock) {LOCKOUT (4s)};
    \node[state, fill=milred!20, below of=ret] (jam) {JAM DETECTED};
    \node[state, fill=milyellow!20, below of=jam] (fault) {FAULT};

    \draw[arrow] (idle) -- node[right, font=\tiny] {Charge button} (ext);
    \draw[arrow] (ext) -- node[right, font=\tiny] {Full extend + hold} (exd);
    \draw[arrow] (ext.south east) -- ++(0.8, 0) -- node[right, font=\tiny] {Full extend + release} ++(0,-1.7) -- (ret.east);
    \draw[arrow] (exd) -- node[right, font=\tiny] {Button released} (ret);
    \draw[arrow] (ret) -- node[above, font=\tiny] {Full retract} (lock);
    \draw[arrow] (lock) -- node[above, font=\tiny] {4 sec timer} ++(0,3.9) -- (idle.east);
    \draw[arrow, red, dashed] (ext) -- node[left, font=\tiny] {Jam detected} (jam);
    \draw[arrow, red, dashed] (ret) -- (jam);
    \draw[arrow] (jam) -- node[right, font=\tiny] {Backoff complete} (fault);
    \draw[arrow] (fault) -- node[left, font=\tiny] {Operator reset} ++(-2.5,0) |- (idle.west);
\end{tikzpicture}
\end{center}

\subsection{CHARGING STATE REFERENCE}

\small
\begin{longtable}{|L{0.15\textwidth}|L{0.25\textwidth}|L{0.50\textwidth}|}
\hline
\rowcolor{milblue!20}
\textbf{State} & \textbf{OSD Display} & \textbf{Description} \\
\hline
IDLE & CHARGING: Ready & Weapon ready to charge \\
\hline
EXTENDING & CHARGING: Extending & Actuator moving forward, pulling bolt \\
\hline
EXTENDED & CHARGING: Extended (HOLD) & Bolt pulled back, awaiting button release \\
\hline
RETRACTING & CHARGING: Retracting & Actuator returning, releasing bolt \\
\hline
LOCKOUT & CHARGING: Lockout (4s) & El-7aress H100 4-second post-charge safety period \\
\hline
JAM DETECTED & CHARGING: JAM! & High torque + no movement = mechanical jam \\
\hline
FAULT & CHARGING: FAULT & Requires operator acknowledgment and reset \\
\hline
\end{longtable}

\subsection{4-SECOND EL-7ARESS H100 LOCKOUT}

\begin{warningbox}[LOCKOUT PERIOD]
Per El-7aress H100 specification, a \textbf{4-second lockout period} follows each completed charging cycle. During this time:
\begin{itemize}
    \item Charging button is \textbf{IGNORED}
    \item Prevents accidental double-charge
    \item Allows weapon to fully seat round
    \item OSD displays countdown or "Lockout" status
\end{itemize}
\end{warningbox}

\subsection{JAM DETECTION AND RECOVERY}

The charging system includes automatic jam detection:

\subsubsection{Jam Detection Criteria}
\begin{itemize}
    \item \textbf{High Torque}: Actuator motor exceeds torque threshold (>80\%)
    \item \textbf{No Movement}: Position not changing despite motor effort
    \item \textbf{Confirmation}: Condition persists for multiple samples (anti-false-positive)
\end{itemize}

\subsubsection{Automatic Jam Response}
\begin{enumerate}
    \item \textbf{IMMEDIATE STOP}: Motor halts to prevent damage
    \item \textbf{STATE TRANSITION}: System enters JAM DETECTED state
    \item \textbf{ALARM}: OSD displays "CHARGING: JAM!"
    \item \textbf{DELAYED BACKOFF}: After 200ms stabilization, actuator retracts to home
    \item \textbf{FAULT STATE}: System enters FAULT state, awaits operator reset
\end{enumerate}

\subsubsection{Jam Clearing Procedure}

\begin{procedurebox}[CLEARING A CHARGING JAM]
\begin{enumerate}
    \item \textbf{STOP}: Do NOT repeatedly press charge button
    \item \textbf{OBSERVE}: Note jam position on OSD (if displayed)
    \item \textbf{WAIT}: Allow automatic backoff to complete
    \item \textbf{VERIFY}: Check OSD shows "FAULT" state
    \item \textbf{INSPECT}: If possible, visually inspect weapon for obstruction
    \item \textbf{CLEAR}: Remove any visible obstruction (per weapon manual)
    \item \textbf{RESET}: Press \button{Charge} button once to reset fault
    \item \textbf{RETRY}: Actuator performs safe retraction to home position
    \item \textbf{RE-CHARGE}: After successful reset, perform normal charging
\end{enumerate}
\end{procedurebox}

\begin{warningbox}[SAFETY]
If jam persists after 2 reset attempts, \textbf{STOP} and request maintenance support. Repeated jam cycling can damage the weapon or actuator.
\end{warningbox}

\subsection{CHARGING SAFETY INTERLOCKS}

Charging is \textbf{BLOCKED} when:

\small
\begin{longtable}{|L{0.30\textwidth}|L{0.60\textwidth}|}
\hline
\rowcolor{milblue!20}
\textbf{Condition} & \textbf{Reason/Resolution} \\
\hline
Emergency Stop Active & Clear E-stop before charging \\
\hline
Station Disabled & Enable station with Station Enable switch \\
\hline
Charge Already In Progress & Wait for current cycle to complete \\
\hline
Lockout Active & Wait 4 seconds after previous charge \\
\hline
Fault State & Reset fault by pressing Charge button \\
\hline
\end{longtable}

\subsection{STARTUP ACTUATOR CHECK}

On system startup, if the actuator is found in an extended position:

\begin{itemize}
    \item System automatically performs \textbf{safe retraction}
    \item Returns actuator to home (retracted) position
    \item Shorter timeout used (best-effort recovery)
    \item Prevents leaving weapon in unsafe partially-charged state
\end{itemize}

% ============================================================================
\section{WEAPONS ENGAGEMENT SEQUENCE}

\subsection{PRE-ENGAGEMENT CHECKLIST}

Before engaging, verify \textbf{ALL} items:

\begin{checklist}
    \item Target positively identified (\textbf{PID})
    \item Target valid (meets ROE)
    \item Fire authorization received (if required)
    \item NOT in no-fire zone (check OSD for zone warnings)
    \item Friendly forces clear
    \item Weapon loaded and ready
    \item Zeroing active (if applicable)
    \item Environmental parameters set (if applicable)
    \item Track established (if using tracking)
\end{checklist}

\begin{warningbox}
IF ANY ITEM CANNOT BE CHECKED, DO NOT FIRE.
\end{warningbox}

\subsection{ENGAGEMENT PROCEDURE (STEP-BY-STEP)}

\subsubsection{STEP 1: Acquire and Track Target}

\begin{enumerate}
    \item Detect and identify target (PID mandatory)
    \item \button{Button 4} → Enter acquisition mode
    \item \button{D-Pad} → Size gate to frame target (10-30\% margin)
    \item \button{Button 4} → Request lock-on
    \item Wait for \textbf{ACTIVE LOCK} (green gate, OSD: TRACKING)
    \item Monitor track quality (confidence >70\%, green gate)
\end{enumerate}

\subsubsection{STEP 2: Range Target}

\begin{enumerate}
    \item Fire Laser Range Finder (LRF trigger)
    \item Wait for range reading (OSD: \osd{RNG: xxxx m})
    \item Verify range reasonable
    \item Range used for ballistic calculations (CCIP)
\end{enumerate}

\begin{notebox}
LRF may fire automatically during tracking (configuration-dependent). Verify range displayed.
\end{notebox}

\subsubsection{STEP 3: Enable Lead Angle Compensation (If Moving Target)}

\textbf{If target is moving} (lateral motion, speed >5 m/s, range >100m):

\begin{enumerate}
    \item Hold \button{Dead Man Switch (Button 3)}
    \item Press \button{LAC Toggle (Button 2)}
    \item Verify LAC status:
    \begin{itemize}
        \item ✅ \textbf{"LEAD ANGLE ON"} (green) = Ready
        \item ⚠️ \textbf{"LEAD ANGLE LAG"} (yellow) = Wait for tracking data
        \item ❌ \textbf{"ZOOM OUT"} (red) = FOV too narrow, zoom out
    \end{itemize}
    \item Observe CCIP reticle offset ahead of target (lead point)
\end{enumerate}

\textbf{If target stationary:} LAC not necessary (CCIP at target center)

\begin{notebox}
Detailed LAC procedures in Lesson 6.
\end{notebox}

\subsubsection{STEP 4: Final Safety Checks}

\begin{checklist}
    \item Verify target still valid
    \item Verify tracking active (green gate, good confidence)
    \item Check OSD for warnings:
    \begin{itemize}
        \item ❌ \textbf{"ZONE VIOLATION"} = DO NOT FIRE
        \item ❌ \textbf{"NO-FIRE ZONE"} = DO NOT FIRE
        \item ✅ No warnings = Clear to fire
    \end{itemize}
    \item Verify friendly forces clear
    \item Verify backstop (if required)
\end{checklist}

\subsubsection{STEP 5: Engage Master Arm}

\begin{enumerate}
    \item Pull trigger to \textbf{Stage 1} (half-pull) → \textbf{Master Arm} (\button{Button 0})
    \begin{itemize}
        \item OR toggle Master Arm switch on Control Panel
    \end{itemize}
    \item Verify \textbf{"ARMED"} indicator light ON (red)
    \item OSD may display: \osd{WEAPON ARMED}
\end{enumerate}

\begin{warningbox}
WEAPON IS NOW HOT
\end{warningbox}

\subsubsection{STEP 6: Fire Weapon}

\begin{enumerate}
    \item Final aim verification (CCIP on target or lead point)
    \item Pull trigger to \textbf{Stage 2} (full-pull) → \textbf{Fire} (\button{Button 5})
    \item Weapon fires
    \item Hold trigger for desired burst:
    \begin{itemize}
        \item \textbf{Single Shot:} Quick press/release (1 round)
        \item \textbf{Burst:} Hold 2-3 seconds (controlled burst)
        \item \textbf{Sustained:} Hold longer (use cautiously)
    \end{itemize}
\end{enumerate}

\textbf{During Firing:}
\begin{itemize}
    \item Tracking keeps reticle on target
    \item Gimbal compensates for recoil
    \item Observe rounds impacting
    \item Adjust fire as needed
\end{itemize}

\textbf{System State:}
\begin{itemize}
    \item Tracking phase → \textbf{FIRING}
    \item OSD: \osd{MODE: TRACKING (FIRING)}
    \item Gate: Green solid outline
    \item Enhanced stabilization active
\end{itemize}

\subsubsection{Dead Reckoning During Firing (El-7aress Doctrine)}

\begin{warningbox}[IMPORTANT - DEAD RECKONING]
Per El-7aress H100 specification: "When firing is initiated, the system aborts Target Tracking. Instead the system moves according to the speed and direction of the WS just prior to pulling the trigger. The system will not automatically compensate for changes in speed or direction of the tracked target during firing."
\end{warningbox}

\textbf{Dead Reckoning Behavior}:
\begin{itemize}
    \item When trigger is pulled, system captures \textbf{last known target velocity}
    \item Gimbal continues moving at captured velocity (azimuth and elevation rates)
    \item Tracker is \textbf{NOT actively following} target during firing
    \item System \textbf{predicts} target position based on last velocity
    \item If target maneuvers during firing, rounds may miss
\end{itemize}

\textbf{Operator Implications}:
\begin{itemize}
    \item ✅ Best for targets with \textbf{constant velocity} (vehicles on road)
    \item ⚠️ Less effective against \textbf{maneuvering targets}
    \item ⚠️ Fire short bursts, reassess, fire again for erratic targets
    \item After firing stops, must \textbf{re-acquire} target to resume tracking
\end{itemize}

\subsubsection{STEP 7: Cease Fire}

\begin{enumerate}
    \item Release \textbf{Fire button} (\button{Button 5} / trigger stage 2)
    \item Weapon stops firing
    \item Release \textbf{Master Arm} (\button{Button 0} / trigger stage 1)
    \item Verify \textbf{"ARMED"} indicator OFF
    \item Tracking continues (unless aborted)
\end{enumerate}

\subsubsection{STEP 8: Assess Target (BDA)}

\textbf{Battle Damage Assessment:}
\begin{itemize}
    \item ✅ Target destroyed? → Stop tracking, report success
    \item ⚠️ Target damaged? → Re-engage (repeat Steps 5-7)
    \item ❌ Target missed? → Check zeroing/environmental/LAC, re-engage
    \item �� Target suppressed? → Maintain track, ready to re-engage
\end{itemize}

\subsubsection{STEP 9: Post-Engagement Actions}

\begin{enumerate}
    \item If target neutralized: \textbf{Stop tracking} (double-click \button{Button 4})
    \item Report engagement results to command
    \item Update ammunition count
    \item Scan for additional targets
    \item Resume surveillance or follow orders
\end{enumerate}

% ============================================================================
\section{ENGAGEMENT BEST PRACTICES}

\subsection{TARGET SELECTION FOR TRACKING}

\subsubsection{Good Targets}
\begin{itemize}
    \item ✅ High contrast against background
    \item ✅ Clearly defined edges
    \item ✅ Sufficient size (>30 pixels)
    \item ✅ Relatively stable motion
\end{itemize}

\subsubsection{Difficult Targets}
\begin{itemize}
    \item ⚠️ Low contrast (camouflaged)
    \item ⚠️ Very small (distant)
    \item ⚠️ Erratic motion (evasive)
    \item ⚠️ Partially obscured
\end{itemize}

\subsubsection{If Tracking Fails}
\begin{itemize}
    \item Try manual engagement (no tracking)
    \item Improve contrast (switch camera or thermal LUT)
    \item Wait for better tracking opportunity
\end{itemize}

\subsection{LEAD ANGLE COMPENSATION TIPS}

\subsubsection{When to Use LAC}
\begin{itemize}
    \item ✅ Target moving laterally (crossing FOV)
    \item ✅ Target speed >5 m/s (~10 mph)
    \item ✅ Range >100 meters
\end{itemize}

\subsubsection{When NOT Needed}
\begin{itemize}
    \item ❌ Stationary targets
    \item ❌ Targets moving radially (toward/away from you)
    \item ❌ Very close range (<50m)
\end{itemize}

\subsubsection{LAC Limitations}
\begin{itemize}
    \item Requires tracking active
    \item Requires sufficient FOV (may need to zoom out)
    \item Assumes constant target velocity (less accurate if maneuvering)
\end{itemize}

\subsection{AMMUNITION CONSERVATION}

\subsubsection{Fire Discipline}
\begin{itemize}
    \item Use controlled bursts (2-5 rounds) vs. full-auto spray
    \item Assess after each burst before re-engaging
    \item Precision over volume
\end{itemize}

\subsubsection{Round Count}
\begin{itemize}
    \item Track ammunition expenditure
    \item Report when low (<20\% remaining)
    \item Conserve for high-priority targets
\end{itemize}

\subsection{SAFETY REMINDERS}

\subsubsection{ALWAYS}
\begin{itemize}
    \item ✅ Verify target before engaging (PID mandatory)
    \item ✅ Check for friendly forces
    \item ✅ Verify NOT in no-fire zone
    \item ✅ Follow Rules of Engagement (ROE)
    \item ✅ Have fire authorization (if required)
\end{itemize}

\subsubsection{NEVER}
\begin{itemize}
    \item ❌ Fire without positive identification
    \item ❌ Fire into no-fire zones
    \item ❌ Fire if friendlies potentially in line of fire
    \item ❌ Fire without authorization (if required)
    \item ❌ Assume tracking is infallible (monitor track quality)
\end{itemize}

% ============================================================================
\section{TRACKING QUICK REFERENCE}

\begin{quickref}
\small
\begin{longtable}{|L{0.25\textwidth}|L{0.35\textwidth}|L{0.25\textwidth}|}
\hline
\rowcolor{milblue!20}
\textbf{Situation} & \textbf{Action} & \textbf{Button/Control} \\
\hline
\endfirsthead

\hline
\rowcolor{milblue!20}
\textbf{Situation} & \textbf{Action} & \textbf{Button/Control} \\
\hline
\endhead

Start tracking & Enter acquisition mode & \button{Button 4} (1st press) \\
\hline
Size gate larger & Increase dimensions & \button{D-Pad ▼} (height) / \button{►} (width) \\
\hline
Size gate smaller & Decrease dimensions & \button{D-Pad ▲} (height) / \button{◄} (width) \\
\hline
Request lock-on & Lock onto target & \button{Button 4} (2nd press) \\
\hline
Monitor track & Observe gate color \& confidence & Visual (OSD) \\
\hline
Track degrading & Prepare for coast or abort & Stand by \button{Button 4} \\
\hline
Emergency abort & Stop tracking immediately & \button{Button 4} (double-click <500ms) \\
\hline
Enable LAC & Activate lead compensation & \button{Button 3 + Button 2} \\
\hline
Arm weapon & Engage Master Arm & \button{Button 0} (or trigger stage 1) \\
\hline
Fire weapon & Discharge weapon & \button{Button 5} (or trigger stage 2) \\
\hline
Cease fire & Safe weapon & Release \button{Button 5 \& Button 0} \\
\hline
\end{longtable}
\end{quickref}

% ============================================================================
% END OF LESSON 5
% ============================================================================