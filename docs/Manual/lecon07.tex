% ============================================================================
% LESSON 7 - SYSTEM STATUS & MONITORING
% ============================================================================
\lesson{7 - SYSTEM STATUS \& MONITORING}

\begin{tabular}{@{}ll@{}}
\textbf{Duration:} & 2 hours \\
\textbf{Type:} & Classroom \\
\textbf{References:} & System status manual, device specifications \\
\end{tabular}

\section{LEARNING OBJECTIVES}

Upon completion of this lesson, operators will be able to:
\begin{itemize}
    \item Access System Status display
    \item Interpret device status indicators
    \item Identify fault conditions
    \item Determine when to escalate to maintenance
\end{itemize}

% ============================================================================
\section{ACCESSING SYSTEM STATUS}

\subsection{SYSTEM STATUS MENU ACCESS}

\begin{procedurebox}[ACCESSING SYSTEM STATUS]
\begin{enumerate}
    \item Press \button{MENU ✓}
    \item Navigate to \textbf{"System Status"}
    \item Press \button{VAL}
\end{enumerate}
\end{procedurebox}

\subsection{Display Sections}

The System Status display provides real-time monitoring of all critical subsystems:

\begin{itemize}
    \item Azimuth/Elevation Servos
    \item IMU (Inertial Measurement Unit)
    \item Laser Range Finder (LRF)
    \item Day/Night Cameras
    \item Control Panels
    \item Servo Actuator
    \item Homing Status
    \item Charging Status
    \item Motion Mode
    \item Alarms/Warnings
\end{itemize}

% ----------------------------------------------------------------------------
\subsection{HOMING SEQUENCE STATUS}

During system startup or when homing is requested, the OSD displays the current homing phase:

\begin{longtable}{|L{0.20\textwidth}|L{0.45\textwidth}|L{0.25\textwidth}|}
\hline
\rowcolor{milblue!20}
\textbf{Status} & \textbf{Meaning} & \textbf{Operator Action} \\
\hline
\endfirsthead

\hline
\rowcolor{milblue!20}
\textbf{Status} & \textbf{Meaning} & \textbf{Operator Action} \\
\hline
\endhead

\textbf{Idle} & No homing operation active & Normal ops \\
\hline
\textbf{Requested} & Homing command sent to servos & Wait \\
\hline
\textbf{In Progress} & Motors actively seeking home position & Wait, do not interrupt \\
\hline
\textbf{Completed} & Home position found successfully & System ready \\
\hline
\textbf{Failed} & Homing timeout or error occurred & May need to retry \\
\hline
\textbf{Aborted} & Homing cancelled (E-Stop or other) & Check safety, may retry \\
\hline
\end{longtable}

\begin{notebox}
Motion modes are blocked during homing. Wait for homing to complete before operating the gimbal.
\end{notebox}

% ----------------------------------------------------------------------------
\subsection{MOTION MODE STATUS}

The OSD displays the current motion mode. The following modes may be displayed:

\begin{longtable}{|L{0.18\textwidth}|L{0.50\textwidth}|L{0.22\textwidth}|}
\hline
\rowcolor{milblue!20}
\textbf{Mode} & \textbf{Description} & \textbf{Gimbal Control} \\
\hline
\endfirsthead

\hline
\rowcolor{milblue!20}
\textbf{Mode} & \textbf{Description} & \textbf{Gimbal Control} \\
\hline
\endhead

\textbf{Manual} & Operator joystick control with stabilization & Operator (joystick) \\
\hline
\textbf{AutoSectorScan} & Automated scanning between two azimuth points & Automatic \\
\hline
\textbf{TRPScan} & Sequential scan through Target Reference Points & Automatic \\
\hline
\textbf{RadarSlew} & Slewing to radar-designated target & Automatic \\
\hline
\textbf{Tracking} & Vision-based target tracking active & Automatic (tracker) \\
\hline
\textbf{FREE} & Gimbal brakes released, manual positioning & None (unpowered) \\
\hline
\end{longtable}

\begin{warningbox}[FREE MODE]
When the local FREE toggle switch is ON, motion mode displays as \textbf{FREE}. The gimbal brakes are released and no servo power is applied. The gimbal can be manually positioned by hand. Ensure the area is clear before physically moving the gimbal.
\end{warningbox}

% ============================================================================
\section{DEVICE STATUS REFERENCE}

\subsection{GIMBAL SERVOS (AZIMUTH \& ELEVATION)}

\begin{longtable}{|L{0.18\textwidth}|L{0.15\textwidth}|L{0.15\textwidth}|L{0.15\textwidth}|L{0.28\textwidth}|}
\hline
\rowcolor{milblue!20}
\textbf{Parameter} & \textbf{Normal} & \textbf{Warning} & \textbf{Fault} & \textbf{Action} \\
\hline
\endfirsthead

\hline
\rowcolor{milblue!20}
\textbf{Parameter} & \textbf{Normal} & \textbf{Warning} & \textbf{Fault} & \textbf{Action} \\
\hline
\endhead

\textbf{Connected} & ✓ & - & ✗ & Notify maintenance \\
\hline
\textbf{Torque} & 0-50\% & 50-80\% & >80\% sustained & Allow cooling, reduce motion \\
\hline
\textbf{Motor Temp} & 20-60\degree C & 60-70\degree C & >70\degree C & Halt operations, allow cooling \\
\hline
\textbf{Driver Temp} & 20-60\degree C & 60-70\degree C & >70\degree C & Halt operations, allow cooling \\
\hline
\textbf{Fault Flag} & No & - & Yes & E-Stop, notify maintenance \\
\hline
\end{longtable}

\subsubsection{Servo Monitoring Guidelines}

\textbf{Normal Operation:}
\begin{itemize}
    \item Torque values fluctuate with gimbal motion
    \item Higher torque during rapid movements or when carrying loads
    \item Temperatures rise gradually during extended operations
\end{itemize}

\textbf{Warning Signs:}
\begin{itemize}
    \item Sustained high torque (>50\%) during light movements
    \item Rapidly rising temperatures
    \item Unusual vibrations or noises
\end{itemize}

% ----------------------------------------------------------------------------
\subsection{IMU (INERTIAL MEASUREMENT UNIT)}

\begin{longtable}{|L{0.18\textwidth}|L{0.15\textwidth}|L{0.15\textwidth}|L{0.15\textwidth}|L{0.28\textwidth}|}
\hline
\rowcolor{milblue!20}
\textbf{Parameter} & \textbf{Normal} & \textbf{Warning} & \textbf{Fault} & \textbf{Action} \\
\hline
\endfirsthead

\hline
\rowcolor{milblue!20}
\textbf{Parameter} & \textbf{Normal} & \textbf{Warning} & \textbf{Fault} & \textbf{Action} \\
\hline
\endhead

\textbf{Connected} & ✓ & - & ✗ & Stabilization offline, notify maint \\
\hline
\textbf{Roll/Pitch} & ±30\degree & ±30-45\degree & >45\degree & Platform unstable, secure vessel \\
\hline
\textbf{Temperature} & 20-60\degree C & 60-70\degree C & >70\degree C & Allow cooling \\
\hline
\end{longtable}

\begin{warningbox}[IMU CRITICAL FUNCTION]
IMU offline = No stabilization. Manual mode only.
\end{warningbox}

\subsubsection{IMU Role in System}

The IMU provides:
\begin{itemize}
    \item Platform orientation data (roll, pitch, yaw)
    \item Angular rate measurements for stabilization
    \item Motion compensation for tracking
    \item Essential data for ballistic calculations
\end{itemize}

\textbf{Impact of IMU Failure:}
\begin{itemize}
    \item No gyrostabilization (image shake during platform motion)
    \item Degraded tracking performance
    \item Reduced ballistic accuracy
    \item Manual mode operations only
\end{itemize}

% ----------------------------------------------------------------------------
\subsection{LASER RANGE FINDER (LRF)}

\begin{longtable}{|L{0.18\textwidth}|L{0.15\textwidth}|L{0.15\textwidth}|L{0.15\textwidth}|L{0.28\textwidth}|}
\hline
\rowcolor{milblue!20}
\textbf{Parameter} & \textbf{Normal} & \textbf{Warning} & \textbf{Fault} & \textbf{Action} \\
\hline
\endfirsthead

\hline
\rowcolor{milblue!20}
\textbf{Parameter} & \textbf{Normal} & \textbf{Warning} & \textbf{Fault} & \textbf{Action} \\
\hline
\endhead

\textbf{Connected} & ✓ & - & ✗ & Manual range estimation, notify maint \\
\hline
\textbf{Temperature} & 20-50\degree C & 50-60\degree C & >60\degree C & Reduce firing rate, allow cooling \\
\hline
\textbf{Laser Count} & Incrementing & - & Not incrementing & LRF malfunction \\
\hline
\textbf{No Echo} & Occasional & Frequent & Always & Target too far, obscured, or LRF fault \\
\hline
\end{longtable}

\begin{warningbox}[LRF CRITICAL FUNCTION]
LRF offline = No ballistics, no LAC. Manual engagement only.
\end{warningbox}

\subsubsection{LRF Status Interpretation}

\textbf{Laser Count:}
\begin{itemize}
    \item Increments with each successful ranging attempt
    \item Should increment even if "NO ECHO" received
    \item If frozen: LRF communication failure
\end{itemize}

\textbf{"NO ECHO" Conditions:}
\begin{itemize}
    \item Target beyond maximum range (~5km for hard targets)
    \item Target obscured by smoke, fog, rain
    \item Target surface non-reflective (absorbs laser)
    \item Laser beam blocked by obstruction
    \item LRF receiver malfunction (if persistent)
\end{itemize}

% ----------------------------------------------------------------------------
\subsection{CAMERAS (DAY \& THERMAL)}

\begin{longtable}{|L{0.18\textwidth}|L{0.15\textwidth}|L{0.15\textwidth}|L{0.15\textwidth}|L{0.28\textwidth}|}
\hline
\rowcolor{milblue!20}
\textbf{Parameter} & \textbf{Normal} & \textbf{Warning} & \textbf{Fault} & \textbf{Action} \\
\hline
\endfirsthead

\hline
\rowcolor{milblue!20}
\textbf{Parameter} & \textbf{Normal} & \textbf{Warning} & \textbf{Fault} & \textbf{Action} \\
\hline
\endhead

\textbf{Connected} & ✓ & - & ✗ & Switch cameras if available \\
\hline
\textbf{Error Flag} & No & - & Yes & Switch cameras, notify maint \\
\hline
\textbf{Thermal FFC} & Occasional & Frequent (<5min) & Constant & Thermal camera fault \\
\hline
\textbf{Focus} & Clear & Slightly blurred & Very blurred & Autofocus off, manual focus \\
\hline
\end{longtable}

\begin{warningbox}[CAMERA CRITICAL FUNCTION]
Both cameras offline = Mission abort. Cannot engage without visual.
\end{warningbox}

\subsubsection{Camera Status Notes}

\textbf{Thermal FFC (Flat Field Correction):}
\begin{itemize}
    \item Normal: Occurs every 5-15 minutes
    \item Recalibrates thermal sensor for accurate temperature readings
    \item Brief image freeze (1-2 seconds) during FFC
    \item Frequent FFC (<5 min intervals): Sensor instability
\end{itemize}

\textbf{Focus Issues:}
\begin{itemize}
    \item Autofocus may struggle with low-contrast scenes
    \item Manual focus available via menu controls
    \item Persistent blur: Lens contamination or camera fault
\end{itemize}

% ----------------------------------------------------------------------------
\subsection{CONTROL PANELS}

\spectable{
\begin{tabular}{|L{0.15\textwidth}|L{0.28\textwidth}|L{0.25\textwidth}|L{0.23\textwidth}|}
\hline
\rowcolor{milblue!20}
\textbf{Device} & \textbf{Function} & \textbf{Fault Impact} & \textbf{Action} \\
\hline
\textbf{DCU} & Buttons, switches, lights & No local control, joystick only & Notify maint (ops continue) \\
\hline
\textbf{Joystick} & Gimbal control, tracking, fire & Mission abort if no backup & Emergency: Use DCU manual control \\
\hline
\end{tabular}
}

\subsubsection{Control Panel Redundancy}

\textbf{Primary Control: Joystick}
\begin{itemize}
    \item All operator functions accessible
    \item Preferred for normal operations
    \item Ergonomic, intuitive control
\end{itemize}

\textbf{Backup Control: DCU (Direct Control Unit)}
\begin{itemize}
    \item Limited functionality (basic gimbal control)
    \item Use if joystick fails
    \item Slower, less precise than joystick
\end{itemize}

% ============================================================================
\section{ALARM INTERPRETATION}

\subsection{CRITICAL ALARMS (RED)}

\begin{longtable}{|L{0.28\textwidth}|L{0.30\textwidth}|L{0.32\textwidth}|}
\hline
\rowcolor{milred!20}
\textbf{Alarm} & \textbf{Meaning} & \textbf{Immediate Action} \\
\hline
\endfirsthead

\hline
\rowcolor{milred!20}
\textbf{Alarm} & \textbf{Meaning} & \textbf{Immediate Action} \\
\hline
\endhead

\textbf{EMERGENCY STOP ACTIVE} & E-Stop engaged & Identify cause, reset when safe \\
\hline
\textbf{WEAPON ARMED (NO AUTH)} & Safety violation & Disarm immediately, investigate \\
\hline
\textbf{ZONE VIOLATION} & Entering No-Fire/No-Traverse & Halt motion, verify zone config \\
\hline
\textbf{SERVO FAULT} & Gimbal malfunction & E-Stop, notify maintenance \\
\hline
\end{longtable}

\subsubsection{Critical Alarm Response Priority}

\textbf{Priority 1: Immediate Safety Threats}
\begin{enumerate}
    \item \textbf{WEAPON ARMED (NO AUTH):} Disarm weapon immediately
    \item \textbf{ZONE VIOLATION:} Halt all motion, assess situation
    \item \textbf{SERVO FAULT:} E-Stop if gimbal behaving erratically
\end{enumerate}

\textbf{Priority 2: System Protection}
\begin{enumerate}
    \item \textbf{EMERGENCY STOP ACTIVE:} Determine if intentional or fault-triggered
\end{enumerate}

% ----------------------------------------------------------------------------
\subsection{WARNING ALARMS (YELLOW)}

\begin{longtable}{|L{0.28\textwidth}|L{0.30\textwidth}|L{0.32\textwidth}|}
\hline
\rowcolor{milyellow!20}
\textbf{Alarm} & \textbf{Meaning} & \textbf{Action} \\
\hline
\endfirsthead

\hline
\rowcolor{milyellow!20}
\textbf{Alarm} & \textbf{Meaning} & \textbf{Action} \\
\hline
\endhead

\textbf{HIGH TEMPERATURE} & Device overheating & Reduce operations, monitor \\
\hline
\textbf{LOW CONFIDENCE TRACKING} & Track quality poor & Verify target, consider manual \\
\hline
\textbf{ZOOM OUT} & LAC FOV insufficient & Zoom out (\button{Button 8}) \\
\hline
\textbf{LEAD ANGLE LAG} & Tracking data insufficient & Wait 2-5 seconds \\
\hline
\end{longtable}

\subsubsection{Warning Alarm Management}

\textbf{HIGH TEMPERATURE:}
\begin{itemize}
    \item Identify which device is overheating (check System Status)
    \item Reduce operational tempo (slower movements, less frequent firing)
    \item Monitor temperature trend (rising or stabilizing?)
    \item If temperature continues to rise: Cease operations, allow cooling
\end{itemize}

\textbf{LOW CONFIDENCE TRACKING:}
\begin{itemize}
    \item Verify target is still visible and properly framed
    \item Check for obstructions or lighting changes
    \item Consider switching to manual engagement
    \item Re-acquire track if confidence drops below 50\%
\end{itemize}

% ----------------------------------------------------------------------------
\subsection{INFO MESSAGES (GREEN/BLUE)}

\begin{longtable}{|L{0.28\textwidth}|L{0.30\textwidth}|L{0.32\textwidth}|}
\hline
\rowcolor{milgreen!20}
\textbf{Message} & \textbf{Meaning} & \textbf{Action} \\
\hline
\endfirsthead

\hline
\rowcolor{milgreen!20}
\textbf{Message} & \textbf{Meaning} & \textbf{Action} \\
\hline
\endhead

\textbf{ZEROING APPLIED} & Zero active & Normal ops \\
\hline
\textbf{ENV APPLIED} & Environmental settings active & Normal ops \\
\hline
\textbf{TRACKING} & Tracking active & Monitor \\
\hline
\end{longtable}

\subsubsection{Informational Message Interpretation}

These messages confirm that optional features are active:
\begin{itemize}
    \item \textbf{ZEROING APPLIED:} Weapon boresight correction in effect
    \item \textbf{ENV APPLIED:} Ballistic corrections for temperature, altitude, wind active
    \item \textbf{TRACKING:} Automatic target tracking engaged
\end{itemize}

\textbf{No Action Required:} These are status confirmations, not warnings.

% ============================================================================
\section{WHEN TO ESCALATE TO MAINTENANCE}

\subsection{OPERATOR-LEVEL ISSUES (Fix Yourself)}

The following issues can be resolved by the operator:

\begin{itemize}
    \item ✅ Reset E-Stop (after clearing danger)
    \item ✅ Power cycle system (soft reboot)
    \item ✅ Switch cameras (if available)
    \item ✅ Adjust zoom/focus
    \item ✅ Re-zero weapon (if needed)
\end{itemize}

\subsection{MAINTENANCE-LEVEL ISSUES (Notify Immediately)}

The following issues require maintenance personnel:

\begin{itemize}
    \item ❌ Device disconnected (and doesn't reconnect after reboot)
    \item ❌ Persistent fault flags
    \item ❌ Servo malfunction (erratic motion, no motion)
    \item ❌ Physical damage observed
    \item ❌ Thermal runaway (temperature >80\degree C)
    \item ❌ Unusual noises (grinding, clicking, squealing)
\end{itemize}

\subsection{Escalation Decision Matrix}

\spectable{
\small
\begin{tabular}{|L{0.25\textwidth}|L{0.30\textwidth}|L{0.35\textwidth}|}
\hline
\rowcolor{milblue!20}
\textbf{Symptom} & \textbf{Operator Action} & \textbf{Escalate If} \\
\hline
Device offline & Power cycle, check connections & Still offline after reboot \\
\hline
High temperature & Reduce operations, allow cooling & Temp >80\degree C or rising \\
\hline
Tracking failure & Adjust gate, improve contrast & Fails on all targets \\
\hline
LRF no reading & Re-fire, try different target & Persistent "NO ECHO" on all targets \\
\hline
Poor video & Switch cameras, adjust focus & Both cameras offline or error \\
\hline
Servo slow/jerky & Check torque, allow cooling & Fault flag or erratic motion \\
\hline
Unusual noises & Identify source, check status & Grinding, clicking, or squealing \\
\hline
\end{tabular}
}

\subsection{Documentation for Maintenance}

When escalating to maintenance, provide:

\begin{checklist}
    \item Specific symptoms observed
    \item When problem first occurred
    \item Any recent changes (maintenance, configuration, environment)
    \item System Status screenshot (if possible)
    \item Any error messages or alarm codes
    \item Operator-level troubleshooting already attempted
\end{checklist}

\begin{notebox}
Better documentation = Faster repair. Be specific and thorough.
\end{notebox}

% ============================================================================
\section{SYSTEM STATUS MONITORING BEST PRACTICES}

\subsection{Regular Monitoring Schedule}

\textbf{Before Each Mission:}
\begin{itemize}
    \item Check System Status display
    \item Verify all devices connected
    \item Note any warnings or elevated temperatures
    \item Document baseline status
\end{itemize}

\textbf{During Operations:}
\begin{itemize}
    \item Monitor temperature trends (especially during extended operations)
    \item Watch for new alarms or warnings
    \item Check device status after any unusual events
\end{itemize}

\textbf{After Mission:}
\begin{itemize}
    \item Final System Status check
    \item Note any new faults or warnings
    \item Document any issues for maintenance
    \item Report any degraded performance
\end{itemize}

\subsection{Proactive Monitoring}

\textbf{Prevent Problems Before They Occur:}
\begin{itemize}
    \item Monitor temperature trends (rising temps indicate potential issues)
    \item Note gradual performance degradation (slower tracking, reduced accuracy)
    \item Report intermittent faults (even if they self-clear)
    \item Watch torque levels (sustained high torque indicates mechanical problems)
\end{itemize}

\begin{cautionbox}
Don't ignore intermittent problems. They often become permanent failures.
\end{cautionbox}

% ============================================================================
% END OF LESSON 7
% ============================================================================