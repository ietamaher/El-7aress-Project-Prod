%!TeX encoding = UTF-8
\UseRawInputEncoding
\documentclass[10pt,a4paper,twoside,openright]{book}

% ============================================================================
% PACKAGE IMPORTS
% ============================================================================
\usepackage[utf8]{inputenc}
\usepackage[T1]{fontenc}
\usepackage[english]{babel}
\usepackage{geometry}
\usepackage{fancyhdr}
\usepackage{titlesec}
\usepackage{tocloft}
\usepackage{graphicx}
\usepackage{xcolor}
\usepackage{colortbl}
\usepackage{longtable}
\usepackage{booktabs}
\usepackage{array}
\usepackage{enumitem}
\usepackage{tikz}
\usetikzlibrary{shapes,arrows,positioning,calc,decorations.pathreplacing}
\usepackage{listings}
\usepackage{hyperref}
\usepackage{bookmark}
\usepackage{tcolorbox}
\tcbuselibrary{breakable,skins}
\usepackage{amsmath}
\usepackage{amssymb}
\usepackage{textcomp}
\usepackage{wasysym}
\usepackage{fontawesome5}
\usepackage{pmboxdraw}
\usepackage{inconsolata}
\usepackage{multicol}
\usepackage{wrapfig}
\usepackage{caption}
\usepackage{subcaption}
\usepackage{setspace}

% Compact spacing
\setstretch{0.9}

% Optimized list spacing
\usepackage{enumitem}
\setlist{itemsep=0pt,parsep=0pt,topsep=2pt,leftmargin=*}
\setlist[itemize]{label=\textbullet,nosep}
\setlist[enumerate]{nosep}

% Icons
\newcommand{\warningsign}{\faExclamationTriangle}
\newcommand{\cautionsign}{\faExclamationCircle}
\newcommand{\notesign}{\faInfoCircle}

% Box drawing
\lstset{
  basicstyle=\ttfamily\small,
  extendedchars=true,
  literate=
    {│}{{\textSFxi}}1
    {├}{{\textSFviii}}1
    {─}{{\textSFx}}1
    {└}{{\textSFii}}1
    {┐}{{\textSFi}}1
    {┌}{{\textSFiii}}1
    {┤}{{\textSFiv}}1
    {┘}{{\textSFvi}}1
    {┴}{{\textSFxii}}1
    {┬}{{\textSFxiii}}1
    {┼}{{\textSFxv}}1
}

% ============================================================================
% PAGE GEOMETRY - MILITARY STANDARD
% ============================================================================
\geometry{
    a4paper,
    top=0.75in,
    bottom=0.75in,
    left=0.75in,
    right=0.75in,
    headheight=0.5in,
    headsep=0.2in,
    footskip=0.4in
}

% ============================================================================
% COLORS
% ============================================================================
\definecolor{milred}{RGB}{153,0,0}
\definecolor{milyellow}{RGB}{204,153,0}
\definecolor{milblue}{RGB}{0,51,102}
\definecolor{milgreen}{RGB}{0,102,51}
\definecolor{lightgray}{RGB}{245,245,245}
\definecolor{mediumgray}{RGB}{200,200,200}
\definecolor{darkgray}{RGB}{100,100,100}

% ============================================================================
% HYPERREF SETUP
% ============================================================================
\hypersetup{
    colorlinks=true,
    linkcolor=milblue,
    urlcolor=milblue,
    citecolor=milblue,
    bookmarksnumbered=true,
    bookmarksopen=true,
    pdftitle={EL 7ARRESS RCWS Operator Manual},
    pdfauthor={Tunisian Ministry of Defense},
    pdfsubject={Remote Controlled Weapon Station Operator Manual},
    pdfkeywords={RCWS, Operator Manual, Military, Weapon System}
}

% ============================================================================
% HEADER AND FOOTER SETUP
% ============================================================================
\pagestyle{fancy}
\fancyhf{}

\fancyhead[L]{\footnotesize\textcolor{milred}{\textbf{RESTRICTED}}}
\fancyhead[C]{\footnotesize\textit{El 7arress RCWS Operator Manual}}
\fancyhead[R]{\footnotesize\textcolor{milred}{\textbf{RESTRICTED}}}
\fancyfoot[L]{\footnotesize RCWS-OM-002-EN v1.0}
\fancyfoot[C]{\footnotesize\thepage}
\fancyfoot[R]{\footnotesize Jan 2025}

\fancypagestyle{plain}{
    \fancyhf{}
    \fancyhead[L]{\footnotesize\textcolor{milred}{\textbf{RESTRICTED}}}
    \fancyhead[R]{\footnotesize\textcolor{milred}{\textbf{RESTRICTED}}}
    \fancyfoot[C]{\footnotesize\thepage}
    \renewcommand{\headrulewidth}{0.3pt}
    \renewcommand{\footrulewidth}{0.3pt}
}

\renewcommand{\headrulewidth}{0.3pt}
\renewcommand{\footrulewidth}{0.3pt}

% ============================================================================
% TITLE FORMATTING
% ============================================================================
\titleformat{\chapter}[display]
  {\normalfont\LARGE\bfseries\color{milblue}}
  {\chaptertitlename\ \thechapter}{15pt}{\LARGE}
\titlespacing*{\chapter}{0pt}{20pt}{15pt}

\titleformat{\section}
  {\normalfont\large\bfseries\color{milblue}}
  {\thesection}{0.8em}{}
\titlespacing*{\section}{0pt}{10pt}{5pt}

\titleformat{\subsection}
  {\normalfont\normalsize\bfseries\color{milblue}}
  {\thesubsection}{0.8em}{}
\titlespacing*{\subsection}{0pt}{8pt}{4pt}

\titleformat{\subsubsection}
  {\normalfont\normalsize\bfseries\color{milblue}}
  {\thesubsubsection}{0.8em}{}
\titlespacing*{\subsubsection}{0pt}{6pt}{3pt}

% ============================================================================
% TABLE OF CONTENTS FORMATTING
% ============================================================================
\renewcommand{\cfttoctitlefont}{\LARGE\bfseries\color{milblue}}
\renewcommand{\cftchapfont}{\bfseries}
\renewcommand{\cftchappagefont}{\bfseries}
\setlength{\cftbeforechapskip}{6pt}
\setlength{\cftbeforesecskip}{2pt}

% ============================================================================
% WARNING/CAUTION BOXES
% ============================================================================
\newtcolorbox{warningbox}[1][WARNING]{
    colback=milred!5,
    colframe=milred,
    fonttitle=\bfseries\small,
    title={\large\warningsign\ #1},
    breakable,
    enhanced,
    sharp corners,
    boxrule=1.5pt,
    left=6pt,
    right=6pt,
    top=6pt,
    bottom=6pt,
    toptitle=3pt,
    bottomtitle=3pt
}

\newtcolorbox{cautionbox}[1][CAUTION]{
    colback=milyellow!10,
    colframe=milyellow,
    fonttitle=\bfseries\small,
    title={\large\cautionsign\ #1},
    breakable,
    enhanced,
    sharp corners,
    boxrule=1.5pt,
    left=6pt,
    right=6pt,
    top=6pt,
    bottom=6pt,
    toptitle=3pt,
    bottomtitle=3pt
}

\newtcolorbox{notebox}[1][NOTE]{
    colback=milblue!5,
    colframe=milblue,
    fonttitle=\bfseries\small,
    title={\large\notesign\ #1},
    breakable,
    enhanced,
    sharp corners,
    boxrule=1pt,
    left=6pt,
    right=6pt,
    top=5pt,
    bottom=5pt,
    toptitle=2pt,
    bottomtitle=2pt
}

\newtcolorbox{procedurebox}[1][PROCEDURE]{
    colback=milgreen!5,
    colframe=milgreen,
    fonttitle=\bfseries\small,
    title={#1},
    breakable,
    enhanced,
    sharp corners,
    boxrule=1pt,
    left=6pt,
    right=6pt,
    top=5pt,
    bottom=5pt
}

% ============================================================================
% LISTINGS SETUP
% ============================================================================
\lstset{
    basicstyle=\ttfamily\footnotesize,
    breaklines=true,
    frame=single,
    backgroundcolor=\color{lightgray},
    commentstyle=\color{milgreen},
    keywordstyle=\color{milblue}\bfseries,
    stringstyle=\color{milred},
    aboveskip=5pt,
    belowskip=5pt,
    xleftmargin=5pt,
    xrightmargin=5pt
}

% ============================================================================
% CUSTOM COMMANDS
% ============================================================================
\newcommand{\lesson}[1]{\chapter{LESSON #1}}
\newcommand{\appendixchapter}[1]{\chapter{APPENDIX #1}}

% Button and display notation
\newcommand{\button}[1]{\texttt{\textbf{[#1]}}}
\newcommand{\osd}[1]{\texttt{\textbf{#1}}}
\newcommand{\degree}{$^\circ$}

% Checkboxes
\newcommand{\checkbox}{$\square$\ }
\newcommand{\checkedbox}{$\boxtimes$\ }

% Checklist environment
\newlist{checklist}{itemize}{1}
\setlist[checklist]{label=$\square$,leftmargin=1.5em}

% Table column types
\newcolumntype{L}[1]{>{\raggedright\arraybackslash}p{#1}}
\newcolumntype{C}[1]{>{\centering\arraybackslash}p{#1}}
\newcolumntype{R}[1]{>{\raggedleft\arraybackslash}p{#1}}

\newenvironment{quickref}
{
    \begin{tcolorbox}[
        colback=lightgray,
        colframe=darkgray,
        sharp corners,
        boxrule=0.5pt,
        left=4pt,
        right=4pt,
        top=4pt,
        bottom=4pt
    ]
    \small
}
{
    \end{tcolorbox}
}

\newenvironment{twocol}
{
    \begin{multicols}{2}
    \small
}
{
    \end{multicols}
}

%============================================================================
% DOCUMENT BEGINS
% ============================================================================
\begin{document}

% ============================================================================
% TITLE PAGE
% ============================================================================
\begin{titlepage}
    \centering
    \vspace*{0.3cm}
    
    {\LARGE\textcolor{milred}{\textbf{RESTRICTED}}}\par
    \vspace{0.8cm}
    
    {\LARGE\bfseries\textcolor{milblue}{EL 7ARRESS RCWS}\par}
    \vspace{0.3cm}
    {\Large\bfseries REMOTE CONTROLLED WEAPON STATION\par}
    \vspace{0.3cm}
    {\Large\bfseries OPERATOR MANUAL (CONDENSED)\par}
    
    \vspace{1.5cm}
    
    % Image placeholder
    \vspace{1.5cm}
    
    {\Large\bfseries TUNISIAN MINISTRY OF DEFENSE\par}
    
    \vfill
    
    {\large VERSION 1.0\par}
    \vspace{0.2cm}
    {\large CLASSIFICATION: RESTRICTED\par}
    \vspace{0.2cm}
    {\large DATE: January 2025\par}
    
    \vspace{0.8cm}
    
    {\LARGE\textcolor{milred}{\textbf{RESTRICTED}}}\par
\end{titlepage}

% ============================================================================
% DOCUMENT CONTROL
% ============================================================================
\chapter*{DOCUMENT CONTROL}
\addcontentsline{toc}{chapter}{Document Control}

\small
\begin{longtable}{|L{0.28\textwidth}|L{0.67\textwidth}|}
\hline
\rowcolor{milblue!20}
\textbf{Item} & \textbf{Details} \\
\hline
\textbf{Document Title} & El 7arress RCWS Operator Manual (Condensed) \\
\hline
\textbf{Document Number} & RCWS-OM-002-EN \\
\hline
\textbf{Version} & 1.0 \\
\hline
\textbf{Date} & January 2025 \\
\hline
\textbf{Classification} & RESTRICTED \\
\hline
\textbf{Prepared By} & El 7arress Project Team \\
\hline
\textbf{Approved By} & Ministry of Defense - Equipment Division \\
\hline
\textbf{Language} & English \\
\hline
\textbf{Pages} & 95 \\
\hline
\end{longtable}

\section*{REVISION HISTORY}

\small
\begin{longtable}{|C{0.12\textwidth}|C{0.18\textwidth}|L{0.45\textwidth}|L{0.18\textwidth}|}
\hline
\rowcolor{milblue!20}
\textbf{Version} & \textbf{Date} & \textbf{Description} & \textbf{Author} \\
\hline
1.0 & Jan 2025 & Initial Condensed Release & El 7arress Team \\
\hline
\end{longtable}

\section*{DISTRIBUTION LIST}

\normalsize
This document is RESTRICTED and shall only be distributed to:
\begin{itemize}
    \item Authorized military operators
    \item Training personnel
    \item Maintenance personnel (as required)
    \item Command staff
\end{itemize}

\vspace{0.5cm}

\begin{warningbox}[DESTRUCTION NOTICE]
\textbf{DESTROY BY:} Shredding or burning when superseded or no longer needed.
\end{warningbox}

% ============================================================================
% WARNINGS AND CAUTIONS
% ============================================================================
\chapter*{WARNINGS AND CAUTIONS}
\addcontentsline{toc}{chapter}{Warnings and Cautions}

\begin{warningbox}[LETHAL WEAPON SYSTEM]
This system controls a live weapon platform. Failure to follow safety procedures can result in death or serious injury. Always treat the weapon as if it is loaded and armed.
\end{warningbox}

\vspace{0.3cm}

\begin{warningbox}[NO-FIRE ZONES]
Always verify reticle position is NOT in a no-fire zone before engaging. Violating no-fire zones may result in friendly casualties or civilian harm.
\end{warningbox}

\vspace{0.3cm}

\begin{warningbox}[EMERGENCY STOP]
The emergency stop button must be accessible at all times. Know its location and be prepared to activate immediately if unsafe conditions develop.
\end{warningbox}

\vspace{0.3cm}

\begin{cautionbox}[ELECTRICAL HAZARDS]
The RCWS operates on high-voltage power systems. Only qualified personnel shall perform maintenance. Disconnect power before servicing.
\end{cautionbox}

\vspace{0.3cm}

\begin{warningbox}[MOVING PARTS]
The gimbal system contains high-speed rotating components. Keep clear of moving parts during operation. Do not place hands or objects near turret during movement.
\end{warningbox}

\vspace{0.3cm}

\begin{cautionbox}[CAMERA DAMAGE]
Do not point cameras directly at the sun or intense light sources. Permanent sensor damage may occur.
\end{cautionbox}

\vspace{0.3cm}

\begin{warningbox}[DEAD MAN SWITCH]
The Dead Man Switch must be held during all weapon operations. Release immediately in emergency situations to safe the weapon.
\end{warningbox}

% ============================================================================
% TABLE OF CONTENTS
% ============================================================================
\tableofcontents

% ============================================================================
% QUICK REFERENCE CARD
% ============================================================================
\chapter*{QUICK REFERENCE CARD - CRITICAL CONTROLS}
\addcontentsline{toc}{chapter}{Quick Reference Card}

\section*{Emergency Controls}

\small
\begin{longtable}{|L{0.25\textwidth}|L{0.18\textwidth}|L{0.50\textwidth}|}
\hline
\rowcolor{milblue!20}
\textbf{Control} & \textbf{Location} & \textbf{Function} \\
\hline
\textbf{EMERGENCY STOP} & DCU (RED button) & Immediately stops all motion and safes weapon \\
\hline
\textbf{Dead Man Switch} & Joystick grip & Must be held for weapon operation; release to safe \\
\hline
\textbf{Gun Arm/Safe Switch} & DCU control panel & Arms or safes the weapon system \\
\hline
\textbf{Station Enable/Disable} & DCU control panel & Enables/disables entire system \\
\hline
\end{longtable}

\section*{Joystick Controls (Essential)}

\small
\begin{longtable}{|L{0.30\textwidth}|L{0.65\textwidth}|}
\hline
\rowcolor{milblue!20}
\textbf{Control} & \textbf{Function} \\
\hline
\textbf{Main Stick} & Gimbal slew (azimuth left/right, elevation up/down) \\
\hline
\textbf{Trigger} & Fire weapon (when armed and authorized) \\
\hline
\textbf{Track Select Button} & Initiate/abort tracking \\
\hline
\textbf{Camera Switch Button} & Toggle between Day and Thermal cameras \\
\hline
\textbf{LRF Button} & Activate laser rangefinder \\
\hline
\textbf{Zoom Control} & Camera zoom in/out \\
\hline
\end{longtable}

\section*{DCU Indicator Lights (Critical Status)}

\small
\begin{longtable}{|L{0.25\textwidth}|C{0.15\textwidth}|L{0.53\textwidth}|}
\hline
\rowcolor{milblue!20}
\textbf{Light} & \textbf{Color} & \textbf{Meaning} \\
\hline
\textbf{System Ready} & Green & System operational and ready \\
\hline
\textbf{Gun Armed} & Red & Weapon is armed - CAUTION \\
\hline
\textbf{Ammo Loaded} & Yellow & Ammunition detected \\
\hline
\textbf{Fault/Alarm} & Red (flash) & System fault - check status display \\
\hline
\textbf{Emergency Active} & Red (solid) & Emergency stop engaged \\
\hline
\end{longtable}

\section*{On-Screen Display (OSD) - Key Elements}

\begin{lstlisting}
┌─────────────────────────────────────────────────┐
│ AZ: 045°  EL: +12°           DAY  FOV: 9°      │
│                                                  │
│                    [RETICLE]                     │
│                       +                          │
│                      (*)  ← CCIP Pipper         │
│                                                  │
│ RNG: 850m            TRACK: ACTIVE              │
│ MODE: Manual         STATUS: ARMED              │
└─────────────────────────────────────────────────┘
\end{lstlisting}

\section*{Safety Rules (Memorize)}

\begin{enumerate}
    \item \textbf{NEVER} point weapon at friendly forces or civilians
    \item \textbf{ALWAYS} verify no-fire zones before engagement
    \item \textbf{ALWAYS} hold Dead Man Switch during operation
    \item \textbf{ALWAYS} confirm target identification before firing
    \item \textbf{IMMEDIATELY} press Emergency Stop if unsafe conditions develop
\end{enumerate}

% ============================================================================
% PROGRAM OF INSTRUCTION
% ============================================================================
\chapter*{PROGRAM OF INSTRUCTION (POI)}
\addcontentsline{toc}{chapter}{Program of Instruction}

\section*{Course Overview}

\noindent
\textbf{Course Title}: El 7arress RCWS Operator Qualification Course\\
\textbf{Course Code}: RCWS-OQC-001\\
\textbf{Course Length}: 40 hours (condensed training program)\\
\textbf{Class Size}: Maximum 8 students per instructor

\subsection*{Prerequisites}
\begin{itemize}
    \item Active military service
    \item Basic weapons qualification
    \item Security clearance (RESTRICTED level or higher)
    \item Medical fitness for duty
    \item Visual acuity 20/20 (corrected or uncorrected)
\end{itemize}

\subsection*{Course Description}

This condensed course qualifies military operators to safely and effectively operate the El 7arress Remote Controlled Weapon Station (RCWS). Students will learn system operation, targeting procedures, tracking techniques, ballistic compensation, emergency procedures, and live fire engagement through intensive practical training.

\section*{Training Hours Breakdown}

\small
\begin{longtable}{|C{0.08\textwidth}|L{0.45\textwidth}|C{0.12\textwidth}|L{0.25\textwidth}|}
\hline
\rowcolor{milblue!20}
\textbf{Lesson} & \textbf{Title} & \textbf{Hours} & \textbf{Type} \\
\hline
1 & Safety Brief \& System Overview & 3 & Classroom + Walk-Around \\
\hline
2 & Basic Operation & 4 & Classroom + Practical \\
\hline
3 & Menu System \& Settings & 2 & Practical \\
\hline
4 & Motion Modes \& Surveillance & 3 & Classroom + Practical \\
\hline
5 & Target Engagement Process & 4 & Classroom + Simulator \\
\hline
6 & Ballistics \& Fire Control & 4 & Classroom + Practical \\
\hline
7 & System Status \& Monitoring & 2 & Classroom + Practical \\
\hline
8 & Emergency Procedures & 2 & Classroom + Practical \\
\hline
9 & Operator Maintenance \& Troubleshooting & 2 & Classroom + Practical \\
\hline
10 & Practical Training \& Evaluation & 6 & Practical + Exam \\
\hline
\textbf{Live Fire} & Live Fire Qualification & 8 & Range \\
\hline
\rowcolor{milblue!20}
\textbf{TOTAL} & & \textbf{40} & \\
\hline
\end{longtable}

\section*{Graduation Requirements}

To successfully complete the course, students must:

\begin{enumerate}
    \item \textbf{Attend} all classroom and practical sessions (100\% attendance required)
    \item \textbf{Pass} written examination with minimum 80\% score
    \item \textbf{Pass} hands-on performance evaluation (GO/NO-GO on all critical tasks)
    \item \textbf{Demonstrate} safe weapons handling throughout training
    \item \textbf{Complete} live fire qualification with minimum 70\% accuracy
\end{enumerate}

\noindent
\textbf{Certificate}: Upon successful completion, graduates receive the \textbf{RCWS Operator Qualification Certificate} valid for 24 months. Annual refresher training required. Full recertification required every 24 months.

\noindent
\textbf{Failure to Meet Standards}: Students who fail to meet graduation requirements will be recycled to the next available course or removed from training per command guidance.

% ============================================================================
% LEARNING OBJECTIVES
% ============================================================================
\chapter*{LEARNING OBJECTIVES}
\addcontentsline{toc}{chapter}{Learning Objectives}

\section*{Terminal Learning Objective (TLO)}

\textbf{At the end of this course, the student will be able to:}

Safely and effectively operate the El 7arress RCWS system to acquire, track, and engage targets in accordance with military standards and safety regulations, while performing operator-level maintenance and troubleshooting procedures.

\section*{Enabling Learning Objectives (ELO)}

\begin{twocol}
\subsection*{LESSON 1}
\begin{itemize}
    \item \textbf{ELO 1.1}: Identify safety hazards
    \item \textbf{ELO 1.2}: Demonstrate emergency stop
    \item \textbf{ELO 1.3}: Explain zone restrictions
    \item \textbf{ELO 1.4}: Identify system components
    \item \textbf{ELO 1.5}: Describe component functions
\end{itemize}

\subsection*{LESSON 2}
\begin{itemize}
    \item \textbf{ELO 2.1}: Perform system startup
    \item \textbf{ELO 2.2}: Operate DCU controls
    \item \textbf{ELO 2.3}: Control gimbal movement
    \item \textbf{ELO 2.4}: Switch cameras
    \item \textbf{ELO 2.5}: Operate zoom controls
    \item \textbf{ELO 2.6}: Interpret OSD elements
    \item \textbf{ELO 2.7}: Perform shutdown
\end{itemize}

\subsection*{LESSON 3}
\begin{itemize}
    \item \textbf{ELO 3.1}: Navigate menus
    \item \textbf{ELO 3.2}: Configure reticle
    \item \textbf{ELO 3.3}: Access system status
    \item \textbf{ELO 3.4}: Modify settings safely
\end{itemize}

\subsection*{LESSON 4}
\begin{itemize}
    \item \textbf{ELO 4.1}: Explain motion modes
    \item \textbf{ELO 4.2}: Switch modes safely
    \item \textbf{ELO 4.3}: Operate sector scan
    \item \textbf{ELO 4.4}: Utilize TRP scan
    \item \textbf{ELO 4.5}: Define no-fire zones
    \item \textbf{ELO 4.6}: Define no-traverse zones
    \item \textbf{ELO 4.7}: Save/load configurations
\end{itemize}

\subsection*{LESSON 5}
\begin{itemize}
    \item \textbf{ELO 5.1}: Describe engagement sequence
    \item \textbf{ELO 5.2}: Initiate tracking
    \item \textbf{ELO 5.3}: Adjust acquisition box
    \item \textbf{ELO 5.4}: Interpret tracking status
    \item \textbf{ELO 5.5}: Execute engagement
    \item \textbf{ELO 5.6}: Abort tracking
\end{itemize}

\subsection*{LESSON 6}
\begin{itemize}
    \item \textbf{ELO 6.1}: Perform zeroing
    \item \textbf{ELO 6.2}: Adjust zeroing offsets
    \item \textbf{ELO 6.3}: Configure environment
    \item \textbf{ELO 6.4}: Apply corrections
    \item \textbf{ELO 6.5}: Enable lead angle
    \item \textbf{ELO 6.6}: Interpret CCIP
    \item \textbf{ELO 6.7}: Engage moving targets
\end{itemize}

\subsection*{LESSON 7}
\begin{itemize}
    \item \textbf{ELO 7.1}: Access status displays
    \item \textbf{ELO 7.2}: Interpret device status
    \item \textbf{ELO 7.3}: Monitor temperatures
    \item \textbf{ELO 7.4}: Recognize abnormalities
    \item \textbf{ELO 7.5}: Read diagnostics
\end{itemize}

\subsection*{LESSON 8}
\begin{itemize}
    \item \textbf{ELO 8.1}: Execute emergency stop
    \item \textbf{ELO 8.2}: Perform tracking abort
    \item \textbf{ELO 8.3}: Weapon safe procedures
    \item \textbf{ELO 8.4}: Respond to failures
    \item \textbf{ELO 8.5}: Emergency shutdown
\end{itemize}

\subsection*{LESSON 9}
\begin{itemize}
    \item \textbf{ELO 9.1}: Pre-operation checks
    \item \textbf{ELO 9.2}: Post-operation procedures
    \item \textbf{ELO 9.3}: Diagnose issues
    \item \textbf{ELO 9.4}: Troubleshoot problems
    \item \textbf{ELO 9.5}: Escalate to maintenance
    \item \textbf{ELO 9.6}: Complete logs
\end{itemize}

\subsection*{LESSON 10}
\begin{itemize}
    \item \textbf{ELO 10.1}: Demonstrate proficiency
    \item \textbf{ELO 10.2}: Execute simulations
    \item \textbf{ELO 10.3}: Maintain safety compliance
    \item \textbf{ELO 10.4}: Meet time standards
    \item \textbf{ELO 10.5}: Qualify live fire
\end{itemize}
\end{twocol}

% ============================================================================
% ABBREVIATIONS & GLOSSARY
% ============================================================================
\chapter*{ABBREVIATIONS \& GLOSSARY}
\addcontentsline{toc}{chapter}{Abbreviations \& Glossary}

\section*{Acronyms}

\begin{twocol}
\small
\begin{itemize}
    \item \textbf{AHRS} - Attitude and Heading Reference System
    \item \textbf{AZ} - Azimuth
    \item \textbf{BDA} - Battle Damage Assessment
    \item \textbf{CCIP} - Continuously Computed Impact Point
    \item \textbf{DCU} - Display and Control Unit
    \item \textbf{EL} - Elevation
    \item \textbf{ELO} - Enabling Learning Objective
    \item \textbf{FFC} - Flat Field Correction
    \item \textbf{FOV} - Field of View
    \item \textbf{HFOV} - Horizontal Field of View
    \item \textbf{LAC} - Lead Angle Compensation
    \item \textbf{LRF} - Laser Range Finder
    \item \textbf{OSD} - On-Screen Display
    \item \textbf{POI} - Program of Instruction
    \item \textbf{RCWS} - Remote Controlled Weapon Station
    \item \textbf{RPM} - Revolutions Per Minute
    \item \textbf{TLO} - Terminal Learning Objective
    \item \textbf{TRP} - Target Reference Point
    \item \textbf{VFOV} - Vertical Field of View
\end{itemize}
\end{twocol}

\section*{Key Terms}

\textbf{Acquisition Box}: Yellow box displayed on screen for selecting tracking target area.

\textbf{Active Lock}: Tracking state where system is successfully following target.

\textbf{Azimuth}: Horizontal angle measured clockwise from North (0\degree\ to 360\degree).

\textbf{Ballistics}: Science of projectile flight and trajectory.

\textbf{Boresight}: Alignment between camera line of sight and weapon bore axis.

\textbf{CCIP Pipper}: Impact prediction point with all ballistic corrections applied.

\textbf{Coast Mode}: Temporary tracking state when target is briefly obscured.

\textbf{Dead Man Switch}: Safety switch that must be held for weapon operation.

\textbf{Elevation}: Vertical angle measured from horizon (-20\degree\ to +60\degree\ typical).

\textbf{Gimbal}: Two-axis rotating mount for cameras and weapon.

\textbf{No-Fire Zone}: Defined area where weapon discharge is prohibited.

\textbf{No-Traverse Zone}: Defined area where gimbal movement is restricted.

\textbf{Reticle}: Aiming crosshair displayed on screen.

\textbf{Sector Scan}: Automatic scanning pattern between two defined points.

\textbf{Slew}: Rotate or move the gimbal.

\textbf{Stabilization}: Automatic compensation for platform movement.

\textbf{Tracking}: Automatic following of a selected target.

\textbf{Zeroing}: Process of aligning weapon point of impact with camera crosshair.

% ============================================================================
% PART I: SYSTEM FUNDAMENTALS
% ============================================================================
\part{SYSTEM FUNDAMENTALS}

% ============================================================================
% LESSON 1 - SAFETY BRIEF & SYSTEM OVERVIEW
% ============================================================================
\lesson{1: SAFETY BRIEF \& SYSTEM OVERVIEW}

\noindent
\begin{tabular}{@{}ll@{}}
\textbf{Duration:} & 3 hours \\
\textbf{Type:} & Classroom + Walk-Around \\
\textbf{References:} & Safety manual, system schematics \\
\end{tabular}

\section{Introduction}

\textbf{Lesson Purpose}: This lesson establishes the critical safety foundation for RCWS operations and provides an overview of system components and capabilities.

\textbf{Learning Objectives}:
\begin{itemize}
    \item Identify all safety hazards associated with RCWS operation
    \item Demonstrate emergency stop procedures
    \item Explain no-fire and no-traverse zone restrictions
    \item Identify all major system components during walk-around
    \item Describe the function of each major component
\end{itemize}

\section{Safety Brief}

\subsection{Five Fundamental Safety Rules}

\textbf{RULE 1: TREAT EVERY WEAPON AS LOADED}
\begin{itemize}
    \item Always assume the RCWS weapon is loaded and armed
    \item Never point the system at anything you do not intend to destroy
    \item Maintain constant awareness of muzzle direction
\end{itemize}

\textbf{RULE 2: KNOW YOUR TARGET AND WHAT LIES BEYOND}
\begin{itemize}
    \item Positively identify targets before engagement
    \item Consider overpenetration and ricochets
    \item Be aware of civilians, friendly forces, and infrastructure
    \item Verify range and backstop conditions
\end{itemize}

\textbf{RULE 3: VERIFY NO-FIRE AND NO-TRAVERSE ZONES}
\begin{itemize}
    \item Check zone status on OSD before every engagement
    \item Do not override zones without proper authorization
    \item Understand that zones protect friendly forces and civilians
    \item Report zone violations immediately
\end{itemize}

\textbf{RULE 4: KEEP DEAD MAN SWITCH ENGAGED ONLY WHEN READY TO FIRE}
\begin{itemize}
    \item Release Dead Man Switch immediately when not engaging
    \item Dead Man Switch is on joystick grip
    \item System automatically safes when switch is released
    \item Practice rapid release in training
\end{itemize}

\textbf{RULE 5: USE EMERGENCY STOP WHEN IN DOUBT}
\begin{itemize}
    \item Large RED button on DCU panel
    \item Activates immediately - no confirmation required
    \item Stops all gimbal motion and safes weapon
    \item Use without hesitation if any unsafe condition develops
\end{itemize}

\subsection{Specific Hazards}

\subsubsection{HAZARD 1: Weapon Discharge}

\begin{itemize}
    \item \textbf{Risk}: Death or serious injury from live ammunition
    \item \textbf{Mitigation}:
    \begin{itemize}
        \item Follow all five safety rules
        \item Verify Gun Armed light before trigger pull
        \item Clear weapon per Appendix A before maintenance
        \item Never bypass safety interlocks
    \end{itemize}
\end{itemize}

\subsubsection{HAZARD 2: Gimbal Movement}

\begin{itemize}
    \item \textbf{Risk}: Crushing injury from rotating turret
    \item \textbf{Mitigation}:
    \begin{itemize}
        \item Stay clear of turret during operation (minimum 2 meters)
        \item Use Emergency Stop if personnel enter hazard zone
        \item Never place hands or tools near moving parts
        \item Disable Station Enable switch before approaching turret
    \end{itemize}
\end{itemize}

\subsubsection{HAZARD 3: Electrical Shock}

\begin{itemize}
    \item \textbf{Risk}: Electrocution from high voltage (110-240V AC)
    \item \textbf{Mitigation}:
    \begin{itemize}
        \item Only qualified maintenance personnel open panels
        \item Disconnect power before any internal work
        \item Do not operate with damaged cables
        \item Report exposed wiring immediately
    \end{itemize}
\end{itemize}

\subsubsection{HAZARD 4: Laser Rangefinder}

\begin{itemize}
    \item \textbf{Risk}: Eye damage from laser (Class 3B laser device)
    \item \textbf{Mitigation}:
    \begin{itemize}
        \item Never look directly into laser aperture
        \item Do not aim at reflective surfaces at close range
        \item Laser safety goggles required for maintenance
        \item LRF automatically times out after 5 seconds
    \end{itemize}
\end{itemize}

\subsubsection{HAZARD 5: Thermal Camera Overheating}

\begin{itemize}
    \item \textbf{Risk}: Sensor damage from excessive heat
    \item \textbf{Mitigation}:
    \begin{itemize}
        \item Never point thermal camera at sun
        \item Do not aim at fires or intense heat sources
        \item Allow Flat Field Correction (FFC) to complete
        \item Monitor camera temperature on status display
    \end{itemize}
\end{itemize}

\subsection{Emergency Procedures (Quick Reference)}

\small
\begin{longtable}{|L{0.25\textwidth}|L{0.35\textwidth}|L{0.33\textwidth}|}
\hline
\rowcolor{milblue!20}
\textbf{Emergency} & \textbf{Immediate Action} & \textbf{Follow-Up} \\
\hline
\textbf{Runaway Gun} & Press EMERGENCY STOP & Notify command, clear area \\
\hline
\textbf{Misfire} & Maintain aim 30 sec, safe weapon & Follow misfire procedures \\
\hline
\textbf{Injury} & Press EMERGENCY STOP & Administer first aid, call medic \\
\hline
\textbf{Fire/Smoke} & Press EMERGENCY STOP, evacuate & Use fire extinguisher if safe \\
\hline
\textbf{Zone Violation} & Release trigger, safe weapon & Report incident immediately \\
\hline
\textbf{Loss of Video} & Press EMERGENCY STOP & Check connections, restart system \\
\hline
\textbf{Jammed Weapon} & Safe weapon, engage manual mode & Clear jam per weapon manual \\
\hline
\end{longtable}

\begin{notebox}[MEMORIZE]
Emergency Stop button location and Dead Man Switch release are your PRIMARY safety controls.
\end{notebox}

\section{System Overview}

\subsection{System Description}

The El 7arress RCWS (Remote Controlled Weapon Station) is a stabilized, remotely operated weapon platform designed for vehicle-mounted applications. The system provides:

\begin{twocol}
\begin{itemize}
    \item \textbf{360\degree\ azimuth rotation} (continuous)
    \item \textbf{-20\degree\ to +60\degree\ elevation} range
    \item \textbf{Day and thermal imaging} capability
    \item \textbf{Automatic target tracking}
    \item \textbf{Laser rangefinding} (50m to 4000m)
    \item \textbf{Ballistic compensation} for accurate fire
    \item \textbf{Zone protection} for safety
\end{itemize}
\end{twocol}

\textbf{Typical Applications}:
\begin{itemize}
    \item Perimeter defense
    \item Convoy protection
    \item Border surveillance
    \item Force protection
    \item Area denial
\end{itemize}

\subsection{Major System Components}

The RCWS consists of three main subsystems:

\subsubsection{1. Display and Control Unit (DCU)}

\begin{itemize}
    \item \textbf{Location}: Inside vehicle, operator station
    \item \textbf{Functions}:
    \begin{itemize}
        \item Video display (1024×768 resolution)
        \item Control buttons and switches
        \item Status indicator lights
        \item Menu navigation
        \item System settings
    \end{itemize}
\end{itemize}

\subsubsection{2. Joystick Controller}

\begin{itemize}
    \item \textbf{Location}: Inside vehicle, operator's right hand position
    \item \textbf{Functions}:
    \begin{itemize}
        \item Gimbal slew control (azimuth/elevation)
        \item Camera zoom
        \item Weapon trigger
        \item Tracking control
        \item Function buttons
        \item Dead Man Switch (safety)
    \end{itemize}
\end{itemize}

\subsubsection{3. Turret Assembly}

\begin{itemize}
    \item \textbf{Location}: Exterior vehicle roof mount
    \item \textbf{Components}:
    \begin{itemize}
        \item Electro-Optical System (cameras + laser rangefinder)
        \item Gimbal mechanism (2-axis stabilized)
        \item Weapon mount
        \item Drive motors and actuators
        \item Internal sensors and electronics
    \end{itemize}
\end{itemize}

\section{Component Walk-Around}

\subsection{Pre-Operation Inspection Sequence}

Perform this walk-around before every operation. Use checklist in Appendix C.

\subsubsection{STATION 1: Display and Control Unit (DCU)}

\textbf{Visual Inspection}:
\begin{enumerate}
    \item Check display screen for cracks or damage
    \item Verify all buttons and switches move freely
    \item Confirm indicator lights are not broken
    \item Check cable connections are secure
    \item Ensure ventilation ports are not blocked
\end{enumerate}

\textbf{DCU Control Panel Layout}:

\begin{lstlisting}
┌─────────────────────────────────────────────┐
│         VIDEO DISPLAY SCREEN                │
│                1024 × 768                    │
│                                              │
└─────────────────────────────────────────────┘
┌─────────────────────────────────────────────┐
│  [�� EMERGENCY]  [POWER]   [SYSTEM READY]   │
│     STOP                                     │
│                                              │
│  [STATION]  [HOME]  [GUN    [FIRE MODE]     │
│   ENABLE           ARM/SAFE]                │
│                                              │
│  [SPEED]    [STABIL.] [DETECT] [AMMO]       │
│   SELECT     ON/OFF    ON/OFF  [LOADED]     │
│                                              │
│  [MENU ▲]  [MENU ▼]  [MENU ✓]  [AUTHOR.]   │
│                                              │
└─────────────────────────────────────────────┘
\end{lstlisting}

\textbf{Button Functions} (detailed in Lesson 2):
\begin{twocol}
\begin{itemize}
    \item \textbf{Emergency Stop} (RED): Immediate system shutdown
    \item \textbf{Station Enable}: Master power for system
    \item \textbf{Home Position}: Return gimbal to forward (0\degree\ AZ, 0\degree\ EL)
    \item \textbf{Gun Arm/Safe}: Toggle weapon arming
    \item \textbf{Fire Mode Selector}: Single / Short Burst / Long Burst
    \item \textbf{Speed Select}: Low / Medium / High gimbal slew speed
    \item \textbf{Stabilization On/Off}: Enable/disable platform stabilization
    \item \textbf{Detection On/Off}: Enable/disable automatic target detection
    \item \textbf{Menu ▲/▼/✓}: Navigate menus
\end{itemize}
\end{twocol}

\textbf{Indicator Lights}:
\begin{itemize}
    \item \textbf{Power} (Green): System powered on
    \item \textbf{System Ready} (Green): All subsystems operational
    \item \textbf{Gun Armed} (Red): Weapon is armed - DANGER
    \item \textbf{Ammo Loaded} (Yellow): Ammunition detected
    \item \textbf{Authorized} (Green): Operator authorization active
    \item \textbf{Fault/Alarm} (Red Flashing): System error - check status
\end{itemize}

\textbf{GO/NO-GO Criteria}:
\begin{itemize}
    \item \checkedbox\ All lights illuminate during power-up self-test
    \item \checkedbox\ No physical damage to display or controls
    \item \checkedbox\ All buttons click and return properly
    \item \checkbox\ Any cracked screen → NO-GO (maintenance required)
    \item \checkbox\ Stuck buttons → NO-GO (maintenance required)
\end{itemize}

\subsubsection{STATION 2: Joystick Controller}

\textbf{Visual Inspection}:
\begin{enumerate}
    \item Check joystick moves smoothly in all directions
    \item Verify trigger guard is intact
    \item Test Dead Man Switch spring-back
    \item Confirm all buttons click properly
    \item Check cable connection is secure
\end{enumerate}

\textbf{Joystick Layout}:

\begin{lstlisting}
        ┌─────────────┐
        │  Zoom Rocker│  ← Camera Zoom In/Out
        │    ▲ ▼      │
        └─────────────┘
              │
        ┌─────────────┐
        │  [CAM] [TRK]│  ← Camera / Track Select
        │             │
        │  [LRF] [FN] │  ← Laser Range / Function
        │      ★      │  ← Hat Switch (8-way)
        │             │
        │   STICK     │  ← Main Stick (AZ/EL)
        │      │      │
        │      │      │
        └──────┼──────┘
               │
          ┌────┴────┐
          │ TRIGGER │  ← Weapon Trigger
          │  GUARD  │
          └─────────┘

    [DEAD MAN SWITCH] ← On grip (must hold)
\end{lstlisting}

\textbf{Control Functions} (detailed in Lesson 2):
\begin{itemize}
    \item \textbf{Main Stick}: Gimbal slew (left/right = azimuth, up/down = elevation)
    \item \textbf{Trigger}: Fire weapon (when armed)
    \item \textbf{Dead Man Switch}: Must be held for weapon operation
    \item \textbf{CAM Button}: Toggle Day/Thermal camera
    \item \textbf{TRK Button}: Initiate/abort tracking
    \item \textbf{LRF Button}: Activate laser rangefinder
    \item \textbf{FN Button}: Context-sensitive function
    \item \textbf{Hat Switch}: Tracking box control / menu navigation
    \item \textbf{Zoom Rocker}: Camera zoom in/out
\end{itemize}

\textbf{GO/NO-GO Criteria}:
\begin{itemize}
    \item \checkedbox\ Stick returns to center when released
    \item \checkedbox\ Trigger has smooth pull with positive click
    \item \checkedbox\ Dead Man Switch springs back when released
    \item \checkedbox\ All buttons respond to press
    \item \checkbox\ Sticky or binding stick → NO-GO
    \item \checkbox\ Dead Man Switch does not return → NO-GO (CRITICAL SAFETY)
    \item \checkbox\ Trigger does not return → NO-GO (CRITICAL SAFETY)
\end{itemize}

\subsubsection{STATION 3: Turret Assembly - Exterior Inspection}

\begin{warningbox}
Ensure Station Enable switch is OFF before approaching turret.
\end{warningbox}

\textbf{3A. Electro-Optical System}

The Electro-Optical (EO) System is an integrated assembly containing:
\begin{itemize}
    \item Day Camera (visible spectrum)
    \item Thermal Camera (infrared)
    \item Laser Rangefinder (LRF)
\end{itemize}

\begin{lstlisting}
┌───────────────────────────────────┐
│    ELECTRO-OPTICAL SYSTEM         │
│  ┌─────────┐     ┌─────────┐     │
│  │  DAY    │     │ THERMAL │     │
│  │ CAMERA  │     │ CAMERA  │     │
│  │ (Sony)  │     │ (FLIR)  │     │
│  └─────────┘     └─────────┘     │
│         ┌─────────┐               │
│         │   LRF   │               │
│         │ (Laser) │               │
│         └─────────┘               │
└───────────────────────────────────┘
\end{lstlisting}

\textbf{Inspection Checklist}:

\textbf{Day Camera}:
\begin{itemize}
    \item Lens is clean and unscratched
    \item Lens cap removed (if installed)
    \item No cracks in protective housing
    \item Cable connections secure
\end{itemize}

\textbf{Thermal Camera}:
\begin{itemize}
    \item Lens is clean (use lens cloth only)
    \item No moisture or condensation visible
    \item Protective cover removed
    \item Camera not pointed at sun
\end{itemize}

\textbf{Laser Rangefinder}:
\begin{itemize}
    \item Aperture is clean
    \item No obstructions in front of lens
    \item Warning labels intact
\end{itemize}

\begin{warningbox}
NEVER look directly into LRF aperture
\end{warningbox}

\textbf{GO/NO-GO Criteria}:
\begin{itemize}
    \item \checkedbox\ All lenses clean and clear
    \item \checkedbox\ No visible damage to housings
    \item \checkedbox\ No loose cables or connections
    \item \checkbox\ Cracked lens → NO-GO (maintenance required)
    \item \checkbox\ Moisture inside camera → NO-GO (maintenance required)
    \item \checkbox\ Obstructed field of view → NO-GO (clear obstruction first)
\end{itemize}

\textbf{3B. Gimbal Mechanism}

\textbf{Visual Inspection}:
\begin{enumerate}
    \item Check for fluid leaks (hydraulic/oil)
    \item Verify no loose bolts or fasteners
    \item Confirm cables are properly routed (not pinched)
    \item Look for signs of impact damage
    \item Check that gimbal rotates freely by hand (power OFF only)
\end{enumerate}

\textbf{Gimbal Axes}:
\begin{itemize}
    \item \textbf{Azimuth Axis} (horizontal rotation): 360\degree\ continuous
    \item \textbf{Elevation Axis} (vertical tilt): -20\degree\ to +60\degree
\end{itemize}

\textbf{Mechanical Limits}:
\begin{itemize}
    \item Hard stops prevent over-rotation
    \item Limit sensors detect end of travel
    \item Software limits prevent sensor contact
\end{itemize}

\textbf{GO/NO-GO Criteria}:
\begin{itemize}
    \item \checkedbox\ No fluid leaks
    \item \checkedbox\ Gimbal moves smoothly by hand (power off)
    \item \checkedbox\ No grinding or binding noises
    \item \checkedbox\ All cables secured with proper strain relief
    \item \checkbox\ Fluid leaks → NO-GO (maintenance required)
    \item \checkbox\ Binding or resistance → NO-GO (maintenance required)
    \item \checkbox\ Loose mounting bolts → NO-GO (torque to spec)
\end{itemize}

\textbf{3C. Weapon Mount}

\begin{warningbox}
Treat all weapons as loaded. Follow weapon-specific clearing procedures (Appendix A).
\end{warningbox}

\textbf{Visual Inspection}:
\begin{enumerate}
    \item Weapon is properly secured in mount
    \item Feed system (belt/magazine) is intact
    \item No obstructions in barrel or ejection port
    \item Mounting bolts are tight
    \item Weapon safety is engaged (if applicable)
\end{enumerate}

\textbf{GO/NO-GO Criteria}:
\begin{itemize}
    \item \checkedbox\ Weapon securely mounted
    \item \checkedbox\ Feed mechanism functions properly
    \item \checkedbox\ Barrel clear of obstructions
    \item \checkbox\ Loose weapon → NO-GO (re-secure per manual)
    \item \checkbox\ Damaged feed system → NO-GO (repair/replace)
    \item \checkbox\ Barrel obstruction → NO-GO (clear and inspect)
\end{itemize}

\textbf{3D. Environmental Protection}

\textbf{Check}:
\begin{enumerate}
    \item Weatherproof covers are intact
    \item Drainage holes are not blocked
    \item Cable glands are sealed
    \item No corrosion on exposed metal
    \item Protective covers removed before operation
\end{enumerate}

\textbf{GO/NO-GO Criteria}:
\begin{itemize}
    \item \checkedbox\ All seals intact
    \item \checkedbox\ No water ingress visible
    \item \checkedbox\ Drainage holes clear
    \item \checkbox\ Water pooling inside → NO-GO (dry and check seals)
    \item \checkbox\ Severe corrosion → NO-GO (maintenance required)
\end{itemize}

\subsection{Walk-Around Completion}

After completing all inspection stations:

\begin{enumerate}
    \item \textbf{Document Results}: Mark checklist (Appendix C) with GO/NO-GO for each item
    \item \textbf{Report Discrepancies}: Any NO-GO items must be reported to maintenance immediately
    \item \textbf{Supervisor Review}: Have supervisor verify inspection before operation
    \item \textbf{Clear Area}: Ensure all personnel clear of turret before power-up
    \item \textbf{Proceed to Startup}: If all items are GO, proceed with Lesson 2 startup procedure
\end{enumerate}

\begin{notebox}[SAFETY]
Never operate RCWS with any NO-GO items. Equipment failure can result in injury or death.
\end{notebox}

\section{System Architecture (Simplified)}

Understanding how information flows through the system helps with troubleshooting.

\subsection{Data Flow Diagram}

\begin{lstlisting}
┌─────────────────────────────────────────────────────┐
│                   OPERATOR                          │
│              (Eyes on DCU Screen)                   │
│              (Hands on Joystick)                    │
└────────────┬────────────────────────┬───────────────┘
             │                        │
             ▼                        ▼
    ┌────────────────┐      ┌────────────────┐
    │   DCU DISPLAY  │      │   JOYSTICK     │
    │   • Video      │      │   • Slew Cmds  │
    │   • OSD Info   │      │   • Buttons    │
    │   • Menus      │      │   • Trigger    │
    └────────┬───────┘      └────────┬───────┘
             │                       │
             └───────────┬───────────┘
                         │
                         ▼
              ┌──────────────────────┐
              │   CONTROL COMPUTER   │
              │   • Processes inputs │
              │   • Updates displays │
              │   • Manages tracking │
              │   • Applies ballistics│
              └──────────┬───────────┘
                         │
         ┌───────────────┼───────────────┐
         │               │               │
         ▼               ▼               ▼
┌────────────────┐ ┌────────────┐ ┌────────────┐
│ ELECTRO-OPTICAL│ │   GIMBAL   │ │   WEAPON   │
│    SYSTEM      │ │   MOTORS   │ │  ACTUATOR  │
│ • Day Camera   │ │ • Azimuth  │ │ • Trigger  │
│ • Thermal Cam  │ │ • Elevation│ │ • Feed     │
│ • LRF          │ └────────────┘ └────────────┘
└────────────────┘
         │
         │ (Video Feedback)
         ▼
   [Back to DCU Display]
\end{lstlisting}

\textbf{Key Points}:
\begin{itemize}
    \item Operator sees video on DCU and controls gimbal with joystick
    \item Control computer processes all inputs and manages subsystems
    \item Cameras provide real-time video feedback
    \item System is a closed-loop: operator adjusts based on what they see
\end{itemize}

\begin{notebox}
You don't need to understand the electronics, just the concept:
\begin{itemize}
    \item \textbf{INPUT}: Your joystick commands
    \item \textbf{PROCESSING}: Computer calculates aim point with ballistics
    \item \textbf{OUTPUT}: Gimbal moves, weapon fires
    \item \textbf{FEEDBACK}: You see results on screen and adjust
\end{itemize}
\end{notebox}

\section{Safety Zone Concepts}

\subsection{No-Fire Zones}

\textbf{Definition}: Geographic areas where weapon discharge is absolutely prohibited.

\textbf{Purpose}:
\begin{itemize}
    \item Protect friendly forces
    \item Protect civilians and infrastructure
    \item Prevent fratricide
    \item Comply with rules of engagement (ROE)
\end{itemize}

\textbf{How It Works}:
\begin{itemize}
    \item Zones are pre-programmed by command
    \item System monitors gimbal position continuously
    \item \textbf{OSD displays "NO-FIRE ZONE" warning} when reticle enters zone
    \item Trigger is \textbf{software locked} when in no-fire zone
    \item Override requires commander authorization code
\end{itemize}

\textbf{Example No-Fire Zones}:
\begin{itemize}
    \item Friendly vehicle positions
    \item Civilian buildings (schools, hospitals, mosques)
    \item Infrastructure (power plants, water treatment)
    \item Friendly patrol routes
\end{itemize}

\textbf{Operator Responsibility}:
\begin{itemize}
    \item \textbf{Always check OSD for no-fire zone warning before engaging}
    \item Do not attempt to fire if warning is displayed
    \item Report zone boundary errors to command
    \item Never share override codes
\end{itemize}

\subsection{No-Traverse Zones}

\textbf{Definition}: Geographic areas where gimbal movement is restricted or prohibited.

\textbf{Purpose}:
\begin{itemize}
    \item Prevent gimbal from hitting vehicle structure
    \item Protect antennas, equipment, or personnel on vehicle
    \item Prevent pointing weapon at vehicle crew positions
    \item Avoid damaging cables or sensors
\end{itemize}

\textbf{How It Works}:
\begin{itemize}
    \item Zones are defined during system installation
    \item Gimbal slew is \textbf{automatically stopped} at zone boundary
    \item You will feel joystick resistance near boundary
    \item \textbf{OSD displays "NO-TRAVERSE" warning} when approaching zone
    \item System prevents entry even if you force joystick
\end{itemize}

\textbf{Example No-Traverse Zones}:
\begin{itemize}
    \item Rear 90\degree\ arc (to avoid vehicle cabin)
    \item Areas with antennas or equipment
    \item Personnel access hatches
    \item Cable routing areas
\end{itemize}

\textbf{Operator Responsibility}:
\begin{itemize}
    \item Learn your vehicle's no-traverse zones
    \item Do not fight the system if gimbal stops
    \item Report if zones are too restrictive for mission
    \item Never disable no-traverse zones without authorization
\end{itemize}

\subsection{Zone Violation Procedures}

\textbf{If you accidentally enter a zone}:

\begin{enumerate}
    \item \textbf{Release Trigger Immediately} (if weapon armed)
    \item \textbf{Slew gimbal out of zone} using joystick
    \item \textbf{Verify OSD warning clears}
    \item \textbf{Report incident} to supervisor
    \item \textbf{Do not re-enter zone} unless mission requires and authorized
\end{enumerate}

\textbf{If system prevents zone entry but mission requires it}:

\begin{enumerate}
    \item \textbf{Do NOT force the system}
    \item \textbf{Report to commander} immediately
    \item \textbf{Request zone boundary adjustment} if appropriate
    \item \textbf{Obtain override authorization} if permitted by ROE
    \item \textbf{Document all overrides} in mission log
\end{enumerate}

\begin{warningbox}[REMEMBER]
Zones exist for safety. Violating zones can kill friendlies.
\end{warningbox}

\section{Student Review Questions}

\begin{enumerate}
    \item What are the five fundamental weapon safety rules?
    \item Where is the emergency stop button located?
    \item What is the azimuth range of the El 7arress RCWS?
    \item What must be held continuously during weapon arming?
    \item Name three physical hazards associated with RCWS operation.
    \item What are the three major system components?
    \item What is the purpose of no-fire zones?
    \item What happens when the gimbal approaches a no-traverse zone?
    \item What is the GO/NO-GO criteria for the Dead Man Switch inspection?
    \item When should the Emergency Stop be used?
\end{enumerate}

% ============================================================================
% END OF LESSON 1
% ============================================================================
% ============================================================================
% LESSON 2 - BASIC OPERATION
% ============================================================================
\lesson{2: BASIC OPERATION}

\noindent
\begin{tabular}{@{}ll@{}}
\textbf{Duration:} & 4 hours \\
\textbf{Type:} & Classroom + Practical \\
\textbf{References:} & Operator manual, system startup checklist \\
\end{tabular}

\section{Introduction}

\textbf{Lesson Purpose}: This lesson teaches complete system startup, DCU and joystick operations, OSD interpretation, and proper shutdown procedures.

\textbf{Learning Objectives}:
\begin{itemize}
    \item Perform complete system startup procedure
    \item Operate all DCU buttons, switches, and controls
    \item Control gimbal movement using joystick
    \item Switch between day and thermal cameras
    \item Operate camera zoom controls
    \item Interpret all OSD elements correctly
    \item Perform normal system shutdown
\end{itemize}

\section{System Startup Procedure}

\subsection{Pre-Startup Checklist}

Before powering on the system, verify:

\begin{checklist}
    \item Walk-around inspection complete (Lesson 1) - all items GO
    \item Weapon cleared per Appendix A
    \item Ammunition removed or accounted for
    \item All personnel clear of turret (minimum 2 meters)
    \item Vehicle power available (check voltage: 20-30V DC nominal)
    \item Operator qualified and authorized
    \item Mission briefing received (zones, ROE, threats)
    \item Communication established with command
\end{checklist}

\begin{warningbox}
Do not start system if any checklist item is not complete.
\end{warningbox}

\subsection{Startup Sequence (10 Steps)}

Perform steps in order. Do not skip steps.

\begin{procedurebox}[STEP 1: INITIAL POWER-UP]
\textbf{Action}:
\begin{itemize}
    \item Ensure \button{Station Enable} switch is in OFF position
    \item Apply vehicle power to RCWS (circuit breaker ON or power cable connected)
\end{itemize}

\textbf{Expected Result}:
\begin{itemize}
    \item \textbf{Power} indicator light illuminates (Green)
    \item DCU screen displays boot logo
    \item System begins self-test (approximately 30 seconds)
\end{itemize}

\textbf{If NO Power Light}:
\begin{itemize}
    \item Check vehicle power supply (voltage 20-30V DC)
    \item Check circuit breaker
    \item Check cable connections
    \item Report to maintenance if power available but no light
\end{itemize}
\end{procedurebox}

\begin{procedurebox}[STEP 2: BOOT SELF-TEST]
\textbf{Action}:
\begin{itemize}
    \item Observe DCU screen during boot
    \item Wait for self-test to complete (DO NOT interrupt)
\end{itemize}

\textbf{Expected Display Sequence}:
\begin{lstlisting}
Checking Devices...
[ OK ] Day Camera
[ OK ] Thermal Camera
[ OK ] Laser Rangefinder
[ OK ] Azimuth Motor
[ OK ] Elevation Motor
[ OK ] Joystick Controller
[ -- ] Weapon Actuator (if not installed)

System Ready
Press STATION ENABLE to continue
\end{lstlisting}

\textbf{If Any Device Shows [FAIL]}:
\begin{itemize}
    \item DO NOT PROCEED with startup
    \item Note which device failed
    \item Report to maintenance immediately
    \item System may operate in degraded mode but requires supervisor approval
\end{itemize}
\end{procedurebox}

\begin{procedurebox}[STEP 3: ENABLE STATION]
\textbf{Action}:
\begin{itemize}
    \item Move \button{Station Enable} switch from OFF to ON
\end{itemize}

\textbf{Expected Result}:
\begin{itemize}
    \item \textbf{System Ready} light illuminates (Green)
    \item Gimbal motors energize (you may hear a soft hum)
    \item Video feed appears on DCU screen
    \item OSD overlay displays system information
    \item Gimbal automatically moves to Home Position (0\degree\ AZ, 0\degree\ EL)
\end{itemize}

\begin{cautionbox}
Gimbal will move during this step. Ensure area is clear.
\end{cautionbox}
\end{procedurebox}

\begin{procedurebox}[STEP 4: VERIFY HOME POSITION]
\textbf{Action}:
\begin{itemize}
    \item Observe OSD azimuth and elevation readings
    \item If not at home (AZ: 000\degree, EL: 00\degree), press \button{Home} button
\end{itemize}

\textbf{Expected Result}:
\begin{itemize}
    \item OSD displays: \osd{AZ: 000° EL: 00°} (±2\degree\ tolerance)
    \item Reticle is centered on screen
    \item Gimbal points directly forward relative to vehicle
\end{itemize}
\end{procedurebox}

\subsubsection{Detailed Homing Sequence (50ms Control Loop)}

The homing process follows a precise timing sequence:

\begin{lstlisting}
Cycle 0:   Operator presses HOME button
           → State = Requested
           → OSD displays: "homing init"

Cycle 1:   (50ms later)
           → GimbalController sends HOME command to PLC42
           → State = InProgress
           → OSD displays: "HOMING..."
           → 30-second timeout timer starts

Wait:      Oriental Motor servos execute homing sequence
           → Azimuth motor moves to home position
           → Elevation motor moves to home position
           → Motors send HOME-END signals when complete
           → Azimuth HOME-END (DI6), Elevation HOME-END (DI7)

Complete:  Both HOME-END signals received
           → State = Completed
           → OSD displays: "HOME COMPLETE" (5 seconds)
\end{lstlisting}

\textbf{Homing Timeout Handling}:
\begin{itemize}
    \item \textbf{30-second timeout} if HOME-END signals not received
    \item OSD displays: \osd{HOMING TIMEOUT - FAULT} (red flashing)
    \item \textbf{Recovery procedure}:
    \begin{enumerate}
        \item Press Emergency Stop
        \item Clear any gimbal obstruction
        \item Release Emergency Stop
        \item Retry homing procedure
    \end{enumerate}
\end{itemize}

\begin{warningbox}
If homing fails repeatedly, report to maintenance. Do not operate gimbal without successful homing.
\end{warningbox}

\begin{procedurebox}[STEP 5: SELECT CAMERA]
\textbf{Action}:
\begin{itemize}
    \item Press \button{CAM} button on joystick to select Day or Thermal camera
    \item Default camera is DAY (visible spectrum)
\end{itemize}

\textbf{Expected Result}:
\begin{itemize}
    \item OSD displays camera type: \osd{DAY} or \osd{THERMAL}
    \item Video image switches between color (day) and grayscale/colorized (thermal)
    \item FOV (Field of View) value updates on OSD
\end{itemize}
\end{procedurebox}

\begin{procedurebox}[STEP 6: TEST GIMBAL MOVEMENT]
\textbf{Action}:
\begin{itemize}
    \item Gently move joystick in all directions (left/right/up/down)
    \item Verify gimbal responds smoothly
    \item Return joystick to center (gimbal stops)
\end{itemize}

\textbf{Expected Result}:
\begin{itemize}
    \item Gimbal slews in direction of joystick movement
    \item OSD azimuth/elevation values update in real-time
    \item No grinding, binding, or unusual noises
    \item Gimbal stops when joystick returns to center
\end{itemize}
\end{procedurebox}

\begin{procedurebox}[STEP 7: TEST CAMERA ZOOM]
\textbf{Action}:
\begin{itemize}
    \item Press Zoom Rocker UP (zoom in) and DOWN (zoom out)
    \item Observe video image magnification change
    \item Observe FOV value on OSD
\end{itemize}

\textbf{Expected Result}:
\begin{itemize}
    \item Image magnifies when zooming in (FOV decreases)
    \item Image wide-angle when zooming out (FOV increases)
    \item Zoom is smooth with no jerking
    \item OSD FOV updates continuously
\end{itemize}
\end{procedurebox}

\begin{procedurebox}[STEP 8: TEST LASER RANGEFINDER (LRF)]
\textbf{Action}:
\begin{itemize}
    \item Aim reticle at a known object 100m+ away
    \item Press and hold \button{LRF} button on joystick
\end{itemize}

\textbf{Expected Result}:
\begin{itemize}
    \item Laser fires (you will NOT see visible beam - infrared)
    \item OSD displays range reading: \osd{RNG: XXXm}
    \item Range updates within 1 second
    \item LRF automatically times out after 5 seconds
\end{itemize}

\begin{warningbox}
Do not aim LRF at people, animals, or reflective surfaces at close range. Eye damage can occur.
\end{warningbox}
\end{procedurebox}

\begin{procedurebox}[STEP 9: ENABLE STABILIZATION]
\textbf{Action}:
\begin{itemize}
    \item Set \button{Stabilization} switch to ON
\end{itemize}

\textbf{Expected Result}:
\begin{itemize}
    \item OSD displays \osd{STAB: ON}
    \item Gimbal compensates for vehicle movement
    \item Reticle remains steady on target even if vehicle rocks
\end{itemize}
\end{procedurebox}

\begin{procedurebox}[STEP 10: SYSTEM READY - FINAL CHECK]
\textbf{Verify all indicator lights}:
\begin{itemize}
    \item \textbf{Power}: Green (ON)
    \item \textbf{System Ready}: Green (ON)
    \item \textbf{Gun Armed}: OFF (system is SAFE)
    \item \textbf{Fault/Alarm}: OFF (no errors)
\end{itemize}

\textbf{Verify OSD displays}:
\begin{itemize}
    \item Live video feed (day or thermal)
    \item Azimuth and elevation values
    \item Current FOV
    \item System mode (Manual)
    \item No warning messages
\end{itemize}

\textbf{If All Checks Pass}: System is ready for operation
\end{procedurebox}

\subsection{Startup Troubleshooting}

\small
\begin{longtable}{|L{0.25\textwidth}|L{0.30\textwidth}|L{0.38\textwidth}|}
\hline
\rowcolor{milblue!20}
\textbf{Problem} & \textbf{Possible Cause} & \textbf{Action} \\
\hline
No power light & No vehicle power & Check circuit breaker, voltage \\
\hline
Self-test fails & Device malfunction & Note failed device, report to maintenance \\
\hline
No video feed & Camera error & Check camera connections, restart system \\
\hline
Gimbal won't move & Motors disabled or fault & Check Station Enable, check for faults \\
\hline
Erratic gimbal & Joystick calibration & Recalibrate joystick (maintenance task) \\
\hline
LRF no reading & Out of range or bad target & Aim at closer/better reflective target \\
\hline
Thermal frozen & FFC in progress & Wait 5 seconds for FFC to complete \\
\hline
\end{longtable}

\section{Display and Control Unit (DCU) Operations}

\subsection{DCU Button and Switch Functions}

\subsubsection{Emergency Stop Button (RED)}

\textbf{Location}: Top left of DCU panel, large RED button

\textbf{Function}: Immediate system shutdown for safety emergencies

\textbf{Operation}:
\begin{enumerate}
    \item Press button (no confirmation required)
    \item System immediately:
    \begin{itemize}
        \item Stops all gimbal movement
        \item Safes weapon (trigger disabled)
        \item Locks servos in place
        \item Displays "EMERGENCY STOP ACTIVE" on OSD
    \end{itemize}
\end{enumerate}

\textbf{To Reset}:
\begin{enumerate}
    \item Twist/pull button to release
    \item Verify emergency condition is resolved
    \item Press \button{Station Enable} OFF then ON to restart
\end{enumerate}

\begin{warningbox}[CRITICAL]
Do NOT hesitate to use Emergency Stop. Better safe than sorry.
\end{warningbox}

\subsubsection{Station Enable Switch}

\textbf{Positions}: OFF / ON

\textbf{OFF Position}:
\begin{itemize}
    \item Gimbal motors disabled (turret cannot move)
    \item Video still displays (cameras remain powered)
    \item Weapon is safed
    \item Safe to approach turret for inspection
\end{itemize}

\textbf{ON Position}:
\begin{itemize}
    \item Gimbal motors enabled
    \item All subsystems operational
    \item Turret can move if joystick input received
    \item Stay clear of turret
\end{itemize}

\subsubsection{Gun Arm/Safe Switch}

\textbf{Positions}: SAFE / ARM

\textbf{SAFE Position} (Default):
\begin{itemize}
    \item Weapon trigger disabled
    \item Gun Armed light is OFF
    \item Trigger pull has no effect
    \item Safe for non-combat operations
\end{itemize}

\textbf{ARM Position}:
\begin{itemize}
    \item Weapon trigger enabled
    \item Gun Armed light illuminates RED
    \item Trigger pull will fire weapon (if other conditions met)
    \item Only use during combat or live fire training
\end{itemize}

\begin{warningbox}
When Gun Armed light is RED, treat weapon as HOT. One trigger pull away from firing.
\end{warningbox}

\subsubsection{Fire Mode Selector}

\textbf{Positions}: SINGLE / SHORT BURST / LONG BURST

\begin{twocol}
\textbf{SINGLE}:
\begin{itemize}
    \item One round per trigger pull
    \item Most accurate mode
    \item Use for precision engagement
\end{itemize}

\textbf{SHORT BURST}:
\begin{itemize}
    \item 3-5 rounds per trigger pull
    \item Good balance of accuracy and firepower
    \item Use for moving targets
\end{itemize}

\textbf{LONG BURST}:
\begin{itemize}
    \item Continuous fire while trigger held
    \item Less accurate due to recoil
    \item Use for area suppression
\end{itemize}
\end{twocol}

\subsubsection{Speed Select Switch}

\textbf{Positions}: LOW / MEDIUM / HIGH

\small
\begin{longtable}{|L{0.20\textwidth}|L{0.35\textwidth}|L{0.38\textwidth}|}
\hline
\rowcolor{milblue!20}
\textbf{Speed} & \textbf{Use Case} & \textbf{Max Slew Rate} \\
\hline
LOW & Zeroing, fine adjustments, precision & ~5\degree/second \\
\hline
MEDIUM & Normal surveillance, target acquisition & ~20\degree/second \\
\hline
HIGH & Close-range threats, emergencies & ~60\degree/second \\
\hline
\end{longtable}

\subsection{DCU Indicator Lights}

\begin{twocol}
\begin{itemize}
    \item \textbf{Power} (Green): Vehicle power supplied
    \item \textbf{System Ready} (Green): All subsystems operational
    \item \textbf{Gun Armed} (Red): Weapon is armed
    \item \textbf{Ammo Loaded} (Yellow): Ammunition detected
    \item \textbf{Authorized} (Green): Operator authorized
    \item \textbf{Fault/Alarm} (Red): System error detected
\end{itemize}
\end{twocol}

\section{Joystick Controller Operations}

\subsection{Joystick Control Techniques}

\subsubsection{Proper Grip}

\textbf{Right Hand Position}:
\begin{enumerate}
    \item Wrap fingers around joystick grip
    \item Index finger rests on trigger (outside trigger guard when not firing)
    \item Thumb on top, near CAM and TRK buttons
    \item Dead Man Switch on rear of grip - squeeze with palm/fingers to engage
\end{enumerate}

\subsubsection{Gimbal Slew Technique}

\textbf{Small Movements} (Precision):
\begin{itemize}
    \item Deflect stick slightly from center (10-20\%)
    \item Gimbal moves slowly
    \item Good for: Tracking, zeroing, fine adjustments
\end{itemize}

\textbf{Large Movements} (Rapid Slew):
\begin{itemize}
    \item Deflect stick fully (80-100\%)
    \item Gimbal moves at maximum speed
    \item Good for: Searching, responding to threats, sector scans
\end{itemize}

\subsection{Joystick Button Functions}

The joystick provides 19 programmable buttons (Button 0-18) plus analog axes and a hat switch.

\subsubsection{Complete Joystick Button Reference (19 Buttons)}

\small
\begin{longtable}{|C{0.08\textwidth}|L{0.30\textwidth}|L{0.52\textwidth}|}
\hline
\rowcolor{milblue!20}
\textbf{Button} & \textbf{Function} & \textbf{Operation} \\
\hline
\textbf{0} & Engagement Mode / Master Arm & Press = Enter engagement, Release = Exit engagement \\
\hline
\textbf{1} & LRF (Laser Range Finder) & Single press = Single measurement, \textbf{Double-click = Toggle continuous LRF (5Hz)} \\
\hline
\textbf{2} & Lead Angle Compensation (LAC) & \textbf{Toggle LAC ON/OFF (joystick ONLY - no menu access!)} \\
\hline
\textbf{3} & Dead Man Switch (Palm Switch) & \textbf{MUST HOLD} for weapon operation and LAC toggle \\
\hline
\textbf{4} & Tracking Control & Single = Cycle tracking phase, \textbf{Double-click (<1 sec) = ABORT tracking} \\
\hline
\textbf{5} & Fire Weapon & Trigger - fires weapon when armed \\
\hline
\textbf{6} & Zoom + & Increase camera magnification (hold for continuous) \\
\hline
\textbf{7} & Thermal LUT + & Next thermal video look-up table (thermal camera only) \\
\hline
\textbf{8} & Zoom - & Decrease camera magnification (hold for continuous) \\
\hline
\textbf{9} & Thermal LUT - & Previous thermal look-up table (thermal camera only) \\
\hline
\textbf{10} & LRF Clear & Clear range reading, stop continuous LRF mode \\
\hline
\textbf{11} & Mode Cycle & Cycle: Manual → Sector Scan → TRP → (Radar) → Manual \\
\hline
\textbf{12} & Reserved & Available for future use \\
\hline
\textbf{13} & Mode Cycle & Duplicate of Button 11 (ergonomic placement) \\
\hline
\textbf{14} & Select Next Zone/TRP & Next TRP page or sector scan zone (during surveillance) \\
\hline
\textbf{15} & Reserved & Future use \\
\hline
\textbf{16} & Select Previous Zone/TRP & Previous TRP page or sector scan zone (during surveillance) \\
\hline
\textbf{17} & Reserved & Future use \\
\hline
\textbf{18} & Reserved & Future use \\
\hline
\end{longtable}

\subsubsection{Analog Controls}

\small
\begin{longtable}{|L{0.28\textwidth}|L{0.65\textwidth}|}
\hline
\rowcolor{milblue!20}
\textbf{Control} & \textbf{Function} \\
\hline
\textbf{Main Stick X-Axis} & Gimbal azimuth slew (left/right) \\
\hline
\textbf{Main Stick Y-Axis} & Gimbal elevation slew (up/down) \\
\hline
\textbf{Hat Switch (D-Pad)} & Tracking gate resize during Acquisition phase / Menu navigation \\
\hline
\end{longtable}

\begin{warningbox}[CRITICAL]
\textbf{Button 2 (LAC)} is the \textbf{ONLY} method to activate Lead Angle Compensation. There is NO menu access for LAC. Dead Man Switch (Button 3) must be held when toggling LAC.
\end{warningbox}

\subsubsection{Camera Switch (CAM Button)}

\textbf{When to Use Day Camera}:
\begin{itemize}
    \item Good lighting conditions
    \item Need color information
    \item Need maximum zoom range (2\degree\ to 60\degree\ FOV)
\end{itemize}

\textbf{When to Use Thermal Camera}:
\begin{itemize}
    \item Darkness, dawn, dusk
    \item Smoke, fog, dust
    \item Detecting hidden personnel (heat signatures)
    \item Identifying recently fired weapons (barrel heat)
\end{itemize}

\subsubsection{Laser Rangefinder (Button 1)}

\textbf{Single Measurement Mode} (default):
\begin{enumerate}
    \item Aim reticle at target
    \item Press Button 1 (single press)
    \item Laser fires (invisible infrared beam)
    \item OSD displays range: \osd{RNG: XXXm}
    \item Range used for ballistic calculations
\end{enumerate}

\textbf{Continuous LRF Mode} (5Hz automatic ranging):
\begin{enumerate}
    \item \textbf{Double-click Button 1} (within 1 second)
    \item System announces: "CONTINUOUS LRF ENABLED"
    \item LRF fires automatically at 5Hz (5 times per second)
    \item Range continuously updates on OSD
    \item \textbf{Double-click Button 1 again} to disable
    \item Or press \textbf{Button 10} (LRF Clear) to stop
\end{enumerate}

\begin{notebox}
Continuous LRF mode is useful for tracking moving targets where range is constantly changing. Disable when not needed to preserve laser life.
\end{notebox}

\textbf{Specifications}:\\
\textbf{Range}: 50m to 4000m\\
\textbf{Accuracy}: ±5 meters\\
\textbf{Continuous Rate}: 5Hz (when enabled)

\begin{warningbox}
Class 3B laser. Do not aim at people or reflective surfaces at close range.
\end{warningbox}

\subsubsection{Dead Man Switch}

\textbf{Location}: Rear of joystick grip

\textbf{Purpose}:
\begin{itemize}
    \item Prevents accidental discharge if operator is incapacitated
    \item Automatic safety if operator loses grip
    \item Required safety interlock for firing
\end{itemize}

\begin{warningbox}[CRITICAL SAFETY RULE]
Release Dead Man Switch immediately when not actively engaging a target.
\end{warningbox}

\section{On-Screen Display (OSD) Interpretation}

\subsection{Complete OSD Layout}

\begin{lstlisting}
┌──────────────────────────────────────────────────────┐
│ AZ: 045°  EL: +12°  | DAY  FOV: 9.0° ZOOM: 15× │ ← Top
│                                                       │
│                    [Target Box]                      │
│                        ┌──┐                          │
│                        └──┘                          │
│                          + ← Reticle                 │
│                         (*) ← CCIP Pipper            │
│                                                       │
│ RNG: 850m         MODE: Manual       STATUS: ARMED   │ ← Bottom
│ TEMP: 35°C        TRACK: Off         AMMO: 450      │
└──────────────────────────────────────────────────────┘
\end{lstlisting}

\subsection{Top Bar Elements}

\begin{itemize}
    \item \textbf{AZ: XXX\degree}: Current azimuth position (000\degree\ to 359\degree)
    \item \textbf{EL: ±XX\degree}: Current elevation position (-20\degree\ to +60\degree)
    \item \textbf{DAY / THERMAL}: Active camera type
    \item \textbf{FOV: X.X\degree}: Current field of view
    \item \textbf{ZOOM: XX×}: Zoom magnification factor
\end{itemize}

\subsection{Center Area Elements}

\textbf{Reticle (+)}:
\begin{itemize}
    \item Main aiming point
    \item Where the gun is currently aimed (with zeroing applied)
    \item Fire weapon with reticle on target
\end{itemize}

\textbf{CCIP Pipper ((*))}:
\begin{itemize}
    \item Continuously Computed Impact Point
    \item Shows where bullet will actually hit with all ballistic corrections
    \item May be offset from reticle if ballistics are applied
    \item Always aim with CCIP for accurate hits
\end{itemize}

\subsection{Bottom Bar Elements}

\begin{twocol}
\textbf{Left Side}:
\begin{itemize}
    \item \textbf{RNG: XXXm}: Distance to target
    \item \textbf{TEMP: XX\degree C}: System temperature
\end{itemize}

\textbf{Center}:
\begin{itemize}
    \item \textbf{MODE}: Current motion mode
    \item \textbf{TRACK}: Tracking status
\end{itemize}

\textbf{Right Side}:
\begin{itemize}
    \item \textbf{STATUS}: Weapon system status
    \item \textbf{AMMO}: Remaining rounds
    \item \textbf{STAB}: Stabilization status
\end{itemize}
\end{twocol}

\subsection{Warning Messages}

\textbf{Critical warnings appear in center screen}:

\begin{itemize}
    \item \textbf{NO-FIRE ZONE WARNING}: Red, flashing - Do NOT fire
    \item \textbf{NO-TRAVERSE WARNING}: Yellow - Movement restricted
    \item \textbf{EMERGENCY STOP ACTIVE}: Red, solid - System safed
    \item \textbf{SYSTEM FAULT}: Red/yellow - Check status
    \item \textbf{TARGET LOST}: Yellow - Tracking lost, coast mode
\end{itemize}

\section{System Shutdown Procedure}

\subsection{Shutdown Sequence (7 Steps)}

\begin{procedurebox}[STEP 1: SAFE THE WEAPON]
\begin{itemize}
    \item Move \button{Gun Arm/Safe} switch to SAFE
    \item Verify Gun Armed light is OFF
    \item Release Dead Man Switch on joystick
\end{itemize}
\end{procedurebox}

\begin{procedurebox}[STEP 2: RETURN TO HOME POSITION]
\begin{itemize}
    \item Press \button{Home} button on DCU
    \item Wait for gimbal to slew to 0\degree\ AZ, 0\degree\ EL
\end{itemize}
\end{procedurebox}

\begin{procedurebox}[STEP 3: DISABLE STABILIZATION]
\begin{itemize}
    \item Move \button{Stabilization} switch to OFF
\end{itemize}
\end{procedurebox}

\begin{procedurebox}[STEP 4: ACCESS SHUTDOWN MENU]
\begin{itemize}
    \item Press \button{MENU ✓} button
    \item Navigate to SYSTEM → Shutdown
    \item Select "Shutdown System"
    \item Confirm shutdown
    \item System performs orderly shutdown
    \item Wait for "SHUTDOWN COMPLETE" message
\end{itemize}
\end{procedurebox}

\begin{procedurebox}[STEP 5: DISABLE STATION]
\begin{itemize}
    \item Move \button{Station Enable} switch to OFF
    \item System Ready light turns OFF
    \item Gimbal motors de-energize
\end{itemize}
\end{procedurebox}

\begin{procedurebox}[STEP 6: REMOVE VEHICLE POWER]
\textbf{If end of mission/shift}:
\begin{itemize}
    \item Turn off circuit breaker, OR
    \item Disconnect power cable (if external)
    \item Power light turns OFF
    \item DCU screen goes black
\end{itemize}
\end{procedurebox}

\begin{procedurebox}[STEP 7: SECURE WEAPON AND EQUIPMENT]
\begin{itemize}
    \item Clear weapon per Appendix A (if required)
    \item Remove ammunition (if required by SOP)
    \item Install protective covers on cameras
    \item Lock operator station (if applicable)
    \item Complete operator log entry
\end{itemize}
\end{procedurebox}

\subsection{Post-Shutdown Checks}

\begin{checklist}
    \item Gun Armed light is OFF
    \item Gimbal is at home position (0\degree\ AZ, 0\degree\ EL)
    \item Station Enable is OFF
    \item Weapon is cleared (if required)
    \item Covers installed (if required)
    \item Operator log entry complete
\end{checklist}

\section{Student Review Questions}

\begin{enumerate}
    \item What is the first step in the system startup procedure?
    \item What should you do if a device shows [FAIL] during self-test?
    \item Where is the Emergency Stop button located?
    \item What are the three fire mode settings?
    \item What is the difference between the reticle and CCIP pipper?
    \item When should you use the thermal camera instead of the day camera?
    \item What is the purpose of the Dead Man Switch?
    \item What does \osd{STAB: ON} indicate on the OSD?
    \item What is the proper sequence for system shutdown?
    \item What must be done before removing vehicle power?
\end{enumerate}
% ============================================================================
% LESSON 3 - MENU SYSTEM & SETTINGS
% ============================================================================
\lesson{3: MENU SYSTEM \& SETTINGS}

\noindent
\begin{tabular}{@{}ll@{}}
\textbf{Duration:} & 2 hours \\
\textbf{Type:} & Practical \\
\textbf{References:} & Menu navigation guide, configuration manual \\
\end{tabular}

\section{Introduction}

\textbf{Lesson Purpose}: This lesson teaches complete menu navigation, display configuration, and system settings management.

\textbf{Learning Objectives}:
\begin{itemize}
    \item Navigate all menu structures using DCU controls
    \item Configure reticle types and display colors
    \item Access system status information
    \item Modify operational settings safely
\end{itemize}

\section{Menu Navigation Basics}

\subsection{Accessing the Main Menu}

\textbf{Method}: Press \button{MENU ✓} button on DCU

\textbf{Result}:
\begin{itemize}
    \item Video feed dims (still visible in background)
    \item Menu window appears in center of screen
    \item Current selection is highlighted
    \item Menu title displayed at top
\end{itemize}

\subsection{Menu Controls}

\small
\begin{longtable}{|L{0.25\textwidth}|L{0.68\textwidth}|}
\hline
\rowcolor{milblue!20}
\textbf{Button} & \textbf{Function} \\
\hline
\button{MENU ✓} & Open menu / Confirm selection / Exit menu \\
\hline
\button{MENU ▲} & Move selection up / Increase value \\
\hline
\button{MENU ▼} & Move selection down / Decrease value \\
\hline
\end{longtable}

\textbf{Alternative}: Joystick HAT switch can also be used for menu navigation

\subsection{Navigation Workflow}

\textbf{5-Step Process}:

\begin{enumerate}
    \item \textbf{Open Menu}: Press \button{MENU ✓} - Main menu appears
    \item \textbf{Navigate}: Press ▲ or ▼ to highlight desired option
    \item \textbf{Select}: Press \button{MENU ✓} to enter submenu or activate option
    \item \textbf{Return}: Navigate to "Return..." option, press \button{MENU ✓}
    \item \textbf{Exit}: Continue pressing "Return..." until back at live video
\end{enumerate}

\begin{notebox}[TIP]
If you get lost in menus, keep selecting "Return..." until you exit completely.
\end{notebox}

\section{Main Menu Structure}

\subsection{Complete Menu Tree}

\begin{lstlisting}
MAIN MENU
├── --- RETICLE & DISPLAY ---
│   ├── Personalize Reticle
│   └── Personalize Colors
│
├── --- BALLISTICS ---
│   ├── Zeroing (Lesson 6)
│   ├── Environmental Parameters (Lesson 6)
│   └── Lead Angle Compensation (Lesson 6)
│
├── --- SYSTEM ---
│   ├── Zone Definitions (Lesson 4)
│   ├── System Status (Lesson 7)
│   └── Shutdown System
│
├── --- INFO ---
│   └── About
│
└── Return ...
\end{lstlisting}

\begin{notebox}
Ballistics and Zone Management are covered in detail in later lessons. This lesson focuses on display settings and basic navigation.
\end{notebox}

\section{Reticle \& Display Settings}

\subsection{Personalize Reticle}

\textbf{Access}: Main Menu → "Personalize Reticle"

\textbf{Purpose}: Select reticle type (crosshair style) for aiming

\subsubsection{Available Reticle Types}

\textbf{1. Box Crosshair} (Default)
\begin{lstlisting}
        │
    ┌───┼───┐
────┤   +   ├────
    └───┼───┘
        │
\end{lstlisting}
\begin{itemize}
    \item Cross with surrounding box
    \item Good for general purpose and tracking
    \item Recommended for most operations
\end{itemize}

\textbf{2. Brackets Reticle}
\begin{lstlisting}
    ┌─     ─┐
        +
    └─     ─┘
\end{lstlisting}
\begin{itemize}
    \item Corner brackets with center crosshair
    \item Enhanced visibility
    \item Good for low-contrast targets
\end{itemize}

\textbf{3. Duplex Crosshair}
\begin{lstlisting}
  ████│████
──────+──────
  ████│████
\end{lstlisting}
\begin{itemize}
    \item Thick outer lines, thin center
    \item Sniper/precision style
    \item Good for long-range
\end{itemize}

\textbf{4. Fine Crosshair}
\begin{lstlisting}
      │
    ──+──
      │
\end{lstlisting}
\begin{itemize}
    \item Thin precision crosshair
    \item Minimal obstruction
    \item Best for extreme precision
\end{itemize}

\textbf{5. Chevron Reticle}
\begin{lstlisting}
      ˅
    ──+──
\end{lstlisting}
\begin{itemize}
    \item Downward pointing chevron
    \item CQB (Close Quarters Battle) style
    \item Good for rapid engagement
\end{itemize}

\subsubsection{How to Change Reticle}

\begin{procedurebox}[CHANGING RETICLE]
\begin{enumerate}
    \item Press \button{MENU ✓}
    \item Navigate to "Personalize Reticle" (▼)
    \item Press \button{MENU ✓} to enter
    \item Use ▲/▼ to highlight desired reticle
    \item Press \button{MENU ✓} to select
    \item Reticle changes immediately on screen
    \item Navigate to "Return..." (▼)
    \item Press \button{MENU ✓} to return to Main Menu
\end{enumerate}
\end{procedurebox}

\textbf{Recommendation}: Use \textbf{Box Crosshair} for combat, \textbf{Fine Crosshair} for long-range surveillance.

\subsection{Personalize Colors}

\textbf{Access}: Main Menu → "Personalize Colors"

\textbf{Purpose}: Change OSD color scheme for visibility

\subsubsection{Available Color Themes}

\small
\begin{longtable}{|L{0.18\textwidth}|L{0.25\textwidth}|L{0.50\textwidth}|}
\hline
\rowcolor{milblue!20}
\textbf{Theme} & \textbf{Primary Color} & \textbf{Use Case} \\
\hline
Green & Bright green & Default - good for day and night \\
\hline
Red & Red & Night vision compatible \\
\hline
Yellow & Yellow & High contrast, bright conditions \\
\hline
Cyan & Cyan & Alternative for user preference \\
\hline
White & White & Maximum contrast \\
\hline
\end{longtable}

\textbf{What Changes}:
\begin{itemize}
    \item Reticle color
    \item OSD text color
    \item Menu text color
    \item Tracking box color (when not status-coded)
\end{itemize}

\textbf{What Does NOT Change}:
\begin{itemize}
    \item Warning messages (always red or yellow)
    \item Tracking status colors (yellow/red/green based on state)
    \item CCIP pipper color
\end{itemize}

\subsubsection{How to Change Color}

\begin{procedurebox}[CHANGING COLOR SCHEME]
\begin{enumerate}
    \item Press \button{MENU ✓}
    \item Navigate to "Personalize Colors" (▼)
    \item Press \button{MENU ✓} to enter
    \item Use ▲/▼ to highlight desired color
    \item Press \button{MENU ✓} to select
    \item OSD changes immediately
    \item Evaluate visibility
    \item Change again if needed, or select "Return..."
\end{enumerate}
\end{procedurebox}

\textbf{Best Practices}:
\begin{itemize}
    \item \textbf{Green}: Good all-around choice (default)
    \item \textbf{Red}: Use with night vision equipment
    \item \textbf{Yellow}: Use in bright sunlight or snow
    \item \textbf{White}: Maximum contrast on dark backgrounds
\end{itemize}

\section{System Menu Options}

\subsection{Zone Definitions}

\textbf{Access}: Main Menu → "Zone Definitions"

\textbf{Purpose}: Manage no-fire zones, no-traverse zones, sector scans, and TRPs

\textbf{Detailed Coverage}: See Lesson 4 (Motion Modes \& Surveillance)

\textbf{Quick Access Functions}:
\begin{itemize}
    \item View active zones
    \item Enable/disable zones
    \item Navigate to zone editor (supervisor/commander function)
\end{itemize}

\begin{notebox}[OPERATOR NOTE]
Zone modification usually requires supervisor authorization. Operators can VIEW zones but typically cannot CHANGE them.
\end{notebox}

\subsection{System Status}

\textbf{Access}: Main Menu → "System Status"

\textbf{Purpose}: View detailed system health and diagnostics

\textbf{Detailed Coverage}: See Lesson 7 (System Status \& Monitoring)

\textbf{Quick Preview}:

Displays status of all subsystems:
\begin{twocol}
\begin{itemize}
    \item Cameras (Day/Night)
    \item Servos (Azimuth/Elevation)
    \item Laser Rangefinder
    \item Joystick
    \item Stabilization System
    \item Tracking System
    \item Weapon Actuator
\end{itemize}
\end{twocol}

\textbf{When to Check}:
\begin{itemize}
    \item At startup (verify all green)
    \item When Fault light illuminates
    \item Before critical operations
    \item During troubleshooting
\end{itemize}

\subsection{Shutdown System}

\textbf{Access}: Main Menu → "Shutdown System"

\textbf{Purpose}: Perform orderly software shutdown before powering off

\textbf{Why Use Menu Shutdown}:
\begin{itemize}
    \item Saves configuration settings
    \item Saves zone data
    \item Closes log files properly
    \item Prevents data corruption
    \item Powers down subsystems in correct sequence
\end{itemize}

\subsubsection{Shutdown Procedure}

\begin{procedurebox}[MENU SHUTDOWN]
\begin{enumerate}
    \item Press \button{MENU ✓}
    \item Navigate to "Shutdown System" (▼ multiple times)
    \item Press \button{MENU ✓}
    \item Confirmation prompt appears: "SHUTDOWN SYSTEM?"
    \item Select "YES, Shutdown"
    \item Press \button{MENU ✓} to confirm
    \item System shuts down:
    \begin{itemize}
        \item "SHUTTING DOWN..." message appears
        \item Progress indicator shows shutdown steps
        \item Cameras power off
        \item Motors de-energize
        \item "SHUTDOWN COMPLETE - Safe to power off" message
    \end{itemize}
    \item Disable \button{Station Enable} switch on DCU
    \item Cut vehicle power (if end of shift)
\end{enumerate}
\end{procedurebox}

\begin{warningbox}[IMPORTANT]
Wait for "SHUTDOWN COMPLETE" message before cutting power. Interrupting shutdown can corrupt configuration files.
\end{warningbox}

\subsection{About / Info}

\textbf{Access}: Main Menu → "About"

\textbf{Purpose}: Display system information for troubleshooting and support

\textbf{Information Displayed}:
\begin{itemize}
    \item System name: "El 7arress RCWS"
    \item Software version (e.g., "v4.5.2")
    \item Build date
    \item Serial number (if configured)
    \item Uptime (hours since power-on)
    \item Operator name (if logged in)
\end{itemize}

\textbf{Use Case}: Provide this information when reporting issues to maintenance.

\section{Ballistics Menu (Overview)}

\textbf{Access}: Main Menu → "--- BALLISTICS ---" section

The ballistics menu provides access to fire control settings. These are covered in detail in \textbf{Lesson 6} but are introduced here for awareness.

\subsection{Ballistics Submenu Options}

\subsubsection{1. Zeroing}
\begin{itemize}
    \item Align weapon point of impact with camera crosshair
    \item Adjust azimuth and elevation offsets
    \item Save/load zeroing profiles
    \item \textbf{Detailed in Lesson 6.1}
\end{itemize}

\subsubsection{2. Environmental Parameters}
\begin{itemize}
    \item Set temperature (\degree C)
    \item Set altitude (meters above sea level)
    \item Set crosswind speed and direction
    \item Apply environmental corrections to ballistics
    \item \textbf{Detailed in Lesson 6.2}
\end{itemize}

\subsubsection{3. Lead Angle Compensation (View Only)}
\begin{itemize}
    \item \textbf{View} lead angle status (Off/On/Lag/ZoomOut)
    \item Displays current LAC system state
    \item \textbf{NOTE: LAC cannot be toggled from menu!}
    \item \textbf{LAC is activated ONLY via joystick Button 2}
    \item See Lesson 6.3 for LAC activation procedure
\end{itemize}

\begin{warningbox}[IMPORTANT]
Lead Angle Compensation (LAC) can \textbf{ONLY} be toggled using joystick \textbf{Button 2} while holding the Dead Man Switch (Button 3). There is NO menu option to enable/disable LAC.
\end{warningbox}

\begin{cautionbox}[OPERATOR NOTE]
Do not modify ballistics settings unless trained. Incorrect settings can cause missed shots or dangerous ricochets. Zeroing and environmental settings are usually performed by designated personnel.
\end{cautionbox}

\section{Menu Quick Reference}

\subsection{Common Menu Tasks}

\subsubsection{Task 1: Change Reticle}

\begin{quickref}
\texttt{MENU ✓ → "Personalize Reticle" → MENU ✓}\\
\texttt{→ Select reticle (▲/▼) → MENU ✓}\\
\texttt{→ "Return..." → MENU ✓}\\
\textbf{Time}: ~10 seconds
\end{quickref}

\subsubsection{Task 2: Change Color Scheme}

\begin{quickref}
\texttt{MENU ✓ → "Personalize Colors" → MENU ✓}\\
\texttt{→ Select color (▲/▼) → MENU ✓}\\
\texttt{→ "Return..." → MENU ✓}\\
\textbf{Time}: ~10 seconds
\end{quickref}

\subsubsection{Task 3: Check System Status}

\begin{quickref}
\texttt{MENU ✓ → "System Status" → MENU ✓}\\
\texttt{→ Review status → "Return..." → MENU ✓}\\
\textbf{Time}: ~15 seconds (plus review time)
\end{quickref}

\subsubsection{Task 4: Shutdown via Menu}

\begin{quickref}
\texttt{MENU ✓ → "Shutdown System" → MENU ✓}\\
\texttt{→ "YES, Shutdown" → MENU ✓}\\
\texttt{→ Wait for "SHUTDOWN COMPLETE"}\\
\texttt{→ Disable Station Enable → Cut power}\\
\textbf{Time}: ~45 seconds
\end{quickref}

\subsection{Menu Navigation Tips}

\begin{enumerate}
    \item \textbf{Muscle Memory}: Practice menu navigation until you can do it without looking at button labels
    \item \textbf{HAT Switch Alternative}: Use joystick HAT switch if your hands are already on the joystick
    \item \textbf{Quick Exit}: If lost in menus, repeatedly press \button{MENU ✓} on section headers to back out quickly
    \item \textbf{Video Still Visible}: Menu is semi-transparent - you can still monitor situation while in menu
    \item \textbf{Menu Timeout}: Some menus auto-exit after 60 seconds of inactivity
    \item \textbf{Combat Discipline}: Minimize menu time during operations
\end{enumerate}

\subsection{Menu Troubleshooting}

\small
\begin{longtable}{|L{0.25\textwidth}|L{0.30\textwidth}|L{0.38\textwidth}|}
\hline
\rowcolor{milblue!20}
\textbf{Problem} & \textbf{Possible Cause} & \textbf{Solution} \\
\hline
Menu won't open & Button stuck or fault & Try joystick HAT switch, restart system \\
\hline
Can't select option & On section header & Use ▲/▼ to move to selectable item \\
\hline
Menu frozen & Software hang & Press Emergency Stop, restart system \\
\hline
Settings don't save & Shutdown without menu & Always use "Shutdown System" menu \\
\hline
Menu text unreadable & Color scheme issue & Change to White or Yellow theme \\
\hline
\end{longtable}

\section{Menu Best Practices}

\subsection{When to Use Menus}

\textbf{DO use menus for}:
\begin{itemize}
    \item Changing display preferences (reticle, color)
    \item Checking system status
    \item Reviewing zone definitions
    \item Configuring ballistics (when trained)
    \item Orderly system shutdown
\end{itemize}

\textbf{DO NOT use menus during}:
\begin{itemize}
    \item Active engagement
    \item Emergency situations
    \item When gimbal must be controlled continuously
    \item Under time pressure
\end{itemize}

\begin{notebox}[RULE OF THUMB]
Menus are for setup and configuration, not combat operations.
\end{notebox}

\subsection{Settings That Persist}

\textbf{Saved Between Power Cycles}:
\begin{itemize}
    \item Reticle type selection
    \item Color scheme
    \item Zeroing offsets (if saved)
    \item Environmental parameters (if saved)
    \item Zone definitions
\end{itemize}

\textbf{NOT Saved} (reset on power-up):
\begin{itemize}
    \item Gimbal position (returns to home)
    \item Active tracking (aborted)
    \item Temporary warnings
    \item Menu navigation position
\end{itemize}

\subsection{Operator vs. Supervisor Functions}

\textbf{Operator Can}:
\begin{itemize}
    \item Change reticle and colors
    \item View system status
    \item View zones
    \item Access ballistics menus (view)
    \item Shutdown system
\end{itemize}

\textbf{Operator Usually CANNOT} (requires authorization):
\begin{itemize}
    \item Modify zone boundaries
    \item Override no-fire zones
    \item Change ballistics profiles
    \item Access maintenance menus
    \item Modify system configuration files
\end{itemize}

\textbf{Consult your unit SOP for specific authorization levels.}

\section{Student Review Questions}

\begin{enumerate}
    \item What button is used to access the main menu?
    \item Name three available reticle types.
    \item Which color scheme is recommended for night vision operations?
    \item What menu option would you use to view system diagnostics?
    \item Why is it important to use the menu shutdown before cutting power?
    \item Can operators modify no-fire zone boundaries?
    \item What settings are saved between power cycles?
    \item What is the recommended reticle for general combat operations?
    \item How do you exit from a submenu?
    \item What should you do if the menu becomes frozen?
\end{enumerate}
% ============================================================================
% LESSON 4 - MOTION MODES & SURVEILLANCE
% ============================================================================
\lesson{4: MOTION MODES \& SURVEILLANCE}

\noindent
\begin{tabular}{@{}ll@{}}
\textbf{Duration:} & 3 hours \\
\textbf{Type:} & Classroom + Practical \\
\textbf{References:} & Zone management guide, surveillance procedures \\
\end{tabular}

\section{Introduction}

\textbf{Lesson Purpose}: This lesson teaches motion mode selection, automated surveillance patterns, and zone management for safe and effective operations.

\textbf{Learning Objectives}:
\begin{itemize}
    \item Explain the purpose of each motion mode
    \item Switch between motion modes safely
    \item Operate automatic sector scan mode
    \item Utilize Target Reference Point (TRP) scan mode
    \item Define and manage no-fire zones
    \item Define and manage no-traverse zones
    \item Save and load zone configurations
\end{itemize}

\section{Motion Modes Overview}

\subsection{What Are Motion Modes?}

Motion modes control \textbf{how the gimbal moves} during operations:

\begin{itemize}
    \item \textbf{Manual Mode}: You control gimbal directly with joystick
    \item \textbf{Auto Sector Scan}: System automatically scans between two points
    \item \textbf{TRP Scan}: System sequentially visits pre-defined Target Reference Points
    \item \textbf{Radar Slew} (if equipped): Gimbal follows radar detections
\end{itemize}

\textbf{Purpose}: Different missions require different surveillance patterns. Motion modes let you switch between direct control and automated surveillance.

\subsection{Mode Selection}

\textbf{How to Change Modes}:
\begin{itemize}
    \item Press joystick \button{Button 11} or \button{Button 13} (either button cycles modes)
    \item Modes cycle in sequence:
\end{itemize}

\begin{lstlisting}
Manual → AutoSectorScan → TRP Scan → Radar Slew → Manual
\end{lstlisting}

\begin{notebox}
\textbf{Radar Slew} is only available if radar hardware is installed. On systems without radar, the mode cycle is: \texttt{Manual → Sector Scan → TRP → Manual}
\end{notebox}

\textbf{Current Mode Display}: OSD bottom center shows: \osd{MODE: Manual}, \osd{MODE: Sector Scan}, \osd{MODE: TRP}, or \osd{MODE: Radar}

\begin{cautionbox}[RESTRICTION]
Cannot change modes during tracking acquisition. Must abort tracking first (press TRK button).
\end{cautionbox}

\section{Manual Mode}

\textbf{Description}: Direct operator control via joystick (default mode)

\textbf{When to Use}:
\begin{itemize}
    \item Direct target engagement
    \item Precise aiming
    \item Immediate threat response
    \item Search operations requiring operator judgment
\end{itemize}

\textbf{Operation}:
\begin{itemize}
    \item Joystick LEFT/RIGHT → Azimuth control
    \item Joystick UP/DOWN → Elevation control
    \item Stick deflection = gimbal speed
    \item Center stick = gimbal stops
\end{itemize}

\begin{notebox}
Already Covered: See Lesson 2, Section 2.3 for detailed joystick control
\end{notebox}

\section{Auto Sector Scan Mode}

\subsection{What Is Sector Scan?}

\textbf{Definition}: Automated gimbal movement that continuously scans between two pre-defined points (left and right limits)

\textbf{Purpose}:
\begin{itemize}
    \item Perimeter surveillance
    \item Monitoring a defined sector without operator input
    \item Frees operator to monitor other systems or threats
\end{itemize}

\textbf{Visual}:
\begin{lstlisting}
    Left Limit              Right Limit
        ↓                       ↓
        ●←←←←←←←scan←←←←←←←●
        ●→→→→→→→scan→→→→→→→●
             (repeats)
\end{lstlisting}

Gimbal continuously scans left-to-right, then right-to-left, repeat.

\subsection{Activating Sector Scan}

\textbf{Prerequisites}:
\begin{enumerate}
    \item Sector scan zone must be defined
    \item At least one sector scan zone exists
    \item System is in AutoSectorScan mode
\end{enumerate}

\begin{procedurebox}[SECTOR SCAN ACTIVATION]
\begin{enumerate}
    \item \textbf{Cycle to Sector Scan Mode}: Press Button 11 or 13 until OSD shows: \osd{MODE: Sector Scan}
    \item \textbf{System Behavior}:
    \begin{itemize}
        \item Gimbal automatically slews to left limit of active sector
        \item Scans to right limit (slow, smooth movement)
        \item Reverses, scans back to left limit
        \item Repeats continuously
    \end{itemize}
    \item \textbf{Scan Speed}: Default: ~5\degree/second (configurable)
    \item \textbf{Elevation}: Scans at elevation defined in zone (usually 0\degree\ horizon)
\end{enumerate}
\end{procedurebox}

\subsection{While Sector Scanning}

\textbf{Operator Actions}:

\textbf{CAN DO}:
\begin{itemize}
    \item Monitor video feed as sector scans
    \item Switch cameras (CAM button)
    \item Zoom in/out (Zoom rocker)
    \item Fire LRF (LRF button) at targets of interest
    \item Initiate tracking (TRK button) - aborts sector scan, switches to Manual
\end{itemize}

\textbf{CANNOT DO}:
\begin{itemize}
    \item Joystick axes do NOT control gimbal (ignored)
    \item Cannot manually slew gimbal (must exit mode first)
\end{itemize}

\textbf{To Override}:
\begin{itemize}
    \item Press Button 11/13 to return to Manual mode
    \item OR: Press TRK to start tracking (auto-switches to Manual)
\end{itemize}

\subsection{Multiple Sector Zones}

\textbf{If Multiple Sectors Defined}:
\begin{itemize}
    \item System scans active sector (last selected via menu or joystick)
    \item To change active sector via menu:
    \begin{enumerate}
        \item Cycle to Manual mode
        \item Access menu: Zone Definitions → Sector Scans
        \item Select desired sector
        \item Return to AutoSectorScan mode
    \end{enumerate}
\end{itemize}

\subsection{Zone Selection via Joystick (Buttons 14/16)}

\textbf{NEW CAPABILITY}: Rapid zone switching during surveillance \textbf{without menu access}.

\begin{procedurebox}[JOYSTICK ZONE SELECTION]
\textbf{During AutoSectorScan Mode}:
\begin{itemize}
    \item \textbf{Button 14}: Select NEXT sector scan zone
    \item \textbf{Button 16}: Select PREVIOUS sector scan zone
    \item Zone switches immediately, scanning resumes in new zone
\end{itemize}

\textbf{During TRP Scan Mode}:
\begin{itemize}
    \item \textbf{Button 14}: Select NEXT TRP location page
    \item \textbf{Button 16}: Select PREVIOUS TRP location page
    \item TRP scan continues with new page's target points
\end{itemize}
\end{procedurebox}

\textbf{Operational Benefit}:
\begin{itemize}
    \item Rapid zone switching \textbf{without leaving surveillance mode}
    \item Hands stay on joystick, eyes stay on video
    \item Faster reaction to changing threat sectors
    \item No menu navigation required
\end{itemize}

\textbf{Example - Sector Scan Zone Cycling}:
\begin{lstlisting}
Currently scanning: "Front Gate" (Zone 1)
Press Button 14 → Switch to "East Perimeter" (Zone 2)
Press Button 14 → Switch to "Rear Area" (Zone 3)
Press Button 14 → Wrap to "Front Gate" (Zone 1)
\end{lstlisting}

\textbf{Example - TRP Page Cycling}:
\begin{lstlisting}
Currently on: "Page 1: Checkpoints"
Press Button 14 → Switch to "Page 2: Hilltops"
Press Button 14 → Switch to "Page 3: Route Monitoring"
Press Button 14 → Wrap to "Page 1: Checkpoints"
\end{lstlisting}

\begin{notebox}
Zone/TRP selection via joystick only works during \textbf{Surveillance} operational mode while in AutoSectorScan or TRPScan motion modes. During Tracking or Engagement, these buttons perform tracking gate control instead.
\end{notebox}

\section{TRP Scan Mode}

\subsection{What Is TRP Scan?}

\textbf{TRP} = \textbf{Target Reference Point} (pre-defined location of interest)

\textbf{Definition}: System sequentially slews to each TRP, dwells (pauses) for observation, then moves to next TRP in list.

\textbf{Purpose}:
\begin{itemize}
    \item Checkpoint verification (e.g., gate 1, gate 2, gate 3)
    \item Known threat areas (e.g., sniper hide sites)
    \item Periodic scans of fixed locations
    \item More efficient than manual searching
\end{itemize}

\textbf{Visual}:
\begin{lstlisting}
TRP 1 (30s dwell) → TRP 2 (30s dwell) → TRP 3 (30s dwell) → TRP 1
\end{lstlisting}

\subsection{Activating TRP Scan}

\textbf{Prerequisites}:
\begin{enumerate}
    \item At least one TRP must be defined
    \item System is in TRP Scan mode
\end{enumerate}

\begin{procedurebox}[TRP SCAN ACTIVATION]
\begin{enumerate}
    \item \textbf{Cycle to TRP Scan Mode}: Press Button 11 or 13 until OSD shows: \osd{MODE: TRP}
    \item \textbf{System Behavior}:
    \begin{itemize}
        \item Gimbal slews to first TRP in list
        \item Dwells for configured time (default: 30 seconds)
        \item Slews to next TRP
        \item Dwells again
        \item Repeats through entire TRP list, then loops back
    \end{itemize}
    \item \textbf{Dwell Time}: Configurable per TRP (5-120 seconds)
\end{enumerate}
\end{procedurebox}

\subsection{While TRP Scanning}

\textbf{During Dwell} (gimbal stationary at TRP):
\begin{itemize}
    \item Observe video
    \item Switch cameras
    \item Zoom
    \item Fire LRF
    \item Initiate tracking (if threat detected)
\end{itemize}

\textbf{During Slew} (gimbal moving between TRPs):
\begin{itemize}
    \item Gimbal is in motion
    \item Video may be blurred (fast slew)
    \item Wait for next dwell to observe
\end{itemize}

\subsection{TRP Scan Best Practices}

\begin{enumerate}
    \item \textbf{Define TRPs During Planning}:
    \begin{itemize}
        \item Pre-mission: Identify key locations
        \item Enter TRP coordinates via menu
        \item Test TRP scan before mission start
    \end{itemize}
    \item \textbf{Prioritize TRPs}: Place highest-threat areas first in list
    \item \textbf{Appropriate Dwell Times}:
    \begin{itemize}
        \item Short dwell (10-15s) for quick checks
        \item Long dwell (60s+) for detailed observation
    \end{itemize}
    \item \textbf{Combine with Manual}: Use TRP scan for routine surveillance, switch to Manual when threat detected
\end{enumerate}

\section{Radar Slew Mode (Optional)}

\textbf{Availability}: Only if radar system is integrated

\textbf{Description}: Gimbal automatically slews to radar-detected targets

\textbf{Purpose}:
\begin{itemize}
    \item Rapid threat response
    \item Automated cueing from radar
    \item Reduces operator workload
\end{itemize}

\begin{procedurebox}[RADAR SLEW OPERATION]
\begin{enumerate}
    \item \textbf{Cycle to Radar Slew Mode}: Press Button 11/13 until OSD shows: \osd{MODE: Radar}
    \item \textbf{System Behavior}:
    \begin{itemize}
        \item Waits for radar detection
        \item When radar detects target → Gimbal slews to radar coordinates
        \item OSD displays: \osd{RADAR CUE} or \osd{SLEWING TO RADAR}
        \item Operator confirms threat visually on video
    \end{itemize}
    \item \textbf{Operator Decision}:
    \begin{itemize}
        \item If threat confirmed → Initiate tracking or engage
        \item If false alarm → Wait for next radar cue or cycle to Manual
    \end{itemize}
\end{enumerate}
\end{procedurebox}

\begin{notebox}
Most El 7arress RCWS systems do NOT have radar. This mode will show "RADAR NOT AVAILABLE" if no radar connected.
\end{notebox}

\section{Motion Mode Quick Reference}

\small
\begin{longtable}{|L{0.20\textwidth}|L{0.25\textwidth}|L{0.20\textwidth}|L{0.28\textwidth}|}
\hline
\rowcolor{milblue!20}
\textbf{Mode} & \textbf{Use Case} & \textbf{Gimbal Control} & \textbf{Exit to Manual} \\
\hline
Manual & Direct engagement & Joystick & N/A (already manual) \\
\hline
AutoSectorScan & Perimeter surveillance & Automatic & Cycle mode or press TRK \\
\hline
TRP Scan & Checkpoint monitoring & Automatic & Cycle mode or press TRK \\
\hline
Radar Slew & Radar integration & Automatic & Cycle mode or press TRK \\
\hline
\end{longtable}

\textbf{Emergency Return to Manual}: Press Button 11/13 repeatedly until \osd{MODE: Manual} displays

\section{Zone Management - Sector Scans}

\subsection{Defining Sector Scan Zones}

Sector scan zones set the \textbf{left and right limits} for AutoSectorScan mode.

\textbf{Access}: Main Menu → Zone Definitions → Sector Scans

\subsubsection{Sector Scan Zone Parameters}

\small
\begin{longtable}{|L{0.25\textwidth}|L{0.35\textwidth}|L{0.33\textwidth}|}
\hline
\rowcolor{milblue!20}
\textbf{Parameter} & \textbf{Description} & \textbf{Typical Value} \\
\hline
Name & Zone identifier & "Front Gate", "Perimeter East" \\
\hline
Left Limit (Az) & Starting azimuth & 045\degree \\
\hline
Right Limit (Az) & Ending azimuth & 135\degree \\
\hline
Elevation & Scan elevation angle & 0\degree\ (horizon) \\
\hline
Scan Speed & Degrees per second & 5\degree/sec \\
\hline
Active & Enable/disable zone & ON / OFF \\
\hline
\end{longtable}

\subsubsection{Creating a Sector Scan Zone}

\textbf{Method 1: Manual Entry} (via menu)

\begin{procedurebox}[MANUAL SECTOR CREATION]
\begin{enumerate}
    \item \textbf{Access Menu}: MENU ✓ → Zone Definitions → Sector Scans → Add New Sector
    \item \textbf{Enter Name}: Use ▲/▼ to enter name characters, MENU ✓ to confirm
    \item \textbf{Set Left Limit}:
    \begin{itemize}
        \item Method A: Slew gimbal to desired left position, press MENU ✓ to "Capture Current Position"
        \item Method B: Manually enter azimuth value using ▲/▼
    \end{itemize}
    \item \textbf{Set Right Limit}: Same as left limit
    \item \textbf{Set Elevation}: Enter elevation angle (typically 0\degree)
    \item \textbf{Set Scan Speed}: Enter degrees/second (5\degree/sec recommended)
    \item \textbf{Enable Zone}: Set Active = ON
    \item \textbf{Save}: MENU ✓ → "Save Sector Scan"
\end{enumerate}
\end{procedurebox}

\textbf{Method 2: Quick Capture} (using joystick)

\begin{procedurebox}[QUICK SECTOR CAPTURE]
\begin{enumerate}
    \item \textbf{Position Gimbal}:
    \begin{itemize}
        \item Use Manual mode to slew to desired left limit
        \item Press \button{FN Button} + Hold for 2 seconds
        \item OSD displays: "LEFT LIMIT CAPTURED"
    \end{itemize}
    \item \textbf{Position Right Limit}:
    \begin{itemize}
        \item Slew to desired right limit
        \item Press \button{FN Button} + Hold for 2 seconds
        \item OSD displays: "RIGHT LIMIT CAPTURED, SECTOR SCAN ZONE CREATED"
    \end{itemize}
    \item \textbf{System Auto-Creates Zone}:
    \begin{itemize}
        \item Default name: "Sector X"
        \item Default elevation: Current elevation
        \item Default speed: 5\degree/sec
    \end{itemize}
    \item \textbf{Edit if Needed}: Access menu to rename or adjust parameters
\end{enumerate}
\end{procedurebox}

\subsection{Activating / Deactivating Sector Zones}

\textbf{Multiple Zones}:
\begin{itemize}
    \item You can define multiple sector scan zones
    \item Only ONE can be active at a time
\end{itemize}

\textbf{To Activate a Zone}:
\begin{enumerate}
    \item MENU ✓ → Zone Definitions → Sector Scans
    \item Use ▲/▼ to select desired zone
    \item MENU ✓ → "Set as Active"
    \item OSD displays: "SECTOR ZONE [Name] ACTIVE"
\end{enumerate}

\textbf{To Deactivate All Sectors}:
\begin{enumerate}
    \item MENU ✓ → Zone Definitions → Sector Scans
    \item Select active zone
    \item MENU ✓ → "Deactivate"
\end{enumerate}

\section{Zone Management - Target Reference Points}

\subsection{Defining TRPs}

TRPs are \textbf{fixed locations} the system can automatically slew to.

\textbf{Access}: Main Menu → Zone Definitions → TRPs

\subsubsection{TRP Parameters}

\small
\begin{longtable}{|L{0.25\textwidth}|L{0.35\textwidth}|L{0.33\textwidth}|}
\hline
\rowcolor{milblue!20}
\textbf{Parameter} & \textbf{Description} & \textbf{Typical Value} \\
\hline
Name & TRP identifier & "Gate 1", "Bunker", "Hill 203" \\
\hline
Azimuth & Direction to TRP & 090\degree \\
\hline
Elevation & Angle to TRP & +5\degree \\
\hline
Dwell Time & Observation time & 30 seconds \\
\hline
Active & Enable/disable & ON / OFF \\
\hline
\end{longtable}

\subsubsection{Creating a TRP}

\textbf{Method 1: Capture Current Position}

\begin{procedurebox}[TRP CAPTURE]
\begin{enumerate}
    \item \textbf{Manual Mode}: Slew gimbal to desired TRP location, zoom/focus on exact point
    \item \textbf{Access Menu}: MENU ✓ → Zone Definitions → TRPs → Add TRP
    \item \textbf{Capture Position}: Select "Capture Current Position", MENU ✓ to confirm
    \item \textbf{Enter Name}: Use ▲/▼ to enter TRP name, MENU ✓ to confirm
    \item \textbf{Set Dwell Time}: Enter seconds (5-120), Default: 30 seconds
    \item \textbf{Enable}: Set Active = ON
    \item \textbf{Save}: MENU ✓ → "Save TRP"
\end{enumerate}
\end{procedurebox}

\textbf{Method 2: Manual Coordinate Entry}

\begin{procedurebox}[TRP MANUAL ENTRY]
\begin{enumerate}
    \item \textbf{Access Menu}: MENU ✓ → Zone Definitions → TRPs → Add TRP
    \item \textbf{Enter Azimuth}: Use ▲/▼ to enter degrees (000-359)
    \item \textbf{Enter Elevation}: Use ▲/▼ to enter degrees (-20 to +60)
    \item \textbf{Continue}: Enter name, dwell time, enable, and save
\end{enumerate}
\end{procedurebox}

\subsection{Managing TRP List}

\textbf{TRP Sequence}:
\begin{itemize}
    \item TRPs are visited in the order they appear in list
    \item To reorder:
    \begin{enumerate}
        \item MENU ✓ → Zone Definitions → TRPs
        \item Select TRP
        \item "Move Up" or "Move Down"
    \end{enumerate}
\end{itemize}

\textbf{Editing TRPs}:
\begin{itemize}
    \item Select TRP from list
    \item MENU ✓ → "Edit TRP"
    \item Modify parameters
    \item Save changes
\end{itemize}

\textbf{Deleting TRPs}:
\begin{itemize}
    \item Select TRP from list
    \item MENU ✓ → "Delete TRP"
    \item Confirm deletion
\end{itemize}

\section{Zone Management - No-Fire \& No-Traverse}

\subsection{Viewing No-Fire Zones}

\textbf{Access}: Main Menu → Zone Definitions → No-Fire Zones

\textbf{Display}:
\begin{itemize}
    \item List of all defined no-fire zones
    \item Each zone shows:
    \begin{itemize}
        \item Name (e.g., "Friendly FOB", "Civilian Area 1")
        \item Boundary type (Polygon, Circle, Arc)
        \item Active status (ON/OFF)
    \end{itemize}
\end{itemize}

\textbf{Operator Permission}:
\begin{itemize}
    \item \textbf{CAN}: View zones, see boundaries on map overlay
    \item \textbf{CANNOT}: Modify boundaries, delete zones, override zones
\end{itemize}

\begin{notebox}
Modification usually requires commander/supervisor authorization.
\end{notebox}

\subsection{Viewing No-Traverse Zones}

\textbf{Access}: Main Menu → Zone Definitions → No-Traverse Zones

\textbf{Display}:
\begin{itemize}
    \item List of all defined no-traverse zones
    \item Each zone shows:
    \begin{itemize}
        \item Name (e.g., "Rear 90\degree", "Antenna Area")
        \item Azimuth limits
        \item Active status
    \end{itemize}
\end{itemize}

\textbf{Purpose Reminder}:
\begin{itemize}
    \item No-traverse zones prevent gimbal movement into restricted areas
    \item Protects vehicle structure, equipment, personnel
\end{itemize}

\section{Saving \& Loading Zone Configurations}

\subsection{Saving Zone Configuration}

\textbf{Purpose}: Save all zones (sectors, TRPs, no-fire, no-traverse) to file for later use

\begin{procedurebox}[SAVE CONFIGURATION]
\begin{enumerate}
    \item \textbf{Access Menu}: MENU ✓ → Zone Definitions → Save/Load → Save Configuration
    \item \textbf{Enter Filename}: Use ▲/▼ to enter filename (e.g., "MISSION\_20250115")
    \item \textbf{Confirm Save}: MENU ✓ → "Save"
    \item OSD displays: "ZONE CONFIG SAVED"
\end{enumerate}
\end{procedurebox}

\textbf{File Location}: Saved to internal storage (typically /configs/zones/)

\subsection{Loading Zone Configuration}

\textbf{Purpose}: Load previously saved zone configuration

\begin{procedurebox}[LOAD CONFIGURATION]
\begin{enumerate}
    \item \textbf{Access Menu}: MENU ✓ → Zone Definitions → Save/Load → Load Configuration
    \item \textbf{Select File}: Use ▲/▼ to browse saved configurations
    \item \textbf{Confirm Load}: MENU ✓ → "Load"
    \item OSD displays: "ZONE CONFIG LOADED"
\end{enumerate}
\end{procedurebox}

\begin{warningbox}
Loading a configuration OVERWRITES current zones. Save current zones first if needed.
\end{warningbox}

\subsection{Default Zone Configuration}

\textbf{Default Zones}:
\begin{itemize}
    \item System ships with default no-traverse zones (vehicle-specific)
    \item Default no-fire zones may be empty (mission-dependent)
\end{itemize}

\textbf{Restoring Defaults}:
\begin{enumerate}
    \item MENU ✓ → Zone Definitions → Save/Load → Restore Defaults
    \item Confirm: "RESTORE DEFAULT ZONES?"
    \item MENU ✓ → "YES"
\end{enumerate}

\section{Surveillance Best Practices}

\subsection{Choosing the Right Mode}

\small
\begin{longtable}{|L{0.30\textwidth}|L{0.25\textwidth}|L{0.38\textwidth}|}
\hline
\rowcolor{milblue!20}
\textbf{Situation} & \textbf{Recommended Mode} & \textbf{Rationale} \\
\hline
Direct threat engagement & Manual & Full control, immediate response \\
\hline
Perimeter watch (quiet) & AutoSectorScan & Automated, frees attention \\
\hline
Checkpoint routine & TRP Scan & Efficient for fixed locations \\
\hline
High-threat area scan & Manual & Requires operator judgment \\
\hline
Radar-integrated ops & Radar Slew & Rapid response to radar cues \\
\hline
\end{longtable}

\subsection{Combining Modes with Tracking}

\textbf{Workflow Example}:
\begin{enumerate}
    \item Start in AutoSectorScan (perimeter surveillance)
    \item Threat detected during scan
    \item Press TRK → System switches to Manual, starts tracking acquisition
    \item Lock onto threat (second TRK press)
    \item Engage or monitor as threat tracked
    \item Abort tracking (third TRK press)
    \item Resume surveillance: Press Button 11/13 to return to AutoSectorScan
\end{enumerate}

\subsection{Zone Discipline}

\textbf{Before Mission}:
\begin{checklist}
    \item Load appropriate zone configuration
    \item Verify no-fire zones match current ROE
    \item Test sector scans and TRPs
    \item Brief all operators on zones
\end{checklist}

\textbf{During Mission}:
\begin{checklist}
    \item Respect all zone warnings
    \item Never attempt to override no-fire zones without authorization
    \item Report zone boundary errors to command
    \item Update TRPs as mission evolves (if authorized)
\end{checklist}

\textbf{After Mission}:
\begin{checklist}
    \item Save zone configuration if modified
    \item Debrief on zone effectiveness
    \item Recommend adjustments for future missions
\end{checklist}

\section{Student Review Questions}

\begin{enumerate}
    \item What are the four motion modes available on the RCWS?
    \item How do you cycle between motion modes?
    \item What is the purpose of Auto Sector Scan mode?
    \item What does TRP stand for?
    \item What is the typical dwell time at a TRP?
    \item How do you create a sector scan zone using the quick capture method?
    \item Can operators modify no-fire zone boundaries?
    \item What happens when you load a zone configuration?
    \item Which motion mode is recommended for direct threat engagement?
    \item How do you reorder TRPs in the scan sequence?
\end{enumerate}
% ============================================================================
% LESSON 5 - TARGET ENGAGEMENT PROCESS
% ============================================================================
\lesson{5 - TARGET ENGAGEMENT PROCESS}

\begin{tabular}{@{}ll@{}}
\textbf{Duration:} & 4 hours (Classroom 1h + Simulator 3h) \\
\textbf{Type:} & Classroom + Practical \\
\textbf{References:} & Operator manual, tracking system documentation \\
\end{tabular}

\section{LEARNING OBJECTIVES}

Upon completion of this lesson, operators will be able to:
\begin{itemize}
    \item Execute complete target engagement sequence
    \item Operate tracking system through all phases
    \item Adjust acquisition gate for target selection
    \item Perform emergency tracking abort
    \item Execute simulated weapons engagement
\end{itemize}

% ============================================================================
\section{TARGET ENGAGEMENT SEQUENCE}

\subsection{THE SIX-PHASE ENGAGEMENT CYCLE}

The complete engagement process follows six distinct phases:

\begin{center}
\texttt{DETECT → IDENTIFY → ACQUIRE → TRACK → ENGAGE → ASSESS}
\end{center}

\spectable{
\begin{tabular}{|L{0.15\textwidth}|L{0.45\textwidth}|L{0.25\textwidth}|}
\hline
\rowcolor{milblue!20}
\textbf{Phase} & \textbf{Operator Actions} & \textbf{Expected Duration} \\
\hline
\textbf{1. DETECT} & Scan area (manual/auto modes), visual detection & Continuous \\
\hline
\textbf{2. IDENTIFY} & Slew to target, zoom in, PID (Positive ID), verify ROE & 5-15 seconds \\
\hline
\textbf{3. ACQUIRE} & Enter tracking acquisition, position gate, size gate & 5-10 seconds \\
\hline
\textbf{4. TRACK} & Lock-on, monitor track quality & Continuous until engagement \\
\hline
\textbf{5. ENGAGE} & Range, LAC (if moving), Master Arm, Fire, Observe & 2-30 seconds \\
\hline
\textbf{6. ASSESS} & BDA (Battle Damage Assessment), re-engage or cease & 5-10 seconds \\
\hline
\end{tabular}
}

\subsection{DETECT \& IDENTIFY (PHASES 1-2)}

\subsubsection{Detection Methods}
\begin{itemize}
    \item Manual scan (joystick control)
    \item Auto Sector Scan (Lesson 4)
    \item TRP Scan (Lesson 4)
    \item Radar cues (if available)
\end{itemize}

\subsubsection{Identification Requirements}
\begin{itemize}
    \item \textbf{PID (Positive Identification)} mandatory before engagement
    \item Use zoom to magnify target (\button{Button 6/8})
    \item Thermal camera may aid identification (night, obscurants)
    \item Verify target meets Rules of Engagement (ROE)
    \item Confirm NOT friendly forces
\end{itemize}

\begin{warningbox}
Failure to achieve PID before engagement violates ROE and may result in fratricide.
\end{warningbox}

% ============================================================================
\section{TRACKING SYSTEM OVERVIEW}

\subsection{TRACKING PHASE STATE MACHINE}

The tracking system operates through discrete phases:

\begin{center}
\begin{tikzpicture}[
    node distance=1.5cm,
    state/.style={rectangle, rounded corners, draw=milblue, fill=milblue!10, thick, minimum width=3cm, minimum height=0.8cm, align=center},
    arrow/.style={->, thick, >=stealth}
]
    \node[state] (off) {OFF};
    \node[state, below of=off] (acq) {ACQUISITION};
    \node[state, below of=acq] (pend) {LOCK PENDING};
    \node[state, below of=pend] (lock) {ACTIVE LOCK};
    \node[state, right of=lock, node distance=4cm] (coast) {COAST};
    \node[state, below of=lock] (fire) {FIRING};
    
    \draw[arrow] (off) -- node[right, font=\tiny] {Button 4} (acq);
    \draw[arrow] (acq) -- node[right, font=\tiny] {Size gate + Button 4} (pend);
    \draw[arrow] (pend) -- node[right, font=\tiny] {Lock success} (lock);
    \draw[arrow] (lock) -- node[above, font=\tiny] {Target lost} (coast);
    \draw[arrow] (coast) -- node[below, font=\tiny] {Re-acquired} (lock);
    \draw[arrow] (lock) -- (fire);
    \draw[arrow] (fire) -- node[left, font=\tiny] {Cease fire} (lock);
    \draw[arrow, dashed] (fire.east) -- ++(2,0) |- node[right, font=\tiny] {Abort} (off.east);
\end{tikzpicture}
\end{center}

\subsection{TRACKING PHASE REFERENCE}

\begin{longtable}{|L{0.13\textwidth}|L{0.13\textwidth}|L{0.10\textwidth}|L{0.10\textwidth}|L{0.11\textwidth}|L{0.18\textwidth}|}
\hline
\rowcolor{milblue!20}
\textbf{Phase} & \textbf{OSD Status} & \textbf{Box Color} & \textbf{Box Style} & \textbf{Duration} & \textbf{Gimbal Control} \\
\hline
\endfirsthead

\hline
\rowcolor{milblue!20}
\textbf{Phase} & \textbf{OSD Status} & \textbf{Box Color} & \textbf{Box Style} & \textbf{Duration} & \textbf{Gimbal Control} \\
\hline
\endhead

\textbf{Off} & MODE: Manual & None & N/A & Until acquisition & Operator (joystick) \\
\hline
\textbf{Acquisition} & ACQUISITION & Yellow & Solid & 5-10 sec & Operator (joystick) \\
\hline
\textbf{Lock Pending} & LOCK PENDING & Cyan & Solid/Flash & 0.5-2 sec & Operator (hold steady) \\
\hline
\textbf{Active Lock} & TRACKING & Green & Dashed & Continuous & Automatic (tracker) \\
\hline
\textbf{Coast} & TRACKING (COAST) & Yellow & Dashed & 1-5 sec & Automatic (predict) \\
\hline
\textbf{Firing} & TRACKING (FIRING) & Green & Solid & While firing & Automatic (hold) \\
\hline
\end{longtable}

\subsection{TRACKING CONTROLS}

\spectable{
\begin{tabular}{|L{0.20\textwidth}|L{0.40\textwidth}|L{0.30\textwidth}|}
\hline
\rowcolor{milblue!20}
\textbf{Control} & \textbf{Function} & \textbf{Active Phase(s)} \\
\hline
\button{Button 4} (single) & Start Acquisition / Request Lock-On & Off → Acq, Acq → Lock Pending \\
\hline
\button{Button 4} (double <1 sec) & \textbf{EMERGENCY ABORT} & Any phase → Off \\
\hline
\button{D-Pad ▲} & Decrease gate height (-4 px) & Acquisition only \\
\hline
\button{D-Pad ▼} & Increase gate height (+4 px) & Acquisition only \\
\hline
\button{D-Pad ◄} & Decrease gate width (-4 px) & Acquisition only \\
\hline
\button{D-Pad ►} & Increase gate width (+4 px) & Acquisition only \\
\hline
\end{tabular}
}

% ============================================================================
\section{ACQUISITION PHASE}

\subsection{ENTERING ACQUISITION MODE}

\subsubsection{Prerequisites}
\begin{checklist}
    \item Motion mode: Manual
    \item Target on-screen and identified
    \item Target approximately centered in reticle
\end{checklist}

\begin{procedurebox}[ENTERING ACQUISITION]
\textbf{1. Press \button{Button 4} (Track Select)}

\textbf{Result:}
\begin{itemize}
    \item System enters \textbf{ACQUISITION} phase
    \item Yellow acquisition gate appears (centered on reticle)
    \item OSD displays: \osd{MODE: ACQUISITION}
    \item Joystick gimbal control remains active
    \item D-Pad now controls gate size
\end{itemize}
\end{procedurebox}

\subsection{SIZING THE ACQUISITION GATE}

\textbf{Objective:} Frame target with 10-30\% margin on all sides

\subsubsection{D-Pad Controls}
\begin{itemize}
    \item \button{UP ▲}: Decrease height
    \item \button{DOWN ▼}: Increase height
    \item \button{LEFT ◄}: Decrease width
    \item \button{RIGHT ►}: Increase width
    \item \textbf{Step size:} 4 pixels per press
\end{itemize}

\subsubsection{Gate Sizing Guidelines}

\spectable{
\begin{tabular}{|L{0.22\textwidth}|L{0.35\textwidth}|L{0.23\textwidth}|}
\hline
\rowcolor{milblue!20}
\textbf{Target Framing} & \textbf{Effect on Tracking} & \textbf{Recommendation} \\
\hline
Too Tight (<10\% margin) & Tracker may lose target if target rotates/expands & ❌ Avoid \\
\hline
Optimal (10-30\% margin) & Best tracking performance & ✅ Ideal \\
\hline
Too Loose (>50\% margin) & Background clutter may confuse tracker & ❌ Avoid \\
\hline
\end{tabular}
}

\subsubsection{Best Practices}
\begin{itemize}
    \item \textbf{Vehicles:} Frame entire hull, exclude ground
    \item \textbf{Personnel:} Include torso/legs, minimize background
    \item \textbf{Moving targets:} Size slightly larger (anticipate motion)
    \item \textbf{Default size:} 100×100 pixels (adjust as needed)
\end{itemize}

\subsection{REQUESTING LOCK-ON}

\subsubsection{When Ready}
\begin{checklist}
    \item Target fully visible and framed in gate
    \item Target has good contrast against background
    \item Target not moving erratically
\end{checklist}

\textbf{Action:} Press \button{Button 4} (second press)

\textbf{Result:}
\begin{itemize}
    \item System → \textbf{LOCK PENDING} phase
    \item Gate color: Yellow → Cyan (or flashing)
    \item OSD: \osd{MODE: LOCK PENDING}
    \item Tracker initialization begins (0.5-2 seconds)
\end{itemize}

\begin{cautionbox}
Do NOT make rapid gimbal movements during Lock Pending. Hold gimbal steady.
\end{cautionbox}

% ============================================================================
\section{LOCK PENDING → ACTIVE LOCK}

\subsection{LOCK PENDING PHASE}

\textbf{Purpose:} Tracking system initializes on target

\textbf{Duration:} 0.5 to 2 seconds (typically ~1 second)

\textbf{Operator Action:} \textbf{WAIT} - maintain steady gimbal

\subsubsection{What Happens}
\begin{enumerate}
    \item System captures reference image of target
    \item Initializes tracker algorithm
    \item Calculates target features
    \item Begins tracking target in video stream
\end{enumerate}

\subsubsection{Transition Outcomes}

\textbf{✅ SUCCESS:}
\begin{itemize}
    \item System → \textbf{ACTIVE LOCK} phase
    \item Gate color: Green
    \item Gate style: Dashed outline
    \item OSD: \osd{MODE: TRACKING} or \osd{ACTIVE LOCK}
    \item Gimbal control switches to automatic
\end{itemize}

\textbf{❌ FAILURE (rare):}
\begin{itemize}
    \item System → \textbf{ACQUISITION} phase (retry)
    \item Possible causes: Poor contrast, target too small, target motion
    \item Operator: Adjust gate size or abort and restart
\end{itemize}

\subsection{ACTIVE LOCK PHASE}

\subsubsection{System Behavior}
\begin{itemize}
    \item Tracker follows target at 30 Hz (30 times/second)
    \item Gimbal automatically moves to keep target centered
    \item Reticle stays on target center
    \item \textbf{Joystick axis inputs IGNORED} (tracking in control)
\end{itemize}

\subsubsection{Visual Indicators}
\begin{itemize}
    \item \textbf{Tracking Gate:} Green dashed outline around target
    \item \textbf{OSD:} \osd{MODE: TRACKING} or \osd{ACTIVE LOCK}
    \item \textbf{Control Panel:} TRACKING light ON (green)
    \item \textbf{Track Confidence:} >70\% (good), 50-70\% (marginal), <50\% (poor)
\end{itemize}

\subsection{OPERATOR ROLE DURING ACTIVE LOCK}

\textbf{Primary Role:} \textbf{MONITOR} - system is tracking automatically

\subsubsection{Monitor For}

\textbf{1. Track Quality:}
\begin{itemize}
    \item ✅ Green gate = Good track
    \item ⚠️ Yellow gate = Marginal (may lose soon)
    \item Track confidence >70\% (good), <50\% (prepare for coast)
\end{itemize}

\textbf{2. Target Status:}
\begin{itemize}
    \item Target correctly identified (didn't jump to wrong object)
    \item Target still valid (meets engagement criteria)
    \item Target not obscured or about to be obscured
\end{itemize}

\textbf{3. Gimbal Position:}
\begin{itemize}
    \item Staying within operational limits
    \item Not approaching no-traverse zones
    \item Elevation within -20\degree\ to +60\degree
\end{itemize}

\subsubsection{Controls Still Active}

\spectable{
\begin{tabular}{|L{0.30\textwidth}|L{0.60\textwidth}|}
\hline
\rowcolor{milgreen!20}
\multicolumn{2}{|c|}{\textbf{ACTIVE CONTROLS}} \\
\hline
\button{Button 0} & Master Arm \\
\hline
\button{Button 2} & LAC toggle \\
\hline
\button{Button 3} & Dead Man Switch \\
\hline
\button{Button 4} & Double-click abort \\
\hline
\button{Button 5} & Fire \\
\hline
\button{Button 6/8} & Zoom (use cautiously, may affect track) \\
\hline
\hline
\rowcolor{milred!20}
\multicolumn{2}{|c|}{\textbf{BLOCKED CONTROLS}} \\
\hline
\button{Button 11/13} & Mode cycle - BLOCKED during tracking \\
\hline
Joystick axes (X/Y) & IGNORED during tracking \\
\hline
\end{tabular}
}

% ============================================================================
\section{COAST MODE}

\subsection{WHEN COAST ACTIVATES}

\textbf{Triggers:}
\begin{itemize}
    \item Target temporarily obscured (passes behind object)
    \item Target leaves field of view briefly
    \item Tracker loses visual lock
    \item Dust, smoke, or other obscurants
\end{itemize}

\subsection{System Behavior}
\begin{itemize}
    \item Tracker \textbf{predicts} target position based on last known velocity
    \item Gimbal continues to predicted position
    \item System attempts to re-acquire target
    \item Tracking gate: Green → Yellow (dashed)
\end{itemize}

\subsection{Display Changes}
\begin{itemize}
    \item \textbf{OSD:} \osd{MODE: COAST} or \osd{TRACKING (COAST)}
    \item \textbf{Gate:} Yellow/amber dashed outline
    \item \textbf{Warning:} "COASTING - TARGET LOST" may display
\end{itemize}

\subsection{COAST OUTCOMES}

\textbf{Typical Duration:} 1-5 seconds

\subsubsection{Outcome 1: Target Re-Acquired (✅ Success)}
\begin{itemize}
    \item Target reappears in field of view
    \item Tracker re-locks on target
    \item System → \textbf{ACTIVE LOCK} phase
    \item Gate: Yellow → Green
    \item Tracking continues normally
\end{itemize}

\subsubsection{Outcome 2: Coast Timeout (❌ Failure)}
\begin{itemize}
    \item Target not re-acquired within timeout (~5 seconds)
    \item System gives up
    \item System → \textbf{OFF} phase
    \item Tracking stops
    \item Operator must restart tracking if desired
\end{itemize}

\subsection{Operator Action During Coast}

\begin{itemize}
    \item ✅ Wait patiently (system attempting re-acquisition)
    \item ✅ Be ready for track to resume
    \item ❌ Do NOT abort prematurely (give system time)
    \item ❌ Do NOT make manual gimbal movements (joystick ignored)
\end{itemize}

\subsubsection{Abort Coast If:}
\begin{itemize}
    \item Target definitely not coming back (destroyed, permanently obscured)
    \item Tracker coasting in wrong direction (confused)
    \item Mission changed
\end{itemize}

% ============================================================================
\section{TRACKING ABORT (EMERGENCY)}

\subsection{WHEN TO ABORT TRACKING}

Abort tracking \textbf{IMMEDIATELY} if:
\begin{itemize}
    \item ❌ Tracking wrong target (friendly, civilian, incorrect target)
    \item ❌ Target no longer valid (fails ROE)
    \item ❌ Safety concern (entering no-fire zone, gimbal obstruction)
    \item ❌ Mission change (new priority, orders to cease)
    \item ❌ Tracking erratic (unexpected gimbal behavior)
\end{itemize}

\subsection{ABORT PROCEDURE}

\begin{procedurebox}[EMERGENCY ABORT]
\textbf{Action:} \textbf{DOUBLE-CLICK \button{Button 4}} (<1 second between presses)

\textbf{Effect (IMMEDIATE):}
\begin{enumerate}
    \item Tracking \textbf{STOPS}
    \item System → \textbf{OFF} phase
    \item Gimbal holds current position
    \item Tracking gate disappears
    \item Weapon fire \textbf{INHIBITED} (even if Master Arm engaged)
    \item OSD: \osd{MODE: MANUAL}
\end{enumerate}
\end{procedurebox}

\subsubsection{Timing}
\begin{itemize}
    \item Press 1 → Press 2 within \textbf{1000 milliseconds} (one second)
    \item \textbf{Too slow} (>1 second): System interprets as two single presses (may restart tracking)
\end{itemize}

\begin{warningbox}[CRITICAL]
Practice double-click timing during training until muscle memory established.
\end{warningbox}

\subsection{AFTER ABORT}

\textbf{System State:}
\begin{itemize}
    \item Manual mode active
    \item Joystick gimbal control restored
    \item No tracking active
\end{itemize}

\textbf{Next Actions (mission-dependent):}
\begin{itemize}
    \item Re-acquire correct target and restart tracking
    \item Return to surveillance mode
    \item Engage different target
    \item Follow commander's orders
\end{itemize}

% ============================================================================
\section{WEAPON CHARGING (COCKING ACTUATOR)}

\subsection{COCKING ACTUATOR OVERVIEW}

The cocking actuator is a servo-controlled mechanism that cycles the weapon's bolt to chamber a round. This must be performed before the weapon can fire.

\subsubsection{Weapon-Specific Charging Cycles}

\small
\begin{longtable}{|L{0.18\textwidth}|L{0.18\textwidth}|L{0.55\textwidth}|}
\hline
\rowcolor{milblue!20}
\textbf{Weapon Type} & \textbf{Cycles Req.} & \textbf{Notes} \\
\hline
\textbf{M2HB (.50 cal)} & \textbf{2 cycles} & Closed bolt weapon - Cycle 1: pulls bolt, picks up round; Cycle 2: chambers round, weapon ready \\
\hline
M240B & 1 cycle & Open bolt weapon - single cycle chambers round \\
\hline
M249 SAW & 1 cycle & Open bolt weapon \\
\hline
MK19 (40mm) & 1 cycle & Grenade launcher \\
\hline
\end{longtable}

\begin{warningbox}[M2HB CRITICAL]
The M2HB requires \textbf{TWO} complete charging cycles before the weapon is ready to fire. A single cycle will NOT chamber a round.
\end{warningbox}

\subsection{CHARGING MODES}

The charging system supports two operational modes:

\subsubsection{Mode 1: Short Press (Automatic Cycle)}

\begin{procedurebox}[SHORT PRESS CHARGING]
\begin{enumerate}
    \item Press \button{Charge} button (joystick Button 9)
    \item Release button immediately (short press)
    \item System \textbf{automatically} cycles: Extend → Retract
    \item For M2HB: System automatically performs second cycle
    \item \textbf{Charging Complete} when retraction finishes
    \item 4-second lockout begins (prevents immediate re-charge)
\end{enumerate}
\end{procedurebox}

\textbf{Advantages}: Faster, automatic multi-cycle for M2HB
\textbf{Best For}: Standard charging operations

\subsubsection{Mode 2: Continuous Hold (Manual Control)}

\begin{procedurebox}[CONTINUOUS HOLD CHARGING]
\begin{enumerate}
    \item Press and \textbf{HOLD} \button{Charge} button
    \item Actuator extends (pulls bolt back)
    \item \textbf{Hold position} while button is held
    \item \textbf{Release} button to initiate retraction
    \item Actuator retracts (releases bolt)
    \item Repeat for additional cycles (M2HB)
    \item 4-second lockout begins after final cycle
\end{enumerate}
\end{procedurebox}

\textbf{Advantages}: Operator controls timing, can hold bolt back
\textbf{Best For}: Clearing jams, inspection, controlled operations

\subsection{CHARGING STATE MACHINE}

\begin{center}
\begin{tikzpicture}[
    node distance=1.3cm,
    state/.style={rectangle, rounded corners, draw=milblue, fill=milblue!10, thick, minimum width=2.5cm, minimum height=0.7cm, align=center, font=\small},
    arrow/.style={->, thick, >=stealth}
]
    \node[state] (idle) {IDLE};
    \node[state, below of=idle] (ext) {EXTENDING};
    \node[state, below of=ext] (exd) {EXTENDED (HOLD)};
    \node[state, below of=exd] (ret) {RETRACTING};
    \node[state, right of=ret, node distance=4cm] (lock) {LOCKOUT (4s)};
    \node[state, fill=milred!20, below of=ret] (jam) {JAM DETECTED};
    \node[state, fill=milyellow!20, below of=jam] (fault) {FAULT};

    \draw[arrow] (idle) -- node[right, font=\tiny] {Charge button} (ext);
    \draw[arrow] (ext) -- node[right, font=\tiny] {Full extend + hold} (exd);
    \draw[arrow] (ext.south east) -- ++(0.8, 0) -- node[right, font=\tiny] {Full extend + release} ++(0,-1.7) -- (ret.east);
    \draw[arrow] (exd) -- node[right, font=\tiny] {Button released} (ret);
    \draw[arrow] (ret) -- node[above, font=\tiny] {Full retract} (lock);
    \draw[arrow] (lock) -- node[above, font=\tiny] {4 sec timer} ++(0,3.9) -- (idle.east);
    \draw[arrow, red, dashed] (ext) -- node[left, font=\tiny] {Jam detected} (jam);
    \draw[arrow, red, dashed] (ret) -- (jam);
    \draw[arrow] (jam) -- node[right, font=\tiny] {Backoff complete} (fault);
    \draw[arrow] (fault) -- node[left, font=\tiny] {Operator reset} ++(-2.5,0) |- (idle.west);
\end{tikzpicture}
\end{center}

\subsection{CHARGING STATE REFERENCE}

\small
\begin{longtable}{|L{0.15\textwidth}|L{0.25\textwidth}|L{0.50\textwidth}|}
\hline
\rowcolor{milblue!20}
\textbf{State} & \textbf{OSD Display} & \textbf{Description} \\
\hline
IDLE & CHARGING: Ready & Weapon ready to charge \\
\hline
EXTENDING & CHARGING: Extending & Actuator moving forward, pulling bolt \\
\hline
EXTENDED & CHARGING: Extended (HOLD) & Bolt pulled back, awaiting button release \\
\hline
RETRACTING & CHARGING: Retracting & Actuator returning, releasing bolt \\
\hline
LOCKOUT & CHARGING: Lockout (4s) & El-7aress H100 4-second post-charge safety period \\
\hline
JAM DETECTED & CHARGING: JAM! & High torque + no movement = mechanical jam \\
\hline
FAULT & CHARGING: FAULT & Requires operator acknowledgment and reset \\
\hline
\end{longtable}

\subsection{4-SECOND EL-7ARESS H100 LOCKOUT}

\begin{warningbox}[LOCKOUT PERIOD]
Per El-7aress H100 specification, a \textbf{4-second lockout period} follows each completed charging cycle. During this time:
\begin{itemize}
    \item Charging button is \textbf{IGNORED}
    \item Prevents accidental double-charge
    \item Allows weapon to fully seat round
    \item OSD displays countdown or "Lockout" status
\end{itemize}
\end{warningbox}

\subsection{JAM DETECTION AND RECOVERY}

The charging system includes automatic jam detection:

\subsubsection{Jam Detection Criteria}
\begin{itemize}
    \item \textbf{High Torque}: Actuator motor exceeds torque threshold (>80\%)
    \item \textbf{No Movement}: Position not changing despite motor effort
    \item \textbf{Confirmation}: Condition persists for multiple samples (anti-false-positive)
\end{itemize}

\subsubsection{Automatic Jam Response}
\begin{enumerate}
    \item \textbf{IMMEDIATE STOP}: Motor halts to prevent damage
    \item \textbf{STATE TRANSITION}: System enters JAM DETECTED state
    \item \textbf{ALARM}: OSD displays "CHARGING: JAM!"
    \item \textbf{DELAYED BACKOFF}: After 200ms stabilization, actuator retracts to home
    \item \textbf{FAULT STATE}: System enters FAULT state, awaits operator reset
\end{enumerate}

\subsubsection{Jam Clearing Procedure}

\begin{procedurebox}[CLEARING A CHARGING JAM]
\begin{enumerate}
    \item \textbf{STOP}: Do NOT repeatedly press charge button
    \item \textbf{OBSERVE}: Note jam position on OSD (if displayed)
    \item \textbf{WAIT}: Allow automatic backoff to complete
    \item \textbf{VERIFY}: Check OSD shows "FAULT" state
    \item \textbf{INSPECT}: If possible, visually inspect weapon for obstruction
    \item \textbf{CLEAR}: Remove any visible obstruction (per weapon manual)
    \item \textbf{RESET}: Press \button{Charge} button once to reset fault
    \item \textbf{RETRY}: Actuator performs safe retraction to home position
    \item \textbf{RE-CHARGE}: After successful reset, perform normal charging
\end{enumerate}
\end{procedurebox}

\begin{warningbox}[SAFETY]
If jam persists after 2 reset attempts, \textbf{STOP} and request maintenance support. Repeated jam cycling can damage the weapon or actuator.
\end{warningbox}

\subsection{CHARGING SAFETY INTERLOCKS}

Charging is \textbf{BLOCKED} when:

\small
\begin{longtable}{|L{0.30\textwidth}|L{0.60\textwidth}|}
\hline
\rowcolor{milblue!20}
\textbf{Condition} & \textbf{Reason/Resolution} \\
\hline
Emergency Stop Active & Clear E-stop before charging \\
\hline
Station Disabled & Enable station with Station Enable switch \\
\hline
Charge Already In Progress & Wait for current cycle to complete \\
\hline
Lockout Active & Wait 4 seconds after previous charge \\
\hline
Fault State & Reset fault by pressing Charge button \\
\hline
\end{longtable}

\subsection{STARTUP ACTUATOR CHECK}

On system startup, if the actuator is found in an extended position:

\begin{itemize}
    \item System automatically performs \textbf{safe retraction}
    \item Returns actuator to home (retracted) position
    \item Shorter timeout used (best-effort recovery)
    \item Prevents leaving weapon in unsafe partially-charged state
\end{itemize}

% ============================================================================
\section{WEAPONS ENGAGEMENT SEQUENCE}

\subsection{PRE-ENGAGEMENT CHECKLIST}

Before engaging, verify \textbf{ALL} items:

\begin{checklist}
    \item Target positively identified (\textbf{PID})
    \item Target valid (meets ROE)
    \item Fire authorization received (if required)
    \item NOT in no-fire zone (check OSD for zone warnings)
    \item Friendly forces clear
    \item Weapon loaded and ready
    \item Zeroing active (if applicable)
    \item Environmental parameters set (if applicable)
    \item Track established (if using tracking)
\end{checklist}

\begin{warningbox}
IF ANY ITEM CANNOT BE CHECKED, DO NOT FIRE.
\end{warningbox}

\subsection{ENGAGEMENT PROCEDURE (STEP-BY-STEP)}

\subsubsection{STEP 1: Acquire and Track Target}

\begin{enumerate}
    \item Detect and identify target (PID mandatory)
    \item \button{Button 4} → Enter acquisition mode
    \item \button{D-Pad} → Size gate to frame target (10-30\% margin)
    \item \button{Button 4} → Request lock-on
    \item Wait for \textbf{ACTIVE LOCK} (green gate, OSD: TRACKING)
    \item Monitor track quality (confidence >70\%, green gate)
\end{enumerate}

\subsubsection{STEP 2: Range Target}

\begin{enumerate}
    \item Fire Laser Range Finder (LRF trigger)
    \item Wait for range reading (OSD: \osd{RNG: xxxx m})
    \item Verify range reasonable
    \item Range used for ballistic calculations (CCIP)
\end{enumerate}

\begin{notebox}
LRF may fire automatically during tracking (configuration-dependent). Verify range displayed.
\end{notebox}

\subsubsection{STEP 3: Enable Lead Angle Compensation (If Moving Target)}

\textbf{If target is moving} (lateral motion, speed >5 m/s, range >100m):

\begin{enumerate}
    \item Hold \button{Dead Man Switch (Button 3)}
    \item Press \button{LAC Toggle (Button 2)}
    \item Verify LAC status:
    \begin{itemize}
        \item ✅ \textbf{"LEAD ANGLE ON"} (green) = Ready
        \item ⚠️ \textbf{"LEAD ANGLE LAG"} (yellow) = Wait for tracking data
        \item ❌ \textbf{"ZOOM OUT"} (red) = FOV too narrow, zoom out
    \end{itemize}
    \item Observe CCIP reticle offset ahead of target (lead point)
\end{enumerate}

\textbf{If target stationary:} LAC not necessary (CCIP at target center)

\begin{notebox}
Detailed LAC procedures in Lesson 6.
\end{notebox}

\subsubsection{STEP 4: Final Safety Checks}

\begin{checklist}
    \item Verify target still valid
    \item Verify tracking active (green gate, good confidence)
    \item Check OSD for warnings:
    \begin{itemize}
        \item ❌ \textbf{"ZONE VIOLATION"} = DO NOT FIRE
        \item ❌ \textbf{"NO-FIRE ZONE"} = DO NOT FIRE
        \item ✅ No warnings = Clear to fire
    \end{itemize}
    \item Verify friendly forces clear
    \item Verify backstop (if required)
\end{checklist}

\subsubsection{STEP 5: Engage Master Arm}

\begin{enumerate}
    \item Pull trigger to \textbf{Stage 1} (half-pull) → \textbf{Master Arm} (\button{Button 0})
    \begin{itemize}
        \item OR toggle Master Arm switch on Control Panel
    \end{itemize}
    \item Verify \textbf{"ARMED"} indicator light ON (red)
    \item OSD may display: \osd{WEAPON ARMED}
\end{enumerate}

\begin{warningbox}
WEAPON IS NOW HOT
\end{warningbox}

\subsubsection{STEP 6: Fire Weapon}

\begin{enumerate}
    \item Final aim verification (CCIP on target or lead point)
    \item Pull trigger to \textbf{Stage 2} (full-pull) → \textbf{Fire} (\button{Button 5})
    \item Weapon fires
    \item Hold trigger for desired burst:
    \begin{itemize}
        \item \textbf{Single Shot:} Quick press/release (1 round)
        \item \textbf{Burst:} Hold 2-3 seconds (controlled burst)
        \item \textbf{Sustained:} Hold longer (use cautiously)
    \end{itemize}
\end{enumerate}

\textbf{During Firing:}
\begin{itemize}
    \item Tracking keeps reticle on target
    \item Gimbal compensates for recoil
    \item Observe rounds impacting
    \item Adjust fire as needed
\end{itemize}

\textbf{System State:}
\begin{itemize}
    \item Tracking phase → \textbf{FIRING}
    \item OSD: \osd{MODE: TRACKING (FIRING)}
    \item Gate: Green solid outline
    \item Enhanced stabilization active
\end{itemize}

\subsubsection{Dead Reckoning During Firing (El-7aress Doctrine)}

\begin{warningbox}[IMPORTANT - DEAD RECKONING]
Per El-7aress H100 specification: "When firing is initiated, the system aborts Target Tracking. Instead the system moves according to the speed and direction of the WS just prior to pulling the trigger. The system will not automatically compensate for changes in speed or direction of the tracked target during firing."
\end{warningbox}

\textbf{Dead Reckoning Behavior}:
\begin{itemize}
    \item When trigger is pulled, system captures \textbf{last known target velocity}
    \item Gimbal continues moving at captured velocity (azimuth and elevation rates)
    \item Tracker is \textbf{NOT actively following} target during firing
    \item System \textbf{predicts} target position based on last velocity
    \item If target maneuvers during firing, rounds may miss
\end{itemize}

\textbf{Operator Implications}:
\begin{itemize}
    \item ✅ Best for targets with \textbf{constant velocity} (vehicles on road)
    \item ⚠️ Less effective against \textbf{maneuvering targets}
    \item ⚠️ Fire short bursts, reassess, fire again for erratic targets
    \item After firing stops, must \textbf{re-acquire} target to resume tracking
\end{itemize}

\subsubsection{STEP 7: Cease Fire}

\begin{enumerate}
    \item Release \textbf{Fire button} (\button{Button 5} / trigger stage 2)
    \item Weapon stops firing
    \item Release \textbf{Master Arm} (\button{Button 0} / trigger stage 1)
    \item Verify \textbf{"ARMED"} indicator OFF
    \item Tracking continues (unless aborted)
\end{enumerate}

\subsubsection{STEP 8: Assess Target (BDA)}

\textbf{Battle Damage Assessment:}
\begin{itemize}
    \item ✅ Target destroyed? → Stop tracking, report success
    \item ⚠️ Target damaged? → Re-engage (repeat Steps 5-7)
    \item ❌ Target missed? → Check zeroing/environmental/LAC, re-engage
    \item �� Target suppressed? → Maintain track, ready to re-engage
\end{itemize}

\subsubsection{STEP 9: Post-Engagement Actions}

\begin{enumerate}
    \item If target neutralized: \textbf{Stop tracking} (double-click \button{Button 4})
    \item Report engagement results to command
    \item Update ammunition count
    \item Scan for additional targets
    \item Resume surveillance or follow orders
\end{enumerate}

% ============================================================================
\section{ENGAGEMENT BEST PRACTICES}

\subsection{TARGET SELECTION FOR TRACKING}

\subsubsection{Good Targets}
\begin{itemize}
    \item ✅ High contrast against background
    \item ✅ Clearly defined edges
    \item ✅ Sufficient size (>30 pixels)
    \item ✅ Relatively stable motion
\end{itemize}

\subsubsection{Difficult Targets}
\begin{itemize}
    \item ⚠️ Low contrast (camouflaged)
    \item ⚠️ Very small (distant)
    \item ⚠️ Erratic motion (evasive)
    \item ⚠️ Partially obscured
\end{itemize}

\subsubsection{If Tracking Fails}
\begin{itemize}
    \item Try manual engagement (no tracking)
    \item Improve contrast (switch camera or thermal LUT)
    \item Wait for better tracking opportunity
\end{itemize}

\subsection{LEAD ANGLE COMPENSATION TIPS}

\subsubsection{When to Use LAC}
\begin{itemize}
    \item ✅ Target moving laterally (crossing FOV)
    \item ✅ Target speed >5 m/s (~10 mph)
    \item ✅ Range >100 meters
\end{itemize}

\subsubsection{When NOT Needed}
\begin{itemize}
    \item ❌ Stationary targets
    \item ❌ Targets moving radially (toward/away from you)
    \item ❌ Very close range (<50m)
\end{itemize}

\subsubsection{LAC Limitations}
\begin{itemize}
    \item Requires tracking active
    \item Requires sufficient FOV (may need to zoom out)
    \item Assumes constant target velocity (less accurate if maneuvering)
\end{itemize}

\subsection{AMMUNITION CONSERVATION}

\subsubsection{Fire Discipline}
\begin{itemize}
    \item Use controlled bursts (2-5 rounds) vs. full-auto spray
    \item Assess after each burst before re-engaging
    \item Precision over volume
\end{itemize}

\subsubsection{Round Count}
\begin{itemize}
    \item Track ammunition expenditure
    \item Report when low (<20\% remaining)
    \item Conserve for high-priority targets
\end{itemize}

\subsection{SAFETY REMINDERS}

\subsubsection{ALWAYS}
\begin{itemize}
    \item ✅ Verify target before engaging (PID mandatory)
    \item ✅ Check for friendly forces
    \item ✅ Verify NOT in no-fire zone
    \item ✅ Follow Rules of Engagement (ROE)
    \item ✅ Have fire authorization (if required)
\end{itemize}

\subsubsection{NEVER}
\begin{itemize}
    \item ❌ Fire without positive identification
    \item ❌ Fire into no-fire zones
    \item ❌ Fire if friendlies potentially in line of fire
    \item ❌ Fire without authorization (if required)
    \item ❌ Assume tracking is infallible (monitor track quality)
\end{itemize}

% ============================================================================
\section{TRACKING QUICK REFERENCE}

\begin{quickref}
\small
\begin{longtable}{|L{0.25\textwidth}|L{0.35\textwidth}|L{0.25\textwidth}|}
\hline
\rowcolor{milblue!20}
\textbf{Situation} & \textbf{Action} & \textbf{Button/Control} \\
\hline
\endfirsthead

\hline
\rowcolor{milblue!20}
\textbf{Situation} & \textbf{Action} & \textbf{Button/Control} \\
\hline
\endhead

Start tracking & Enter acquisition mode & \button{Button 4} (1st press) \\
\hline
Size gate larger & Increase dimensions & \button{D-Pad ▼} (height) / \button{►} (width) \\
\hline
Size gate smaller & Decrease dimensions & \button{D-Pad ▲} (height) / \button{◄} (width) \\
\hline
Request lock-on & Lock onto target & \button{Button 4} (2nd press) \\
\hline
Monitor track & Observe gate color \& confidence & Visual (OSD) \\
\hline
Track degrading & Prepare for coast or abort & Stand by \button{Button 4} \\
\hline
Emergency abort & Stop tracking immediately & \button{Button 4} (double-click <500ms) \\
\hline
Enable LAC & Activate lead compensation & \button{Button 3 + Button 2} \\
\hline
Arm weapon & Engage Master Arm & \button{Button 0} (or trigger stage 1) \\
\hline
Fire weapon & Discharge weapon & \button{Button 5} (or trigger stage 2) \\
\hline
Cease fire & Safe weapon & Release \button{Button 5 \& Button 0} \\
\hline
\end{longtable}
\end{quickref}

% ============================================================================
% END OF LESSON 5
% ============================================================================
% ============================================================================
% LESSON 6 - BALLISTICS & FIRE CONTROL
% ============================================================================
\lesson{6 - BALLISTICS \& FIRE CONTROL}

\begin{tabular}{@{}ll@{}}
\textbf{Duration:} & 5 hours (Classroom 2h + Practical 3h) \\
\textbf{Type:} & Classroom + Practical \\
\textbf{References:} & Ballistics manual, environmental procedures, LAC guide \\
\end{tabular}

\section{LEARNING OBJECTIVES}

Upon completion of this lesson, operators will be able to:
\begin{itemize}
    \item Perform weapon zeroing (boresight alignment)
    \item Configure environmental parameters for ballistic corrections
    \item Activate and employ Lead Angle Compensation (LAC) for moving targets
    \item Interpret fire control status indicators
    \item Combine zeroing, environmental settings, and LAC for accurate engagements
\end{itemize}

% ============================================================================
\section{WEAPON ZEROING (BORESIGHT ALIGNMENT)}

\subsection{THE BORESIGHT OFFSET PROBLEM}

\subsubsection{Physical Reality}
\begin{itemize}
    \item Camera and weapon barrel are physically separated (typically 15-30cm)
    \item Camera points at reticle center
    \item Weapon points at different location
    \item \textbf{Without correction:} Weapon impacts below/beside reticle aim point
\end{itemize}

\subsubsection{Diagram}

\begin{center}
\begin{tikzpicture}
    \node[draw, circle, minimum size=0.5cm] (reticle) at (0,0) {⊙};
    \node[left of=reticle, node distance=2cm] (camera) {Camera →};
    \node[below of=reticle, node distance=1.5cm] (impact) {•};
    \node[left of=impact, node distance=2cm] (weapon) {Weapon →};
    
    \draw[->, thick] (camera) -- (reticle);
    \draw[->, thick, dashed] (weapon) -- (impact);
    \draw[<->, red] (reticle) -- node[right] {offset} (impact);
    
    \node[right of=reticle, node distance=2.5cm, align=left] {reticle center};
    \node[right of=impact, node distance=2.5cm, align=left] {actual impact point};
\end{tikzpicture}
\end{center}

\subsubsection{Zeroing Solution}
\begin{itemize}
    \item Applies angular offsets to compensate for physical separation
    \item \textbf{Azimuth Offset:} Horizontal correction (left/right)
    \item \textbf{Elevation Offset:} Vertical correction (up/down)
    \item \textbf{After Zeroing:} Reticle shows where weapon will actually hit
\end{itemize}

% ----------------------------------------------------------------------------
\subsection{ZEROING PROCEDURE}

\subsubsection{Pre-Zeroing Requirements}

\textbf{Environmental Conditions:}
\begin{checklist}
    \item Calm wind (<5 knots)
    \item Good visibility (daylight preferred)
    \item Stable platform (vehicle stationary)
    \item Temperature moderate
\end{checklist}

\textbf{Range Setup:}
\begin{checklist}
    \item Known range to target (recommended 100-300m)
    \item Fixed target (large, visible, safe backstop)
    \item No obstructions
\end{checklist}

\textbf{System Status:}
\begin{checklist}
    \item Station powered and initialized
    \item Camera operational (day camera for initial zero)
    \item Weapon loaded and ready
    \item Manual mode (no active tracking or motion modes)
\end{checklist}

% ----------------------------------------------------------------------------
\subsubsection{COMPLETE ZEROING PROCEDURE (STEP-BY-STEP)}

\begin{procedurebox}[STEP 1: Access Zeroing Menu]
\begin{enumerate}
    \item Press \button{MENU ✓} on Control Panel
    \item Navigate to \textbf{"Zeroing"} option (\button{▲/▼})
    \item Press \button{VAL} to enter
    \item Zeroing screen appears with instructions
\end{enumerate}
\end{procedurebox}

\begin{procedurebox}[STEP 2: Aim at Target]
\begin{enumerate}
    \item Use joystick to center reticle on target center
    \item Ensure stable aim
    \item Target should be large (minimum 30cm × 30cm)
    \item Target at known range (100-300m recommended)
\end{enumerate}
\end{procedurebox}

\begin{procedurebox}[STEP 3: Fire Test Shot(s)]
\begin{enumerate}
    \item Hold \button{Button 0} (Master Arm)
    \item Press \button{Button 5} (Fire) - single shot or short burst
    \item Observe impact point on target
    \item Note offset from reticle center (direction and distance)
\end{enumerate}

\textbf{Example:}
\begin{itemize}
    \item Reticle centered on target bullseye
    \item Impact observed 20cm low and 10cm right
    \item This offset will be corrected in next steps
\end{itemize}
\end{procedurebox}

\begin{procedurebox}[STEP 4: Adjust Reticle to Impact]
\textbf{OSD displays:} \osd{Use JOYSTICK to move main RETICLE to ACTUAL IMPACT POINT}

\begin{enumerate}
    \item Use joystick to move reticle from target center to \textbf{actual impact location}
    \item Position reticle exactly where weapon hit
    \item This movement is captured as zeroing offset
    \item Do NOT fire again during this step
\end{enumerate}

\textbf{Example Continued:}
\begin{itemize}
    \item Move joystick down and right
    \item Reticle now positioned at impact point (20cm low, 10cm right of center)
    \item System records this offset
\end{itemize}
\end{procedurebox}

\begin{procedurebox}[STEP 5: Apply Zero]
\begin{enumerate}
    \item Press \button{MENU ✓ / VAL} button to apply
    \item System calculates offsets (Azimuth, Elevation)
    \item Completion screen shows: \osd{Zeroing Adjustment Applied!}
    \item Displays final offsets (e.g., "Az: 0.85\degree, El: -1.23\degree")
    \item \textbf{"Z"} indicator appears on OSD (confirms zero active)
\end{enumerate}
\end{procedurebox}

\begin{procedurebox}[STEP 6: Verify Zero]
\begin{enumerate}
    \item \textbf{Return reticle to target center} (joystick)
    \item Fire verification shot
    \item Impact should now match reticle center
    \item \textbf{If offset remains:} Repeat procedure
    \item \textbf{If accurate:} Zero complete
\end{enumerate}
\end{procedurebox}

\begin{notebox}
Zero values saved to configuration file, loaded automatically on startup.
\end{notebox}

% ----------------------------------------------------------------------------
\subsection{MULTI-RANGE ZERO VALIDATION}

\subsubsection{Why Multiple Ranges Matter}
\begin{itemize}
    \item Ballistic arc is curved, not straight
    \item Zero at 100m may not be accurate at 500m
    \item Single-range zero is a compromise
\end{itemize}

\textbf{Recommended Zero Range:} \textbf{200-250m} (best for general-purpose zero, acceptable accuracy 50m-600m)

\subsubsection{MULTI-RANGE VALIDATION PROCEDURE}

After initial zero at 200m, test at multiple ranges:

\spectable{
\begin{tabular}{|L{0.15\textwidth}|L{0.50\textwidth}|L{0.25\textwidth}|}
\hline
\rowcolor{milblue!20}
\textbf{Range} & \textbf{Procedure} & \textbf{Acceptable Error} \\
\hline
\textbf{100m} & Aim at center, fire test shot & ±5cm \\
\hline
\textbf{300m} & Aim at center, fire test shot & ±10cm \\
\hline
\textbf{500m} & Aim at center, fire test shot & ±20cm (may be low due to drop) \\
\hline
\end{tabular}
}

\textbf{If Errors Exceed Acceptable:}
\begin{itemize}
    \item Re-zero at primary engagement range (e.g., 250m)
    \item Compromise between short and long range accuracy
\end{itemize}

% ----------------------------------------------------------------------------
\subsection{CLEARING ACTIVE ZERO}

\subsubsection{When to Clear Zero}
\begin{itemize}
    \item Weapon or camera repositioned (maintenance, replacement)
    \item Zero no longer accurate (tested and failed verification)
    \item Different weapon or ammunition type installed
\end{itemize}

\begin{procedurebox}[CLEARING ZERO PROCEDURE]
\begin{enumerate}
    \item \button{MENU ✓} → Zeroing → \textbf{Clear Zero}
    \item Confirm: \osd{CLEAR ACTIVE ZERO?}
    \item \button{MENU ✓} → \osd{YES}
    \item OSD \osd{Z} indicator disappears
    \item System returns to factory boresight (no offset)
    \item Must perform new zeroing procedure before live fire
\end{enumerate}
\end{procedurebox}

% ============================================================================
\section{ENVIRONMENTAL PARAMETERS}

\subsection{ENVIRONMENTAL EFFECTS ON BALLISTICS}

Ballistic trajectory is affected by environmental conditions:

\spectable{
\begin{tabular}{|L{0.18\textwidth}|L{0.35\textwidth}|L{0.35\textwidth}|}
\hline
\rowcolor{milblue!20}
\textbf{Parameter} & \textbf{Effect on Trajectory} & \textbf{System Correction} \\
\hline
\textbf{Temperature} & Air density changes → drag changes & Ballistic LUT adjustment \\
\hline
\textbf{Altitude} & Lower air pressure → less drag & LUT adjustment for altitude \\
\hline
\textbf{Crosswind} & Lateral projectile drift & Azimuth offset correction \\
\hline
\end{tabular}
}

\subsubsection{System Fields (SystemStateData)}
\begin{itemize}
    \item \texttt{environmentalTemperatureCelsius} (default: 15\degree C)
    \item \texttt{environmentalAltitudeMeters} (default: 0m / sea level)
    \item \texttt{environmentalCrosswindMS} (meters per second)
    \item \texttt{environmentalAppliedToBallistics} (boolean flag)
\end{itemize}

% ----------------------------------------------------------------------------
\subsection{ENVIRONMENTAL PARAMETERS SETUP}

\subsubsection{ACCESSING ENVIRONMENTAL MENU}

\begin{enumerate}
    \item Press \button{MENU ✓}
    \item Navigate to \textbf{"Environmental Parameters"} (\button{▲/▼})
    \item Press \button{VAL} to enter
    \item Environmental configuration screen appears
\end{enumerate}

% ----------------------------------------------------------------------------
\subsubsection{SETTING TEMPERATURE}

\textbf{Purpose:} Correct for air density changes due to temperature

\begin{procedurebox}[TEMPERATURE CONFIGURATION]
\begin{enumerate}
    \item Select \textbf{"Temperature"} option
    \item Use \button{▲/▼} to adjust value (range: -40\degree C to +60\degree C)
    \item Set to \textbf{current ambient temperature}
    \item Press \button{VAL} to confirm
\end{enumerate}
\end{procedurebox}

\textbf{Temperature Guidelines:}

\spectable{
\begin{tabular}{|L{0.15\textwidth}|L{0.15\textwidth}|L{0.25\textwidth}|L{0.30\textwidth}|}
\hline
\rowcolor{milblue!20}
\textbf{Condition} & \textbf{Temp.} & \textbf{Air Density} & \textbf{Trajectory Effect} \\
\hline
\textbf{Cold} & <0\degree C & High (dense air) & More drag, shorter range \\
\hline
\textbf{Standard} & 15\degree C & Standard (ISO) & No correction \\
\hline
\textbf{Hot} & >30\degree C & Low (thin air) & Less drag, longer range \\
\hline
\end{tabular}
}

\begin{notebox}
Temperature measured at \textbf{shooter location}, not target location.
\end{notebox}

% ----------------------------------------------------------------------------
\subsubsection{SETTING ALTITUDE}

\textbf{Purpose:} Correct for air pressure changes with elevation

\begin{procedurebox}[ALTITUDE CONFIGURATION]
\begin{enumerate}
    \item Select \textbf{"Altitude"} option
    \item Use \button{▲/▼} to adjust value (range: -500m to +4000m)
    \item Set to \textbf{current altitude above sea level} (meters)
    \item Press \button{VAL} to confirm
\end{enumerate}
\end{procedurebox}

\textbf{Altitude Guidelines:}

\spectable{
\begin{tabular}{|L{0.22\textwidth}|L{0.25\textwidth}|L{0.40\textwidth}|}
\hline
\rowcolor{milblue!20}
\textbf{Altitude} & \textbf{Air Pressure} & \textbf{Trajectory Effect} \\
\hline
\textbf{Sea Level (0m)} & Standard (101.3 kPa) & No correction \\
\hline
\textbf{1000m} & Lower (~90 kPa) & Less drag, longer range, less drop \\
\hline
\textbf{2000m+} & Much lower & Significantly less drag, flatter trajectory \\
\hline
\end{tabular}
}

\begin{notebox}[TIP]
Use GPS, map, or barometric altimeter to determine altitude.
\end{notebox}

% ----------------------------------------------------------------------------
\subsubsection{SETTING CROSSWIND}

\textbf{Purpose:} Correct for lateral projectile drift due to wind

\begin{procedurebox}[CROSSWIND CONFIGURATION]
\begin{enumerate}
    \item Select \textbf{"Crosswind"} option
    \item Use \button{▲/▼} to adjust value (range: 0-25 m/s)
    \item Set to \textbf{estimated crosswind speed} (perpendicular to fire direction)
    \item Press \button{VAL} to confirm
\end{enumerate}
\end{procedurebox}

\textbf{Crosswind Assessment:}

\spectable{
\small
\begin{tabular}{|C{0.15\textwidth}|C{0.15\textwidth}|L{0.55\textwidth}|}
\hline
\rowcolor{milblue!20}
\textbf{Wind (m/s)} & \textbf{Wind (knots)} & \textbf{Visual Cues} \\
\hline
\textbf{0-2} & 0-4 & Calm, smoke rises vertically \\
\hline
\textbf{3-5} & 6-10 & Light breeze, leaves rustle, flags extended \\
\hline
\textbf{6-10} & 12-20 & Moderate wind, small branches move, dust raised \\
\hline
\textbf{11-15} & 21-30 & Fresh wind, small trees sway \\
\hline
\textbf{16+} & 31+ & Strong wind, large branches move \\
\hline
\end{tabular}
}

\subsubsection{Crosswind Direction}
\begin{itemize}
    \item \textbf{Full crosswind:} Wind perpendicular to fire direction (maximum effect)
    \item \textbf{Partial crosswind:} Wind at angle to fire direction (reduced effect)
    \item \textbf{Headwind/Tailwind:} Wind along fire direction (minimal lateral drift, affects range only)
\end{itemize}

\textbf{System Assumption:} Crosswind value entered represents \textbf{full crosswind component} (perpendicular). Operator should estimate effective crosswind speed accounting for wind angle.

% ----------------------------------------------------------------------------
\subsubsection{APPLYING ENVIRONMENTAL SETTINGS}

\begin{procedurebox}[APPLYING SETTINGS]
\textbf{STEP 1: Configure All Parameters}
\begin{itemize}
    \item Temperature set
    \item Altitude set
    \item Crosswind set
\end{itemize}

\textbf{STEP 2: Apply to Ballistics}
\begin{enumerate}
    \item Select \textbf{"Apply Environmental Settings"}
    \item Confirm: \osd{APPLY TO BALLISTICS?}
    \item Press \button{VAL} → \osd{YES}
    \item OSD displays: \osd{ENV} indicator (confirms environmental corrections active)
\end{enumerate}

\textbf{STEP 3: Verify Active}
\begin{itemize}
    \item Check OSD for \osd{ENV} indicator
    \item Environmental corrections now applied to CCIP reticle
    \item All subsequent shots use environmental-corrected ballistics
\end{itemize}
\end{procedurebox}

% ----------------------------------------------------------------------------
\subsubsection{CLEARING ENVIRONMENTAL SETTINGS}

\textbf{When to Clear:}
\begin{itemize}
    \item Environmental conditions changed significantly
    \item Moving to different location (altitude, temperature change)
    \item Wind conditions changed
    \item Returning to standard conditions
\end{itemize}

\begin{procedurebox}[CLEARING ENVIRONMENTAL SETTINGS]
\begin{enumerate}
    \item \button{MENU ✓} → Environmental Parameters → \textbf{Clear Settings}
    \item Confirm: \osd{CLEAR ENVIRONMENTAL SETTINGS?}
    \item \button{MENU ✓} → \osd{YES}
    \item OSD \osd{ENV} indicator disappears
    \item System returns to standard conditions (15\degree C, 0m altitude, 0 m/s wind)
\end{enumerate}
\end{procedurebox}

% ----------------------------------------------------------------------------
\subsection{ENVIRONMENTAL PARAMETERS QUICK REFERENCE}

\spectable{
\small
\begin{tabular}{|L{0.20\textwidth}|C{0.12\textwidth}|C{0.12\textwidth}|C{0.12\textwidth}|C{0.10\textwidth}|}
\hline
\rowcolor{milblue!20}
\textbf{Situation} & \textbf{Temp.} & \textbf{Alt.} & \textbf{Wind} & \textbf{Apply?} \\
\hline
Sea level, standard day, calm & 15\degree C & 0m & 0 m/s & No (default) \\
\hline
Desert, hot, calm & 40\degree C & 0m & 0 m/s & Yes (temp) \\
\hline
Mountain, cold, windy & -10\degree C & 2000m & 8 m/s & Yes (all) \\
\hline
Temperate, mild, light breeze & 20\degree C & 300m & 3 m/s & Optional \\
\hline
\end{tabular}
}

% ============================================================================
\section{LEAD ANGLE COMPENSATION (LAC)}

\begin{warningbox}[CRITICAL - JOYSTICK ONLY]
Lead Angle Compensation (LAC) can \textbf{ONLY} be toggled using joystick \textbf{Button 2} while holding the Dead Man Switch (Button 3). There is \textbf{NO menu access} for LAC activation. The menu only displays LAC status.
\end{warningbox}

\subsection{THE MOVING TARGET PROBLEM}

\subsubsection{Without Lead Compensation}

\begin{center}
\begin{verbatim}
Time T=0 (Fire)          Time T=TOF (Impact)

Target: [X]              Target: -----> [X]
         ↓
Bullet:  •               Bullet:         •

Result: MISS (bullet hits where target WAS)
\end{verbatim}
\end{center}

\subsubsection{With Lead Compensation}

\begin{center}
\begin{verbatim}
Time T=0 (Fire)          Time T=TOF (Impact)

Target: [X]              Target: -----> [X]
         ↓ (aim ahead)                   ↓
Bullet:  →→→ •           Bullet:         •

Result: HIT (bullet meets target at predicted position)
\end{verbatim}
\end{center}

% ----------------------------------------------------------------------------
\subsection{LEAD ANGLE FUNDAMENTALS}

\textbf{Lead Angle:} Angular offset between target's current position and predicted intercept point

\subsubsection{Factors Affecting Lead Angle}
\begin{enumerate}
    \item \textbf{Target Velocity:} Faster target = more lead
    \item \textbf{Target Direction:} Crossing target = maximum lead, approaching/receding = minimal
    \item \textbf{Range to Target:} Greater range = longer TOF = more lead
    \item \textbf{Projectile Velocity:} Slower projectile = longer TOF = more lead
    \item \textbf{Target Angular Rate:} How fast target crosses FOV
\end{enumerate}

\subsubsection{System Calculation Process (30 Hz update rate)}
\begin{enumerate}
    \item Measure target motion (tracking system provides angular rates)
    \item Determine range (LRF)
    \item Calculate Time-of-Flight (TOF)
    \item Predict target position (current position + angular rate × TOF)
    \item Calculate lead angle
    \item Apply offset to CCIP reticle
\end{enumerate}

% ----------------------------------------------------------------------------
\subsection{LAC ACTIVATION REQUIREMENTS}

\subsubsection{Prerequisites}
\begin{checklist}
    \item Active target track established (Tracking Phase = Active Lock)
    \item Valid range data from LRF
    \item Target exhibiting motion (angular rate > threshold)
    \item Sufficient camera FOV (not zoomed in excessively)
    \item System initialized and operational
\end{checklist}

\subsubsection{Safety Interlock}
\begin{itemize}
    \item \textbf{Dead Man Switch (\button{Button 3}) MUST be held} during activation
\end{itemize}

% ----------------------------------------------------------------------------
\subsection{LAC ACTIVATION PROCEDURE}

\subsubsection{PRE-CONDITIONS}
\begin{checklist}
    \item Target tracked (Active Lock achieved)
    \item Target moving at measurable velocity
    \item Range data valid (LRF fired successfully)
    \item Camera FOV adequate (avoid max zoom)
\end{checklist}

\begin{procedurebox}[LAC ACTIVATION]
\textbf{ACTIVATION:}
\begin{enumerate}
    \item Hold \button{Button 3} (Dead Man Switch)
    \item Press \button{Button 2} (LAC Toggle)
    \item Release Button 2
    \item Release Button 3
\end{enumerate}

\textbf{VERIFICATION:}
\begin{checklist}
    \item \osd{LAC: ON} indicator appears (GREEN)
    \item CCIP reticle shifts to lead position (ahead of target)
    \item Track confidence remains >70\%
\end{checklist}
\end{procedurebox}

\subsubsection{El-7aress-Compliant LAC Latching Behavior}

Per El-7aress H100 specification, LAC follows specific latching rules:

\begin{warningbox}[2-SECOND MINIMUM INTERVAL]
"A minimum of 2 seconds must be waited before reuse of lead angle compensation feature."
\end{warningbox}

\textbf{LAC Latching Behavior:}
\begin{itemize}
    \item When LAC is toggled ON, system \textbf{latches} current target tracking rate
    \item Lead calculation uses latched rate, not continuously updated rate
    \item If operator toggles LAC OFF then ON again, must wait \textbf{2 seconds minimum}
    \item System will \textbf{block} LAC toggle if attempted too soon
\end{itemize}

\textbf{Warning - Target Switching:}
\begin{warningbox}
"If target \#2 is not properly acquired, the WS will fire outside the desired engagement area by continuing to apply the lead angle acquired from target \#1."
\end{warningbox}

\textbf{Operator Procedure for Target Switch:}
\begin{enumerate}
    \item Toggle LAC OFF (Button 2 while holding Button 3)
    \item Acquire new target (tracking lock on target \#2)
    \item Wait minimum 2 seconds
    \item Toggle LAC ON for new target
    \item Verify \osd{LAC: ON} before firing
\end{enumerate}

% ----------------------------------------------------------------------------
\subsection{LAC STATUS INDICATORS}

The OSD displays LAC status in the following formats:

\begin{lstlisting}
LAC: OFF       (gray)   - LAC disabled
LAC: ON        (green)  - LAC active and functioning
LAC: LAG       (yellow) - Insufficient tracking data, wait 2-5 sec
LAC: ZOOM OUT  (red)    - FOV too narrow, zoom out required
\end{lstlisting}

\subsubsection{STATUS: "LAC: ON" (GREEN)}

\textbf{Meaning:} LAC active and functioning correctly

\textbf{Display:}
\begin{itemize}
    \item OSD shows: \osd{LAC: ON} in green
    \item CCIP reticle offset ahead of target
    \item Lead offset calculated and applied
\end{itemize}

\textbf{Operator Action:}
\begin{itemize}
    \item Aim at CCIP reticle (not at target directly)
    \item CCIP shows where to aim for predicted intercept
    \item Fire when CCIP on target (system handles lead)
\end{itemize}

% ----------------------------------------------------------------------------
\subsubsection{STATUS: "LAC: LAG" (YELLOW)}

\textbf{Meaning:} Insufficient tracking data for accurate lead calculation

\textbf{Causes:}
\begin{itemize}
    \item Track recently established (<2 seconds)
    \item Target velocity not yet stable
    \item Tracking confidence fluctuating
\end{itemize}

\textbf{Display:}
\begin{itemize}
    \item OSD shows: \osd{LAC: LAG} in YELLOW
    \item Reticle may show partial lead offset (unreliable)
\end{itemize}

\textbf{Operator Action:}
\begin{itemize}
    \item \textbf{WAIT} (2-5 seconds) for tracking to stabilize
    \item Do NOT fire until status changes to \osd{LAC: ON} (GREEN)
    \item If LAG persists >10 seconds: Check tracking quality, verify target moving
\end{itemize}

% ----------------------------------------------------------------------------
\subsubsection{STATUS: "LAC: ZOOM OUT" (RED)}

\textbf{Meaning:} Camera FOV too narrow for accurate lead calculation

\textbf{Causes:}
\begin{itemize}
    \item Zoomed in too far (max zoom or near-max zoom)
    \item Target angular rate too high for current FOV
    \item System cannot measure target velocity accurately
\end{itemize}

\textbf{Display:}
\begin{itemize}
    \item OSD shows: \osd{LAC: ZOOM OUT} in RED
    \item LAC non-functional (lead calculation disabled)
\end{itemize}

\textbf{Operator Action:}
\begin{enumerate}
    \item \textbf{Zoom out} (\button{Button 8}) gradually
    \item Wait for status to change to \osd{LAC: ON} (GREEN)
    \item If problem persists: Zoom out more
    \item Acceptable FOV: Typically Wide or Mid zoom (not Tele/Max)
\end{enumerate}

\begin{notebox}
Extremely fast-moving targets may require wide FOV even at close range.
\end{notebox}

% ----------------------------------------------------------------------------
\subsection{LAC DEACTIVATION PROCEDURE}

\subsubsection{When to Deactivate}
\begin{itemize}
    \item Target stops moving
    \item Lost track of target
    \item "ZOOM OUT" warning persists
    \item "LEAD ANGLE LAG" persists
    \item Engagement complete
    \item Switching to stationary target
\end{itemize}

\begin{procedurebox}[LAC DEACTIVATION]
\textbf{DEACTIVATION:}
\begin{enumerate}
    \item Hold \button{Button 3} (Dead Man Switch)
    \item Press \button{Button 2} (LAC Toggle)
    \item Release Button 2
    \item Release Button 3
\end{enumerate}

\textbf{VERIFICATION:}
\begin{checklist}
    \item \osd{LEAD ANGLE ON} indicator disappears
    \item \osd{LAC} bracket removed from CCIP reticle
    \item Reticle pipper returns to boresight alignment
    \item System ready for stationary target engagement
\end{checklist}
\end{procedurebox}

% ----------------------------------------------------------------------------
\subsection{USING LAC IN ENGAGEMENT}

\subsubsection{MOVING TARGET ENGAGEMENT (WITH LAC)}

\begin{procedurebox}[COMPLETE PROCEDURE]
\textbf{1. Acquire and Track Target} (Lesson 5)
\begin{itemize}
    \item Enter acquisition mode (\button{Button 4})
    \item Size gate, request lock-on
    \item Achieve Active Lock (green gate, tracking)
\end{itemize}

\textbf{2. Range Target}
\begin{itemize}
    \item Fire LRF (or automatic ranging during tracking)
    \item Verify range displayed (OSD: \osd{RNG: xxxx m})
\end{itemize}

\textbf{3. Enable LAC}
\begin{itemize}
    \item Hold \button{Button 3} + Press \button{Button 2}
    \item Verify \osd{LEAD ANGLE ON} (GREEN)
    \item Wait if \osd{LEAD ANGLE LAG} (YELLOW)
    \item Zoom out if \osd{ZOOM OUT} (RED)
\end{itemize}

\textbf{4. Observe Lead Offset}
\begin{itemize}
    \item CCIP reticle offset ahead of target in direction of motion
    \item Lead offset adjusts continuously as target moves
    \item \textbf{Aim at CCIP reticle} (not at target center)
\end{itemize}

\textbf{5. Fire}
\begin{itemize}
    \item Master Arm (\button{Button 0})
    \item Fire (\button{Button 5})
    \item Rounds impact at predicted intercept point
\end{itemize}

\textbf{6. Observe Effect}
\begin{itemize}
    \item Rounds should hit target (not behind target)
    \item If missing: Check tracking quality, verify LAC ON, verify range accurate
\end{itemize}
\end{procedurebox}

% ----------------------------------------------------------------------------
\subsection{LAC LIMITATIONS}

\subsubsection{When LAC Works Best}
\begin{itemize}
    \item ✅ Target moving laterally (crossing FOV)
    \item ✅ Target speed >5 m/s (~10 mph)
    \item ✅ Range >100 meters
    \item ✅ Constant target velocity
\end{itemize}

\subsubsection{When LAC is NOT Needed}
\begin{itemize}
    \item ❌ Stationary targets (LAC shows no offset)
    \item ❌ Targets moving radially (toward/away) - minimal lateral lead
    \item ❌ Very close range (<50m) - TOF too short for significant lead
\end{itemize}

\subsubsection{LAC Limitations}
\begin{itemize}
    \item Requires active tracking (cannot use LAC without track)
    \item Requires sufficient FOV (may need to zoom out)
    \item Assumes constant target velocity (less accurate if target maneuvering)
    \item Not effective for erratic or unpredictable motion
\end{itemize}

% ============================================================================
\section{TWO INDEPENDENT BALLISTIC CORRECTION SYSTEMS}

\begin{warningbox}[CRITICAL UNDERSTANDING]
The El 7arress RCWS employs \textbf{TWO INDEPENDENT} ballistic correction systems. Understanding the difference is essential for accurate fire.
\end{warningbox}

\subsection{SYSTEM 1: BALLISTIC DROP COMPENSATION (AUTOMATIC)}

\textbf{Activation:} \textbf{AUTOMATIC} when LRF has valid range data

\textbf{What It Corrects:}
\begin{itemize}
    \item \textbf{Gravity drop} - Bullet falls due to gravity over distance
    \item \textbf{Wind deflection} - Crosswind pushes bullet laterally
    \item \textbf{Environmental factors} - Temperature, altitude affect trajectory
\end{itemize}

\textbf{System Variables:}
\begin{itemize}
    \item \texttt{ballisticDropOffsetAz} - Azimuth correction (wind deflection)
    \item \texttt{ballisticDropOffsetEl} - Elevation correction (gravity drop)
    \item \texttt{ballisticDropActive} - True when valid LRF range exists
\end{itemize}

\textbf{Operator Action:} \textbf{NONE REQUIRED} - System applies corrections automatically when range is valid.

\textbf{OSD Indicator:} \osd{ENV} appears when environmental parameters are active.

% ----------------------------------------------------------------------------
\subsection{SYSTEM 2: MOTION LEAD COMPENSATION (MANUAL TOGGLE)}

\textbf{Activation:} \textbf{MANUAL} via joystick Button 2 (LAC toggle)

\textbf{What It Corrects:}
\begin{itemize}
    \item \textbf{Moving target lead} - Aim ahead of moving target
    \item Based on target angular velocity from tracker
    \item Based on range (LRF) and time-of-flight calculation
\end{itemize}

\textbf{System Variables:}
\begin{itemize}
    \item \texttt{motionLeadOffsetAz} - Azimuth lead offset for moving target
    \item \texttt{motionLeadOffsetEl} - Elevation lead offset for moving target
    \item \texttt{leadAngleCompensationActive} - True when operator toggles LAC ON
\end{itemize}

\textbf{Operator Action:} \textbf{REQUIRED} - Must press Button 2 to enable for moving targets.

\textbf{OSD Indicator:} \osd{LAC: ON} / \osd{LAC: LAG} / \osd{LAC: ZOOM OUT}

% ----------------------------------------------------------------------------
\subsection{COMBINED CCIP CALCULATION}

The CCIP (Continuously Computed Impact Point) pipper shows where the bullet will actually hit:

\begin{center}
\fbox{
\parbox{0.9\textwidth}{
\centering
\textbf{CCIP (*) = Gun Boresight + Zeroing Offset}\\
\textbf{+ Ballistic Drop Compensation (auto when range valid)}\\
\textbf{+ Motion Lead Compensation (only if Button 2 toggled ON)}
}
}
\end{center}

\begin{notebox}[KEY DISTINCTION]
\textbf{Ballistic Drop} is for \textbf{stationary} targets (compensates for gravity/wind).\\
\textbf{Motion Lead} is for \textbf{moving} targets (compensates for target velocity).\\
Both can be active simultaneously for moving targets at range with environmental factors.
\end{notebox}

% ============================================================================
\section{COMBINING FIRE CONTROL SYSTEMS}

\subsection{FIRE CONTROL SOLUTION HIERARCHY}

The complete fire control solution combines multiple corrections:

\begin{center}
\texttt{FINAL AIM POINT = Gun Boresight + Zeroing Offset +}\\
\texttt{Environmental Corrections + Lead Angle Offset}
\end{center}

\subsubsection{System Integration}
\begin{enumerate}
    \item \textbf{Gun Boresight:} Factory default (camera-to-weapon offset)
    \item \textbf{+ Zeroing Offset:} Operator-configured (corrects boresight error)
    \item \textbf{+ Environmental Corrections:} Ballistic LUT adjustments (temperature, altitude, wind)
    \item \textbf{+ Lead Angle Offset:} Real-time moving target compensation
\end{enumerate}

\subsubsection{OSD Indicators}
\begin{itemize}
    \item \osd{Z}: Zeroing active
    \item \osd{ENV}: Environmental parameters active
    \item \osd{LEAD ANGLE ON}: LAC active
\end{itemize}

\textbf{All Active Example:}
\begin{itemize}
    \item OSD displays: \osd{Z ENV LEAD ANGLE ON}
    \item CCIP reticle shows: Zeroing + Environmental + Lead corrections
    \item Most accurate engagement solution
\end{itemize}

% ----------------------------------------------------------------------------
\subsection{ENGAGEMENT SCENARIOS}

\subsubsection{SCENARIO 1: Stationary Target, Standard Conditions}

\textbf{Configuration:}
\begin{itemize}
    \item Zeroing: Active (Z)
    \item Environmental: Not needed (standard conditions)
    \item LAC: Not needed (stationary target)
\end{itemize}

\textbf{OSD:} \osd{Z}

\textbf{Fire Control:} Zeroing offset only

% ----------------------------------------------------------------------------
\subsubsection{SCENARIO 2: Stationary Target, Hot Desert, High Altitude}

\textbf{Configuration:}
\begin{itemize}
    \item Zeroing: Active (Z)
    \item Environmental: Active (Temp: 45\degree C, Alt: 1500m, Wind: 0 m/s)
    \item LAC: Not needed (stationary target)
\end{itemize}

\textbf{OSD:} \osd{Z ENV}

\textbf{Fire Control:} Zeroing + Environmental (temperature, altitude)

% ----------------------------------------------------------------------------
\subsubsection{SCENARIO 3: Moving Target, Windy Conditions, Mountain}

\textbf{Configuration:}
\begin{itemize}
    \item Zeroing: Active (Z)
    \item Environmental: Active (Temp: -5\degree C, Alt: 2500m, Wind: 10 m/s)
    \item LAC: Active (target moving 15 m/s lateral)
\end{itemize}

\textbf{OSD:} \osd{Z ENV LEAD ANGLE ON}

\textbf{Fire Control:} Zeroing + Environmental + Lead Angle (full solution)

% ----------------------------------------------------------------------------
\subsubsection{SCENARIO 4: Close Range, Moving Target, Standard Conditions}

\textbf{Configuration:}
\begin{itemize}
    \item Zeroing: Active (Z)
    \item Environmental: Not needed
    \item LAC: Active (target moving, but close range <100m may not require much lead)
\end{itemize}

\textbf{OSD:} \osd{Z LEAD ANGLE ON}

\textbf{Fire Control:} Zeroing + Lead Angle

% ----------------------------------------------------------------------------
\subsection{FIRE CONTROL QUICK REFERENCE}

\begin{longtable}{|L{0.13\textwidth}|C{0.08\textwidth}|L{0.15\textwidth}|C{0.08\textwidth}|C{0.10\textwidth}|C{0.06\textwidth}|L{0.18\textwidth}|}
\hline
\rowcolor{milblue!20}
\textbf{Target Type} & \textbf{Range} & \textbf{Conditions} & \textbf{Zero} & \textbf{Env.} & \textbf{LAC} & \textbf{OSD} \\
\hline
\endfirsthead

\hline
\rowcolor{milblue!20}
\textbf{Target Type} & \textbf{Range} & \textbf{Conditions} & \textbf{Zero} & \textbf{Env.} & \textbf{LAC} & \textbf{OSD} \\
\hline
\endhead

Stationary & Any & Standard & ✅ & ❌ & ❌ & Z \\
\hline
Stationary & Any & Extreme temp/alt & ✅ & ✅ & ❌ & Z ENV \\
\hline
Stationary & Any & Windy & ✅ & ✅ (wind) & ❌ & Z ENV \\
\hline
Moving (slow) & <100m & Standard & ✅ & ❌ & Optional & Z (LEAD ANGLE ON) \\
\hline
Moving (fast) & >100m & Standard & ✅ & ❌ & ✅ & Z LEAD ANGLE ON \\
\hline
Moving (fast) & >100m & Extreme & ✅ & ✅ & ✅ & Z ENV LEAD ANGLE ON \\
\hline
\end{longtable}

% ============================================================================
\section{FIRE CONTROL BEST PRACTICES}

\subsection{ZEROING}

\subsubsection{Best Practices}
\begin{itemize}
    \item ✅ Zero weapon before first operational use
    \item ✅ Verify zero periodically (weekly or after transport)
    \item ✅ Re-zero if weapon or camera serviced/replaced
    \item ✅ Use consistent ammunition type for zeroing and operations
    \item ✅ Zero at primary engagement range (200-250m recommended)
\end{itemize}

\subsubsection{Common Errors}
\begin{itemize}
    \item ❌ Skipping multi-range validation (zero may be inaccurate at long range)
    \item ❌ Moving gimbal between firing test shot and adjusting reticle (invalidates offset measurement)
    \item ❌ Assuming zero is permanent (mechanical drift can occur over time)
\end{itemize}

% ----------------------------------------------------------------------------
\subsection{ENVIRONMENTAL PARAMETERS}

\subsubsection{Best Practices}
\begin{itemize}
    \item ✅ Update environmental settings at mission start
    \item ✅ Re-assess if conditions change significantly during mission
    \item ✅ Use handheld weather instruments (thermometer, anemometer, altimeter) for accuracy
    \item ✅ Estimate conservatively (err toward standard conditions if unsure)
\end{itemize}

\subsubsection{Common Errors}
\begin{itemize}
    \item ❌ Using old/stale environmental data from previous mission
    \item ❌ Ignoring significant temperature or wind changes
    \item ❌ Over-correcting for minor environmental variations (<5\degree C temp change, <2 m/s wind)
\end{itemize}

% ----------------------------------------------------------------------------
\subsection{LEAD ANGLE COMPENSATION}

\subsubsection{Best Practices}
\begin{itemize}
    \item ✅ Only use LAC for targets moving >5 m/s at ranges >100m
    \item ✅ Wait for \osd{LEAD ANGLE ON} (GREEN) before firing (ignore "LAG" yellow)
    \item ✅ Zoom out if \osd{ZOOM OUT} (RED) warning appears
    \item ✅ Trust the system - aim at CCIP reticle, not at target
\end{itemize}

\subsubsection{Common Errors}
\begin{itemize}
    \item ❌ Activating LAC for stationary targets (adds unnecessary complexity)
    \item ❌ Firing while \osd{LEAD ANGLE LAG} (YELLOW) - lead calculation incomplete
    \item ❌ Ignoring \osd{ZOOM OUT} (RED) warning - LAC non-functional
    \item ❌ "Kentucky windage" (manually aiming off-target) while LAC active - double-correcting
\end{itemize}

% ----------------------------------------------------------------------------
\subsection{SAFETY REMINDERS}

\subsubsection{ALWAYS}
\begin{itemize}
    \item ✅ Verify zero before live fire operations
    \item ✅ Update environmental parameters for current conditions
    \item ✅ Check OSD indicators before firing (Z, ENV, LEAD ANGLE ON)
    \item ✅ Use Dead Man Switch when activating/deactivating LAC (safety interlock)
\end{itemize}

\subsubsection{NEVER}
\begin{itemize}
    \item ❌ Fire without valid zero (impacts will be off-target)
    \item ❌ Assume environmental corrections are active (verify "ENV" indicator)
    \item ❌ Fire with \osd{ZOOM OUT} (RED) warning (LAC non-functional)
    \item ❌ Bypass Dead Man Switch safety interlock
\end{itemize}

% ============================================================================
% END OF LESSON 6
% ============================================================================
% ============================================================================
% LESSON 7 - SYSTEM STATUS & MONITORING
% ============================================================================
\lesson{7 - SYSTEM STATUS \& MONITORING}

\begin{tabular}{@{}ll@{}}
\textbf{Duration:} & 2 hours \\
\textbf{Type:} & Classroom \\
\textbf{References:} & System status manual, device specifications \\
\end{tabular}

\section{LEARNING OBJECTIVES}

Upon completion of this lesson, operators will be able to:
\begin{itemize}
    \item Access System Status display
    \item Interpret device status indicators
    \item Identify fault conditions
    \item Determine when to escalate to maintenance
\end{itemize}

% ============================================================================
\section{ACCESSING SYSTEM STATUS}

\subsection{SYSTEM STATUS MENU ACCESS}

\begin{procedurebox}[ACCESSING SYSTEM STATUS]
\begin{enumerate}
    \item Press \button{MENU ✓}
    \item Navigate to \textbf{"System Status"}
    \item Press \button{VAL}
\end{enumerate}
\end{procedurebox}

\subsection{Display Sections}

The System Status display provides real-time monitoring of all critical subsystems:

\begin{itemize}
    \item Azimuth/Elevation Servos
    \item IMU (Inertial Measurement Unit)
    \item Laser Range Finder (LRF)
    \item Day/Night Cameras
    \item Control Panels
    \item Servo Actuator
    \item Homing Status
    \item Charging Status
    \item Motion Mode
    \item Alarms/Warnings
\end{itemize}

% ----------------------------------------------------------------------------
\subsection{HOMING SEQUENCE STATUS}

During system startup or when homing is requested, the OSD displays the current homing phase:

\begin{longtable}{|L{0.20\textwidth}|L{0.45\textwidth}|L{0.25\textwidth}|}
\hline
\rowcolor{milblue!20}
\textbf{Status} & \textbf{Meaning} & \textbf{Operator Action} \\
\hline
\endfirsthead

\hline
\rowcolor{milblue!20}
\textbf{Status} & \textbf{Meaning} & \textbf{Operator Action} \\
\hline
\endhead

\textbf{Idle} & No homing operation active & Normal ops \\
\hline
\textbf{Requested} & Homing command sent to servos & Wait \\
\hline
\textbf{In Progress} & Motors actively seeking home position & Wait, do not interrupt \\
\hline
\textbf{Completed} & Home position found successfully & System ready \\
\hline
\textbf{Failed} & Homing timeout or error occurred & May need to retry \\
\hline
\textbf{Aborted} & Homing cancelled (E-Stop or other) & Check safety, may retry \\
\hline
\end{longtable}

\begin{notebox}
Motion modes are blocked during homing. Wait for homing to complete before operating the gimbal.
\end{notebox}

% ----------------------------------------------------------------------------
\subsection{MOTION MODE STATUS}

The OSD displays the current motion mode. The following modes may be displayed:

\begin{longtable}{|L{0.18\textwidth}|L{0.50\textwidth}|L{0.22\textwidth}|}
\hline
\rowcolor{milblue!20}
\textbf{Mode} & \textbf{Description} & \textbf{Gimbal Control} \\
\hline
\endfirsthead

\hline
\rowcolor{milblue!20}
\textbf{Mode} & \textbf{Description} & \textbf{Gimbal Control} \\
\hline
\endhead

\textbf{Manual} & Operator joystick control with stabilization & Operator (joystick) \\
\hline
\textbf{AutoSectorScan} & Automated scanning between two azimuth points & Automatic \\
\hline
\textbf{TRPScan} & Sequential scan through Target Reference Points & Automatic \\
\hline
\textbf{RadarSlew} & Slewing to radar-designated target & Automatic \\
\hline
\textbf{Tracking} & Vision-based target tracking active & Automatic (tracker) \\
\hline
\textbf{FREE} & Gimbal brakes released, manual positioning & None (unpowered) \\
\hline
\end{longtable}

\begin{warningbox}[FREE MODE]
When the local FREE toggle switch is ON, motion mode displays as \textbf{FREE}. The gimbal brakes are released and no servo power is applied. The gimbal can be manually positioned by hand. Ensure the area is clear before physically moving the gimbal.
\end{warningbox}

% ============================================================================
\section{DEVICE STATUS REFERENCE}

\subsection{GIMBAL SERVOS (AZIMUTH \& ELEVATION)}

\begin{longtable}{|L{0.18\textwidth}|L{0.15\textwidth}|L{0.15\textwidth}|L{0.15\textwidth}|L{0.28\textwidth}|}
\hline
\rowcolor{milblue!20}
\textbf{Parameter} & \textbf{Normal} & \textbf{Warning} & \textbf{Fault} & \textbf{Action} \\
\hline
\endfirsthead

\hline
\rowcolor{milblue!20}
\textbf{Parameter} & \textbf{Normal} & \textbf{Warning} & \textbf{Fault} & \textbf{Action} \\
\hline
\endhead

\textbf{Connected} & ✓ & - & ✗ & Notify maintenance \\
\hline
\textbf{Torque} & 0-50\% & 50-80\% & >80\% sustained & Allow cooling, reduce motion \\
\hline
\textbf{Motor Temp} & 20-60\degree C & 60-70\degree C & >70\degree C & Halt operations, allow cooling \\
\hline
\textbf{Driver Temp} & 20-60\degree C & 60-70\degree C & >70\degree C & Halt operations, allow cooling \\
\hline
\textbf{Fault Flag} & No & - & Yes & E-Stop, notify maintenance \\
\hline
\end{longtable}

\subsubsection{Servo Monitoring Guidelines}

\textbf{Normal Operation:}
\begin{itemize}
    \item Torque values fluctuate with gimbal motion
    \item Higher torque during rapid movements or when carrying loads
    \item Temperatures rise gradually during extended operations
\end{itemize}

\textbf{Warning Signs:}
\begin{itemize}
    \item Sustained high torque (>50\%) during light movements
    \item Rapidly rising temperatures
    \item Unusual vibrations or noises
\end{itemize}

% ----------------------------------------------------------------------------
\subsection{IMU (INERTIAL MEASUREMENT UNIT)}

\begin{longtable}{|L{0.18\textwidth}|L{0.15\textwidth}|L{0.15\textwidth}|L{0.15\textwidth}|L{0.28\textwidth}|}
\hline
\rowcolor{milblue!20}
\textbf{Parameter} & \textbf{Normal} & \textbf{Warning} & \textbf{Fault} & \textbf{Action} \\
\hline
\endfirsthead

\hline
\rowcolor{milblue!20}
\textbf{Parameter} & \textbf{Normal} & \textbf{Warning} & \textbf{Fault} & \textbf{Action} \\
\hline
\endhead

\textbf{Connected} & ✓ & - & ✗ & Stabilization offline, notify maint \\
\hline
\textbf{Roll/Pitch} & ±30\degree & ±30-45\degree & >45\degree & Platform unstable, secure vessel \\
\hline
\textbf{Temperature} & 20-60\degree C & 60-70\degree C & >70\degree C & Allow cooling \\
\hline
\end{longtable}

\begin{warningbox}[IMU CRITICAL FUNCTION]
IMU offline = No stabilization. Manual mode only.
\end{warningbox}

\subsubsection{IMU Role in System}

The IMU provides:
\begin{itemize}
    \item Platform orientation data (roll, pitch, yaw)
    \item Angular rate measurements for stabilization
    \item Motion compensation for tracking
    \item Essential data for ballistic calculations
\end{itemize}

\textbf{Impact of IMU Failure:}
\begin{itemize}
    \item No gyrostabilization (image shake during platform motion)
    \item Degraded tracking performance
    \item Reduced ballistic accuracy
    \item Manual mode operations only
\end{itemize}

% ----------------------------------------------------------------------------
\subsection{LASER RANGE FINDER (LRF)}

\begin{longtable}{|L{0.18\textwidth}|L{0.15\textwidth}|L{0.15\textwidth}|L{0.15\textwidth}|L{0.28\textwidth}|}
\hline
\rowcolor{milblue!20}
\textbf{Parameter} & \textbf{Normal} & \textbf{Warning} & \textbf{Fault} & \textbf{Action} \\
\hline
\endfirsthead

\hline
\rowcolor{milblue!20}
\textbf{Parameter} & \textbf{Normal} & \textbf{Warning} & \textbf{Fault} & \textbf{Action} \\
\hline
\endhead

\textbf{Connected} & ✓ & - & ✗ & Manual range estimation, notify maint \\
\hline
\textbf{Temperature} & 20-50\degree C & 50-60\degree C & >60\degree C & Reduce firing rate, allow cooling \\
\hline
\textbf{Laser Count} & Incrementing & - & Not incrementing & LRF malfunction \\
\hline
\textbf{No Echo} & Occasional & Frequent & Always & Target too far, obscured, or LRF fault \\
\hline
\end{longtable}

\begin{warningbox}[LRF CRITICAL FUNCTION]
LRF offline = No ballistics, no LAC. Manual engagement only.
\end{warningbox}

\subsubsection{LRF Status Interpretation}

\textbf{Laser Count:}
\begin{itemize}
    \item Increments with each successful ranging attempt
    \item Should increment even if "NO ECHO" received
    \item If frozen: LRF communication failure
\end{itemize}

\textbf{"NO ECHO" Conditions:}
\begin{itemize}
    \item Target beyond maximum range (~5km for hard targets)
    \item Target obscured by smoke, fog, rain
    \item Target surface non-reflective (absorbs laser)
    \item Laser beam blocked by obstruction
    \item LRF receiver malfunction (if persistent)
\end{itemize}

% ----------------------------------------------------------------------------
\subsection{CAMERAS (DAY \& THERMAL)}

\begin{longtable}{|L{0.18\textwidth}|L{0.15\textwidth}|L{0.15\textwidth}|L{0.15\textwidth}|L{0.28\textwidth}|}
\hline
\rowcolor{milblue!20}
\textbf{Parameter} & \textbf{Normal} & \textbf{Warning} & \textbf{Fault} & \textbf{Action} \\
\hline
\endfirsthead

\hline
\rowcolor{milblue!20}
\textbf{Parameter} & \textbf{Normal} & \textbf{Warning} & \textbf{Fault} & \textbf{Action} \\
\hline
\endhead

\textbf{Connected} & ✓ & - & ✗ & Switch cameras if available \\
\hline
\textbf{Error Flag} & No & - & Yes & Switch cameras, notify maint \\
\hline
\textbf{Thermal FFC} & Occasional & Frequent (<5min) & Constant & Thermal camera fault \\
\hline
\textbf{Focus} & Clear & Slightly blurred & Very blurred & Autofocus off, manual focus \\
\hline
\end{longtable}

\begin{warningbox}[CAMERA CRITICAL FUNCTION]
Both cameras offline = Mission abort. Cannot engage without visual.
\end{warningbox}

\subsubsection{Camera Status Notes}

\textbf{Thermal FFC (Flat Field Correction):}
\begin{itemize}
    \item Normal: Occurs every 5-15 minutes
    \item Recalibrates thermal sensor for accurate temperature readings
    \item Brief image freeze (1-2 seconds) during FFC
    \item Frequent FFC (<5 min intervals): Sensor instability
\end{itemize}

\textbf{Focus Issues:}
\begin{itemize}
    \item Autofocus may struggle with low-contrast scenes
    \item Manual focus available via menu controls
    \item Persistent blur: Lens contamination or camera fault
\end{itemize}

% ----------------------------------------------------------------------------
\subsection{CONTROL PANELS}

\spectable{
\begin{tabular}{|L{0.15\textwidth}|L{0.28\textwidth}|L{0.25\textwidth}|L{0.23\textwidth}|}
\hline
\rowcolor{milblue!20}
\textbf{Device} & \textbf{Function} & \textbf{Fault Impact} & \textbf{Action} \\
\hline
\textbf{DCU} & Buttons, switches, lights & No local control, joystick only & Notify maint (ops continue) \\
\hline
\textbf{Joystick} & Gimbal control, tracking, fire & Mission abort if no backup & Emergency: Use DCU manual control \\
\hline
\end{tabular}
}

\subsubsection{Control Panel Redundancy}

\textbf{Primary Control: Joystick}
\begin{itemize}
    \item All operator functions accessible
    \item Preferred for normal operations
    \item Ergonomic, intuitive control
\end{itemize}

\textbf{Backup Control: DCU (Direct Control Unit)}
\begin{itemize}
    \item Limited functionality (basic gimbal control)
    \item Use if joystick fails
    \item Slower, less precise than joystick
\end{itemize}

% ============================================================================
\section{ALARM INTERPRETATION}

\subsection{CRITICAL ALARMS (RED)}

\begin{longtable}{|L{0.28\textwidth}|L{0.30\textwidth}|L{0.32\textwidth}|}
\hline
\rowcolor{milred!20}
\textbf{Alarm} & \textbf{Meaning} & \textbf{Immediate Action} \\
\hline
\endfirsthead

\hline
\rowcolor{milred!20}
\textbf{Alarm} & \textbf{Meaning} & \textbf{Immediate Action} \\
\hline
\endhead

\textbf{EMERGENCY STOP ACTIVE} & E-Stop engaged & Identify cause, reset when safe \\
\hline
\textbf{WEAPON ARMED (NO AUTH)} & Safety violation & Disarm immediately, investigate \\
\hline
\textbf{ZONE VIOLATION} & Entering No-Fire/No-Traverse & Halt motion, verify zone config \\
\hline
\textbf{SERVO FAULT} & Gimbal malfunction & E-Stop, notify maintenance \\
\hline
\end{longtable}

\subsubsection{Critical Alarm Response Priority}

\textbf{Priority 1: Immediate Safety Threats}
\begin{enumerate}
    \item \textbf{WEAPON ARMED (NO AUTH):} Disarm weapon immediately
    \item \textbf{ZONE VIOLATION:} Halt all motion, assess situation
    \item \textbf{SERVO FAULT:} E-Stop if gimbal behaving erratically
\end{enumerate}

\textbf{Priority 2: System Protection}
\begin{enumerate}
    \item \textbf{EMERGENCY STOP ACTIVE:} Determine if intentional or fault-triggered
\end{enumerate}

% ----------------------------------------------------------------------------
\subsection{WARNING ALARMS (YELLOW)}

\begin{longtable}{|L{0.28\textwidth}|L{0.30\textwidth}|L{0.32\textwidth}|}
\hline
\rowcolor{milyellow!20}
\textbf{Alarm} & \textbf{Meaning} & \textbf{Action} \\
\hline
\endfirsthead

\hline
\rowcolor{milyellow!20}
\textbf{Alarm} & \textbf{Meaning} & \textbf{Action} \\
\hline
\endhead

\textbf{HIGH TEMPERATURE} & Device overheating & Reduce operations, monitor \\
\hline
\textbf{LOW CONFIDENCE TRACKING} & Track quality poor & Verify target, consider manual \\
\hline
\textbf{ZOOM OUT} & LAC FOV insufficient & Zoom out (\button{Button 8}) \\
\hline
\textbf{LEAD ANGLE LAG} & Tracking data insufficient & Wait 2-5 seconds \\
\hline
\end{longtable}

\subsubsection{Warning Alarm Management}

\textbf{HIGH TEMPERATURE:}
\begin{itemize}
    \item Identify which device is overheating (check System Status)
    \item Reduce operational tempo (slower movements, less frequent firing)
    \item Monitor temperature trend (rising or stabilizing?)
    \item If temperature continues to rise: Cease operations, allow cooling
\end{itemize}

\textbf{LOW CONFIDENCE TRACKING:}
\begin{itemize}
    \item Verify target is still visible and properly framed
    \item Check for obstructions or lighting changes
    \item Consider switching to manual engagement
    \item Re-acquire track if confidence drops below 50\%
\end{itemize}

% ----------------------------------------------------------------------------
\subsection{INFO MESSAGES (GREEN/BLUE)}

\begin{longtable}{|L{0.28\textwidth}|L{0.30\textwidth}|L{0.32\textwidth}|}
\hline
\rowcolor{milgreen!20}
\textbf{Message} & \textbf{Meaning} & \textbf{Action} \\
\hline
\endfirsthead

\hline
\rowcolor{milgreen!20}
\textbf{Message} & \textbf{Meaning} & \textbf{Action} \\
\hline
\endhead

\textbf{ZEROING APPLIED} & Zero active & Normal ops \\
\hline
\textbf{ENV APPLIED} & Environmental settings active & Normal ops \\
\hline
\textbf{TRACKING} & Tracking active & Monitor \\
\hline
\end{longtable}

\subsubsection{Informational Message Interpretation}

These messages confirm that optional features are active:
\begin{itemize}
    \item \textbf{ZEROING APPLIED:} Weapon boresight correction in effect
    \item \textbf{ENV APPLIED:} Ballistic corrections for temperature, altitude, wind active
    \item \textbf{TRACKING:} Automatic target tracking engaged
\end{itemize}

\textbf{No Action Required:} These are status confirmations, not warnings.

% ============================================================================
\section{WHEN TO ESCALATE TO MAINTENANCE}

\subsection{OPERATOR-LEVEL ISSUES (Fix Yourself)}

The following issues can be resolved by the operator:

\begin{itemize}
    \item ✅ Reset E-Stop (after clearing danger)
    \item ✅ Power cycle system (soft reboot)
    \item ✅ Switch cameras (if available)
    \item ✅ Adjust zoom/focus
    \item ✅ Re-zero weapon (if needed)
\end{itemize}

\subsection{MAINTENANCE-LEVEL ISSUES (Notify Immediately)}

The following issues require maintenance personnel:

\begin{itemize}
    \item ❌ Device disconnected (and doesn't reconnect after reboot)
    \item ❌ Persistent fault flags
    \item ❌ Servo malfunction (erratic motion, no motion)
    \item ❌ Physical damage observed
    \item ❌ Thermal runaway (temperature >80\degree C)
    \item ❌ Unusual noises (grinding, clicking, squealing)
\end{itemize}

\subsection{Escalation Decision Matrix}

\spectable{
\small
\begin{tabular}{|L{0.25\textwidth}|L{0.30\textwidth}|L{0.35\textwidth}|}
\hline
\rowcolor{milblue!20}
\textbf{Symptom} & \textbf{Operator Action} & \textbf{Escalate If} \\
\hline
Device offline & Power cycle, check connections & Still offline after reboot \\
\hline
High temperature & Reduce operations, allow cooling & Temp >80\degree C or rising \\
\hline
Tracking failure & Adjust gate, improve contrast & Fails on all targets \\
\hline
LRF no reading & Re-fire, try different target & Persistent "NO ECHO" on all targets \\
\hline
Poor video & Switch cameras, adjust focus & Both cameras offline or error \\
\hline
Servo slow/jerky & Check torque, allow cooling & Fault flag or erratic motion \\
\hline
Unusual noises & Identify source, check status & Grinding, clicking, or squealing \\
\hline
\end{tabular}
}

\subsection{Documentation for Maintenance}

When escalating to maintenance, provide:

\begin{checklist}
    \item Specific symptoms observed
    \item When problem first occurred
    \item Any recent changes (maintenance, configuration, environment)
    \item System Status screenshot (if possible)
    \item Any error messages or alarm codes
    \item Operator-level troubleshooting already attempted
\end{checklist}

\begin{notebox}
Better documentation = Faster repair. Be specific and thorough.
\end{notebox}

% ============================================================================
\section{SYSTEM STATUS MONITORING BEST PRACTICES}

\subsection{Regular Monitoring Schedule}

\textbf{Before Each Mission:}
\begin{itemize}
    \item Check System Status display
    \item Verify all devices connected
    \item Note any warnings or elevated temperatures
    \item Document baseline status
\end{itemize}

\textbf{During Operations:}
\begin{itemize}
    \item Monitor temperature trends (especially during extended operations)
    \item Watch for new alarms or warnings
    \item Check device status after any unusual events
\end{itemize}

\textbf{After Mission:}
\begin{itemize}
    \item Final System Status check
    \item Note any new faults or warnings
    \item Document any issues for maintenance
    \item Report any degraded performance
\end{itemize}

\subsection{Proactive Monitoring}

\textbf{Prevent Problems Before They Occur:}
\begin{itemize}
    \item Monitor temperature trends (rising temps indicate potential issues)
    \item Note gradual performance degradation (slower tracking, reduced accuracy)
    \item Report intermittent faults (even if they self-clear)
    \item Watch torque levels (sustained high torque indicates mechanical problems)
\end{itemize}

\begin{cautionbox}
Don't ignore intermittent problems. They often become permanent failures.
\end{cautionbox}

% ============================================================================
% END OF LESSON 7
% ============================================================================
% ============================================================================
% LESSON 8 - EMERGENCY PROCEDURES
% ============================================================================
\lesson{8 - EMERGENCY PROCEDURES}

\begin{tabular}{@{}ll@{}}
\textbf{Duration:} & 3 hours (Classroom 1h + Practical 2h) \\
\textbf{Type:} & Classroom + Practical \\
\textbf{References:} & Emergency procedures manual, safety guidelines \\
\end{tabular}

\section{LEARNING OBJECTIVES}

Upon completion of this lesson, operators will be able to:
\begin{itemize}
    \item Execute Emergency Stop (E-Stop)
    \item Perform emergency tracking abort
    \item Respond to weapon malfunctions
    \item Handle runaway gimbal
    \item Execute emergency system shutdown
\end{itemize}

% ============================================================================
\section{EMERGENCY STOP (E-STOP)}

\subsection{E-STOP BUTTON}

\subsubsection{Location}
\textbf{Red mushroom button on Control Panel}

\subsubsection{When to Use}

Use E-Stop \textbf{IMMEDIATELY} when:
\begin{itemize}
    \item ❌ Personnel in line of fire
    \item ❌ Gimbal moving toward restricted area
    \item ❌ Runaway/uncontrolled motion
    \item ❌ Any immediate danger
\end{itemize}

\subsection{E-STOP ACTIVATION}

\begin{procedurebox}[EMERGENCY STOP ACTIVATION]
\textbf{Action:} \textbf{STRIKE E-STOP BUTTON}

\textbf{Effect (IMMEDIATE):}
\begin{enumerate}
    \item All gimbal motion halted
    \item Weapon fire control disabled
    \item Tracking stopped
    \item System functions locked out
    \item OSD: \textcolor{milred}{\osd{EMERGENCY STOP ACTIVE}} (RED)
\end{enumerate}
\end{procedurebox}

\begin{warningbox}[E-STOP IS IRREVERSIBLE]
Once activated, E-Stop cannot be overridden. System remains locked until manually reset.
\end{warningbox}

\subsubsection{E-Stop System Behavior}

\textbf{What Stops:}
\begin{itemize}
    \item All servo motor power
    \item Weapon arming circuits
    \item Tracking algorithms
    \item Automatic motion modes
\end{itemize}

\textbf{What Continues:}
\begin{itemize}
    \item Power to cameras (video feed maintained)
    \item Power to control panels
    \item System monitoring (can view System Status)
    \item Communication systems
\end{itemize}

% ----------------------------------------------------------------------------
\subsection{E-STOP RESET}

\subsubsection{Pre-Reset Safety Verification}

Only reset after verifying \textbf{ALL} items:

\begin{checklist}
    \item Immediate danger cleared
    \item Personnel clear
    \item No equipment damage
    \item Cause identified
    \item Safe to resume
\end{checklist}

\begin{procedurebox}[E-STOP RESET PROCEDURE]
\begin{enumerate}
    \item Twist E-STOP button clockwise
    \item Button pops out
    \item System resets (takes 5-10 seconds)
    \item Check System Status for faults
    \item Perform health check before resuming ops
\end{enumerate}
\end{procedurebox}

\begin{warningbox}[DO NOT RESET IF]
\textbf{DO NOT RESET E-STOP IF:}
\begin{itemize}
    \item Cause unknown
    \item Damage visible
    \item Personnel in danger
    \item Maintenance required
\end{itemize}
\end{warningbox}

\subsubsection{Post-Reset Verification}

After E-Stop reset, verify:
\begin{checklist}
    \item All devices show "Connected" in System Status
    \item No fault flags present
    \item Gimbal responds to joystick inputs (test in manual mode)
    \item Tracking system functional (test acquisition)
    \item Safety systems operational (test Master Arm/Disarm)
\end{checklist}

% ============================================================================
\section{EMERGENCY TRACKING ABORT}

\subsection{WHEN TO ABORT TRACKING}

Abort tracking \textbf{immediately} when:

\begin{itemize}
    \item ❌ Tracking wrong target
    \item ❌ Target enters restricted zone
    \item ❌ Friendlies near target
    \item ❌ Lost positive ID
    \item ❌ System erratic
    \item ❌ Civilian identified
\end{itemize}

\subsection{TRACKING ABORT PROCEDURE}

\begin{procedurebox}[EMERGENCY TRACKING ABORT]
\textbf{Procedure:} \textbf{DOUBLE-CLICK \button{Button 4}} (<500ms)

\textbf{Effect:}
\begin{itemize}
    \item Tracking stops immediately
    \item Tracking gate disappears
    \item Returns to Manual mode
    \item Weapon fire inhibited
\end{itemize}
\end{procedurebox}

\begin{notebox}
No Dead Man Switch required for abort. Works from any phase.
\end{notebox}

\subsubsection{Tracking Abort vs. E-Stop}

\textbf{Use Tracking Abort When:}
\begin{itemize}
    \item Wrong target being tracked
    \item Need to quickly disengage tracking
    \item System otherwise functioning normally
\end{itemize}

\textbf{Use E-Stop When:}
\begin{itemize}
    \item Immediate danger to personnel
    \item System malfunction (runaway gimbal)
    \item Need to halt \textbf{all} system motion
\end{itemize}

\subsection{Post-Abort Actions}

After aborting tracking:

\begin{enumerate}
    \item Verify gimbal in Manual mode
    \item Assess situation (why was abort necessary?)
    \item Re-acquire correct target (if applicable)
    \item Report incident (if target misidentification occurred)
\end{enumerate}

% ============================================================================
\section{WEAPON EMERGENCY PROCEDURES}

\subsection{ACCIDENTAL DISCHARGE}

\subsubsection{Recognition}

\textbf{If weapon fires unintentionally:}
\begin{itemize}
    \item Fire button not pressed, but weapon firing
    \item Single shot when burst intended (or vice versa)
    \item Weapon fires on Master Arm engagement
\end{itemize}

\subsubsection{Immediate Actions}

\begin{procedurebox}[ACCIDENTAL DISCHARGE RESPONSE]
\textbf{Immediate Actions (in order):}
\begin{enumerate}
    \item \textbf{Release \button{Button 5}} (Fire) - cease fire
    \item \textbf{Release \button{Button 0}} (Master Arm) - disarm
    \item \textbf{Point gimbal to safe direction} (skyward)
    \item \textbf{E-STOP} if motion continues
    \item \textbf{Turn Station power OFF}
\end{enumerate}
\end{procedurebox}

\begin{warningbox}[CRITICAL SAFETY VIOLATION]
Accidental discharge is a \textbf{CRITICAL SAFETY EVENT}. All operations must cease until investigation complete.
\end{warningbox}

\subsubsection{Post-Incident Actions}

\begin{checklist}
    \item Ensure weapon safe
    \item Check for damage/casualties
    \item Notify chain of command immediately
    \item Preserve system logs
    \item Do NOT resume operations until investigation complete
\end{checklist}

\textbf{Investigation Requirements:}
\begin{itemize}
    \item Maintenance personnel must inspect fire control circuits
    \item Software logs must be reviewed
    \item Trigger mechanism must be tested
    \item Safety interlocks must be verified
    \item Commander's authorization required before resuming operations
\end{itemize}

% ----------------------------------------------------------------------------
\subsection{WEAPON JAM / MALFUNCTION}

\subsubsection{Symptoms}

\begin{itemize}
    \item Weapon fires then stops mid-burst
    \item Unusual sounds (click, no firing)
    \item Fire button pressed, no response
    \item Partial firing (rounds not feeding properly)
\end{itemize}

\subsubsection{Immediate Actions}

\begin{procedurebox}[WEAPON JAM RESPONSE]
\begin{enumerate}
    \item \textbf{Release Fire button} (cease trigger)
    \item \textbf{Release Master Arm} (disarm)
    \item \textbf{Turn Gun Arm switch to SAFE}
    \item \textbf{Point gimbal to safe direction}
    \item \textbf{Notify command} - weapon malfunction
\end{enumerate}
\end{procedurebox}

\begin{warningbox}[DO NOT ATTEMPT TO CLEAR JAM]
\textbf{DO NOT:}
\begin{itemize}
    \item ❌ Attempt to clear jam from operator station
    \item ❌ Continue firing attempts
    \item ❌ Inspect weapon without proper clearance
\end{itemize}

\textbf{Action:} Weapon maintenance personnel required.
\end{warningbox}

\subsubsection{Operator Limitations}

\textbf{Operators are NOT authorized to:}
\begin{itemize}
    \item Clear weapon jams
    \item Perform weapon maintenance
    \item Inspect weapon mechanisms
    \item Override weapon safety interlocks
\end{itemize}

\textbf{Operators ARE authorized to:}
\begin{itemize}
    \item Safe the weapon (Gun Arm to SAFE)
    \item Report malfunction symptoms
    \item Secure the area
    \item Await maintenance personnel
\end{itemize}

% ============================================================================
\section{RUNAWAY GIMBAL}

\subsection{SYMPTOMS}

\textbf{Runaway gimbal indicators:}
\begin{itemize}
    \item Gimbal moves without joystick input
    \item Cannot stop gimbal with joystick
    \item Gimbal moving toward no-traverse zone
    \item Erratic, unpredictable motion
    \item Gimbal oscillating rapidly
\end{itemize}

\subsection{IMMEDIATE ACTIONS}

\begin{procedurebox}[RUNAWAY GIMBAL RESPONSE]
\textbf{Immediate Actions (in order):}
\begin{enumerate}
    \item \textbf{E-STOP} (strike immediately)
    \item \textbf{Verify E-Stop engaged} (button latched, motion stopped)
    \item \textbf{Turn Station power OFF}
    \item \textbf{Notify maintenance} - servo malfunction
\end{enumerate}
\end{procedurebox}

\begin{warningbox}[DO NOT RESET]
Do NOT reset E-Stop until:
\begin{itemize}
    \item Maintenance personnel inspect system
    \item Cause identified and resolved
    \item Safety verified
\end{itemize}
\end{warningbox}

\subsection{Runaway Gimbal Causes}

\textbf{Common Causes:}
\begin{itemize}
    \item Servo driver malfunction
    \item Software control loop error
    \item Encoder feedback failure
    \item Electrical short circuit
    \item Control signal interference
\end{itemize}

\textbf{All require maintenance intervention.}

\subsection{Prevention}

\textbf{Reduce risk of runaway gimbal:}
\begin{itemize}
    \item Perform daily pre-operation checks
    \item Monitor servo torque and temperature
    \item Report unusual gimbal behavior immediately
    \item Keep E-Stop button clear and accessible
    \item Practice E-Stop activation during training
\end{itemize}

% ============================================================================
\section{LOST COMMUNICATION}

\subsection{OPERATOR TO COMMAND}

\subsubsection{Symptoms}
\begin{itemize}
    \item Radio/intercom dead
    \item No response to radio calls
    \item Static or interference on all channels
\end{itemize}

\subsubsection{Actions}

\begin{procedurebox}[LOST COMMUNICATION WITH COMMAND]
\begin{checklist}
    \item Switch to backup radio
    \item Use hand signals (if line of sight)
    \item Continue mission per SOP (if applicable)
    \item Return to safe position (if no comms recovery)
\end{checklist}
\end{procedurebox}

\textbf{Standing Orders:}
\begin{itemize}
    \item If communications lost during active engagement: Complete engagement, then return to safe position
    \item If communications lost during surveillance: Continue surveillance, attempt recovery every 5 minutes
    \item If communications not recovered within 30 minutes: Return to base/rally point
\end{itemize}

% ----------------------------------------------------------------------------
\subsection{SYSTEM TO DEVICES}

\subsubsection{Symptoms}
\begin{itemize}
    \item Device disconnected (LRF, Camera, Servo)
    \item "Device Offline" message in System Status
    \item Loss of specific functionality (ranging, video, motion)
\end{itemize}

\subsubsection{Actions}

\begin{procedurebox}[LOST DEVICE COMMUNICATION]
\begin{enumerate}
    \item Check System Status (identify device)
    \item Power cycle system (soft reboot)
    \item If device does not reconnect: Operate without device (degraded mode)
    \item Notify maintenance
\end{enumerate}
\end{procedurebox}

\subsubsection{Mission Impact}

\spectable{
\begin{tabular}{|L{0.25\textwidth}|L{0.65\textwidth}|}
\hline
\rowcolor{milblue!20}
\textbf{Device Offline} & \textbf{Mission Impact} \\
\hline
\textbf{LRF offline} & Manual range estimation only, no LAC. Reduced accuracy at long range. \\
\hline
\textbf{Camera offline} & Switch to alternate camera. If both offline: mission abort. \\
\hline
\textbf{Servo offline} & Mission abort - cannot control gimbal. \\
\hline
\textbf{IMU offline} & No stabilization. Manual mode only. Degraded tracking. \\
\hline
\textbf{Joystick offline} & Use DCU backup controls (limited functionality). \\
\hline
\end{tabular}
}

% ============================================================================
\section{EMERGENCY SYSTEM SHUTDOWN}

\subsection{WHEN TO PERFORM FULL SHUTDOWN}

Perform emergency shutdown when:
\begin{itemize}
    \item Fire/smoke from system
    \item Multiple critical faults
    \item Safety concern requiring immediate powerdown
    \item Ordered by command
    \item Electrical burning smell
    \item Unusual arcing or sparking
\end{itemize}

\subsection{EMERGENCY SHUTDOWN PROCEDURE}

\begin{procedurebox}[EMERGENCY SYSTEM SHUTDOWN]
\begin{enumerate}
    \item \textbf{E-STOP} (if not already engaged)
    \item \textbf{Turn Gun Arm to SAFE}
    \item \textbf{Turn Station Enable switch to OFF}
    \item \textbf{Turn main power switch to OFF}
    \item \textbf{Disconnect external power} (if directed)
\end{enumerate}
\end{procedurebox}

\subsection{Post-Shutdown Actions}

\begin{checklist}
    \item Ensure all motion stopped
    \item Ensure weapon safe
    \item Notify command/maintenance
    \item Secure area (prevent unauthorized restart)
    \item Document circumstances leading to shutdown
\end{checklist}

\begin{warningbox}[DO NOT RESTART]
Do NOT attempt to restart system until:
\begin{itemize}
    \item Cause of emergency identified
    \item Maintenance personnel inspect system
    \item Commander authorizes restart
\end{itemize}
\end{warningbox}

\subsection{Fire Response}

\textbf{If fire or smoke observed:}
\begin{enumerate}
    \item Perform emergency shutdown (above procedure)
    \item Evacuate immediate area
    \item Alert fire response team
    \item Use fire extinguisher if:
    \begin{itemize}
        \item Fire is small and contained
        \item Safe to approach
        \item Proper extinguisher type available (electrical fires: CO2 or dry chemical)
    \end{itemize}
    \item Do NOT use water on electrical fires
\end{enumerate}

% ============================================================================
\section{EMERGENCY QUICK REFERENCE}

\begin{quickref}
\begin{longtable}{|L{0.22\textwidth}|L{0.35\textwidth}|L{0.30\textwidth}|}
\hline
\rowcolor{milred!20}
\textbf{Emergency} & \textbf{Action} & \textbf{Button/Switch} \\
\hline
\endfirsthead

\hline
\rowcolor{milred!20}
\textbf{Emergency} & \textbf{Action} & \textbf{Button/Switch} \\
\hline
\endhead

\textbf{Immediate Danger} & Emergency Stop & E-STOP (red mushroom) \\
\hline
\textbf{Wrong Target} & Tracking Abort & \button{Button 4} (double-click) \\
\hline
\textbf{Accidental Fire} & Cease Fire, Disarm & Release \button{Button 5}, Release \button{Button 0} \\
\hline
\textbf{Weapon Jam} & Disarm, Safe Weapon & Release \button{Button 0}, Gun Arm SAFE \\
\hline
\textbf{Runaway Gimbal} & Emergency Stop, Power Off & E-STOP, Station Power OFF \\
\hline
\textbf{Fire/Smoke} & Full Shutdown & E-STOP → Station OFF → Main Power OFF \\
\hline
\end{longtable}
\end{quickref}

% ============================================================================
\section{EMERGENCY RESPONSE TRAINING}

\subsection{Required Emergency Drills}

All operators must practice the following emergency responses:

\subsubsection{Drill 1: E-Stop Activation}
\begin{itemize}
    \item Practice striking E-Stop button from operating position
    \item Time to E-Stop: <1 second from recognition
    \item Verify system lockout
    \item Practice reset procedure
\end{itemize}

\subsubsection{Drill 2: Tracking Abort}
\begin{itemize}
    \item Practice double-click abort during tracking exercise
    \item Time to abort: <500ms
    \item Verify tracking stops immediately
    \item Practice re-acquisition after abort
\end{itemize}

\subsubsection{Drill 3: Accidental Discharge Response}
\begin{itemize}
    \item Simulated accidental discharge scenario
    \item Practice immediate cease-fire and disarm sequence
    \item Practice safe gimbal positioning
    \item Practice emergency shutdown
\end{itemize}

\subsubsection{Drill 4: Runaway Gimbal Response}
\begin{itemize}
    \item Simulated runaway gimbal (instructor-initiated)
    \item Practice immediate E-Stop
    \item Practice emergency shutdown
    \item Verify no attempt to reset without authorization
\end{itemize}

\subsection{Emergency Response Evaluation Criteria}

\textbf{Proficiency Standards:}
\begin{itemize}
    \item E-Stop activation: <1 second recognition-to-action time
    \item Tracking abort: <500ms double-click execution
    \item Correct procedure selection: 100\% (no procedural errors)
    \item Post-emergency verification: All steps completed
\end{itemize}

\begin{warningbox}[TRAINING REQUIREMENT]
Operators must demonstrate proficiency in all emergency procedures before operational qualification.
\end{warningbox}

% ============================================================================
\section{STRESS MANAGEMENT IN EMERGENCIES}

\subsection{Staying Calm Under Pressure}

\textbf{Emergency Response Psychology:}
\begin{itemize}
    \item Recognize stress response (tunnel vision, rapid heartbeat, loss of fine motor control)
    \item Focus on immediate action (don't overthink)
    \item Trust your training (muscle memory)
    \item Breathe (deliberate breathing reduces stress)
\end{itemize}

\subsection{Decision Making in Emergencies}

\textbf{Priority Framework:}
\begin{enumerate}
    \item \textbf{Life Safety:} Prevent injury to personnel (highest priority)
    \item \textbf{Equipment Protection:} Prevent damage to system (secondary)
    \item \textbf{Mission Continuity:} Resume operations (lowest priority)
\end{enumerate}

\textbf{When in Doubt:}
\begin{itemize}
    \item Default to most conservative action (E-Stop)
    \item Request guidance from command
    \item Do NOT resume operations until safe
\end{itemize}

% ============================================================================
% END OF LESSON 8
% ============================================================================
% ============================================================================
% LESSON 9 - OPERATOR MAINTENANCE & TROUBLESHOOTING
% ============================================================================
\lesson{9 - OPERATOR MAINTENANCE \& TROUBLESHOOTING}

\begin{tabular}{@{}ll@{}}
\textbf{Duration:} & 3 hours (Classroom 1h + Practical 2h) \\
\textbf{Type:} & Classroom + Practical \\
\textbf{References:} & Maintenance manual, troubleshooting guide \\
\end{tabular}

\section{LEARNING OBJECTIVES}

Upon completion of this lesson, operators will be able to:
\begin{itemize}
    \item Perform daily operator checks
    \item Troubleshoot common issues
    \item Apply systematic troubleshooting methodology
    \item Determine when to escalate to maintenance
\end{itemize}

% ============================================================================
\section{DAILY OPERATOR CHECKS}

\subsection{PRE-OPERATION CHECKLIST}

\textbf{Perform before each mission / daily:}

\begin{longtable}{|L{0.15\textwidth}|L{0.25\textwidth}|L{0.25\textwidth}|L{0.25\textwidth}|}
\hline
\rowcolor{milblue!20}
\textbf{System} & \textbf{Check} & \textbf{GO Criteria} & \textbf{NO-GO Criteria} \\
\hline
\endfirsthead

\hline
\rowcolor{milblue!20}
\textbf{System} & \textbf{Check} & \textbf{GO Criteria} & \textbf{NO-GO Criteria} \\
\hline
\endhead

\textbf{Power} & Main power, Station power & All ON, no alarms & Power OFF, alarms present \\
\hline
\textbf{Control Panel} & Buttons, switches, lights responsive & All functional & Non-responsive, lights out \\
\hline
\textbf{Joystick} & Axes, buttons, Dead Man Switch & Smooth motion, all buttons work & Sticky, unresponsive \\
\hline
\textbf{Gimbal} & Azimuth/Elevation motion & Smooth, quiet, full ROM & Grinding, jerky, limits hit \\
\hline
\textbf{Cameras} & Day \& Thermal video & Clear video, both cameras & No video, errors \\
\hline
\textbf{LRF} & Ranging test (known target) & Range within ±5m & No reading, error \\
\hline
\textbf{Tracking} & Acquisition test & Gate appears, tracking functional & No gate, tracking fails \\
\hline
\textbf{Weapon} & Arming, Safe & Arm/Safe switches functional & Switches stuck, no response \\
\hline
\textbf{System Status} & All devices connected & All ✓ Connected & Devices disconnected/faulted \\
\hline
\end{longtable}

\begin{warningbox}[NO-GO CRITERIA]
If NO-GO: Do NOT proceed with mission. Notify maintenance.
\end{warningbox}

\subsection{Pre-Operation Check Procedure}

\begin{procedurebox}[DAILY PRE-OPERATION CHECK]
\textbf{1. Visual Inspection}
\begin{itemize}
    \item Inspect gimbal for physical damage
    \item Check cable connections (secure, not frayed)
    \item Verify E-Stop button not engaged
    \item Check for fluid leaks, corrosion, loose parts
\end{itemize}

\textbf{2. Power-Up Sequence}
\begin{itemize}
    \item Turn main power ON
    \item Turn Station Enable ON
    \item Observe startup sequence (all lights illuminate)
    \item Wait for system initialization (~30 seconds)
\end{itemize}

\textbf{3. System Status Check}
\begin{itemize}
    \item Access System Status display
    \item Verify all devices "Connected"
    \item Check for fault flags (should be none)
    \item Note baseline temperatures
\end{itemize}

\textbf{4. Functional Tests}
\begin{itemize}
    \item Joystick: Test all axes and buttons
    \item Gimbal: Test azimuth/elevation motion (smooth, full range)
    \item Cameras: Switch between day/thermal, verify clear video
    \item LRF: Range test on known target (verify ±5m accuracy)
    \item Tracking: Quick acquisition test (gate appears, lock achieved)
    \item Weapon: Test arm/safe switches (DO NOT load weapon)
\end{itemize}

\textbf{5. Documentation}
\begin{itemize}
    \item Record check completion in logbook
    \item Note any discrepancies or warnings
    \item Sign off as "GO" or "NO-GO"
\end{itemize}
\end{procedurebox}

% ============================================================================
\section{TROUBLESHOOTING METHODOLOGY}

\subsection{STOP-LOOK-ASSESS-FIX (SLAF) METHOD}

\subsubsection{S - STOP}
\textbf{Don't rush to "fix" without understanding}
\begin{itemize}
    \item Take E-Stop if safety concern
    \item Document symptoms precisely
    \item Don't make random changes hoping for improvement
    \item Pause and think before acting
\end{itemize}

\subsubsection{L - LOOK}
\textbf{Observe all symptoms carefully}
\begin{itemize}
    \item Check System Status display
    \item Review recent actions (what changed?)
    \item Note all error messages and alarms
    \item Observe physical indicators (lights, sounds, motion)
\end{itemize}

\subsubsection{A - ASSESS}
\textbf{Narrow down to specific subsystem}
\begin{itemize}
    \item Check simplest causes first (power, connections, switches)
    \item Consult troubleshooting charts
    \item Eliminate possibilities systematically
    \item Determine if problem is operator-level or maintenance-level
\end{itemize}

\subsubsection{F - FIX}
\textbf{Apply appropriate solution}
\begin{itemize}
    \item Apply operator-level fix if authorized
    \item Escalate to maintenance if needed
    \item Verify fix resolved issue
    \item Document actions taken
\end{itemize}

\subsection{Troubleshooting Best Practices}

\begin{itemize}
    \item \textbf{Start simple:} Check power, cables, switches before assuming complex failure
    \item \textbf{Change one thing at a time:} If you change multiple things, you won't know what fixed it
    \item \textbf{Verify the fix:} Test that the problem is actually resolved
    \item \textbf{Document everything:} What was the symptom? What did you try? What worked?
\end{itemize}

% ============================================================================
\section{COMMON ISSUES \& FIXES}

\subsection{JOYSTICK NOT RESPONDING}

\subsubsection{Symptoms}
\begin{itemize}
    \item No gimbal motion when joystick moved
    \item Buttons not working
    \item Joystick LED off
\end{itemize}

\subsubsection{Diagnosis}

\begin{procedurebox}[JOYSTICK DIAGNOSIS]
\begin{enumerate}
    \item Check USB cable (both ends firmly connected)
    \item Check joystick power LED (should be illuminated)
    \item Check System Status → Joystick Connected
    \item Try moving joystick (any response?)
\end{enumerate}
\end{procedurebox}

\subsubsection{Operator-Level Fixes}

\begin{checklist}
    \item Reconnect USB cable firmly (both ends)
    \item Power cycle system (turn Station OFF, wait 10 seconds, turn ON)
    \item Try different USB port (if available)
    \item Check for physical damage to cable or joystick
\end{checklist}

\textbf{Escalate if:} Still not detected after power cycle.

% ----------------------------------------------------------------------------
\subsection{NO VIDEO FROM CAMERA}

\subsubsection{Symptoms}
\begin{itemize}
    \item Black screen
    \item "No Signal" message
    \item Frozen image
    \item Severe pixelation or artifacts
\end{itemize}

\subsubsection{Diagnosis}

\begin{procedurebox}[CAMERA DIAGNOSIS]
\begin{enumerate}
    \item Check camera selection (Day vs. Night - Button 10/12)
    \item Check System Status → Camera Connected
    \item Verify lens cap removed (if applicable)
    \item Check for obstructions in front of lens
\end{enumerate}
\end{procedurebox}

\subsubsection{Operator-Level Fixes}

\begin{checklist}
    \item Switch cameras (\button{Button 10/12})
    \item Power cycle system
    \item Check video cable connections (if accessible)
    \item Clean lens (if dirty/foggy)
    \item Adjust focus (if autofocus failed)
\end{checklist}

\textbf{Escalate if:} Both cameras offline.

% ----------------------------------------------------------------------------
\subsection{TRACKING FAILS TO LOCK}

\subsubsection{Symptoms}
\begin{itemize}
    \item Stays in LOCK PENDING indefinitely
    \item Returns to ACQUISITION after lock attempt
    \item Gate flickers or disappears
    \item Lock achieved but immediately lost
\end{itemize}

\subsubsection{Diagnosis}

\begin{procedurebox}[TRACKING DIAGNOSIS]
\begin{enumerate}
    \item Check target contrast (clearly visible on screen?)
    \item Check acquisition gate sizing (10-30\% margin?)
    \item Check gimbal stability (holding steady during lock?)
    \item Check target motion (stationary or slow-moving?)
\end{enumerate}
\end{procedurebox}

\subsubsection{Operator-Level Fixes}

\begin{checklist}
    \item Improve target contrast (switch camera, adjust thermal LUT)
    \item Re-size acquisition gate (ensure 10-30\% margin around target)
    \item Hold gimbal steady during lock attempt (no joystick input)
    \item Select better target (higher contrast, larger, clearer edges)
    \item Switch to manual engagement (no tracking)
\end{checklist}

\textbf{Escalate if:} Tracking fails on all targets (tracker malfunction).

% ----------------------------------------------------------------------------
\subsection{GIMBAL NOT MOVING}

\subsubsection{Symptoms}
\begin{itemize}
    \item Joystick input has no effect
    \item Gimbal completely frozen
    \item Only one axis moving (azimuth OR elevation)
    \item Very slow, sluggish motion
\end{itemize}

\subsubsection{Diagnosis}

\begin{procedurebox}[GIMBAL DIAGNOSIS]
\begin{enumerate}
    \item Check E-Stop (engaged/latched?)
    \item Check Station Enable switch (ON?)
    \item Check motion mode (Manual mode selected?)
    \item Check System Status → Servos Connected
    \item Check for gimbal limit switches (at mechanical limit?)
\end{enumerate}
\end{procedurebox}

\subsubsection{Operator-Level Fixes}

\begin{checklist}
    \item Reset E-Stop (if engaged)
    \item Turn Station Enable ON
    \item Cycle to Manual mode (\button{Button 11/13})
    \item Move gimbal away from limits (if at limit switch)
    \item Power cycle system
\end{checklist}

\textbf{Escalate if:} Servos disconnected or faulted.

% ----------------------------------------------------------------------------
\subsection{LRF NOT RANGING}

\subsubsection{Symptoms}
\begin{itemize}
    \item No range reading displayed
    \item "NO ECHO" message (persistent)
    \item Range reading frozen/stale
    \item Wildly inaccurate range (e.g., 50,000m)
\end{itemize}

\subsubsection{Diagnosis}

\begin{procedurebox}[LRF DIAGNOSIS]
\begin{enumerate}
    \item Check target distance (within max range ~5km for hard targets?)
    \item Check target reflectivity (solid surface, not transparent/absorbent?)
    \item Check LRF temperature (System Status - overheated?)
    \item Check atmospheric conditions (fog, smoke, heavy rain?)
    \item Check for obstructions between LRF and target
\end{enumerate}
\end{procedurebox}

\subsubsection{Operator-Level Fixes}

\begin{checklist}
    \item Re-fire LRF (multiple attempts - 3-5 tries)
    \item Aim at different target (more reflective surface)
    \item Allow LRF to cool (if overheated - wait 5-10 minutes)
    \item Switch to manual range estimation (map, known landmarks)
    \item Wait for atmospheric conditions to improve
\end{checklist}

\textbf{Escalate if:} LRF disconnected or constant fault flag.

% ----------------------------------------------------------------------------
\subsection{LRF OPERATING MODES}

The LRF supports two operating modes:

\subsubsection{Single-Shot Mode (Default)}

\begin{itemize}
    \item Press \button{Button 1} once to trigger a single range measurement
    \item Range displayed on OSD: \osd{RNG: xxxx m}
    \item Best for stationary targets or occasional ranging
\end{itemize}

\subsubsection{Continuous Mode (5Hz Automatic Ranging)}

\begin{procedurebox}[ENABLING CONTINUOUS LRF MODE]
\textbf{To enable:}
\begin{enumerate}
    \item \textbf{Double-press} \button{Button 1} within 1 second
    \item System enters continuous ranging mode (5Hz)
    \item Range updates automatically ~5 times per second
\end{enumerate}

\textbf{To disable:}
\begin{enumerate}
    \item \textbf{Double-press} \button{Button 1} again within 1 second
    \item System returns to single-shot mode
\end{enumerate}
\end{procedurebox}

\begin{notebox}
When continuous mode is active, single presses of Button 1 are \textbf{ignored}. You must double-press again to disable continuous mode and return to single-shot operation.
\end{notebox}

\textbf{When to use Continuous Mode:}
\begin{itemize}
    \item Tracking moving targets where range changes rapidly
    \item When you need constant range updates for ballistic calculations
    \item During active engagements with maneuvering targets
\end{itemize}

\textbf{Caution:}
\begin{itemize}
    \item Continuous mode increases LRF component wear
    \item May cause faster temperature rise
    \item Disable when not needed
\end{itemize}

% ----------------------------------------------------------------------------
\subsection{LRF CLEAR VS ZEROING CLEAR}

These are \textbf{separate functions}:

\begin{itemize}
    \item \textbf{LRF Clear (Button 10):} Clears the current LRF range measurement only. Sets stored range to 0. Does \textbf{NOT} affect weapon zeroing.
    \item \textbf{Clear Active Zero (Menu):} Clears weapon zeroing offsets. Does \textbf{NOT} affect LRF measurements.
\end{itemize}

\begin{notebox}
If you want to clear both LRF range and zeroing, you must perform both actions separately.
\end{notebox}

% ----------------------------------------------------------------------------
\subsection{HIGH TEMPERATURE WARNING}

\subsubsection{Symptoms}
\begin{itemize}
    \item "HIGH TEMP" alarm (yellow or red)
    \item Device temperature >60\degree C in System Status
    \item Thermal shutdown (device stops functioning)
\end{itemize}

\subsubsection{Diagnosis}

\begin{procedurebox}[HIGH TEMPERATURE DIAGNOSIS]
\begin{enumerate}
    \item Check System Status → which device is overheating?
    \item Check ambient temperature (extreme heat environment?)
    \item Check operation tempo (continuous motion/firing?)
    \item Check ventilation (blocked air intakes?)
\end{enumerate}
\end{procedurebox}

\subsubsection{Operator-Level Fixes}

\begin{checklist}
    \item Reduce operations (slower movements, less frequent firing)
    \item Allow cooling period (5-15 minutes, depending on device)
    \item Limit continuous motion/firing (use intermittent operations)
    \item Increase ventilation (if possible - open vents, improve airflow)
    \item Monitor temperature trend (should decrease during cooldown)
\end{checklist}

\textbf{Escalate if:} Temperature remains >80\degree C after cooldown period.

\subsubsection{Temperature Management}

\textbf{Prevention:}
\begin{itemize}
    \item Avoid sustained high-tempo operations in hot environments
    \item Allow periodic cooling breaks (5 minutes every 30 minutes in extreme heat)
    \item Monitor temperature trends proactively
    \item Plan operations during cooler times of day (if possible)
\end{itemize}

% ============================================================================
\section{TROUBLESHOOTING QUICK REFERENCE}

\begin{longtable}{|L{0.18\textwidth}|L{0.22\textwidth}|L{0.25\textwidth}|L{0.25\textwidth}|}
\hline
\rowcolor{milblue!20}
\textbf{Symptom} & \textbf{Likely Cause} & \textbf{Operator Fix} & \textbf{Escalate If} \\
\hline
\endfirsthead

\hline
\rowcolor{milblue!20}
\textbf{Symptom} & \textbf{Likely Cause} & \textbf{Operator Fix} & \textbf{Escalate If} \\
\hline
\endhead

\textbf{No joystick response} & USB disconnected & Reconnect USB, reboot & Still offline after reboot \\
\hline
\textbf{Black video} & Camera fault & Switch cameras & Both cameras offline \\
\hline
\textbf{Tracking won't lock} & Poor contrast, bad gate & Improve contrast, resize gate & Fails on all targets \\
\hline
\textbf{Gimbal frozen} & E-Stop, Station OFF & Reset E-Stop, Station ON & Servo fault flag \\
\hline
\textbf{LRF no reading} & Target too far/obscured & Re-fire, different target & LRF disconnected \\
\hline
\textbf{High temp} & Overuse & Reduce ops, cool down & Temp stays >80\degree C \\
\hline
\textbf{Weapon won't arm} & Safety interlocks & Check switches, authorization & Switches non-functional \\
\hline
\textbf{Lost zero accuracy} & Drift, maintenance & Re-zero weapon & Persistent after re-zero \\
\hline
\end{longtable}

% ============================================================================
\section{ADVANCED TROUBLESHOOTING}

\subsection{INTERMITTENT PROBLEMS}

\subsubsection{Characteristics}
\begin{itemize}
    \item Problem appears and disappears randomly
    \item Works sometimes, fails other times
    \item Difficult to reproduce consistently
\end{itemize}

\subsubsection{Troubleshooting Approach}

\textbf{Document the Pattern:}
\begin{itemize}
    \item When does it occur? (time of day, temperature, specific operation)
    \item What makes it better/worse?
    \item How often does it happen?
    \item Any common factors when it occurs?
\end{itemize}

\textbf{Common Causes:}
\begin{itemize}
    \item Loose connections (intermittent contact)
    \item Temperature-related (works cold, fails hot)
    \item Software timing issues (race conditions)
    \item Electrical interference (when other systems operating)
\end{itemize}

\begin{cautionbox}
Intermittent problems often indicate impending failure. Report immediately even if problem self-clears.
\end{cautionbox}

% ----------------------------------------------------------------------------
\subsection{MULTIPLE SIMULTANEOUS FAULTS}

\subsubsection{Diagnosis Strategy}

When multiple systems fail simultaneously:

\begin{enumerate}
    \item \textbf{Look for common cause:}
    \begin{itemize}
        \item Power supply issue (affects all systems)
        \item Main computer failure (affects all subsystems)
        \item Communication bus failure (multiple devices offline)
    \end{itemize}
    
    \item \textbf{Prioritize troubleshooting:}
    \begin{itemize}
        \item Start with power (most fundamental)
        \item Then communication (affects device connectivity)
        \item Then individual devices
    \end{itemize}
    
    \item \textbf{Consider cascade failures:}
    \begin{itemize}
        \item One device failure may cause others
        \item Example: IMU failure → tracking fails → LAC fails
    \end{itemize}
\end{enumerate}

\textbf{Escalate immediately if:}
\begin{itemize}
    \item >3 devices offline simultaneously
    \item Power supply issues suspected
    \item Unknown common cause
\end{itemize}

% ----------------------------------------------------------------------------
\subsection{DEGRADED PERFORMANCE}

\subsubsection{Symptoms}
\begin{itemize}
    \item System works, but not as well as before
    \item Slower response times
    \item Reduced accuracy
    \item Increased errors/retries needed
\end{itemize}

\subsubsection{Investigation}

\textbf{Compare to Baseline:}
\begin{itemize}
    \item Check tracking confidence (was 90\%, now 70\%?)
    \item Check ranging accuracy (was ±2m, now ±10m?)
    \item Check gimbal smoothness (jerky vs. smooth)
    \item Check response time (lag vs. immediate)
\end{itemize}

\textbf{Possible Causes:}
\begin{itemize}
    \item Mechanical wear (servos, bearings)
    \item Sensor degradation (camera, LRF, IMU)
    \item Software configuration drift
    \item Environmental factors (temperature, contamination)
\end{itemize}

\textbf{Operator Actions:}
\begin{itemize}
    \item Document specific performance degradation
    \item Attempt re-calibration (re-zero, re-configure)
    \item Report to maintenance (even if still operational)
\end{itemize}

% ============================================================================
\section{PREVENTIVE MAINTENANCE}

\subsection{OPERATOR-LEVEL MAINTENANCE}

\subsubsection{Daily Tasks}
\begin{itemize}
    \item Clean camera lenses (soft cloth, approved cleaner)
    \item Inspect cables and connectors (secure, no damage)
    \item Check fluid levels (if applicable)
    \item Wipe down control panels (remove dirt, moisture)
    \item Visual inspection for damage or corrosion
\end{itemize}

\subsubsection{Weekly Tasks}
\begin{itemize}
    \item Verify zero accuracy (test shot at known range)
    \item Exercise full range of motion (azimuth/elevation)
    \item Test all emergency procedures (E-Stop, tracking abort)
    \item Check backup systems (backup radio, backup camera)
    \item Review system logs (check for recurring errors)
\end{itemize}

\subsubsection{Monthly Tasks (or as directed)}
\begin{itemize}
    \item Re-zero weapon
    \item Full system health check (all devices, all functions)
    \item Update system software (if authorized and available)
    \item Lubricate moving parts (if authorized and trained)
\end{itemize}

\begin{notebox}
Preventive maintenance prevents failures. Regular checks catch problems before they become critical.
\end{notebox}

% ----------------------------------------------------------------------------
\subsection{WHAT OPERATORS SHOULD NOT DO}

\textbf{Operators are NOT authorized to:}
\begin{itemize}
    \item ❌ Open sealed enclosures
    \item ❌ Repair circuit boards or electronics
    \item ❌ Replace major components (servos, cameras, computers)
    \item ❌ Modify software or configuration files
    \item ❌ Perform calibrations beyond re-zeroing
    \item ❌ Disable safety interlocks
\end{itemize}

\textbf{Operators ARE authorized to:}
\begin{itemize}
    \item ✅ Clean external surfaces and lenses
    \item ✅ Reconnect loose cables (external only)
    \item ✅ Re-zero weapon
    \item ✅ Adjust user settings (zoom, brightness, environmental parameters)
    \item ✅ Power cycle system
    \item ✅ Replace batteries (if trained and authorized)
\end{itemize}

% ============================================================================
\section{MAINTENANCE ESCALATION}

\subsection{WHEN TO CALL MAINTENANCE}

\textbf{Call maintenance immediately for:}
\begin{itemize}
    \item Device disconnected (doesn't reconnect after reboot)
    \item Persistent fault flags
    \item Servo malfunction (erratic motion, no motion, unusual noises)
    \item Physical damage observed
    \item Thermal runaway (temperature >80\degree C, not decreasing)
    \item Electrical issues (burning smell, smoke, arcing)
    \item Weapon malfunction (jam, accidental discharge, failure to fire)
    \item Safety system failure (E-Stop, interlocks not working)
\end{itemize}

\subsection{INFORMATION TO PROVIDE}

When calling maintenance, provide:

\begin{checklist}
    \item System serial number / unit identifier
    \item Specific symptoms (be detailed)
    \item When problem first occurred
    \item Frequency (constant, intermittent, getting worse?)
    \item Recent events (maintenance, configuration changes, incidents)
    \item System Status information (which devices faulted?)
    \item Error messages or alarm codes
    \item Troubleshooting already attempted
    \item Current system state (powered on, E-Stopped, etc.)
\end{checklist}

\subsection{MAINTENANCE RESPONSE PRIORITIES}

\textbf{Priority 1 - Immediate Response:}
\begin{itemize}
    \item Safety system failure
    \item Fire/smoke/electrical hazard
    \item Weapon malfunction
    \item Multiple critical systems offline
\end{itemize}

\textbf{Priority 2 - Same Day Response:}
\begin{itemize}
    \item Single critical system offline (LRF, camera, servo)
    \item Degraded but operational (reduced capability)
\end{itemize}

\textbf{Priority 3 - Scheduled Maintenance:}
\begin{itemize}
    \item Minor issues (cosmetic damage, non-critical warnings)
    \item Preventive maintenance tasks
    \item Performance optimization
\end{itemize}

% ============================================================================
\section{TROUBLESHOOTING CASE STUDIES}

\subsection{CASE STUDY 1: Tracking Works Initially, Then Fails}

\textbf{Symptoms:}
\begin{itemize}
    \item Tracking locks successfully
    \item After 5-10 minutes, tracking becomes erratic
    \item Eventually loses lock completely
    \item Problem recurs each mission
\end{itemize}

\textbf{SLAF Analysis:}
\begin{itemize}
    \item \textbf{Stop:} Problem is intermittent, temperature-related?
    \item \textbf{Look:} Check System Status → Camera temperature rising
    \item \textbf{Assess:} Camera overheating, thermal FFC becoming more frequent
    \item \textbf{Fix:} Report to maintenance - camera cooling system fault
\end{itemize}

\textbf{Lesson:} Intermittent problems that worsen over time indicate component failure. Don't ignore!

% ----------------------------------------------------------------------------
\subsection{CASE STUDY 2: Gimbal Moves Slowly in One Direction}

\textbf{Symptoms:}
\begin{itemize}
    \item Azimuth movement slow when turning left
    \item Normal speed turning right
    \item Elevation normal
    \item No fault flags in System Status
\end{itemize}

\textbf{SLAF Analysis:}
\begin{itemize}
    \item \textbf{Stop:} Asymmetric problem suggests mechanical issue
    \item \textbf{Look:} Check torque readings → high torque on left turns
    \item \textbf{Assess:} Mechanical resistance (bearing wear, obstruction)
    \item \textbf{Fix:} Escalate to maintenance - mechanical inspection needed
\end{itemize}

\textbf{Lesson:} Directional or asymmetric problems often indicate mechanical issues, not software/electrical.

% ----------------------------------------------------------------------------
\subsection{CASE STUDY 3: System Won't Power Up}

\textbf{Symptoms:}
\begin{itemize}
    \item Main power switch ON, but no lights
    \item No response from any system
    \item Complete dead system
\end{itemize}

\textbf{SLAF Analysis:}
\begin{itemize}
    \item \textbf{Stop:} Total power failure - electrical issue
    \item \textbf{Look:} Check external power source (generator, battery)
    \item \textbf{Assess:} External power available? Cables connected? Circuit breaker tripped?
    \item \textbf{Fix:} 
    \begin{itemize}
        \item If external power issue: Restore external power
        \item If external power OK: Escalate - internal power supply fault
    \end{itemize}
\end{itemize}

\textbf{Lesson:} Start troubleshooting from the outside in. Check simplest causes (external power) before assuming complex internal failures.

% ============================================================================
% END OF LESSON 9
% ============================================================================
% ============================================================================
% LESSON 10 - PRACTICAL TRAINING & EVALUATION
% ============================================================================
\lesson{10 - PRACTICAL TRAINING \& EVALUATION}

\begin{tabular}{@{}ll@{}}
\textbf{Duration:} & 8 hours (Hands-On Training \& Evaluation) \\
\textbf{Type:} & Practical + Evaluation \\
\textbf{References:} & All previous lessons, evaluation checklist \\
\end{tabular}

\section{LEARNING OBJECTIVES}

Upon completion of this lesson, operators will be able to:
\begin{itemize}
    \item Demonstrate proficiency in all operator tasks
    \item Pass written examination (80\% minimum)
    \item Pass practical performance evaluation
    \item Qualify for operational duty
\end{itemize}

% ============================================================================
\section{PRACTICAL TRAINING EXERCISES}

\subsection{EXERCISE 1: SYSTEM STARTUP \& HEALTH CHECK}

\textbf{Duration:} 30 minutes

\subsubsection{Task}
Power up system, perform pre-operation checks

\subsubsection{Performance Standards}

\begin{checklist}
    \item Correct power-up sequence
    \item All devices detected and connected
    \item Pre-operation checklist 100\% complete
    \item Ready for operations within 5 minutes
\end{checklist}

\subsubsection{Evaluation Criteria}

\spectable{
\begin{tabular}{|L{0.40\textwidth}|C{0.15\textwidth}|L{0.35\textwidth}|}
\hline
\rowcolor{milblue!20}
\textbf{Task Element} & \textbf{Weight} & \textbf{Standard} \\
\hline
Power-up sequence & 25\% & All steps in correct order \\
\hline
Device verification & 25\% & All devices confirmed online \\
\hline
Checklist completion & 30\% & All items checked, documented \\
\hline
Time to ready & 20\% & <5 minutes from power-on \\
\hline
\end{tabular}
}

\subsubsection{Common Errors to Avoid}
\begin{itemize}
    \item Skipping System Status check
    \item Not documenting baseline temperatures
    \item Failing to test all motion axes
    \item Rushing through checklist (missing items)
\end{itemize}

% ----------------------------------------------------------------------------
\subsection{EXERCISE 2: MANUAL GIMBAL CONTROL}

\textbf{Duration:} 30 minutes

\subsubsection{Task}
Navigate gimbal to designated azimuth/elevation coordinates

\subsubsection{Performance Standards}

\begin{checklist}
    \item Smooth, controlled movements
    \item Accuracy: ±2\degree\ azimuth, ±2\degree\ elevation
    \item Complete 5 targets within 3 minutes
\end{checklist}

\subsubsection{Target Coordinates}

\spectable{
\begin{tabular}{|C{0.15\textwidth}|C{0.25\textwidth}|C{0.25\textwidth}|C{0.20\textwidth}|}
\hline
\rowcolor{milblue!20}
\textbf{Target} & \textbf{Azimuth} & \textbf{Elevation} & \textbf{Time Limit} \\
\hline
Target 1 & 45\degree & +10\degree & 30 seconds \\
\hline
Target 2 & 180\degree & +20\degree & 30 seconds \\
\hline
Target 3 & 270\degree & -5\degree & 40 seconds \\
\hline
Target 4 & 90\degree & +30\degree & 40 seconds \\
\hline
Target 5 & 0\degree & 0\degree & 40 seconds \\
\hline
\end{tabular}
}

\subsubsection{Evaluation Criteria}

\textbf{GO Criteria:}
\begin{itemize}
    \item All targets acquired within ±2\degree
    \item No jerky or erratic motion
    \item No gimbal limit violations
    \item Completed within time limit
\end{itemize}

\textbf{NO-GO Criteria:}
\begin{itemize}
    \item Accuracy >±5\degree\ on any target
    \item Exceeds time limit by >50\%
    \item Gimbal limit violation
    \item Unsafe control (rapid, uncontrolled movements)
\end{itemize}

% ----------------------------------------------------------------------------
\subsection{EXERCISE 3: TARGET ACQUISITION \& TRACKING}

\textbf{Duration:} 1 hour

\subsubsection{Task}
Acquire and track stationary and moving targets

\subsubsection{Performance Standards}

\begin{checklist}
    \item Acquisition gate correctly sized (10-30\% margin)
    \item Lock achieved within 10 seconds
    \item Track maintained >30 seconds
    \item Clean abort on command
\end{checklist}

\subsubsection{Scenarios}

\textbf{Scenario A: Stationary Target}
\begin{enumerate}
    \item Instructor designates stationary target
    \item Student acquires and locks tracking
    \item Maintain track for 30 seconds
    \item Abort tracking on command
\end{enumerate}

\textbf{Scenario B: Slow-Moving Target}
\begin{enumerate}
    \item Instructor designates moving target (vehicle, simulated)
    \item Student acquires and locks tracking
    \item Maintain track through target motion
    \item Monitor track confidence (should remain >70\%)
\end{enumerate}

\textbf{Scenario C: Fast-Moving Target}
\begin{enumerate}
    \item Instructor designates fast-moving target
    \item Student adjusts FOV (zoom out if needed)
    \item Acquire and lock tracking
    \item Manage "LEAD ANGLE LAG" warning (wait for green)
\end{enumerate}

\subsubsection{Evaluation Criteria}

\spectable{
\begin{tabular}{|L{0.40\textwidth}|C{0.15\textwidth}|L{0.35\textwidth}|}
\hline
\rowcolor{milblue!20}
\textbf{Task Element} & \textbf{Weight} & \textbf{Standard} \\
\hline
Gate sizing & 20\% & 10-30\% margin, target fully visible \\
\hline
Lock achievement & 30\% & <10 seconds from gate sizing \\
\hline
Track maintenance & 30\% & >30 seconds, confidence >70\% \\
\hline
Abort execution & 20\% & Immediate, clean, returns to manual \\
\hline
\end{tabular}
}

% ----------------------------------------------------------------------------
\subsection{EXERCISE 4: SIMULATED ENGAGEMENT}

\textbf{Duration:} 1 hour

\subsubsection{Task}
Complete engagement sequence (no live fire)

\subsubsection{Performance Standards}

\begin{checklist}
    \item PID confirmed before engagement
    \item Tracking established (Active Lock)
    \item LAC activated (moving target)
    \item Safety checks complete
    \item Simulated fire (correct procedure)
    \item BDA (Battle Damage Assessment)
\end{checklist}

\subsubsection{Engagement Sequence}

\begin{procedurebox}[SIMULATED ENGAGEMENT]
\textbf{Phase 1: Target Detection \& Identification}
\begin{enumerate}
    \item Instructor designates target (verbal or visual cue)
    \item Student detects and slews to target
    \item Student identifies target (announces type, description)
    \item Student confirms PID and ROE compliance
\end{enumerate}

\textbf{Phase 2: Target Acquisition}
\begin{enumerate}
    \item Enter acquisition mode (\button{Button 4})
    \item Size gate appropriately
    \item Request lock-on (\button{Button 4})
    \item Achieve Active Lock (green gate)
\end{enumerate}

\textbf{Phase 3: Fire Control Setup}
\begin{enumerate}
    \item Range target (LRF or manual estimation)
    \item Enable LAC if target moving (\button{Button 3 + Button 2})
    \item Verify LAC status (LEAD ANGLE ON if applicable)
    \item Announce range and lead status
\end{enumerate}

\textbf{Phase 4: Safety Checks}
\begin{enumerate}
    \item Verify no zone violations (check OSD)
    \item Verify friendlies clear (announce "CLEAR")
    \item Verify target still valid
    \item Announce "READY TO ENGAGE"
\end{enumerate}

\textbf{Phase 5: Simulated Engagement}
\begin{enumerate}
    \item Engage Master Arm (\button{Button 0})
    \item Announce "ARMED"
    \item Simulate fire (\textit{do not press Button 5, announce "FIRE"})
    \item Announce "CEASE FIRE"
    \item Disarm (\button{Button 0})
\end{enumerate}

\textbf{Phase 6: Battle Damage Assessment}
\begin{enumerate}
    \item Observe simulated impact
    \item Announce BDA result (destroyed/damaged/missed)
    \item Stop tracking if target neutralized
\end{enumerate}
\end{procedurebox}

\subsubsection{Critical Errors (Automatic Failure)}

\begin{itemize}
    \item ❌ No PID before engagement
    \item ❌ Zone violation not recognized
    \item ❌ Friendlies not checked
    \item ❌ Wrong target engaged
    \item ❌ Skipped safety checks
\end{itemize}

% ----------------------------------------------------------------------------
\subsection{EXERCISE 5: WEAPON ZEROING}

\textbf{Duration:} 1 hour

\subsubsection{Task}
Zero weapon at 200m range

\subsubsection{Performance Standards}

\begin{checklist}
    \item Correct zeroing procedure
    \item Zero applied successfully
    \item Verification shot within ±5cm
    \item "Z" indicator appears on OSD
\end{checklist}

\subsubsection{Zeroing Procedure}

\begin{enumerate}
    \item Access Zeroing menu (\button{MENU ✓} → Zeroing → \button{VAL})
    \item Aim at target center (200m range)
    \item Fire test shot (observe impact location)
    \item Move reticle to impact point (joystick)
    \item Apply zero (\button{MENU ✓})
    \item Verify "Z" indicator on OSD
    \item Fire verification shot (should hit reticle center)
\end{enumerate}

\subsubsection{Evaluation Criteria}

\textbf{GO Criteria:}
\begin{itemize}
    \item Correct procedure followed
    \item Zero applied (Z indicator visible)
    \item Verification shot within ±5cm of aim point
\end{itemize}

\textbf{NO-GO Criteria:}
\begin{itemize}
    \item Incorrect procedure (steps out of order)
    \item Zero not applied (no Z indicator)
    \item Verification shot >±10cm from aim point
\end{itemize}

\begin{notebox}
If verification shot fails, student may retry zeroing procedure once.
\end{notebox}

% ----------------------------------------------------------------------------
\subsection{EXERCISE 6: EMERGENCY PROCEDURES}

\textbf{Duration:} 1 hour

\subsubsection{Task}
Respond to simulated emergencies

\subsubsection{Scenarios}

\textbf{Scenario 1: Wrong Target Identified}
\begin{itemize}
    \item \textbf{Setup:} Student tracking target
    \item \textbf{Event:} Instructor announces "FRIENDLY FORCES IN TRACKING GATE"
    \item \textbf{Required Action:} Immediate tracking abort (\button{Button 4} double-click)
    \item \textbf{Standard:} Abort within 500ms of announcement
\end{itemize}

\textbf{Scenario 2: Runaway Gimbal}
\begin{itemize}
    \item \textbf{Setup:} Instructor simulates runaway gimbal (unexpected motion)
    \item \textbf{Event:} Gimbal moves erratically without joystick input
    \item \textbf{Required Action:} Immediate E-Stop, then Station Power OFF
    \item \textbf{Standard:} E-Stop within 1 second, power OFF within 3 seconds
\end{itemize}

\textbf{Scenario 3: Accidental Discharge}
\begin{itemize}
    \item \textbf{Setup:} Simulated weapon armed
    \item \textbf{Event:} Instructor announces "WEAPON FIRING" (unintended)
    \item \textbf{Required Action:} Release fire, release Master Arm, point safe, E-Stop, power OFF
    \item \textbf{Standard:} All steps within 5 seconds
\end{itemize}

\textbf{Scenario 4: Lost Communication}
\begin{itemize}
    \item \textbf{Setup:} Operator performing surveillance
    \item \textbf{Event:} Instructor simulates radio failure
    \item \textbf{Required Action:} Switch backup radio, attempt recovery, follow SOP
    \item \textbf{Standard:} Backup radio within 30 seconds, correct SOP followed
\end{itemize}

\subsubsection{Performance Standards}

\begin{checklist}
    \item Immediate recognition of emergency (<2 seconds)
    \item Correct procedure executed
    \item Reaction time <2 seconds (critical emergencies)
    \item All safety steps completed
\end{checklist}

% ----------------------------------------------------------------------------
\subsection{EXERCISE 7: TROUBLESHOOTING}

\textbf{Duration:} 1 hour

\subsubsection{Task}
Diagnose and resolve simulated faults

\subsubsection{Faults Presented}

\textbf{Fault 1: Joystick Disconnected}
\begin{itemize}
    \item \textbf{Symptom:} No gimbal response, no joystick LED
    \item \textbf{Expected Diagnosis:} USB cable disconnected
    \item \textbf{Expected Fix:} Reconnect USB, verify System Status
\end{itemize}

\textbf{Fault 2: Camera Offline}
\begin{itemize}
    \item \textbf{Symptom:} Black screen, "No Signal"
    \item \textbf{Expected Diagnosis:} Camera disconnected or fault
    \item \textbf{Expected Fix:} Switch to alternate camera, power cycle if both offline
\end{itemize}

\textbf{Fault 3: Tracking Failure}
\begin{itemize}
    \item \textbf{Symptom:} Cannot achieve lock, stays in LOCK PENDING
    \item \textbf{Expected Diagnosis:} Poor target contrast or gate sizing
    \item \textbf{Expected Fix:} Improve contrast (switch camera/LUT), resize gate
\end{itemize}

\textbf{Fault 4: LRF No Reading}
\begin{itemize}
    \item \textbf{Symptom:} Persistent "NO ECHO"
    \item \textbf{Expected Diagnosis:} Target too far or obscured
    \item \textbf{Expected Fix:} Re-fire on closer/more reflective target, or manual range estimation
\end{itemize}

\subsubsection{Performance Standards}

\begin{checklist}
    \item Correct diagnosis using SLAF method
    \item Operator-level fix applied (if applicable)
    \item Proper escalation to maintenance (if needed)
    \item Solution verified (problem resolved)
\end{checklist}

\subsubsection{Evaluation Criteria}

\spectable{
\begin{tabular}{|L{0.40\textwidth}|C{0.15\textwidth}|L{0.35\textwidth}|}
\hline
\rowcolor{milblue!20}
\textbf{Task Element} & \textbf{Weight} & \textbf{Standard} \\
\hline
Problem diagnosis & 40\% & Correct root cause identified \\
\hline
Troubleshooting method & 20\% & SLAF method applied correctly \\
\hline
Fix application & 30\% & Appropriate fix attempted \\
\hline
Escalation decision & 10\% & Correct escalation if needed \\
\hline
\end{tabular}
}

% ============================================================================
\section{WRITTEN EXAMINATION}

\subsection{EXAMINATION FORMAT}

\textbf{Format:} 50 multiple-choice questions

\textbf{Passing Score:} 80\% (40/50 correct)

\textbf{Time Limit:} 60 minutes

\subsection{Topics Covered}

\spectable{
\begin{tabular}{|L{0.50\textwidth}|C{0.25\textwidth}|C{0.15\textwidth}|}
\hline
\rowcolor{milblue!20}
\textbf{Topic} & \textbf{Questions} & \textbf{Weight} \\
\hline
Safety procedures & 10 & 20\% \\
\hline
Basic operation & 10 & 20\% \\
\hline
Tracking \& engagement & 10 & 20\% \\
\hline
Ballistics \& fire control & 10 & 20\% \\
\hline
Emergency procedures & 5 & 10\% \\
\hline
System status \& troubleshooting & 5 & 10\% \\
\hline
\textbf{TOTAL} & \textbf{50} & \textbf{100\%} \\
\hline
\end{tabular}
}

\subsection{Sample Questions}

\subsubsection{Safety Procedures}
\begin{enumerate}
    \item When should the E-Stop button be activated?
    \begin{itemize}
        \item A) Only when ordered by command
        \item B) When personnel are in line of fire
        \item C) After every mission
        \item D) Only during maintenance
    \end{itemize}
    \textbf{Correct Answer: B}
\end{enumerate}

\subsubsection{Tracking \& Engagement}
\begin{enumerate}
    \item What is the correct acquisition gate sizing margin around the target?
    \begin{itemize}
        \item A) 0-5\% margin
        \item B) 10-30\% margin
        \item C) 50-75\% margin
        \item D) Gate should be as tight as possible
    \end{itemize}
    \textbf{Correct Answer: B}
\end{enumerate}

\subsubsection{Emergency Procedures}
\begin{enumerate}
    \item How do you perform an emergency tracking abort?
    \begin{itemize}
        \item A) Press E-Stop button
        \item B) Single-click Button 4
        \item C) Double-click Button 4 (<500ms)
        \item D) Turn Station Power OFF
    \end{itemize}
    \textbf{Correct Answer: C}
\end{enumerate}

\subsection{Examination Rules}

\begin{itemize}
    \item Closed book (no reference materials)
    \item No electronic devices
    \item Questions may be asked for clarification (instructor will not provide answers)
    \item Review allowed if time permits
    \item Minimum 80\% required to pass
\end{itemize}

% ============================================================================
\section{PERFORMANCE EVALUATION}

\subsection{EVALUATION FORMAT}

\textbf{Format:} Instructor-observed practical assessment

\textbf{Duration:} 2-3 hours

\textbf{Passing Standard:} 80\% overall, NO critical errors

\subsection{Evaluation Criteria}

\begin{longtable}{|L{0.35\textwidth}|C{0.15\textwidth}|L{0.40\textwidth}|}
\hline
\rowcolor{milblue!20}
\textbf{Task} & \textbf{Weight} & \textbf{GO/NO-GO Criteria} \\
\hline
\endfirsthead

\hline
\rowcolor{milblue!20}
\textbf{Task} & \textbf{Weight} & \textbf{GO/NO-GO Criteria} \\
\hline
\endhead

\textbf{System Startup} & 10\% & All steps correct, <5 min \\
\hline
\textbf{Manual Control} & 10\% & Smooth, accurate (±2\degree) \\
\hline
\textbf{Tracking} & 20\% & Lock achieved, maintained >30s \\
\hline
\textbf{Simulated Engagement} & 25\% & Full sequence, all safety checks \\
\hline
\textbf{Zeroing} & 15\% & Correct procedure, accurate \\
\hline
\textbf{Emergency Response} & 15\% & Immediate, correct procedure \\
\hline
\textbf{Troubleshooting} & 5\% & Correct diagnosis \& action \\
\hline
\textbf{TOTAL} & \textbf{100\%} & \textbf{≥80\% required} \\
\hline
\end{longtable}

\subsection{Critical Errors (Automatic Failure)}

\textbf{Any of the following results in automatic NO-GO:}

\begin{itemize}
    \item ❌ Safety violation (no-fire zone, friendly fire)
    \item ❌ Failed to E-Stop when required
    \item ❌ Accidental discharge (simulated)
    \item ❌ Wrong target engagement
    \item ❌ Bypassed safety interlocks
    \item ❌ Damaged equipment through negligence
\end{itemize}

\subsection{Scoring}

\subsubsection{Individual Task Scoring}

Each task scored 0-100\%:
\begin{itemize}
    \item \textbf{100\%:} Flawless execution
    \item \textbf{90\%:} Minor errors, corrected immediately
    \item \textbf{80\%:} Meets minimum standard
    \item \textbf{70\%:} Below standard, requires additional training
    \item \textbf{<70\%:} Unsatisfactory, significant deficiencies
\end{itemize}

\subsubsection{Overall Score Calculation}

\begin{enumerate}
    \item Calculate weighted score for each task
    \item Sum all weighted scores = Overall Score
    \item Check for critical errors (automatic failure if any)
    \item Final result: GO (≥80\%, no critical errors) or NO-GO (<80\% or critical error)
\end{enumerate}

\textbf{Example Calculation:}
\begin{itemize}
    \item System Startup: 90\% × 10\% = 9 points
    \item Manual Control: 85\% × 10\% = 8.5 points
    \item Tracking: 95\% × 20\% = 19 points
    \item Simulated Engagement: 80\% × 25\% = 20 points
    \item Zeroing: 90\% × 15\% = 13.5 points
    \item Emergency Response: 100\% × 15\% = 15 points
    \item Troubleshooting: 75\% × 5\% = 3.75 points
    \item \textbf{Overall Score: 88.75\% = GO}
\end{itemize}

% ============================================================================
\section{QUALIFICATION REQUIREMENTS}

\subsection{TO QUALIFY AS EL 7ARRESS RCWS OPERATOR}

\textbf{Must pass ALL:}

\begin{checklist}
    \item Written examination (≥80\%)
    \item Practical evaluation (≥80\%)
    \item Zero critical errors
    \item Instructor recommendation
\end{checklist}

\subsection{Qualification Certificate}

Upon successful completion, operator receives:
\begin{itemize}
    \item Certificate of Qualification
    \item Operator ID card
    \item Entry in training records
    \item Authorization for operational duty
\end{itemize}

\textbf{Qualification Valid:} 12 months from date of issue

% ----------------------------------------------------------------------------
\subsection{REQUALIFICATION REQUIREMENTS}

Requalification required:
\begin{itemize}
    \item \textbf{Annually} (refresher training + re-test)
    \item \textbf{After >6 months away from system}
    \item \textbf{After major system upgrade}
    \item \textbf{If unsafe practices observed}
    \item \textbf{Upon commander's discretion}
\end{itemize}

\subsubsection{Requalification Process}

\textbf{Annual Requalification:}
\begin{enumerate}
    \item 4-hour refresher training (classroom)
    \item Abbreviated written exam (25 questions, 80\% required)
    \item Practical evaluation (key tasks only: engagement, emergencies)
    \item Verification of zero proficiency
\end{enumerate}

\textbf{Extended Absence (>6 months):}
\begin{enumerate}
    \item Full course review recommended
    \item Full written examination
    \item Full practical evaluation
    \item May require full re-qualification at instructor's discretion
\end{enumerate}

% ============================================================================
\section{REMEDIAL TRAINING}

\subsection{IF STUDENT FAILS QUALIFICATION}

\textbf{Failed Written Exam:}
\begin{itemize}
    \item Review incorrect answers with instructor
    \item Additional study (weak areas identified)
    \item Re-test after minimum 24 hours
    \item Maximum 2 re-test attempts
\end{itemize}

\textbf{Failed Practical Evaluation:}
\begin{itemize}
    \item Review performance with instructor (identify deficiencies)
    \item Additional practice on failed tasks
    \item Re-test after minimum 48 hours
    \item Maximum 2 re-test attempts
\end{itemize}

\textbf{Critical Error:}
\begin{itemize}
    \item Immediate training halt
    \item Safety review with instructor and commander
    \item Mandatory additional training (duration determined by commander)
    \item Re-start evaluation from beginning
\end{itemize}

\subsection{Maximum Attempts}

\begin{itemize}
    \item \textbf{Written Exam:} 3 attempts maximum (initial + 2 retakes)
    \item \textbf{Practical Eval:} 3 attempts maximum (initial + 2 retakes)
    \item \textbf{If failed after 3 attempts:} Student removed from training, command decision on future training
\end{itemize}

% ============================================================================
\section{POST-QUALIFICATION}

\subsection{CONTINUING EDUCATION}

Qualified operators are expected to:
\begin{itemize}
    \item Maintain proficiency through regular operations
    \item Attend quarterly refresher training sessions
    \item Review operator manual periodically
    \item Stay current on system updates and modifications
    \item Mentor junior operators (once experienced)
\end{itemize}

\subsection{ADVANCED TRAINING}

Experienced operators may qualify for advanced courses:
\begin{itemize}
    \item \textbf{Instructor Operator:} Train new operators
    \item \textbf{Maintainer Cross-Training:} Operator-level maintenance
    \item \textbf{Tactics Development:} Advanced employment techniques
    \item \textbf{System Specialist:} Deep technical expertise
\end{itemize}

\subsection{PERFORMANCE TRACKING}

Operator performance tracked through:
\begin{itemize}
    \item Mission logs (accuracy, efficiency)
    \item Incident reports (any safety issues)
    \item Peer evaluations (teamwork, professionalism)
    \item Annual requalification scores
\end{itemize}

\textbf{Outstanding performance may result in:}
\begin{itemize}
    \item Commendation
    \item Selection for advanced training
    \item Instructor duty assignment
    \item Leadership positions
\end{itemize}

% ============================================================================
\section{FINAL CHECKLIST}

\subsection{BEFORE FINAL EVALUATION}

\textbf{Student should verify:}

\begin{checklist}
    \item Attended all lessons (1-10)
    \item Completed all practical exercises
    \item Reviewed all reference materials
    \item Practiced emergency procedures
    \item Comfortable with all system functions
    \item Any questions answered by instructor
    \item Well-rested and prepared for evaluation day
\end{checklist}

\subsection{ON EVALUATION DAY}

\textbf{Bring:}
\begin{itemize}
    \item Government-issued ID
    \item Training record
    \item Writing materials (pen/pencil for written exam)
    \item Appropriate uniform
    \item Positive attitude and confidence
\end{itemize}

\textbf{Arrive:}
\begin{itemize}
    \item 15 minutes before scheduled time
    \item Well-rested and alert
    \item Ready to demonstrate proficiency
\end{itemize}

% ============================================================================
\section{CONCLUSION}

\subsection{IMPORTANCE OF QUALIFICATION}

El 7arress RCWS operators hold a position of great responsibility:
\begin{itemize}
    \item Control of lethal weapon system
    \item Protection of friendly forces
    \item Engagement of hostile threats
    \item Safety of personnel and equipment
\end{itemize}

Qualification ensures operators are:
\begin{itemize}
    \item Technically proficient
    \item Safety-conscious
    \item Tactically sound
    \item Ready for operational duty
\end{itemize}

\subsection{FINAL WORDS}

\begin{quotation}
\textit{``A qualified operator is not just someone who knows how to operate the system - it is someone who can be trusted with a weapon, who makes sound decisions under pressure, and who puts safety and mission accomplishment above all else.''}

\hfill --- El 7arress Training Doctrine
\end{quotation}

\vspace{1cm}

\textbf{Good luck on your qualification!}

% ============================================================================
% END OF LESSON 10
% ============================================================================
% ============================================================================
% APPENDIX A - QUICK REFERENCE CARDS
% ============================================================================
\appendixchapter{A - QUICK REFERENCE CARDS}

\section{STARTUP SEQUENCE}

\begin{quickref}
	\textbf{SYSTEM STARTUP PROCEDURE}
	
	\begin{enumerate}
		\item Main Power ON
		\item Station Enable ON
		\item Wait for boot (~60 seconds)
		\item Check System Status (all devices connected)
		\item Perform pre-operation checks
		\item Ready for operations
	\end{enumerate}
\end{quickref}

% ============================================================================
\section{CONTROL REFERENCE}

\subsection{JOYSTICK}

\spectable{
	\begin{tabular}{|L{0.18\textwidth}|L{0.72\textwidth}|}
		\hline
		\rowcolor{milblue!20}
		\textbf{Control} & \textbf{Function} \\
		\hline
		\textbf{X-Axis} & Azimuth (left/right) \\
		\hline
		\textbf{Y-Axis} & Elevation (up/down) \\
		\hline
		\textbf{Button 0} & Fire Intent (half-press) - Engages LAC if armed, enters dead reckoning during tracking \\
		\hline
		\textbf{Button 1} & LRF Trigger (single=measure, double-click=toggle continuous 5Hz mode) \\
		\hline
		\textbf{Button 2} & LAC Toggle (hold Button 3) - Arms/disarms LAC. 2-second minimum between toggles \\
		\hline
		\textbf{Button 3} & Dead Man Switch (Palm Switch) - Must be held for weapon ops, LAC, tracking \\
		\hline
		\textbf{Button 4} & Track Select (single=advance phase, double-click <1sec=abort tracking) \\
		\hline
		\textbf{Button 5} & Fire - Press=fire, Release=stop. Requires all safety conditions \\
		\hline
		\textbf{Button 6} & Zoom In (continuous while held) \\
		\hline
		\textbf{Button 7} & Video LUT Next (thermal camera only) \\
		\hline
		\textbf{Button 8} & Zoom Out (continuous while held) \\
		\hline
		\textbf{Button 9} & Video LUT Previous (thermal camera only) \\
		\hline
		\textbf{Button 10} & LRF Clear - Clears range measurement only (NOT zeroing) \\
		\hline
		\textbf{Button 11/13} & Motion Mode Cycle - Blocked during tracking \\
		\hline
		\textbf{Button 12} & Exit to Manual Mode - Blocked during tracking \\
		\hline
		\textbf{D-Pad} & Acquisition gate sizing (in Acquisition mode only) \\
		\hline
	\end{tabular}
}

% ----------------------------------------------------------------------------
\subsection{CONTROL PANEL (DCU)}

\spectable{
	\begin{tabular}{|L{0.25\textwidth}|L{0.65\textwidth}|}
		\hline
		\rowcolor{milblue!20}
		\textbf{Control} & \textbf{Function} \\
		\hline
		\textbf{E-STOP} & Emergency Stop (red mushroom) \\
		\hline
		\textbf{Station Enable} & Power on/off weapon station \\
		\hline
		\textbf{Gun Arm / Safe} & Weapon arming switch \\
		\hline
		\textbf{MENU ✓} & Menu access / Validate \\
		\hline
		\textbf{▲ / ▼} & Menu navigation \\
		\hline
		\textbf{Authorization Key} & Enable weapon fire \\
		\hline
		\textbf{Indicator Lights} & System status (Armed, Tracking, Faults) \\
		\hline
	\end{tabular}
}

% ============================================================================
\section{OSD INDICATORS}

\begin{longtable}{|L{0.30\textwidth}|L{0.60\textwidth}|}
	\hline
	\rowcolor{milblue!20}
	\textbf{Indicator} & \textbf{Meaning} \\
	\hline
	\endfirsthead

	\hline
	\rowcolor{milblue!20}
	\textbf{Indicator} & \textbf{Meaning} \\
	\hline
	\endhead

	\textbf{Z} & Zeroing active \\
	\hline
	\textbf{ENV} & Environmental parameters active \\
	\hline
	\textcolor{cyan}{\textbf{LAC ARMED}} (CYAN) & LAC armed, awaiting fire trigger. Lead NOT applied \\
	\hline
	\textcolor{milgreen}{\textbf{LEAD ANGLE ON}} (GREEN) & LAC engaged during firing, lead actively applied \\
	\hline
	\textcolor{milyellow}{\textbf{LEAD ANGLE LAG}} (YELLOW) & LAC waiting for tracking data \\
	\hline
	\textcolor{milred}{\textbf{ZOOM OUT}} (RED) & LAC FOV insufficient \\
	\hline
	\textcolor{milred}{\textbf{ZONE VIOLATION}} (RED) & No-Fire/No-Traverse zone warning \\
	\hline
	\textbf{TRACKING} & Tracking active \\
	\hline
	\textbf{ACQUISITION} & Acquisition mode \\
	\hline
	\textbf{COAST} & Tracking lost, predicting target position \\
	\hline
	\textbf{FREE} & Gimbal brakes released, manual positioning \\
	\hline
	\textbf{RNG: xxxx m} & Laser range reading \\
	\hline
\end{longtable}

% ============================================================================
\section{EMERGENCY ACTIONS}

\begin{longtable}{|L{0.30\textwidth}|L{0.60\textwidth}|}
	\hline
	\rowcolor{milred!20}
	\textbf{Emergency} & \textbf{Immediate Action} \\
	\hline
	\endfirsthead
	
	\hline
	\rowcolor{milred!20}
	\textbf{Emergency} & \textbf{Immediate Action} \\
	\hline
	\endhead
	
	\textbf{Personnel in danger} & \textbf{E-STOP} \\
	\hline
	\textbf{Wrong target} & \textbf{\button{Button 4} (double-click)} \\
	\hline
	\textbf{Accidental fire} & \textbf{Release \button{Button 5}, Release \button{Button 0}} \\
	\hline
	\textbf{Runaway gimbal} & \textbf{E-STOP → Station Power OFF} \\
	\hline
	\textbf{Fire/Smoke} & \textbf{E-STOP → Main Power OFF} \\
	\hline
\end{longtable}

% ============================================================================
\section{TRACKING QUICK REFERENCE}

\begin{quickref}
	\textbf{TRACKING SEQUENCE}
	
	\begin{enumerate}
		\item \textbf{Start Acquisition:} \button{Button 4} (single press)
		\item \textbf{Size Gate:} \button{D-Pad} (10-30\% margin)
		\item \textbf{Request Lock:} \button{Button 4} (second press)
		\item \textbf{Monitor Track:} Green gate = good, Yellow = marginal
		\item \textbf{Emergency Abort:} \button{Button 4} (double-click <500ms)
	\end{enumerate}
\end{quickref}

% ============================================================================
\section{ENGAGEMENT CHECKLIST}

\begin{quickref}
	\textbf{PRE-ENGAGEMENT VERIFICATION}
	
	\begin{checklist}
		\item Target positively identified (PID)
		\item Target valid (meets ROE)
		\item Fire authorization received
		\item NOT in no-fire zone
		\item Friendly forces clear
		\item Weapon loaded and ready
		\item Zeroing active (Z indicator)
		\item Track established (if using tracking)
	\end{checklist}
	
	\textbf{IF ANY ITEM CANNOT BE CHECKED: DO NOT FIRE}
\end{quickref}

% ============================================================================
\section{BALLISTICS QUICK REFERENCE}

\subsection{Zeroing Status}

\spectable{
	\small
	\begin{tabular}{|L{0.25\textwidth}|L{0.35\textwidth}|L{0.30\textwidth}|}
		\hline
		\rowcolor{milblue!20}
		\textbf{OSD Indicator} & \textbf{Meaning} & \textbf{Action} \\
		\hline
		\textbf{Z} visible & Zeroing active & Normal ops \\
		\hline
		No \textbf{Z} & No zero applied & Re-zero before firing \\
		\hline
		\textbf{ENV} visible & Environmental corrections active & Normal ops \\
		\hline
		\textbf{LEAD ANGLE ON} & LAC active & Aim at CCIP reticle \\
		\hline
	\end{tabular}
}

% ----------------------------------------------------------------------------
\subsection{Fire Control Combinations}

\spectable{
	\small
	\begin{tabular}{|L{0.22\textwidth}|L{0.22\textwidth}|C{0.08\textwidth}|C{0.08\textwidth}|C{0.08\textwidth}|L{0.20\textwidth}|}
		\hline
		\rowcolor{milblue!20}
		\textbf{Target} & \textbf{Conditions} & \textbf{Z} & \textbf{ENV} & \textbf{LAC} & \textbf{OSD Display} \\
		\hline
		Stationary & Standard & ✓ & & & Z \\
		\hline
		Stationary & Extreme temp/alt & ✓ & ✓ & & Z ENV \\
		\hline
		Moving (fast) & Standard & ✓ & & ✓ & Z LEAD ANGLE ON \\
		\hline
		Moving (fast) & Extreme & ✓ & ✓ & ✓ & Z ENV LEAD ANGLE ON \\
		\hline
	\end{tabular}
}

% ============================================================================
\section{TROUBLESHOOTING QUICK GUIDE}

\begin{longtable}{|L{0.25\textwidth}|L{0.35\textwidth}|L{0.30\textwidth}|}
	\hline
	\rowcolor{milblue!20}
	\textbf{Symptom} & \textbf{Quick Fix} & \textbf{Escalate If} \\
	\hline
	\endfirsthead
	
	\hline
	\rowcolor{milblue!20}
	\textbf{Symptom} & \textbf{Quick Fix} & \textbf{Escalate If} \\
	\hline
	\endhead
	
	No joystick & Reconnect USB, reboot & Still offline \\
	\hline
	Black video & Switch cameras & Both offline \\
	\hline
	Won't track & Resize gate, improve contrast & Fails all targets \\
	\hline
	Gimbal frozen & Reset E-Stop, Station ON & Servo fault \\
	\hline
	LRF no reading & Re-fire, try different target & LRF disconnected \\
	\hline
	High temp & Reduce ops, cool down & Temp >80\degree C \\
	\hline
\end{longtable}

% ============================================================================
\section{SYSTEM STATUS QUICK CHECK}

\begin{quickref}
	\textbf{HEALTHY SYSTEM INDICATORS}
	
	\begin{itemize}
		\item All devices: \textbf{Connected} ✓
		\item Servo torque: \textbf{0-50\%}
		\item Temperatures: \textbf{20-60\degree C}
		\item No fault flags
		\item Track confidence: \textbf{>70\%}
		\item No alarms (RED or YELLOW)
	\end{itemize}
	
	\textbf{DEGRADED SYSTEM INDICATORS}
	
	\begin{itemize}
		\item Servo torque: \textbf{50-80\%} (marginal)
		\item Temperatures: \textbf{60-70\degree C} (warning)
		\item Track confidence: \textbf{50-70\%} (marginal)
		\item YELLOW alarms present
	\end{itemize}
	
	\textbf{FAULTED SYSTEM (NO-GO)}
	
	\begin{itemize}
		\item Any device: \textbf{Disconnected} ✗
		\item Servo torque: \textbf{>80\%} sustained
		\item Temperatures: \textbf{>70\degree C}
		\item Fault flags present
		\item RED alarms present
	\end{itemize}
\end{quickref}

% ============================================================================
\section{ENVIRONMENTAL PARAMETERS GUIDE}

\subsection{When to Apply Environmental Settings}

\spectable{
	\small
	\begin{tabular}{|L{0.30\textwidth}|C{0.12\textwidth}|C{0.12\textwidth}|C{0.12\textwidth}|C{0.12\textwidth}|}
		\hline
		\rowcolor{milblue!20}
		\textbf{Situation} & \textbf{Temp} & \textbf{Alt} & \textbf{Wind} & \textbf{Apply?} \\
		\hline
		Sea level, standard, calm & 15\degree C & 0m & 0 m/s & No \\
		\hline
		Desert, hot, calm & 40\degree C & 0m & 0 m/s & Yes \\
		\hline
		Mountain, cold, windy & -10\degree C & 2000m & 8 m/s & Yes \\
		\hline
		Temperate, mild breeze & 20\degree C & 300m & 3 m/s & Optional \\
		\hline
	\end{tabular}
}

% ============================================================================
\section{LAC (LEAD ANGLE COMPENSATION) GUIDE}

\begin{quickref}
	\textbf{LAC ARMING (Two-Stage Workflow)}

	\begin{enumerate}
		\item Establish Active Lock (green gate)
		\item Hold \button{Button 3} (Dead Man Switch)
		\item Press \button{Button 2} (LAC Toggle)
		\item Release both buttons
		\item Verify: \textcolor{cyan}{\textbf{LAC ARMED}} (CYAN) - Lead NOT yet applied
		\item Press fire trigger (Button 0) to \textbf{engage} LAC
		\item Status changes to \textcolor{milgreen}{\textbf{LEAD ANGLE ON}} (GREEN) - Lead applied
	\end{enumerate}

	\textbf{LAC STATUS MEANINGS}

	\begin{itemize}
		\item \textcolor{cyan}{\textbf{LAC ARMED}} (CYAN): Armed, awaiting fire trigger
		\item \textcolor{milgreen}{\textbf{LEAD ANGLE ON}} (GREEN): Engaged, lead applied during firing
		\item \textcolor{milyellow}{\textbf{LEAD ANGLE LAG}} (YELLOW): Wait 2-5 seconds
		\item \textcolor{milred}{\textbf{ZOOM OUT}} (RED): Zoom out, LAC non-functional
	\end{itemize}

	\textbf{WHEN TO USE LAC}

	\begin{itemize}
		\item ✅ Target moving laterally (crossing FOV)
		\item ✅ Target speed >5 m/s
		\item ✅ Range >100 meters
	\end{itemize}

	\textbf{WHEN NOT TO USE LAC}

	\begin{itemize}
		\item ❌ Stationary targets
		\item ❌ Targets moving radially (toward/away)
		\item ❌ Very close range (<50m)
	\end{itemize}

	\textbf{IMPORTANT:} Wait 2 seconds before re-arming LAC. System blocks early attempts.
\end{quickref}

% ============================================================================
\section{SAFETY REMINDERS}

\begin{warningbox}[CRITICAL SAFETY RULES]
	\textbf{ALWAYS:}
	\begin{itemize}
		\item Verify target before engaging (PID mandatory)
		\item Check for friendly forces
		\item Verify NOT in no-fire zone
		\item Follow Rules of Engagement (ROE)
		\item Have fire authorization (if required)
	\end{itemize}
	
	\textbf{NEVER:}
	\begin{itemize}
		\item Fire without positive identification
		\item Fire into no-fire zones
		\item Fire if friendlies potentially in line of fire
		\item Fire without authorization (if required)
		\item Bypass safety interlocks
	\end{itemize}
\end{warningbox}

% ============================================================================
% END OF APPENDIX A
% ============================================================================
% ============================================================================
% APPENDIX B - TECHNICAL SPECIFICATIONS
% ============================================================================
\appendixchapter{B - TECHNICAL SPECIFICATIONS}

\section{SYSTEM CAPABILITIES}

\subsection{Gimbal Performance}

\spectable{
	\begin{tabular}{|L{0.35\textwidth}|L{0.55\textwidth}|}
		\hline
		\rowcolor{milblue!20}
		\textbf{Parameter} & \textbf{Specification} \\
		\hline
		\textbf{Azimuth Range} & 360\degree\ continuous \\
		\hline
		\textbf{Elevation Range} & -20\degree\ to +60\degree \\
		\hline
		\textbf{Azimuth Speed} & 0-60\degree/sec \\
		\hline
		\textbf{Elevation Speed} & 0-45\degree/sec \\
		\hline
		\textbf{Pointing Accuracy} & ±0.5\degree \\
		\hline
		\textbf{Stabilization} & 3-axis (roll, pitch, yaw) \\
		\hline
		\textbf{Max Weapon Caliber} & 12.7mm (.50 cal) \\
		\hline
	\end{tabular}
}

\subsection{Performance Notes}

\subsubsection{Azimuth Range}
\begin{itemize}
	\item Full 360\degree\ continuous rotation
	\item No physical stops or cable wrap concerns
	\item Slip ring technology for continuous rotation
\end{itemize}

\subsubsection{Elevation Range}
\begin{itemize}
	\item Depression: -20\degree\ (below horizon)
	\item Elevation: +60\degree\ (above horizon)
	\item Typical engagement range: -10\degree\ to +30\degree
	\item Mechanical stops at limits
\end{itemize}

\subsubsection{Slew Rates}
\begin{itemize}
	\item Maximum slew rates achievable in manual mode
	\item Tracking mode: typically 30-40\degree/sec maximum
	\item Variable speed control via joystick
	\item Acceleration limits protect mechanical components
\end{itemize}

% ============================================================================
\section{ELECTRO-OPTICAL SYSTEM}

\subsection{DAY CAMERA (SONY)}

\spectable{
	\begin{tabular}{|L{0.35\textwidth}|L{0.55\textwidth}|}
		\hline
		\rowcolor{milblue!20}
		\textbf{Parameter} & \textbf{Specification} \\
		\hline
		\textbf{Resolution} & 1920×1080 (Full HD) \\
		\hline
		\textbf{Zoom} & Optical continuous \\
		\hline
		\textbf{FOV (Wide)} & ~15\degree\ H \\
		\hline
		\textbf{FOV (Tele)} & ~1.5\degree\ H \\
		\hline
		\textbf{Focus} & Autofocus / Manual \\
		\hline
		\textbf{Frame Rate} & 30 fps \\
		\hline
		\textbf{Low Light} & 0.01 lux (minimum) \\
		\hline
	\end{tabular}
}

\subsubsection{Day Camera Features}

\textbf{Zoom Capabilities:}
\begin{itemize}
	\item Continuous optical zoom
	\item 10:1 zoom ratio (approximately)
	\item No discrete zoom steps
	\item Digital zoom available (reduces quality)
\end{itemize}

\textbf{Focus System:}
\begin{itemize}
	\item Autofocus: Continuous or single-shot
	\item Manual focus: Available via menu
	\item Focus range: 10m to infinity
	\item Close focus: 10m minimum
\end{itemize}

\textbf{Image Quality:}
\begin{itemize}
	\item Full HD resolution (1920×1080)
	\item 30 frames per second
	\item Wide dynamic range (WDR) capable
	\item Low-light capable (0.01 lux)
\end{itemize}

% ----------------------------------------------------------------------------
\subsection{THERMAL CAMERA (FLIR TAU 2)}

\spectable{
	\begin{tabular}{|L{0.35\textwidth}|L{0.55\textwidth}|}
		\hline
		\rowcolor{milblue!20}
		\textbf{Parameter} & \textbf{Specification} \\
		\hline
		\textbf{Resolution} & 640×512 pixels \\
		\hline
		\textbf{Detector Type} & Uncooled VOx microbolometer \\
		\hline
		\textbf{FOV (Wide)} & 10.4\degree\ H × 8\degree\ V \\
		\hline
		\textbf{FOV (Narrow)} & 5.2\degree\ H × 4\degree\ V \\
		\hline
		\textbf{Digital Zoom} & 2× / 4× \\
		\hline
		\textbf{Spectral Range} & 7.5-13.5 μm (long-wave infrared) \\
		\hline
		\textbf{NETD} & <50 mK (thermal sensitivity) \\
		\hline
		\textbf{Frame Rate} & 30 Hz (60 Hz option) \\
		\hline
	\end{tabular}
}

\subsubsection{Thermal Camera Features}

\textbf{Field of View:}
\begin{itemize}
	\item Wide FOV: 10.4\degree\ H × 8\degree\ V (default)
	\item Narrow FOV: 5.2\degree\ H × 4\degree\ V (2× optical zoom)
	\item Digital zoom: 2× and 4× available
	\item Combined zoom: Up to 8× total (2× optical + 4× digital)
\end{itemize}

\textbf{Look-Up Tables (LUTs):}
\begin{itemize}
	\item White-Hot: Hot objects appear white
	\item Black-Hot: Hot objects appear black
	\item Rainbow: Color-coded temperature
	\item Ironbow: Enhanced detail color-coding
	\item Additional LUTs available
\end{itemize}

\textbf{Flat Field Correction (FFC):}
\begin{itemize}
	\item Automatic: Performed every 5-15 minutes
	\item Manual: Can be triggered via menu
	\item Duration: 1-2 second image freeze
	\item Purpose: Recalibrate sensor for accurate temperature reading
\end{itemize}

% ----------------------------------------------------------------------------
\subsection{LASER RANGE FINDER}

\spectable{
	\begin{tabular}{|L{0.35\textwidth}|L{0.55\textwidth}|}
		\hline
		\rowcolor{milblue!20}
		\textbf{Parameter} & \textbf{Specification} \\
		\hline
		\textbf{Range (Hard Target)} & 20m - 5000m \\
		\hline
		\textbf{Range (Soft Target)} & 20m - 2000m (typical) \\
		\hline
		\textbf{Accuracy} & ±5m \\
		\hline
		\textbf{Wavelength} & 1550nm (eye-safe) \\
		\hline
		\textbf{Max Fire Rate} & ~1 Hz (continuous) \\
		\hline
		\textbf{Beam Divergence} & <0.5 mrad \\
		\hline
		\textbf{Laser Class} & Class 1M (eye-safe) \\
		\hline
	\end{tabular}
}

\subsubsection{LRF Performance Notes}

\textbf{Range Performance:}
\begin{itemize}
	\item Hard targets (metal, concrete): Up to 5000m
	\item Soft targets (vegetation, fabric): Up to 2000m
	\item Minimum range: 20m
	\item Accuracy: ±5m across entire range
\end{itemize}

\textbf{Environmental Factors:}
\begin{itemize}
	\item Fog/Rain: Reduces maximum range
	\item Smoke: Severely degrades performance
	\item Heat shimmer: May affect accuracy at long range
	\item Target reflectivity: Dark/absorbent surfaces reduce range
\end{itemize}

\textbf{Eye Safety:}
\begin{itemize}
	\item 1550nm wavelength (eye-safe)
	\item Class 1M laser (safe under normal conditions)
	\item Do not view through magnifying optics
	\item Safe for use near personnel
\end{itemize}

% ============================================================================
\section{INERTIAL MEASUREMENT UNIT (IMU)}

\subsection{IMU Specifications}

\spectable{
	\begin{tabular}{|L{0.35\textwidth}|L{0.55\textwidth}|}
		\hline
		\rowcolor{milblue!20}
		\textbf{Parameter} & \textbf{Specification} \\
		\hline
		\textbf{Model} & MicroStrain 3DM-GX3-25 AHRS \\
		\hline
		\textbf{Update Rate} & 100 Hz \\
		\hline
		\textbf{Roll/Pitch Accuracy} & <0.5\degree\ (static), <2\degree\ (dynamic) \\
		\hline
		\textbf{Yaw Accuracy} & <2\degree\ (with magnetometer) \\
		\hline
		\textbf{Angular Rate Range} & ±300\degree/sec \\
		\hline
		\textbf{Acceleration Range} & ±5g \\
		\hline
		\textbf{Operating Temp} & -40\degree C to +85\degree C \\
		\hline
	\end{tabular}
}

\subsection{IMU Role in System}

\textbf{Stabilization:}
\begin{itemize}
	\item Measures platform motion (roll, pitch, yaw)
	\item Provides real-time orientation data
	\item Enables gyrostabilization (compensates for vehicle motion)
	\item Update rate: 100 Hz (100 times per second)
\end{itemize}

\textbf{Tracking Support:}
\begin{itemize}
	\item Provides angular rate data for target tracking
	\item Compensates for platform motion during tracking
	\item Essential for accurate lead angle calculation
\end{itemize}

\textbf{Ballistics:}
\begin{itemize}
	\item Platform tilt compensation
	\item Motion compensation for moving platforms
	\item Critical for accurate fire control
\end{itemize}

% ============================================================================
\section{POWER REQUIREMENTS}

\subsection{Power Specifications}

\spectable{
	\begin{tabular}{|L{0.35\textwidth}|L{0.55\textwidth}|}
		\hline
		\rowcolor{milblue!20}
		\textbf{Parameter} & \textbf{Specification} \\
		\hline
		\textbf{Input Voltage} & 24-28 VDC \\
		\hline
		\textbf{Nominal Voltage} & 28 VDC \\
		\hline
		\textbf{Max Current} & 40A (peak), 20A (nominal) \\
		\hline
		\textbf{Power Consumption} & ~500W nominal, ~1000W peak \\
		\hline
		\textbf{Startup Inrush} & <60A for <1 second \\
		\hline
	\end{tabular}
}

\subsection{Power Distribution}

\textbf{Power Consumers:}
\begin{itemize}
	\item Servo motors: 300-600W (depending on motion)
	\item Main computer: 100W
	\item Cameras: 50W total
	\item LRF: 20W average, 50W peak (during firing)
	\item Control panels: 30W
	\item IMU and sensors: 20W
\end{itemize}

\textbf{Power Management:}
\begin{itemize}
	\item Soft-start circuitry limits inrush current
	\item Over-current protection (60A trip)
	\item Over-voltage protection (>32V trip)
	\item Under-voltage warning (<22V)
	\item Thermal management (reduces power if overheating)
\end{itemize}

% ============================================================================
\section{ENVIRONMENTAL SPECIFICATIONS}

\subsection{Operating Environment}

\spectable{
	\begin{tabular}{|L{0.35\textwidth}|L{0.55\textwidth}|}
		\hline
		\rowcolor{milblue!20}
		\textbf{Parameter} & \textbf{Specification} \\
		\hline
		\textbf{Operating Temp} & -20\degree C to +55\degree C \\
		\hline
		\textbf{Storage Temp} & -40\degree C to +70\degree C \\
		\hline
		\textbf{Humidity} & 5\% to 95\% RH (non-condensing) \\
		\hline
		\textbf{Ingress Protection} & IP54 (dust/water splash protected) \\
		\hline
		\textbf{Altitude} & -500m to 4000m ASL \\
		\hline
		\textbf{Shock} & 20g (11ms, half-sine) \\
		\hline
		\textbf{Vibration} & MIL-STD-810G \\
		\hline
	\end{tabular}
}

\subsection{Environmental Notes}

\textbf{Temperature:}
\begin{itemize}
	\item Operating range: -20\degree C to +55\degree C
	\item Performance may degrade at temperature extremes
	\item Servo torque increases at low temperatures
	\item Thermal management active above 40\degree C
\end{itemize}

\textbf{Humidity:}
\begin{itemize}
	\item Sealed enclosures protect electronics
	\item Lens fogging possible at temperature extremes
	\item Desiccant packs in critical areas
	\item Regular maintenance in high-humidity environments
\end{itemize}

\textbf{Ingress Protection:}
\begin{itemize}
	\item IP54 rating (dust/water splash)
	\item Protected against rain and dust
	\item NOT waterproof (do not submerge)
	\item Pressure washing not recommended
\end{itemize}

% ============================================================================
\section{MECHANICAL SPECIFICATIONS}

\subsection{Physical Dimensions}

\spectable{
	\begin{tabular}{|L{0.35\textwidth}|L{0.55\textwidth}|}
		\hline
		\rowcolor{milblue!20}
		\textbf{Parameter} & \textbf{Specification} \\
		\hline
		\textbf{Gimbal Diameter} & ~600mm \\
		\hline
		\textbf{Height (Stowed)} & ~500mm \\
		\hline
		\textbf{Total Weight} & ~80 kg (without weapon) \\
		\hline
		\textbf{Weapon Weight} & +25-35 kg (12.7mm MG) \\
		\hline
		\textbf{Mounting Interface} & 4× M12 bolts (square pattern) \\
		\hline
	\end{tabular}
}

\subsection{Materials}

\textbf{Primary Structure:}
\begin{itemize}
	\item Aluminum alloy (7075-T6)
	\item Stainless steel (fasteners, bearings)
	\item Composite materials (non-structural covers)
\end{itemize}

\textbf{Coatings:}
\begin{itemize}
	\item Hard anodized aluminum (corrosion protection)
	\item Powder coat (selected components)
	\item Nickel-plated steel (high-wear areas)
\end{itemize}

% ============================================================================
\section{COMMUNICATION INTERFACES}

\subsection{Control Interfaces}

\spectable{
	\begin{tabular}{|L{0.35\textwidth}|L{0.55\textwidth}|}
		\hline
		\rowcolor{milblue!20}
		\textbf{Interface} & \textbf{Specification} \\
		\hline
		\textbf{Joystick} & USB 2.0 (HID device) \\
		\hline
		\textbf{Control Panel} & RS-485 (Modbus RTU) \\
		\hline
		\textbf{Video Output} & HDMI 1.4 (1920×1080 @ 60Hz) \\
		\hline
		\textbf{Ethernet} & 10/100 Mbps (diagnostics, updates) \\
		\hline
	\end{tabular}
}

\subsection{Device Communication}

\textbf{Internal Communications:}
\begin{itemize}
	\item Servo drives: CAN bus (1 Mbps)
	\item IMU: RS-232 (115200 baud)
	\item LRF: RS-232 (38400 baud)
	\item Cameras: IP network (internal switch)
\end{itemize}

% ============================================================================
\section{BALLISTIC SPECIFICATIONS}

\subsection{Supported Ammunition}

\textbf{Primary: 12.7×99mm NATO (.50 BMG)}
\begin{itemize}
	\item M2 Ball (standard)
	\item M8 API (Armor Piercing Incendiary)
	\item M20 APIT (Armor Piercing Incendiary Tracer)
	\item Mk 211 Mod 0 (Multipurpose)
\end{itemize}

\subsection{Ballistic Computer}

\spectable{
	\begin{tabular}{|L{0.35\textwidth}|L{0.55\textwidth}|}
		\hline
		\rowcolor{milblue!20}
		\textbf{Parameter} & \textbf{Specification} \\
		\hline
		\textbf{Range Calculation} & 20m to 2000m \\
		\hline
		\textbf{Update Rate} & 30 Hz \\
		\hline
		\textbf{Environmental Inputs} & Temp, altitude, wind \\
		\hline
		\textbf{Drag Model} & G1 (standard projectile) \\
		\hline
		\textbf{Ballistic Tables} & Pre-computed LUT \\
		\hline
	\end{tabular}
}

\subsection{Fire Control Accuracy}

\textbf{Expected Accuracy (M2 Ball, 500m):}
\begin{itemize}
	\item With zeroing: ±0.5 mil (±25cm @ 500m)
	\item With zeroing + environmental: ±0.3 mil (±15cm @ 500m)
	\item With zeroing + environmental + LAC: ±0.5 mil (±25cm @ 500m, moving target)
\end{itemize}

\begin{notebox}
	Accuracy specifications assume properly zeroed weapon, accurate range data, and good tracking (if using LAC). Operator skill and weapon dispersion also affect accuracy.
\end{notebox}

% ============================================================================
\section{MEAN TIME BETWEEN FAILURES (MTBF)}

\subsection{Reliability Estimates}

\spectable{
	\begin{tabular}{|L{0.40\textwidth}|L{0.50\textwidth}|}
		\hline
		\rowcolor{milblue!20}
		\textbf{Component} & \textbf{MTBF (hours)} \\
		\hline
		\textbf{Servo Motors} & 10,000 hours \\
		\hline
		\textbf{IMU} & 50,000 hours \\
		\hline
		\textbf{Day Camera} & 20,000 hours \\
		\hline
		\textbf{Thermal Camera} & 15,000 hours \\
		\hline
		\textbf{LRF} & 10,000 hours (laser diode) \\
		\hline
		\textbf{Main Computer} & 40,000 hours \\
		\hline
		\textbf{Overall System} & 5,000 hours (operational) \\
		\hline
	\end{tabular}
}

\subsection{Maintenance Intervals}

\textbf{Operator Maintenance:}
\begin{itemize}
	\item Daily: Pre-operation checks, cleaning
	\item Weekly: Detailed inspection, zero verification
	\item Monthly: Re-zeroing, lubrication (if authorized)
\end{itemize}

\textbf{Maintenance Personnel:}
\begin{itemize}
	\item Quarterly: Detailed inspection, calibration verification
	\item Annually: Major service, component replacement as needed
	\item As-needed: Fault repair, component replacement
\end{itemize}

% ============================================================================
% END OF APPENDIX B
% ============================================================================
% ACRONYMS & GLOSSARY

\section{C.1 ACRONYMS}

% TABLE ROW: | Acronym | Definition |
% TABLE ROW: |---------|------------|
% TABLE ROW: | **RCWS** | Remote Controlled Weapon Station |
% TABLE ROW: | **DCU** | Display and Control Unit |
% TABLE ROW: | **LRF** | Laser Range Finder |
% TABLE ROW: | **IMU** | Inertial Measurement Unit |
% TABLE ROW: | **AHRS** | Attitude and Heading Reference System |
% TABLE ROW: | **FOV** | Field of View |
% TABLE ROW: | **HFOV** | Horizontal Field of View |
% TABLE ROW: | **VFOV** | Vertical Field of View |
% TABLE ROW: | **LAC** | Lead Angle Compensation |
% TABLE ROW: | **CCIP** | Continuously Computed Impact Point |
% TABLE ROW: | **TRP** | Target Reference Point |
% TABLE ROW: | **PID** | Positive Identification |
% TABLE ROW: | **ROE** | Rules of Engagement |
% TABLE ROW: | **BDA** | Battle Damage Assessment |
% TABLE ROW: | **TOF** | Time of Flight |
% TABLE ROW: | **LUT** | Look-Up Table (Thermal) |
% TABLE ROW: | **FFC** | Flat Field Correction (Thermal camera calibration) |
% TABLE ROW: | **E-Stop** | Emergency Stop |
% TABLE ROW: | **OSD** | On-Screen Display |

\section{C.2 GLOSSARY}

\textbf{Acquisition}: The phase where operator positions and sizes the tracking gate over the target before requesting lock-on.

\textbf{Azimuth}: Horizontal angle, measured in degrees (0-360\degree{}). 0\degree{}=North, 90\degree{}=East, 180\degree{}=South, 270\degree{}=West.

\textbf{Boresight}: The alignment between the camera line-of-sight and the weapon bore axis. Zeroing corrects boresight offset.

\textbf{Coast Mode}: Tracking phase where the target is temporarily lost and the system predicts target position to maintain track.

\textbf{Dead Man Switch}: Safety interlock requiring continuous operator pressure to enable certain functions (LAC activation).

\textbf{Elevation}: Vertical angle, measured in degrees. 0\degree{}=horizon, +90\degree{}=zenith, -90\degree{}=nadir.

\textbf{Gimbal}: The stabilized platform that holds the camera and weapon, allowing azimuth and elevation motion.

\textbf{Lead Angle}: The angular offset between a moving target's current position and the predicted intercept point, compensating for projectile time-of-flight.

\textbf{Lock Pending}: Tracking phase where the system is initializing the tracker algorithm on the target.

\textbf{Reticle}: The aiming point displayed on the video feed, showing where the weapon will impact (with zeroing/ballistics applied).

\textbf{Stabilization}: The system's ability to compensate for platform motion (vehicle movement, ship roll/pitch) to keep the gimbal pointed at the target.

\textbf{Zeroing}: The process of correcting the offset between the camera aim point and weapon impact point, ensuring the reticle accurately represents where rounds will hit.

% ============================================================================
% APPENDIX D - MAINTENANCE LOG TEMPLATE
% ============================================================================
\appendixchapter{D - MAINTENANCE LOG TEMPLATE}

\section{DAILY OPERATOR LOG}

\subsection{Log Header}

\begin{tcolorbox}[
	colback=lightgray,
	colframe=milblue,
	title=DAILY OPERATOR LOG - HEADER INFORMATION,
	sharp corners,
	boxrule=1pt
	]
	
	\textbf{Date:} \rule{4cm}{0.4pt} \hfill \textbf{Mission ID:} \rule{4cm}{0.4pt}
	
	\vspace{0.3cm}
	
	\textbf{Operator Name:} \rule{5cm}{0.4pt} \hfill \textbf{Rank:} \rule{3cm}{0.4pt}
	
	\vspace{0.3cm}
	
	\textbf{Unit:} \rule{5cm}{0.4pt} \hfill \textbf{Vehicle ID:} \rule{3cm}{0.4pt}
	
	\vspace{0.3cm}
	
	\textbf{Mission Type:} \rule{8cm}{0.4pt}
	
	\vspace{0.3cm}
	
	\textbf{Weather Conditions:} \rule{10cm}{0.4pt}
	
\end{tcolorbox}

% ============================================================================
\subsection{PRE-OPERATION CHECKS}

\begin{longtable}{|L{0.20\textwidth}|C{0.10\textwidth}|C{0.10\textwidth}|L{0.50\textwidth}|}
	\hline
	\rowcolor{milblue!20}
	\textbf{System} & \textbf{Status} & \textbf{GO/NO-GO} & \textbf{Notes} \\
	\hline
	\endfirsthead
	
	\hline
	\rowcolor{milblue!20}
	\textbf{System} & \textbf{Status} & \textbf{GO/NO-GO} & \textbf{Notes} \\
	\hline
	\endhead
	
	\textbf{Power} & $\square$ ✓ $\square$ ✗ & $\square$ GO $\square$ NO-GO & \\[1.5em]
	\hline
	\textbf{Control Panel} & $\square$ ✓ $\square$ ✗ & $\square$ GO $\square$ NO-GO & \\[1.5em]
	\hline
	\textbf{Joystick} & $\square$ ✓ $\square$ ✗ & $\square$ GO $\square$ NO-GO & \\[1.5em]
	\hline
	\textbf{Gimbal Motion} & $\square$ ✓ $\square$ ✗ & $\square$ GO $\square$ NO-GO & \\[1.5em]
	\hline
	\textbf{Day Camera} & $\square$ ✓ $\square$ ✗ & $\square$ GO $\square$ NO-GO & \\[1.5em]
	\hline
	\textbf{Thermal Camera} & $\square$ ✓ $\square$ ✗ & $\square$ GO $\square$ NO-GO & \\[1.5em]
	\hline
	\textbf{LRF} & $\square$ ✓ $\square$ ✗ & $\square$ GO $\square$ NO-GO & \\[1.5em]
	\hline
	\textbf{Tracking} & $\square$ ✓ $\square$ ✗ & $\square$ GO $\square$ NO-GO & \\[1.5em]
	\hline
	\textbf{Weapon Arm/Safe} & $\square$ ✓ $\square$ ✗ & $\square$ GO $\square$ NO-GO & \\[1.5em]
	\hline
	\textbf{System Status} & $\square$ ✓ $\square$ ✗ & $\square$ GO $\square$ NO-GO & \\[1.5em]
	\hline
\end{longtable}

\begin{tcolorbox}[
	colback=milgreen!10,
	colframe=milgreen,
	sharp corners,
	boxrule=1pt
	]
	\textbf{OVERALL PRE-OPERATION STATUS:} \hspace{1cm} $\square$ \textbf{GO} \hspace{2cm} $\square$ \textbf{NO-GO}
	
	\vspace{0.2cm}
	
	\textbf{If NO-GO:} Do NOT proceed with mission. Notify maintenance immediately.
\end{tcolorbox}

% ============================================================================
\subsection{BASELINE SYSTEM READINGS}

\textbf{Record baseline readings at start of operations:}

\vspace{0.3cm}

\begin{tcolorbox}[
	colback=lightgray,
	colframe=darkgray,
	sharp corners,
	boxrule=0.5pt
	]
	
	\textbf{Time:} \rule{3cm}{0.4pt}
	
	\vspace{0.3cm}
	
	\begin{tabular}{ll}
		\textbf{Azimuth Servo Temp:} \rule{3cm}{0.4pt} & \textbf{Elevation Servo Temp:} \rule{3cm}{0.4pt} \\[0.5em]
		\textbf{IMU Temp:} \rule{3cm}{0.4pt} & \textbf{LRF Temp:} \rule{3cm}{0.4pt} \\[0.5em]
		\textbf{Day Camera Status:} \rule{3cm}{0.4pt} & \textbf{Thermal Camera Status:} \rule{3cm}{0.4pt} \\[0.5em]
	\end{tabular}
	
	\vspace{0.2cm}
	
	\textbf{Zero Status:} $\square$ Z Indicator Present $\square$ No Zero Applied
	
	\textbf{Environmental Settings:} $\square$ ENV Indicator Present $\square$ Not Applied
	
\end{tcolorbox}

% ============================================================================
\section{OPERATIONS LOG}

\subsection{Operational Events}

\begin{longtable}{|C{0.10\textwidth}|L{0.25\textwidth}|L{0.55\textwidth}|}
	\hline
	\rowcolor{milblue!20}
	\textbf{Time} & \textbf{Activity} & \textbf{Notes} \\
	\hline
	\endfirsthead
	
	\hline
	\rowcolor{milblue!20}
	\textbf{Time} & \textbf{Activity} & \textbf{Notes} \\
	\hline
	\endhead
	
	& & \\[2em]
	\hline
	& & \\[2em]
	\hline
	& & \\[2em]
	\hline
	& & \\[2em]
	\hline
	& & \\[2em]
	\hline
	& & \\[2em]
	\hline
	& & \\[2em]
	\hline
	& & \\[2em]
	\hline
	& & \\[2em]
	\hline
	& & \\[2em]
	\hline
	& & \\[2em]
	\hline
	& & \\[2em]
	\hline
\end{longtable}

% ============================================================================
\subsection{ENGAGEMENT LOG}

\textbf{Record all weapon engagements:}

\begin{longtable}{|C{0.08\textwidth}|L{0.18\textwidth}|C{0.10\textwidth}|C{0.10\textwidth}|L{0.12\textwidth}|L{0.32\textwidth}|}
	\hline
	\rowcolor{milblue!20}
	\textbf{Time} & \textbf{Target Type} & \textbf{Range (m)} & \textbf{Rounds} & \textbf{BDA} & \textbf{Notes} \\
	\hline
	\endfirsthead
	
	\hline
	\rowcolor{milblue!20}
	\textbf{Time} & \textbf{Target Type} & \textbf{Range (m)} & \textbf{Rounds} & \textbf{BDA} & \textbf{Notes} \\
	\hline
	\endhead
	
	& & & & & \\[2em]
	\hline
	& & & & & \\[2em]
	\hline
	& & & & & \\[2em]
	\hline
	& & & & & \\[2em]
	\hline
	& & & & & \\[2em]
	\hline
\end{longtable}

\begin{notebox}
	\textbf{BDA Codes:} D=Destroyed, DM=Damaged, M=Missed, S=Suppressed, MK=Mobility Kill, FK=Firepower Kill
\end{notebox}

% ============================================================================
\section{POST-OPERATION CHECKS}

\subsection{System Verification}

\begin{longtable}{|L{0.25\textwidth}|C{0.10\textwidth}|C{0.10\textwidth}|L{0.45\textwidth}|}
	\hline
	\rowcolor{milblue!20}
	\textbf{System} & \textbf{Status} & \textbf{Pass/Fail} & \textbf{Notes} \\
	\hline
	\endfirsthead
	
	\hline
	\rowcolor{milblue!20}
	\textbf{System} & \textbf{Status} & \textbf{Pass/Fail} & \textbf{Notes} \\
	\hline
	\endhead
	
	\textbf{All Devices Connected} & $\square$ ✓ $\square$ ✗ & $\square$ PASS $\square$ FAIL & \\[1.5em]
	\hline
	\textbf{No Fault Warnings} & $\square$ ✓ $\square$ ✗ & $\square$ PASS $\square$ FAIL & \\[1.5em]
	\hline
	\textbf{Zero Still Valid} & $\square$ ✓ $\square$ ✗ & $\square$ PASS $\square$ FAIL & \\[1.5em]
	\hline
	\textbf{System Secured} & $\square$ ✓ $\square$ ✗ & $\square$ PASS $\square$ FAIL & \\[1.5em]
	\hline
	\textbf{Weapon Safed} & $\square$ ✓ $\square$ ✗ & $\square$ PASS $\square$ FAIL & \\[1.5em]
	\hline
	\textbf{Station Powered Down} & $\square$ ✓ $\square$ ✗ & $\square$ PASS $\square$ FAIL & \\[1.5em]
	\hline
\end{longtable}

% ----------------------------------------------------------------------------
\subsection{Final System Readings}

\textbf{Record final readings at end of operations:}

\vspace{0.3cm}

\begin{tcolorbox}[
	colback=lightgray,
	colframe=darkgray,
	sharp corners,
	boxrule=0.5pt
	]
	
	\textbf{Time:} \rule{3cm}{0.4pt}
	
	\vspace{0.3cm}
	
	\begin{tabular}{ll}
		\textbf{Azimuth Servo Temp:} \rule{3cm}{0.4pt} & \textbf{Elevation Servo Temp:} \rule{3cm}{0.4pt} \\[0.5em]
		\textbf{IMU Temp:} \rule{3cm}{0.4pt} & \textbf{LRF Temp:} \rule{3cm}{0.4pt} \\[0.5em]
		\textbf{Total Operating Hours:} \rule{3cm}{0.4pt} & \textbf{Rounds Fired:} \rule{3cm}{0.4pt} \\[0.5em]
	\end{tabular}
	
\end{tcolorbox}

% ============================================================================
\section{FAULTS / ISSUES OBSERVED}

\begin{longtable}{|L{0.25\textwidth}|C{0.10\textwidth}|L{0.30\textwidth}|C{0.12\textwidth}|}
	\hline
	\rowcolor{milred!20}
	\textbf{Issue Description} & \textbf{Time} & \textbf{Action Taken} & \textbf{Escalated?} \\
	\hline
	\endfirsthead
	
	\hline
	\rowcolor{milred!20}
	\textbf{Issue Description} & \textbf{Time} & \textbf{Action Taken} & \textbf{Escalated?} \\
	\hline
	\endhead
	
	& & & $\square$ YES $\square$ NO \\[2em]
	\hline
	& & & $\square$ YES $\square$ NO \\[2em]
	\hline
	& & & $\square$ YES $\square$ NO \\[2em]
	\hline
	& & & $\square$ YES $\square$ NO \\[2em]
	\hline
	& & & $\square$ YES $\square$ NO \\[2em]
	\hline
	& & & $\square$ YES $\square$ NO \\[2em]
	\hline
\end{longtable}

\begin{warningbox}
	If any issues were escalated to maintenance, provide detailed information including:
	\begin{itemize}
		\item Specific symptoms observed
		\item System Status screenshot/data (if available)
		\item Error messages or alarm codes
		\item Actions already attempted
	\end{itemize}
\end{warningbox}

% ============================================================================
\section{AMMUNITION EXPENDITURE}

\begin{tcolorbox}[
	colback=lightgray,
	colframe=milblue,
	sharp corners,
	boxrule=1pt
	]
	
	\textbf{Ammunition Type:} \rule{6cm}{0.4pt}
	
	\vspace{0.3cm}
	
	\begin{tabular}{ll}
		\textbf{Starting Rounds:} \rule{4cm}{0.4pt} & \textbf{Ending Rounds:} \rule{4cm}{0.4pt} \\[0.8em]
		\textbf{Rounds Fired:} \rule{4cm}{0.4pt} & \textbf{Rounds Remaining:} \rule{4cm}{0.4pt} \\[0.8em]
	\end{tabular}
	
	\vspace{0.3cm}
	
	\textbf{Malfunctions:} $\square$ None \hspace{1cm} $\square$ Jams: \rule{2cm}{0.4pt} \hspace{1cm} $\square$ Misfires: \rule{2cm}{0.4pt}
	
	\vspace{0.3cm}
	
	\textbf{Notes:} \rule{12cm}{0.4pt}
	
\end{tcolorbox}

% ============================================================================
\section{OPERATOR CERTIFICATION}

\begin{tcolorbox}[
	colback=milgreen!10,
	colframe=milgreen,
	title=OPERATOR CERTIFICATION,
	sharp corners,
	boxrule=1.5pt
	]
	
	I certify that:
	\begin{itemize}
		\item All pre-operation checks were completed
		\item All system operations were conducted per standard procedures
		\item All engagements met Rules of Engagement (ROE)
		\item All faults/issues were documented and escalated as appropriate
		\item All post-operation checks were completed
		\item System is secured and safe
	\end{itemize}
	
	\vspace{0.5cm}
	
	\textbf{Operator Name (Print):} \rule{6cm}{0.4pt}
	
	\vspace{0.5cm}
	
	\textbf{Operator Signature:} \rule{6cm}{0.4pt} \hfill \textbf{Date:} \rule{4cm}{0.4pt}
	
\end{tcolorbox}

% ============================================================================
\section{SUPERVISOR REVIEW}

\begin{tcolorbox}[
	colback=milblue!10,
	colframe=milblue,
	title=SUPERVISOR REVIEW (If Required),
	sharp corners,
	boxrule=1pt
	]
	
	\textbf{Review Comments:}
	
	\vspace{3cm}
	
	\rule{\textwidth}{0.4pt}
	
	\vspace{0.5cm}
	
	\textbf{Supervisor Name (Print):} \rule{6cm}{0.4pt}
	
	\vspace{0.5cm}
	
	\textbf{Supervisor Signature:} \rule{6cm}{0.4pt} \hfill \textbf{Date:} \rule{4cm}{0.4pt}
	
\end{tcolorbox}

% ============================================================================
\section{LOG RETENTION}

\subsection{Retention Requirements}

\begin{itemize}
	\item \textbf{Daily logs:} Retain for minimum 90 days
	\item \textbf{Logs with engagements:} Retain for minimum 1 year
	\item \textbf{Logs with incidents:} Retain indefinitely (until investigation complete + 5 years)
	\item \textbf{Storage:} Secure location, restricted access
	\item \textbf{Digital copies:} Recommended for long-term retention
\end{itemize}

\subsection{Log Distribution}

Upon completion, provide copies to:
\begin{itemize}
	\item Unit operations office (original)
	\item Maintenance section (if faults reported)
	\item Commander (if engagements occurred)
	\item Training section (for performance tracking)
\end{itemize}

% ============================================================================
\section{MAINTENANCE LOG TEMPLATE}

\subsection{FOR MAINTENANCE PERSONNEL USE}

\begin{tcolorbox}[
	colback=milyellow!10,
	colframe=milyellow,
	title=MAINTENANCE WORK ORDER,
	sharp corners,
	boxrule=1pt
	]
	
	\textbf{Work Order \#:} \rule{4cm}{0.4pt} \hfill \textbf{Date:} \rule{4cm}{0.4pt}
	
	\vspace{0.3cm}
	
	\textbf{System/Vehicle ID:} \rule{6cm}{0.4pt} \hfill \textbf{Priority:} $\square$ 1 $\square$ 2 $\square$ 3
	
	\vspace{0.3cm}
	
	\textbf{Problem Description (from operator):}
	
	\vspace{2cm}
	
	\rule{\textwidth}{0.4pt}
	
	\vspace{0.3cm}
	
	\textbf{Maintenance Actions Taken:}
	
	\vspace{3cm}
	
	\rule{\textwidth}{0.4pt}
	
	\vspace{0.3cm}
	
	\textbf{Parts Replaced:}
	
	\vspace{1.5cm}
	
	\rule{\textwidth}{0.4pt}
	
	\vspace{0.3cm}
	
	\textbf{Status:} $\square$ Repaired $\square$ Requires Further Work $\square$ Parts on Order
	
	\vspace{0.3cm}
	
	\textbf{Technician Name:} \rule{5cm}{0.4pt} \hfill \textbf{Date:} \rule{4cm}{0.4pt}
	
	\vspace{0.3cm}
	
	\textbf{Technician Signature:} \rule{5cm}{0.4pt} \hfill \textbf{Hours:} \rule{3cm}{0.4pt}
	
\end{tcolorbox}

% ============================================================================
\section{NOTES AND BEST PRACTICES}

\subsection{Log Completion Best Practices}

\begin{itemize}
	\item \textbf{Be specific:} "LRF intermittent at 3000m+" vs. "LRF not working"
	\item \textbf{Record times:} All events should have timestamps
	\item \textbf{Document everything:} Even minor issues can become trends
	\item \textbf{Be legible:} Use clear handwriting or digital entry
	\item \textbf{Complete immediately:} Don't rely on memory hours later
	\item \textbf{Report upward:} Ensure chain of command aware of issues
\end{itemize}

\subsection{Common Log Errors to Avoid}

\begin{itemize}
	\item ❌ Incomplete pre-operation checks
	\item ❌ Missing timestamps
	\item ❌ Vague problem descriptions
	\item ❌ Not documenting operator-level troubleshooting attempts
	\item ❌ Forgetting to record ammunition expenditure
	\item ❌ Not signing/dating the log
\end{itemize}

% ============================================================================
% END OF APPENDIX D
% ============================================================================

\vspace{2cm}

\begin{center}
	\rule{0.5\textwidth}{1pt}
	
	\vspace{0.5cm}
	
	\textbf{\Large END OF EL 7ARRESS RCWS OPERATOR MANUAL}
	
	\vspace{0.5cm}
	
	\textit{Tunisian Ministry of Defense}
	
	\textit{Version 1.0 --- October 2025}
	
	\vspace{0.5cm}
	
	\rule{0.5\textwidth}{1pt}
\end{center}
\end{document}
