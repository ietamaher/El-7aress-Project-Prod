% ============================================================================
% LESSON 4 - MOTION MODES & SURVEILLANCE
% ============================================================================
\lesson{4: MOTION MODES \& SURVEILLANCE}

\noindent
\begin{tabular}{@{}ll@{}}
\textbf{Duration:} & 3 hours \\
\textbf{Type:} & Classroom + Practical \\
\textbf{References:} & Zone management guide, surveillance procedures \\
\end{tabular}

\section{Introduction}

\textbf{Lesson Purpose}: This lesson teaches motion mode selection, automated surveillance patterns, and zone management for safe and effective operations.

\textbf{Learning Objectives}:
\begin{itemize}
    \item Explain the purpose of each motion mode
    \item Switch between motion modes safely
    \item Operate automatic sector scan mode
    \item Utilize Target Reference Point (TRP) scan mode
    \item Define and manage no-fire zones
    \item Define and manage no-traverse zones
    \item Save and load zone configurations
\end{itemize}

\section{Motion Modes Overview}

\subsection{What Are Motion Modes?}

Motion modes control \textbf{how the gimbal moves} during operations:

\begin{itemize}
    \item \textbf{Manual Mode}: You control gimbal directly with joystick
    \item \textbf{Auto Sector Scan}: System automatically scans between two points
    \item \textbf{TRP Scan}: System sequentially visits pre-defined Target Reference Points
    \item \textbf{Radar Slew} (if equipped): Gimbal follows radar detections
\end{itemize}

\textbf{Purpose}: Different missions require different surveillance patterns. Motion modes let you switch between direct control and automated surveillance.

\subsection{Mode Selection}

\textbf{How to Change Modes}:
\begin{itemize}
    \item Press joystick \button{Button 11} or \button{Button 13} (either button cycles modes)
    \item Modes cycle in sequence:
\end{itemize}

\begin{lstlisting}
Manual → AutoSectorScan → TRP Scan → Radar Slew → Manual
\end{lstlisting}

\begin{notebox}
\textbf{Radar Slew} is only available if radar hardware is installed. On systems without radar, the mode cycle is: \texttt{Manual → Sector Scan → TRP → Manual}
\end{notebox}

\textbf{Current Mode Display}: OSD bottom center shows: \osd{MODE: Manual}, \osd{MODE: Sector Scan}, \osd{MODE: TRP}, or \osd{MODE: Radar}

\begin{cautionbox}[RESTRICTION]
Cannot change modes during tracking acquisition. Must abort tracking first (press TRK button).
\end{cautionbox}

\section{Manual Mode}

\textbf{Description}: Direct operator control via joystick (default mode)

\textbf{When to Use}:
\begin{itemize}
    \item Direct target engagement
    \item Precise aiming
    \item Immediate threat response
    \item Search operations requiring operator judgment
\end{itemize}

\textbf{Operation}:
\begin{itemize}
    \item Joystick LEFT/RIGHT → Azimuth control
    \item Joystick UP/DOWN → Elevation control
    \item Stick deflection = gimbal speed
    \item Center stick = gimbal stops
\end{itemize}

\begin{notebox}
Already Covered: See Lesson 2, Section 2.3 for detailed joystick control
\end{notebox}

\section{Auto Sector Scan Mode}

\subsection{What Is Sector Scan?}

\textbf{Definition}: Automated gimbal movement that continuously scans between two pre-defined points (left and right limits)

\textbf{Purpose}:
\begin{itemize}
    \item Perimeter surveillance
    \item Monitoring a defined sector without operator input
    \item Frees operator to monitor other systems or threats
\end{itemize}

\textbf{Visual}:
\begin{lstlisting}
    Left Limit              Right Limit
        ↓                       ↓
        ●←←←←←←←scan←←←←←←←●
        ●→→→→→→→scan→→→→→→→●
             (repeats)
\end{lstlisting}

Gimbal continuously scans left-to-right, then right-to-left, repeat.

\subsection{Activating Sector Scan}

\textbf{Prerequisites}:
\begin{enumerate}
    \item Sector scan zone must be defined
    \item At least one sector scan zone exists
    \item System is in AutoSectorScan mode
\end{enumerate}

\begin{procedurebox}[SECTOR SCAN ACTIVATION]
\begin{enumerate}
    \item \textbf{Cycle to Sector Scan Mode}: Press Button 11 or 13 until OSD shows: \osd{MODE: Sector Scan}
    \item \textbf{System Behavior}:
    \begin{itemize}
        \item Gimbal automatically slews to left limit of active sector
        \item Scans to right limit (slow, smooth movement)
        \item Reverses, scans back to left limit
        \item Repeats continuously
    \end{itemize}
    \item \textbf{Scan Speed}: Default: ~5\degree/second (configurable)
    \item \textbf{Elevation}: Scans at elevation defined in zone (usually 0\degree\ horizon)
\end{enumerate}
\end{procedurebox}

\subsection{While Sector Scanning}

\textbf{Operator Actions}:

\textbf{CAN DO}:
\begin{itemize}
    \item Monitor video feed as sector scans
    \item Switch cameras (CAM button)
    \item Zoom in/out (Zoom rocker)
    \item Fire LRF (LRF button) at targets of interest
    \item Initiate tracking (TRK button) - aborts sector scan, switches to Manual
\end{itemize}

\textbf{CANNOT DO}:
\begin{itemize}
    \item Joystick axes do NOT control gimbal (ignored)
    \item Cannot manually slew gimbal (must exit mode first)
\end{itemize}

\textbf{To Override}:
\begin{itemize}
    \item Press Button 11/13 to return to Manual mode
    \item OR: Press TRK to start tracking (auto-switches to Manual)
\end{itemize}

\subsection{Multiple Sector Zones}

\textbf{If Multiple Sectors Defined}:
\begin{itemize}
    \item System scans active sector (last selected via menu or joystick)
    \item To change active sector via menu:
    \begin{enumerate}
        \item Cycle to Manual mode
        \item Access menu: Zone Definitions → Sector Scans
        \item Select desired sector
        \item Return to AutoSectorScan mode
    \end{enumerate}
\end{itemize}

\subsection{Zone Selection via Joystick (Buttons 14/16)}

\textbf{NEW CAPABILITY}: Rapid zone switching during surveillance \textbf{without menu access}.

\begin{procedurebox}[JOYSTICK ZONE SELECTION]
\textbf{During AutoSectorScan Mode}:
\begin{itemize}
    \item \textbf{Button 14}: Select NEXT sector scan zone
    \item \textbf{Button 16}: Select PREVIOUS sector scan zone
    \item Zone switches immediately, scanning resumes in new zone
\end{itemize}

\textbf{During TRP Scan Mode}:
\begin{itemize}
    \item \textbf{Button 14}: Select NEXT TRP location page
    \item \textbf{Button 16}: Select PREVIOUS TRP location page
    \item TRP scan continues with new page's target points
\end{itemize}
\end{procedurebox}

\textbf{Operational Benefit}:
\begin{itemize}
    \item Rapid zone switching \textbf{without leaving surveillance mode}
    \item Hands stay on joystick, eyes stay on video
    \item Faster reaction to changing threat sectors
    \item No menu navigation required
\end{itemize}

\textbf{Example - Sector Scan Zone Cycling}:
\begin{lstlisting}
Currently scanning: "Front Gate" (Zone 1)
Press Button 14 → Switch to "East Perimeter" (Zone 2)
Press Button 14 → Switch to "Rear Area" (Zone 3)
Press Button 14 → Wrap to "Front Gate" (Zone 1)
\end{lstlisting}

\textbf{Example - TRP Page Cycling}:
\begin{lstlisting}
Currently on: "Page 1: Checkpoints"
Press Button 14 → Switch to "Page 2: Hilltops"
Press Button 14 → Switch to "Page 3: Route Monitoring"
Press Button 14 → Wrap to "Page 1: Checkpoints"
\end{lstlisting}

\begin{notebox}
Zone/TRP selection via joystick only works during \textbf{Surveillance} operational mode while in AutoSectorScan or TRPScan motion modes. During Tracking or Engagement, these buttons perform tracking gate control instead.
\end{notebox}

\section{TRP Scan Mode}

\subsection{What Is TRP Scan?}

\textbf{TRP} = \textbf{Target Reference Point} (pre-defined location of interest)

\textbf{Definition}: System sequentially slews to each TRP, dwells (pauses) for observation, then moves to next TRP in list.

\textbf{Purpose}:
\begin{itemize}
    \item Checkpoint verification (e.g., gate 1, gate 2, gate 3)
    \item Known threat areas (e.g., sniper hide sites)
    \item Periodic scans of fixed locations
    \item More efficient than manual searching
\end{itemize}

\textbf{Visual}:
\begin{lstlisting}
TRP 1 (30s dwell) → TRP 2 (30s dwell) → TRP 3 (30s dwell) → TRP 1
\end{lstlisting}

\subsection{Activating TRP Scan}

\textbf{Prerequisites}:
\begin{enumerate}
    \item At least one TRP must be defined
    \item System is in TRP Scan mode
\end{enumerate}

\begin{procedurebox}[TRP SCAN ACTIVATION]
\begin{enumerate}
    \item \textbf{Cycle to TRP Scan Mode}: Press Button 11 or 13 until OSD shows: \osd{MODE: TRP}
    \item \textbf{System Behavior}:
    \begin{itemize}
        \item Gimbal slews to first TRP in list
        \item Dwells for configured time (default: 30 seconds)
        \item Slews to next TRP
        \item Dwells again
        \item Repeats through entire TRP list, then loops back
    \end{itemize}
    \item \textbf{Dwell Time}: Configurable per TRP (5-120 seconds)
\end{enumerate}
\end{procedurebox}

\subsection{While TRP Scanning}

\textbf{During Dwell} (gimbal stationary at TRP):
\begin{itemize}
    \item Observe video
    \item Switch cameras
    \item Zoom
    \item Fire LRF
    \item Initiate tracking (if threat detected)
\end{itemize}

\textbf{During Slew} (gimbal moving between TRPs):
\begin{itemize}
    \item Gimbal is in motion
    \item Video may be blurred (fast slew)
    \item Wait for next dwell to observe
\end{itemize}

\subsection{TRP Scan Best Practices}

\begin{enumerate}
    \item \textbf{Define TRPs During Planning}:
    \begin{itemize}
        \item Pre-mission: Identify key locations
        \item Enter TRP coordinates via menu
        \item Test TRP scan before mission start
    \end{itemize}
    \item \textbf{Prioritize TRPs}: Place highest-threat areas first in list
    \item \textbf{Appropriate Dwell Times}:
    \begin{itemize}
        \item Short dwell (10-15s) for quick checks
        \item Long dwell (60s+) for detailed observation
    \end{itemize}
    \item \textbf{Combine with Manual}: Use TRP scan for routine surveillance, switch to Manual when threat detected
\end{enumerate}

\section{Radar Slew Mode (Optional)}

\textbf{Availability}: Only if radar system is integrated

\textbf{Description}: Gimbal automatically slews to radar-detected targets

\textbf{Purpose}:
\begin{itemize}
    \item Rapid threat response
    \item Automated cueing from radar
    \item Reduces operator workload
\end{itemize}

\begin{procedurebox}[RADAR SLEW OPERATION]
\begin{enumerate}
    \item \textbf{Cycle to Radar Slew Mode}: Press Button 11/13 until OSD shows: \osd{MODE: Radar}
    \item \textbf{System Behavior}:
    \begin{itemize}
        \item Waits for radar detection
        \item When radar detects target → Gimbal slews to radar coordinates
        \item OSD displays: \osd{RADAR CUE} or \osd{SLEWING TO RADAR}
        \item Operator confirms threat visually on video
    \end{itemize}
    \item \textbf{Operator Decision}:
    \begin{itemize}
        \item If threat confirmed → Initiate tracking or engage
        \item If false alarm → Wait for next radar cue or cycle to Manual
    \end{itemize}
\end{enumerate}
\end{procedurebox}

\begin{notebox}
Most El 7arress RCWS systems do NOT have radar. This mode will show "RADAR NOT AVAILABLE" if no radar connected.
\end{notebox}

\section{Motion Mode Quick Reference}

\small
\begin{longtable}{|L{0.20\textwidth}|L{0.25\textwidth}|L{0.20\textwidth}|L{0.28\textwidth}|}
\hline
\rowcolor{milblue!20}
\textbf{Mode} & \textbf{Use Case} & \textbf{Gimbal Control} & \textbf{Exit to Manual} \\
\hline
Manual & Direct engagement & Joystick & N/A (already manual) \\
\hline
AutoSectorScan & Perimeter surveillance & Automatic & Cycle mode or press TRK \\
\hline
TRP Scan & Checkpoint monitoring & Automatic & Cycle mode or press TRK \\
\hline
Radar Slew & Radar integration & Automatic & Cycle mode or press TRK \\
\hline
\end{longtable}

\textbf{Emergency Return to Manual}: Press Button 11/13 repeatedly until \osd{MODE: Manual} displays

\section{Zone Management - Sector Scans}

\subsection{Defining Sector Scan Zones}

Sector scan zones set the \textbf{left and right limits} for AutoSectorScan mode.

\textbf{Access}: Main Menu → Zone Definitions → Sector Scans

\subsubsection{Sector Scan Zone Parameters}

\small
\begin{longtable}{|L{0.25\textwidth}|L{0.35\textwidth}|L{0.33\textwidth}|}
\hline
\rowcolor{milblue!20}
\textbf{Parameter} & \textbf{Description} & \textbf{Typical Value} \\
\hline
Name & Zone identifier & "Front Gate", "Perimeter East" \\
\hline
Left Limit (Az) & Starting azimuth & 045\degree \\
\hline
Right Limit (Az) & Ending azimuth & 135\degree \\
\hline
Elevation & Scan elevation angle & 0\degree\ (horizon) \\
\hline
Scan Speed & Degrees per second & 5\degree/sec \\
\hline
Active & Enable/disable zone & ON / OFF \\
\hline
\end{longtable}

\subsubsection{Creating a Sector Scan Zone}

\textbf{Method 1: Manual Entry} (via menu)

\begin{procedurebox}[MANUAL SECTOR CREATION]
\begin{enumerate}
    \item \textbf{Access Menu}: MENU ✓ → Zone Definitions → Sector Scans → Add New Sector
    \item \textbf{Enter Name}: Use ▲/▼ to enter name characters, MENU ✓ to confirm
    \item \textbf{Set Left Limit}:
    \begin{itemize}
        \item Method A: Slew gimbal to desired left position, press MENU ✓ to "Capture Current Position"
        \item Method B: Manually enter azimuth value using ▲/▼
    \end{itemize}
    \item \textbf{Set Right Limit}: Same as left limit
    \item \textbf{Set Elevation}: Enter elevation angle (typically 0\degree)
    \item \textbf{Set Scan Speed}: Enter degrees/second (5\degree/sec recommended)
    \item \textbf{Enable Zone}: Set Active = ON
    \item \textbf{Save}: MENU ✓ → "Save Sector Scan"
\end{enumerate}
\end{procedurebox}

\textbf{Method 2: Quick Capture} (using joystick)

\begin{procedurebox}[QUICK SECTOR CAPTURE]
\begin{enumerate}
    \item \textbf{Position Gimbal}:
    \begin{itemize}
        \item Use Manual mode to slew to desired left limit
        \item Press \button{FN Button} + Hold for 2 seconds
        \item OSD displays: "LEFT LIMIT CAPTURED"
    \end{itemize}
    \item \textbf{Position Right Limit}:
    \begin{itemize}
        \item Slew to desired right limit
        \item Press \button{FN Button} + Hold for 2 seconds
        \item OSD displays: "RIGHT LIMIT CAPTURED, SECTOR SCAN ZONE CREATED"
    \end{itemize}
    \item \textbf{System Auto-Creates Zone}:
    \begin{itemize}
        \item Default name: "Sector X"
        \item Default elevation: Current elevation
        \item Default speed: 5\degree/sec
    \end{itemize}
    \item \textbf{Edit if Needed}: Access menu to rename or adjust parameters
\end{enumerate}
\end{procedurebox}

\subsection{Activating / Deactivating Sector Zones}

\textbf{Multiple Zones}:
\begin{itemize}
    \item You can define multiple sector scan zones
    \item Only ONE can be active at a time
\end{itemize}

\textbf{To Activate a Zone}:
\begin{enumerate}
    \item MENU ✓ → Zone Definitions → Sector Scans
    \item Use ▲/▼ to select desired zone
    \item MENU ✓ → "Set as Active"
    \item OSD displays: "SECTOR ZONE [Name] ACTIVE"
\end{enumerate}

\textbf{To Deactivate All Sectors}:
\begin{enumerate}
    \item MENU ✓ → Zone Definitions → Sector Scans
    \item Select active zone
    \item MENU ✓ → "Deactivate"
\end{enumerate}

\section{Zone Management - Target Reference Points}

\subsection{Defining TRPs}

TRPs are \textbf{fixed locations} the system can automatically slew to.

\textbf{Access}: Main Menu → Zone Definitions → TRPs

\subsubsection{TRP Parameters}

\small
\begin{longtable}{|L{0.25\textwidth}|L{0.35\textwidth}|L{0.33\textwidth}|}
\hline
\rowcolor{milblue!20}
\textbf{Parameter} & \textbf{Description} & \textbf{Typical Value} \\
\hline
Name & TRP identifier & "Gate 1", "Bunker", "Hill 203" \\
\hline
Azimuth & Direction to TRP & 090\degree \\
\hline
Elevation & Angle to TRP & +5\degree \\
\hline
Dwell Time & Observation time & 30 seconds \\
\hline
Active & Enable/disable & ON / OFF \\
\hline
\end{longtable}

\subsubsection{Creating a TRP}

\textbf{Method 1: Capture Current Position}

\begin{procedurebox}[TRP CAPTURE]
\begin{enumerate}
    \item \textbf{Manual Mode}: Slew gimbal to desired TRP location, zoom/focus on exact point
    \item \textbf{Access Menu}: MENU ✓ → Zone Definitions → TRPs → Add TRP
    \item \textbf{Capture Position}: Select "Capture Current Position", MENU ✓ to confirm
    \item \textbf{Enter Name}: Use ▲/▼ to enter TRP name, MENU ✓ to confirm
    \item \textbf{Set Dwell Time}: Enter seconds (5-120), Default: 30 seconds
    \item \textbf{Enable}: Set Active = ON
    \item \textbf{Save}: MENU ✓ → "Save TRP"
\end{enumerate}
\end{procedurebox}

\textbf{Method 2: Manual Coordinate Entry}

\begin{procedurebox}[TRP MANUAL ENTRY]
\begin{enumerate}
    \item \textbf{Access Menu}: MENU ✓ → Zone Definitions → TRPs → Add TRP
    \item \textbf{Enter Azimuth}: Use ▲/▼ to enter degrees (000-359)
    \item \textbf{Enter Elevation}: Use ▲/▼ to enter degrees (-20 to +60)
    \item \textbf{Continue}: Enter name, dwell time, enable, and save
\end{enumerate}
\end{procedurebox}

\subsection{Managing TRP List}

\textbf{TRP Sequence}:
\begin{itemize}
    \item TRPs are visited in the order they appear in list
    \item To reorder:
    \begin{enumerate}
        \item MENU ✓ → Zone Definitions → TRPs
        \item Select TRP
        \item "Move Up" or "Move Down"
    \end{enumerate}
\end{itemize}

\textbf{Editing TRPs}:
\begin{itemize}
    \item Select TRP from list
    \item MENU ✓ → "Edit TRP"
    \item Modify parameters
    \item Save changes
\end{itemize}

\textbf{Deleting TRPs}:
\begin{itemize}
    \item Select TRP from list
    \item MENU ✓ → "Delete TRP"
    \item Confirm deletion
\end{itemize}

\section{Zone Management - No-Fire \& No-Traverse}

\subsection{Viewing No-Fire Zones}

\textbf{Access}: Main Menu → Zone Definitions → No-Fire Zones

\textbf{Display}:
\begin{itemize}
    \item List of all defined no-fire zones
    \item Each zone shows:
    \begin{itemize}
        \item Name (e.g., "Friendly FOB", "Civilian Area 1")
        \item Boundary type (Polygon, Circle, Arc)
        \item Active status (ON/OFF)
    \end{itemize}
\end{itemize}

\textbf{Operator Permission}:
\begin{itemize}
    \item \textbf{CAN}: View zones, see boundaries on map overlay
    \item \textbf{CANNOT}: Modify boundaries, delete zones, override zones
\end{itemize}

\begin{notebox}
Modification usually requires commander/supervisor authorization.
\end{notebox}

\subsection{Viewing No-Traverse Zones}

\textbf{Access}: Main Menu → Zone Definitions → No-Traverse Zones

\textbf{Display}:
\begin{itemize}
    \item List of all defined no-traverse zones
    \item Each zone shows:
    \begin{itemize}
        \item Name (e.g., "Rear 90\degree", "Antenna Area")
        \item Azimuth limits
        \item Active status
    \end{itemize}
\end{itemize}

\textbf{Purpose Reminder}:
\begin{itemize}
    \item No-traverse zones prevent gimbal movement into restricted areas
    \item Protects vehicle structure, equipment, personnel
\end{itemize}

\section{Saving \& Loading Zone Configurations}

\subsection{Saving Zone Configuration}

\textbf{Purpose}: Save all zones (sectors, TRPs, no-fire, no-traverse) to file for later use

\begin{procedurebox}[SAVE CONFIGURATION]
\begin{enumerate}
    \item \textbf{Access Menu}: MENU ✓ → Zone Definitions → Save/Load → Save Configuration
    \item \textbf{Enter Filename}: Use ▲/▼ to enter filename (e.g., "MISSION\_20250115")
    \item \textbf{Confirm Save}: MENU ✓ → "Save"
    \item OSD displays: "ZONE CONFIG SAVED"
\end{enumerate}
\end{procedurebox}

\textbf{File Location}: Saved to internal storage (typically /configs/zones/)

\subsection{Loading Zone Configuration}

\textbf{Purpose}: Load previously saved zone configuration

\begin{procedurebox}[LOAD CONFIGURATION]
\begin{enumerate}
    \item \textbf{Access Menu}: MENU ✓ → Zone Definitions → Save/Load → Load Configuration
    \item \textbf{Select File}: Use ▲/▼ to browse saved configurations
    \item \textbf{Confirm Load}: MENU ✓ → "Load"
    \item OSD displays: "ZONE CONFIG LOADED"
\end{enumerate}
\end{procedurebox}

\begin{warningbox}
Loading a configuration OVERWRITES current zones. Save current zones first if needed.
\end{warningbox}

\subsection{Default Zone Configuration}

\textbf{Default Zones}:
\begin{itemize}
    \item System ships with default no-traverse zones (vehicle-specific)
    \item Default no-fire zones may be empty (mission-dependent)
\end{itemize}

\textbf{Restoring Defaults}:
\begin{enumerate}
    \item MENU ✓ → Zone Definitions → Save/Load → Restore Defaults
    \item Confirm: "RESTORE DEFAULT ZONES?"
    \item MENU ✓ → "YES"
\end{enumerate}

\section{Surveillance Best Practices}

\subsection{Choosing the Right Mode}

\small
\begin{longtable}{|L{0.30\textwidth}|L{0.25\textwidth}|L{0.38\textwidth}|}
\hline
\rowcolor{milblue!20}
\textbf{Situation} & \textbf{Recommended Mode} & \textbf{Rationale} \\
\hline
Direct threat engagement & Manual & Full control, immediate response \\
\hline
Perimeter watch (quiet) & AutoSectorScan & Automated, frees attention \\
\hline
Checkpoint routine & TRP Scan & Efficient for fixed locations \\
\hline
High-threat area scan & Manual & Requires operator judgment \\
\hline
Radar-integrated ops & Radar Slew & Rapid response to radar cues \\
\hline
\end{longtable}

\subsection{Combining Modes with Tracking}

\textbf{Workflow Example}:
\begin{enumerate}
    \item Start in AutoSectorScan (perimeter surveillance)
    \item Threat detected during scan
    \item Press TRK → System switches to Manual, starts tracking acquisition
    \item Lock onto threat (second TRK press)
    \item Engage or monitor as threat tracked
    \item Abort tracking (third TRK press)
    \item Resume surveillance: Press Button 11/13 to return to AutoSectorScan
\end{enumerate}

\subsection{Zone Discipline}

\textbf{Before Mission}:
\begin{checklist}
    \item Load appropriate zone configuration
    \item Verify no-fire zones match current ROE
    \item Test sector scans and TRPs
    \item Brief all operators on zones
\end{checklist}

\textbf{During Mission}:
\begin{checklist}
    \item Respect all zone warnings
    \item Never attempt to override no-fire zones without authorization
    \item Report zone boundary errors to command
    \item Update TRPs as mission evolves (if authorized)
\end{checklist}

\textbf{After Mission}:
\begin{checklist}
    \item Save zone configuration if modified
    \item Debrief on zone effectiveness
    \item Recommend adjustments for future missions
\end{checklist}

\section{Student Review Questions}

\begin{enumerate}
    \item What are the four motion modes available on the RCWS?
    \item How do you cycle between motion modes?
    \item What is the purpose of Auto Sector Scan mode?
    \item What does TRP stand for?
    \item What is the typical dwell time at a TRP?
    \item How do you create a sector scan zone using the quick capture method?
    \item Can operators modify no-fire zone boundaries?
    \item What happens when you load a zone configuration?
    \item Which motion mode is recommended for direct threat engagement?
    \item How do you reorder TRPs in the scan sequence?
\end{enumerate}