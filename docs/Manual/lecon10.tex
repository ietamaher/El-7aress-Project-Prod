% ============================================================================
% LESSON 10 - PRACTICAL TRAINING & EVALUATION
% ============================================================================
\lesson{10 - PRACTICAL TRAINING \& EVALUATION}

\begin{tabular}{@{}ll@{}}
\textbf{Duration:} & 8 hours (Hands-On Training \& Evaluation) \\
\textbf{Type:} & Practical + Evaluation \\
\textbf{References:} & All previous lessons, evaluation checklist \\
\end{tabular}

\section{LEARNING OBJECTIVES}

Upon completion of this lesson, operators will be able to:
\begin{itemize}
    \item Demonstrate proficiency in all operator tasks
    \item Pass written examination (80\% minimum)
    \item Pass practical performance evaluation
    \item Qualify for operational duty
\end{itemize}

% ============================================================================
\section{PRACTICAL TRAINING EXERCISES}

\subsection{EXERCISE 1: SYSTEM STARTUP \& HEALTH CHECK}

\textbf{Duration:} 30 minutes

\subsubsection{Task}
Power up system, perform pre-operation checks

\subsubsection{Performance Standards}

\begin{checklist}
    \item Correct power-up sequence
    \item All devices detected and connected
    \item Pre-operation checklist 100\% complete
    \item Ready for operations within 5 minutes
\end{checklist}

\subsubsection{Evaluation Criteria}

\spectable{
\begin{tabular}{|L{0.40\textwidth}|C{0.15\textwidth}|L{0.35\textwidth}|}
\hline
\rowcolor{milblue!20}
\textbf{Task Element} & \textbf{Weight} & \textbf{Standard} \\
\hline
Power-up sequence & 25\% & All steps in correct order \\
\hline
Device verification & 25\% & All devices confirmed online \\
\hline
Checklist completion & 30\% & All items checked, documented \\
\hline
Time to ready & 20\% & <5 minutes from power-on \\
\hline
\end{tabular}
}

\subsubsection{Common Errors to Avoid}
\begin{itemize}
    \item Skipping System Status check
    \item Not documenting baseline temperatures
    \item Failing to test all motion axes
    \item Rushing through checklist (missing items)
\end{itemize}

% ----------------------------------------------------------------------------
\subsection{EXERCISE 2: MANUAL GIMBAL CONTROL}

\textbf{Duration:} 30 minutes

\subsubsection{Task}
Navigate gimbal to designated azimuth/elevation coordinates

\subsubsection{Performance Standards}

\begin{checklist}
    \item Smooth, controlled movements
    \item Accuracy: ±2\degree\ azimuth, ±2\degree\ elevation
    \item Complete 5 targets within 3 minutes
\end{checklist}

\subsubsection{Target Coordinates}

\spectable{
\begin{tabular}{|C{0.15\textwidth}|C{0.25\textwidth}|C{0.25\textwidth}|C{0.20\textwidth}|}
\hline
\rowcolor{milblue!20}
\textbf{Target} & \textbf{Azimuth} & \textbf{Elevation} & \textbf{Time Limit} \\
\hline
Target 1 & 45\degree & +10\degree & 30 seconds \\
\hline
Target 2 & 180\degree & +20\degree & 30 seconds \\
\hline
Target 3 & 270\degree & -5\degree & 40 seconds \\
\hline
Target 4 & 90\degree & +30\degree & 40 seconds \\
\hline
Target 5 & 0\degree & 0\degree & 40 seconds \\
\hline
\end{tabular}
}

\subsubsection{Evaluation Criteria}

\textbf{GO Criteria:}
\begin{itemize}
    \item All targets acquired within ±2\degree
    \item No jerky or erratic motion
    \item No gimbal limit violations
    \item Completed within time limit
\end{itemize}

\textbf{NO-GO Criteria:}
\begin{itemize}
    \item Accuracy >±5\degree\ on any target
    \item Exceeds time limit by >50\%
    \item Gimbal limit violation
    \item Unsafe control (rapid, uncontrolled movements)
\end{itemize}

% ----------------------------------------------------------------------------
\subsection{EXERCISE 3: TARGET ACQUISITION \& TRACKING}

\textbf{Duration:} 1 hour

\subsubsection{Task}
Acquire and track stationary and moving targets

\subsubsection{Performance Standards}

\begin{checklist}
    \item Acquisition gate correctly sized (10-30\% margin)
    \item Lock achieved within 10 seconds
    \item Track maintained >30 seconds
    \item Clean abort on command
\end{checklist}

\subsubsection{Scenarios}

\textbf{Scenario A: Stationary Target}
\begin{enumerate}
    \item Instructor designates stationary target
    \item Student acquires and locks tracking
    \item Maintain track for 30 seconds
    \item Abort tracking on command
\end{enumerate}

\textbf{Scenario B: Slow-Moving Target}
\begin{enumerate}
    \item Instructor designates moving target (vehicle, simulated)
    \item Student acquires and locks tracking
    \item Maintain track through target motion
    \item Monitor track confidence (should remain >70\%)
\end{enumerate}

\textbf{Scenario C: Fast-Moving Target}
\begin{enumerate}
    \item Instructor designates fast-moving target
    \item Student adjusts FOV (zoom out if needed)
    \item Acquire and lock tracking
    \item Manage "LEAD ANGLE LAG" warning (wait for green)
\end{enumerate}

\subsubsection{Evaluation Criteria}

\spectable{
\begin{tabular}{|L{0.40\textwidth}|C{0.15\textwidth}|L{0.35\textwidth}|}
\hline
\rowcolor{milblue!20}
\textbf{Task Element} & \textbf{Weight} & \textbf{Standard} \\
\hline
Gate sizing & 20\% & 10-30\% margin, target fully visible \\
\hline
Lock achievement & 30\% & <10 seconds from gate sizing \\
\hline
Track maintenance & 30\% & >30 seconds, confidence >70\% \\
\hline
Abort execution & 20\% & Immediate, clean, returns to manual \\
\hline
\end{tabular}
}

% ----------------------------------------------------------------------------
\subsection{EXERCISE 4: SIMULATED ENGAGEMENT}

\textbf{Duration:} 1 hour

\subsubsection{Task}
Complete engagement sequence (no live fire)

\subsubsection{Performance Standards}

\begin{checklist}
    \item PID confirmed before engagement
    \item Tracking established (Active Lock)
    \item LAC activated (moving target)
    \item Safety checks complete
    \item Simulated fire (correct procedure)
    \item BDA (Battle Damage Assessment)
\end{checklist}

\subsubsection{Engagement Sequence}

\begin{procedurebox}[SIMULATED ENGAGEMENT]
\textbf{Phase 1: Target Detection \& Identification}
\begin{enumerate}
    \item Instructor designates target (verbal or visual cue)
    \item Student detects and slews to target
    \item Student identifies target (announces type, description)
    \item Student confirms PID and ROE compliance
\end{enumerate}

\textbf{Phase 2: Target Acquisition}
\begin{enumerate}
    \item Enter acquisition mode (\button{Button 4})
    \item Size gate appropriately
    \item Request lock-on (\button{Button 4})
    \item Achieve Active Lock (green gate)
\end{enumerate}

\textbf{Phase 3: Fire Control Setup}
\begin{enumerate}
    \item Range target (LRF or manual estimation)
    \item Enable LAC if target moving (\button{Button 3 + Button 2})
    \item Verify LAC status (LEAD ANGLE ON if applicable)
    \item Announce range and lead status
\end{enumerate}

\textbf{Phase 4: Safety Checks}
\begin{enumerate}
    \item Verify no zone violations (check OSD)
    \item Verify friendlies clear (announce "CLEAR")
    \item Verify target still valid
    \item Announce "READY TO ENGAGE"
\end{enumerate}

\textbf{Phase 5: Simulated Engagement}
\begin{enumerate}
    \item Engage Master Arm (\button{Button 0})
    \item Announce "ARMED"
    \item Simulate fire (\textit{do not press Button 5, announce "FIRE"})
    \item Announce "CEASE FIRE"
    \item Disarm (\button{Button 0})
\end{enumerate}

\textbf{Phase 6: Battle Damage Assessment}
\begin{enumerate}
    \item Observe simulated impact
    \item Announce BDA result (destroyed/damaged/missed)
    \item Stop tracking if target neutralized
\end{enumerate}
\end{procedurebox}

\subsubsection{Critical Errors (Automatic Failure)}

\begin{itemize}
    \item ❌ No PID before engagement
    \item ❌ Zone violation not recognized
    \item ❌ Friendlies not checked
    \item ❌ Wrong target engaged
    \item ❌ Skipped safety checks
\end{itemize}

% ----------------------------------------------------------------------------
\subsection{EXERCISE 5: WEAPON ZEROING}

\textbf{Duration:} 1 hour

\subsubsection{Task}
Zero weapon at 200m range

\subsubsection{Performance Standards}

\begin{checklist}
    \item Correct zeroing procedure
    \item Zero applied successfully
    \item Verification shot within ±5cm
    \item "Z" indicator appears on OSD
\end{checklist}

\subsubsection{Zeroing Procedure}

\begin{enumerate}
    \item Access Zeroing menu (\button{MENU ✓} → Zeroing → \button{VAL})
    \item Aim at target center (200m range)
    \item Fire test shot (observe impact location)
    \item Move reticle to impact point (joystick)
    \item Apply zero (\button{MENU ✓})
    \item Verify "Z" indicator on OSD
    \item Fire verification shot (should hit reticle center)
\end{enumerate}

\subsubsection{Evaluation Criteria}

\textbf{GO Criteria:}
\begin{itemize}
    \item Correct procedure followed
    \item Zero applied (Z indicator visible)
    \item Verification shot within ±5cm of aim point
\end{itemize}

\textbf{NO-GO Criteria:}
\begin{itemize}
    \item Incorrect procedure (steps out of order)
    \item Zero not applied (no Z indicator)
    \item Verification shot >±10cm from aim point
\end{itemize}

\begin{notebox}
If verification shot fails, student may retry zeroing procedure once.
\end{notebox}

% ----------------------------------------------------------------------------
\subsection{EXERCISE 6: EMERGENCY PROCEDURES}

\textbf{Duration:} 1 hour

\subsubsection{Task}
Respond to simulated emergencies

\subsubsection{Scenarios}

\textbf{Scenario 1: Wrong Target Identified}
\begin{itemize}
    \item \textbf{Setup:} Student tracking target
    \item \textbf{Event:} Instructor announces "FRIENDLY FORCES IN TRACKING GATE"
    \item \textbf{Required Action:} Immediate tracking abort (\button{Button 4} double-click)
    \item \textbf{Standard:} Abort within 500ms of announcement
\end{itemize}

\textbf{Scenario 2: Runaway Gimbal}
\begin{itemize}
    \item \textbf{Setup:} Instructor simulates runaway gimbal (unexpected motion)
    \item \textbf{Event:} Gimbal moves erratically without joystick input
    \item \textbf{Required Action:} Immediate E-Stop, then Station Power OFF
    \item \textbf{Standard:} E-Stop within 1 second, power OFF within 3 seconds
\end{itemize}

\textbf{Scenario 3: Accidental Discharge}
\begin{itemize}
    \item \textbf{Setup:} Simulated weapon armed
    \item \textbf{Event:} Instructor announces "WEAPON FIRING" (unintended)
    \item \textbf{Required Action:} Release fire, release Master Arm, point safe, E-Stop, power OFF
    \item \textbf{Standard:} All steps within 5 seconds
\end{itemize}

\textbf{Scenario 4: Lost Communication}
\begin{itemize}
    \item \textbf{Setup:} Operator performing surveillance
    \item \textbf{Event:} Instructor simulates radio failure
    \item \textbf{Required Action:} Switch backup radio, attempt recovery, follow SOP
    \item \textbf{Standard:} Backup radio within 30 seconds, correct SOP followed
\end{itemize}

\subsubsection{Performance Standards}

\begin{checklist}
    \item Immediate recognition of emergency (<2 seconds)
    \item Correct procedure executed
    \item Reaction time <2 seconds (critical emergencies)
    \item All safety steps completed
\end{checklist}

% ----------------------------------------------------------------------------
\subsection{EXERCISE 7: TROUBLESHOOTING}

\textbf{Duration:} 1 hour

\subsubsection{Task}
Diagnose and resolve simulated faults

\subsubsection{Faults Presented}

\textbf{Fault 1: Joystick Disconnected}
\begin{itemize}
    \item \textbf{Symptom:} No gimbal response, no joystick LED
    \item \textbf{Expected Diagnosis:} USB cable disconnected
    \item \textbf{Expected Fix:} Reconnect USB, verify System Status
\end{itemize}

\textbf{Fault 2: Camera Offline}
\begin{itemize}
    \item \textbf{Symptom:} Black screen, "No Signal"
    \item \textbf{Expected Diagnosis:} Camera disconnected or fault
    \item \textbf{Expected Fix:} Switch to alternate camera, power cycle if both offline
\end{itemize}

\textbf{Fault 3: Tracking Failure}
\begin{itemize}
    \item \textbf{Symptom:} Cannot achieve lock, stays in LOCK PENDING
    \item \textbf{Expected Diagnosis:} Poor target contrast or gate sizing
    \item \textbf{Expected Fix:} Improve contrast (switch camera/LUT), resize gate
\end{itemize}

\textbf{Fault 4: LRF No Reading}
\begin{itemize}
    \item \textbf{Symptom:} Persistent "NO ECHO"
    \item \textbf{Expected Diagnosis:} Target too far or obscured
    \item \textbf{Expected Fix:} Re-fire on closer/more reflective target, or manual range estimation
\end{itemize}

\subsubsection{Performance Standards}

\begin{checklist}
    \item Correct diagnosis using SLAF method
    \item Operator-level fix applied (if applicable)
    \item Proper escalation to maintenance (if needed)
    \item Solution verified (problem resolved)
\end{checklist}

\subsubsection{Evaluation Criteria}

\spectable{
\begin{tabular}{|L{0.40\textwidth}|C{0.15\textwidth}|L{0.35\textwidth}|}
\hline
\rowcolor{milblue!20}
\textbf{Task Element} & \textbf{Weight} & \textbf{Standard} \\
\hline
Problem diagnosis & 40\% & Correct root cause identified \\
\hline
Troubleshooting method & 20\% & SLAF method applied correctly \\
\hline
Fix application & 30\% & Appropriate fix attempted \\
\hline
Escalation decision & 10\% & Correct escalation if needed \\
\hline
\end{tabular}
}

% ============================================================================
\section{WRITTEN EXAMINATION}

\subsection{EXAMINATION FORMAT}

\textbf{Format:} 50 multiple-choice questions

\textbf{Passing Score:} 80\% (40/50 correct)

\textbf{Time Limit:} 60 minutes

\subsection{Topics Covered}

\spectable{
\begin{tabular}{|L{0.50\textwidth}|C{0.25\textwidth}|C{0.15\textwidth}|}
\hline
\rowcolor{milblue!20}
\textbf{Topic} & \textbf{Questions} & \textbf{Weight} \\
\hline
Safety procedures & 10 & 20\% \\
\hline
Basic operation & 10 & 20\% \\
\hline
Tracking \& engagement & 10 & 20\% \\
\hline
Ballistics \& fire control & 10 & 20\% \\
\hline
Emergency procedures & 5 & 10\% \\
\hline
System status \& troubleshooting & 5 & 10\% \\
\hline
\textbf{TOTAL} & \textbf{50} & \textbf{100\%} \\
\hline
\end{tabular}
}

\subsection{Sample Questions}

\subsubsection{Safety Procedures}
\begin{enumerate}
    \item When should the E-Stop button be activated?
    \begin{itemize}
        \item A) Only when ordered by command
        \item B) When personnel are in line of fire
        \item C) After every mission
        \item D) Only during maintenance
    \end{itemize}
    \textbf{Correct Answer: B}
\end{enumerate}

\subsubsection{Tracking \& Engagement}
\begin{enumerate}
    \item What is the correct acquisition gate sizing margin around the target?
    \begin{itemize}
        \item A) 0-5\% margin
        \item B) 10-30\% margin
        \item C) 50-75\% margin
        \item D) Gate should be as tight as possible
    \end{itemize}
    \textbf{Correct Answer: B}
\end{enumerate}

\subsubsection{Emergency Procedures}
\begin{enumerate}
    \item How do you perform an emergency tracking abort?
    \begin{itemize}
        \item A) Press E-Stop button
        \item B) Single-click Button 4
        \item C) Double-click Button 4 (<500ms)
        \item D) Turn Station Power OFF
    \end{itemize}
    \textbf{Correct Answer: C}
\end{enumerate}

\subsection{Examination Rules}

\begin{itemize}
    \item Closed book (no reference materials)
    \item No electronic devices
    \item Questions may be asked for clarification (instructor will not provide answers)
    \item Review allowed if time permits
    \item Minimum 80\% required to pass
\end{itemize}

% ============================================================================
\section{PERFORMANCE EVALUATION}

\subsection{EVALUATION FORMAT}

\textbf{Format:} Instructor-observed practical assessment

\textbf{Duration:} 2-3 hours

\textbf{Passing Standard:} 80\% overall, NO critical errors

\subsection{Evaluation Criteria}

\begin{longtable}{|L{0.35\textwidth}|C{0.15\textwidth}|L{0.40\textwidth}|}
\hline
\rowcolor{milblue!20}
\textbf{Task} & \textbf{Weight} & \textbf{GO/NO-GO Criteria} \\
\hline
\endfirsthead

\hline
\rowcolor{milblue!20}
\textbf{Task} & \textbf{Weight} & \textbf{GO/NO-GO Criteria} \\
\hline
\endhead

\textbf{System Startup} & 10\% & All steps correct, <5 min \\
\hline
\textbf{Manual Control} & 10\% & Smooth, accurate (±2\degree) \\
\hline
\textbf{Tracking} & 20\% & Lock achieved, maintained >30s \\
\hline
\textbf{Simulated Engagement} & 25\% & Full sequence, all safety checks \\
\hline
\textbf{Zeroing} & 15\% & Correct procedure, accurate \\
\hline
\textbf{Emergency Response} & 15\% & Immediate, correct procedure \\
\hline
\textbf{Troubleshooting} & 5\% & Correct diagnosis \& action \\
\hline
\textbf{TOTAL} & \textbf{100\%} & \textbf{≥80\% required} \\
\hline
\end{longtable}

\subsection{Critical Errors (Automatic Failure)}

\textbf{Any of the following results in automatic NO-GO:}

\begin{itemize}
    \item ❌ Safety violation (no-fire zone, friendly fire)
    \item ❌ Failed to E-Stop when required
    \item ❌ Accidental discharge (simulated)
    \item ❌ Wrong target engagement
    \item ❌ Bypassed safety interlocks
    \item ❌ Damaged equipment through negligence
\end{itemize}

\subsection{Scoring}

\subsubsection{Individual Task Scoring}

Each task scored 0-100\%:
\begin{itemize}
    \item \textbf{100\%:} Flawless execution
    \item \textbf{90\%:} Minor errors, corrected immediately
    \item \textbf{80\%:} Meets minimum standard
    \item \textbf{70\%:} Below standard, requires additional training
    \item \textbf{<70\%:} Unsatisfactory, significant deficiencies
\end{itemize}

\subsubsection{Overall Score Calculation}

\begin{enumerate}
    \item Calculate weighted score for each task
    \item Sum all weighted scores = Overall Score
    \item Check for critical errors (automatic failure if any)
    \item Final result: GO (≥80\%, no critical errors) or NO-GO (<80\% or critical error)
\end{enumerate}

\textbf{Example Calculation:}
\begin{itemize}
    \item System Startup: 90\% × 10\% = 9 points
    \item Manual Control: 85\% × 10\% = 8.5 points
    \item Tracking: 95\% × 20\% = 19 points
    \item Simulated Engagement: 80\% × 25\% = 20 points
    \item Zeroing: 90\% × 15\% = 13.5 points
    \item Emergency Response: 100\% × 15\% = 15 points
    \item Troubleshooting: 75\% × 5\% = 3.75 points
    \item \textbf{Overall Score: 88.75\% = GO}
\end{itemize}

% ============================================================================
\section{QUALIFICATION REQUIREMENTS}

\subsection{TO QUALIFY AS EL 7ARRESS RCWS OPERATOR}

\textbf{Must pass ALL:}

\begin{checklist}
    \item Written examination (≥80\%)
    \item Practical evaluation (≥80\%)
    \item Zero critical errors
    \item Instructor recommendation
\end{checklist}

\subsection{Qualification Certificate}

Upon successful completion, operator receives:
\begin{itemize}
    \item Certificate of Qualification
    \item Operator ID card
    \item Entry in training records
    \item Authorization for operational duty
\end{itemize}

\textbf{Qualification Valid:} 12 months from date of issue

% ----------------------------------------------------------------------------
\subsection{REQUALIFICATION REQUIREMENTS}

Requalification required:
\begin{itemize}
    \item \textbf{Annually} (refresher training + re-test)
    \item \textbf{After >6 months away from system}
    \item \textbf{After major system upgrade}
    \item \textbf{If unsafe practices observed}
    \item \textbf{Upon commander's discretion}
\end{itemize}

\subsubsection{Requalification Process}

\textbf{Annual Requalification:}
\begin{enumerate}
    \item 4-hour refresher training (classroom)
    \item Abbreviated written exam (25 questions, 80\% required)
    \item Practical evaluation (key tasks only: engagement, emergencies)
    \item Verification of zero proficiency
\end{enumerate}

\textbf{Extended Absence (>6 months):}
\begin{enumerate}
    \item Full course review recommended
    \item Full written examination
    \item Full practical evaluation
    \item May require full re-qualification at instructor's discretion
\end{enumerate}

% ============================================================================
\section{REMEDIAL TRAINING}

\subsection{IF STUDENT FAILS QUALIFICATION}

\textbf{Failed Written Exam:}
\begin{itemize}
    \item Review incorrect answers with instructor
    \item Additional study (weak areas identified)
    \item Re-test after minimum 24 hours
    \item Maximum 2 re-test attempts
\end{itemize}

\textbf{Failed Practical Evaluation:}
\begin{itemize}
    \item Review performance with instructor (identify deficiencies)
    \item Additional practice on failed tasks
    \item Re-test after minimum 48 hours
    \item Maximum 2 re-test attempts
\end{itemize}

\textbf{Critical Error:}
\begin{itemize}
    \item Immediate training halt
    \item Safety review with instructor and commander
    \item Mandatory additional training (duration determined by commander)
    \item Re-start evaluation from beginning
\end{itemize}

\subsection{Maximum Attempts}

\begin{itemize}
    \item \textbf{Written Exam:} 3 attempts maximum (initial + 2 retakes)
    \item \textbf{Practical Eval:} 3 attempts maximum (initial + 2 retakes)
    \item \textbf{If failed after 3 attempts:} Student removed from training, command decision on future training
\end{itemize}

% ============================================================================
\section{POST-QUALIFICATION}

\subsection{CONTINUING EDUCATION}

Qualified operators are expected to:
\begin{itemize}
    \item Maintain proficiency through regular operations
    \item Attend quarterly refresher training sessions
    \item Review operator manual periodically
    \item Stay current on system updates and modifications
    \item Mentor junior operators (once experienced)
\end{itemize}

\subsection{ADVANCED TRAINING}

Experienced operators may qualify for advanced courses:
\begin{itemize}
    \item \textbf{Instructor Operator:} Train new operators
    \item \textbf{Maintainer Cross-Training:} Operator-level maintenance
    \item \textbf{Tactics Development:} Advanced employment techniques
    \item \textbf{System Specialist:} Deep technical expertise
\end{itemize}

\subsection{PERFORMANCE TRACKING}

Operator performance tracked through:
\begin{itemize}
    \item Mission logs (accuracy, efficiency)
    \item Incident reports (any safety issues)
    \item Peer evaluations (teamwork, professionalism)
    \item Annual requalification scores
\end{itemize}

\textbf{Outstanding performance may result in:}
\begin{itemize}
    \item Commendation
    \item Selection for advanced training
    \item Instructor duty assignment
    \item Leadership positions
\end{itemize}

% ============================================================================
\section{FINAL CHECKLIST}

\subsection{BEFORE FINAL EVALUATION}

\textbf{Student should verify:}

\begin{checklist}
    \item Attended all lessons (1-10)
    \item Completed all practical exercises
    \item Reviewed all reference materials
    \item Practiced emergency procedures
    \item Comfortable with all system functions
    \item Any questions answered by instructor
    \item Well-rested and prepared for evaluation day
\end{checklist}

\subsection{ON EVALUATION DAY}

\textbf{Bring:}
\begin{itemize}
    \item Government-issued ID
    \item Training record
    \item Writing materials (pen/pencil for written exam)
    \item Appropriate uniform
    \item Positive attitude and confidence
\end{itemize}

\textbf{Arrive:}
\begin{itemize}
    \item 15 minutes before scheduled time
    \item Well-rested and alert
    \item Ready to demonstrate proficiency
\end{itemize}

% ============================================================================
\section{CONCLUSION}

\subsection{IMPORTANCE OF QUALIFICATION}

El 7arress RCWS operators hold a position of great responsibility:
\begin{itemize}
    \item Control of lethal weapon system
    \item Protection of friendly forces
    \item Engagement of hostile threats
    \item Safety of personnel and equipment
\end{itemize}

Qualification ensures operators are:
\begin{itemize}
    \item Technically proficient
    \item Safety-conscious
    \item Tactically sound
    \item Ready for operational duty
\end{itemize}

\subsection{FINAL WORDS}

\begin{quotation}
\textit{``A qualified operator is not just someone who knows how to operate the system - it is someone who can be trusted with a weapon, who makes sound decisions under pressure, and who puts safety and mission accomplishment above all else.''}

\hfill --- El 7arress Training Doctrine
\end{quotation}

\vspace{1cm}

\textbf{Good luck on your qualification!}

% ============================================================================
% END OF LESSON 10
% ============================================================================