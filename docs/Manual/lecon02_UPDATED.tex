% ============================================================================
% LESSON 2 - BASIC OPERATION (UPDATED FOR QT6)
% ============================================================================
\lesson{2: BASIC OPERATION}

\noindent
\begin{tabular}{@{}ll@{}}
\textbf{Duration:} & 4 hours \\
\textbf{Type:} & Classroom + Practical \\
\textbf{References:} & Operator manual, system startup checklist \\
\end{tabular}

\section{Introduction}

\textbf{Lesson Purpose}: This lesson teaches complete system startup, ammunition loading, DCU and joystick operations, OSD interpretation, and proper shutdown procedures.

\textbf{Learning Objectives}:
\begin{itemize}
    \item Perform complete system startup procedure
    \item Execute ammunition feed cycle
    \item Operate all DCU buttons, switches, and controls
    \item Control gimbal movement using joystick
    \item Utilize all 19 joystick buttons effectively
    \item Switch between day and thermal cameras
    \item Operate camera zoom and LUT controls
    \item Interpret all OSD elements correctly including LAC status
    \item Perform normal system shutdown
\end{itemize}

\section{System Startup Procedure}

\subsection{Pre-Startup Checklist}

Before powering on the system, verify:

\begin{checklist}
    \item Walk-around inspection complete (Lesson 1) - all items GO
    \item Weapon cleared per Appendix A
    \item Ammunition removed or accounted for
    \item All personnel clear of turret (minimum 2 meters)
    \item Vehicle power available (check voltage: 20-30V DC nominal)
    \item Operator qualified and authorized
    \item Mission briefing received (zones, ROE, threats)
    \item Communication established with command
\end{checklist}

\begin{warningbox}
Do not start system if any checklist item is not complete.
\end{warningbox}

\subsection{Startup Sequence (10 Steps)}

Perform steps in order. Do not skip steps.

\begin{procedurebox}[STEP 1: INITIAL POWER-UP]
\textbf{Action}:
\begin{itemize}
    \item Ensure \button{Station Enable} switch is in OFF position
    \item Apply vehicle power to RCWS (circuit breaker ON or power cable connected)
\end{itemize}

\textbf{Expected Result}:
\begin{itemize}
    \item \textbf{Power} indicator light illuminates (Green)
    \item DCU screen displays boot logo: "El 7arress RCWS"
    \item System begins self-test (approximately 30 seconds)
\end{itemize}

\textbf{If NO Power Light}:
\begin{itemize}
    \item Check vehicle power supply (voltage 20-30V DC)
    \item Check circuit breaker
    \item Check cable connections
    \item Report to maintenance if power available but no light
\end{itemize}
\end{procedurebox}

\begin{procedurebox}[STEP 2: BOOT SELF-TEST]
\textbf{Action}:
\begin{itemize}
    \item Observe DCU screen during boot
    \item Wait for self-test to complete (DO NOT interrupt)
\end{itemize}

\textbf{Expected Display Sequence}:
\begin{lstlisting}
El 7arress RCWS - System Initialization

Checking Devices...
[ OK ] Day Camera (Sony)
[ OK ] Thermal Camera (FLIR TAU 2)
[ OK ] Laser Rangefinder
[ OK ] IMU/AHRS (Attitude System)
[ OK ] Azimuth Servo
[ OK ] Elevation Servo
[ OK ] Servo Actuator
[ OK ] Joystick Controller
[ OK ] PLC21 (Control Panel)
[ OK ] PLC42 (Gimbal Station)

System Ready
Press STATION ENABLE to continue
\end{lstlisting}

\textbf{If Any Device Shows [FAIL]}:
\begin{itemize}
    \item DO NOT PROCEED with startup
    \item Note which device failed
    \item Report to maintenance immediately
    \item System may operate in degraded mode but requires supervisor approval
\end{itemize}
\end{procedurebox}

\begin{procedurebox}[STEP 3: ENABLE STATION]
\textbf{Action}:
\begin{itemize}
    \item Move \button{Station Enable} switch from OFF to ON
\end{itemize}

\textbf{Expected Result}:
\begin{itemize}
    \item \textbf{System Ready} light illuminates (Green)
    \item Gimbal motors energize (you may hear a soft hum)
    \item Video feed appears on DCU screen
    \item OSD overlay displays system information
    \item OSD displays: \osd{SYSTEM INIT...}
    \item Gimbal automatically initiates homing sequence
\end{itemize}

\begin{cautionbox}
Gimbal will move during this step. Ensure area is clear.
\end{cautionbox}
\end{procedurebox}

\begin{procedurebox}[STEP 4: AUTOMATIC HOMING SEQUENCE]
\textbf{System automatically executes homing to calibrate gimbal position.}

\textbf{Homing Sequence (50ms control loop cycles)}:

\begin{enumerate}
    \item \textbf{Cycle 0 - Initialization}:
    \begin{itemize}
        \item System enters \texttt{Requested} state
        \item OSD displays: \osd{homing init} (orange text)
    \end{itemize}

    \item \textbf{Cycle 1 - Command Sent} (50ms later):
    \begin{itemize}
        \item GimbalController sends HOME command to PLC42
        \item PLC42 sends HOME signal to Oriental Motor servo drives
        \item Servo drives begin mechanical homing sequence
    \end{itemize}

    \item \textbf{Cycle 2 - In Progress} (100ms total):
    \begin{itemize}
        \item State transitions to \texttt{InProgress}
        \item OSD displays: \osd{HOMING...} (yellow text)
        \item 30-second timeout timer starts
        \item Gimbal begins moving to mechanical home positions
    \end{itemize}

    \item \textbf{Waiting for HOME-END Signals}:
    \begin{itemize}
        \item Azimuth servo completes homing → sends HOME-END signal (DI6)
        \item Elevation servo completes homing → sends HOME-END signal (DI7)
        \item Both signals must be received for completion
    \end{itemize}

    \item \textbf{Completion}:
    \begin{itemize}
        \item Both HOME-END signals received
        \item State = \texttt{Completed}
        \item OSD displays: \osd{HOME COMPLETE} (green text, 2 seconds)
        \item Gimbal positions now calibrated to mechanical home
        \item System transitions to operational mode
    \end{itemize}
\end{enumerate}

\textbf{Expected Result}:
\begin{itemize}
    \item OSD displays: \osd{AZ: 000° EL: 00°} (±2° tolerance)
    \item Reticle centered on screen
    \item Gimbal points directly forward relative to vehicle
    \item \textbf{Normal completion time}: 5-15 seconds
\end{itemize}

\textbf{Manual Homing (if needed)}:
\begin{itemize}
    \item If gimbal position drifts or after maintenance, press \button{HOME} button on DCU
    \item Same homing sequence executes as described above
\end{itemize}

\begin{warningbox}[HOMING TIMEOUT]
If homing does not complete within \textbf{30 seconds}, system enters FAULT state:
\begin{itemize}
    \item OSD displays: \osd{HOMING TIMEOUT - FAULT} (red flashing)
    \item Gimbal stops movement
    \item System safes all operations
\end{itemize}

\textbf{Possible Causes}:
\begin{itemize}
    \item Mechanical obstruction preventing movement
    \item Servo drive failure or power loss
    \item HOME sensor failure on Oriental Motor drive
    \item Cable disconnection (DI6/DI7 signals)
    \item Excessive mechanical friction
\end{itemize}

\textbf{Recovery Action}:
\begin{enumerate}
    \item Press \button{EMERGENCY STOP}
    \item Inspect gimbal for obstructions (Station Enable OFF)
    \item Clear any obstructions
    \item Verify cable connections to servo drives
    \item Reset system: Station Enable OFF → ON
    \item System will retry homing
    \item If fault persists after 2 attempts: Report to maintenance
\end{enumerate}
\end{warningbox}

\begin{notebox}[TECHNICAL NOTE]
The Oriental Motor servo drives use optical HOME sensors. When HOME command is sent, each servo rotates until it detects the HOME sensor mark, then sends a HOME-END signal back to PLC42. This ensures absolute position calibration on every startup.
\end{notebox}
\end{procedurebox}

\begin{procedurebox}[STEP 5: SELECT CAMERA]
\textbf{Action}:
\begin{itemize}
    \item System defaults to DAY camera on startup
    \item To switch: Press joystick button (camera toggle - see Section 2.4)
\end{itemize}

\textbf{Expected Result}:
\begin{itemize}
    \item OSD displays camera type: \osd{DAY} or \osd{THERMAL}
    \item Video image switches between color (day) and grayscale/colorized (thermal)
    \item FOV (Field of View) value updates on OSD based on camera type
\end{itemize}
\end{procedurebox}

\begin{procedurebox}[STEP 6: TEST GIMBAL MOVEMENT]
\textbf{Action}:
\begin{itemize}
    \item Gently move joystick in all directions (left/right/up/down)
    \item Verify gimbal responds smoothly
    \item Return joystick to center (gimbal stops)
\end{itemize}

\textbf{Expected Result}:
\begin{itemize}
    \item Gimbal slews in direction of joystick movement
    \item OSD azimuth/elevation values update in real-time
    \item No grinding, binding, or unusual noises
    \item Gimbal stops when joystick returns to center
\end{itemize}
\end{procedurebox}

\begin{procedurebox}[STEP 7: TEST CAMERA ZOOM]
\textbf{Action}:
\begin{itemize}
    \item Press \button{Button 6} (Zoom +) to zoom in
    \item Press \button{Button 8} (Zoom -) to zoom out
    \item Observe video image magnification change
    \item Observe FOV value on OSD
\end{itemize}

\textbf{Expected Result}:
\begin{itemize}
    \item Image magnifies when zooming in (FOV decreases)
    \item Image wide-angle when zooming out (FOV increases)
    \item Zoom is smooth with no jerking
    \item OSD FOV updates continuously
\end{itemize}
\end{procedurebox}

\begin{procedurebox}[STEP 8: TEST LASER RANGEFINDER (LRF)]
\textbf{Action}:
\begin{itemize}
    \item Aim reticle at a known object 100m+ away
    \item Press \button{Button 1} (single-click for single measurement)
\end{itemize}

\textbf{Expected Result}:
\begin{itemize}
    \item Laser fires (you will NOT see visible beam - infrared)
    \item OSD displays range reading: \osd{RNG: XXXm}
    \item Range updates within 1 second
    \item LRF automatically times out after 5 seconds
\end{itemize}

\begin{warningbox}
Do not aim LRF at people, animals, or reflective surfaces at close range. Class 3B laser - eye damage can occur.
\end{warningbox}
\end{procedurebox}

\begin{procedurebox}[STEP 9: ENABLE STABILIZATION]
\textbf{Action}:
\begin{itemize}
    \item Set \button{Stabilization} switch to ON
\end{itemize}

\textbf{Expected Result}:
\begin{itemize}
    \item OSD displays \osd{STAB: ON}
    \item AHRS (Attitude and Heading Reference System) activates
    \item Gimbal compensates for vehicle movement (roll, pitch, yaw)
    \item Reticle remains steady on target even if vehicle rocks
\end{itemize}

\begin{notebox}[AHRS STABILIZATION]
The system uses IMU (Inertial Measurement Unit) data to stabilize the gimbal. When the vehicle is \textbf{stationary} (detected automatically), stabilization accuracy improves significantly. The system displays \osd{VEHICLE STATIONARY} on OSD when enhanced stabilization is active.
\end{notebox}
\end{procedurebox}

\begin{procedurebox}[STEP 10: SYSTEM READY - FINAL CHECK]
\textbf{Verify all indicator lights}:
\begin{itemize}
    \item \textbf{Power}: Green (ON)
    \item \textbf{System Ready}: Green (ON)
    \item \textbf{Gun Armed}: OFF (system is SAFE)
    \item \textbf{Fault/Alarm}: OFF (no errors)
\end{itemize}

\textbf{Verify OSD displays}:
\begin{itemize}
    \item Live video feed (day or thermal)
    \item Azimuth and elevation values (near 000°/00°)
    \item Current FOV
    \item System mode: \osd{MODE: Manual}
    \item Ammunition status: \osd{IDLE} (gray - no belt loaded)
    \item No warning messages
\end{itemize}

\textbf{If All Checks Pass}: System is ready for operation
\end{procedurebox}

\subsection{Startup Troubleshooting}

\small
\begin{longtable}{|L{0.25\textwidth}|L{0.30\textwidth}|L{0.38\textwidth}|}
\hline
\rowcolor{milblue!20}
\textbf{Problem} & \textbf{Possible Cause} & \textbf{Action} \\
\hline
No power light & No vehicle power & Check circuit breaker, voltage \\
\hline
Self-test fails & Device malfunction & Note failed device, report to maintenance \\
\hline
No video feed & Camera error & Check camera connections, restart system \\
\hline
Gimbal won't move & Motors disabled or fault & Check Station Enable, check for faults \\
\hline
Homing timeout & Obstruction, sensor fail & Clear obstruction, check HOME sensors (DI6/DI7) \\
\hline
Erratic gimbal & Joystick calibration & Recalibrate joystick (maintenance task) \\
\hline
LRF no reading & Out of range or bad target & Aim at closer/better reflective target \\
\hline
Thermal frozen & FFC in progress & Wait 5 seconds for FFC to complete \\
\hline
\end{longtable}

% ============================================================================
\section{Ammunition Feed Procedure}
% ============================================================================

\subsection{Ammunition Feed System Overview}

\textbf{System Type}: Servo-actuator based belt feed with encoder position detection

\textbf{Detection Method}:
\begin{itemize}
    \item \textbf{NO physical belt sensor exists}
    \item System confirms ammunition loading via \textbf{servo encoder position feedback}
    \item Servo actuator extends to engage belt, then retracts to home position
    \item Encoder verifies servo reached target positions
\end{itemize}

\subsection{OSD Ammunition Status Indicators}

The OSD displays current ammunition feed status using color-coded indicators:

\small
\begin{longtable}{|L{0.18\textwidth}|C{0.15\textwidth}|L{0.60\textwidth}|}
\hline
\rowcolor{milblue!20}
\textbf{Status} & \textbf{Color} & \textbf{Meaning} \\
\hline
\osd{IDLE} & Gray & No ammunition loaded, feed cycle not run yet \\
\hline
\osd{FEED...} & Orange & Feed cycle in progress (extending/retracting) \\
\hline
\osd{BELT} & Green & Belt successfully loaded (encoder confirms seated position) \\
\hline
\osd{FAULT} & Red (flash) & Timeout occurred, operator reset required \\
\hline
\end{longtable}

\subsection{Ammunition Loading Procedure}

\begin{procedurebox}[LOADING AMMUNITION BELT]
\textbf{Prerequisites}:
\begin{checklist}
    \item System powered and initialized
    \item Station Enable: ON
    \item Weapon system: SAFE
    \item Operator authorized
\end{checklist}

\textbf{Procedure}:
\begin{enumerate}
    \item Manually feed ammunition belt into weapon receiver (per weapon manual)
    \item Press \button{AMMO LOAD} button on DCU panel
    \item System begins automatic feed cycle:
    \begin{itemize}
        \item OSD displays: \osd{FEED...} (orange)
        \item Servo actuator extends to \textbf{63,000 encoder counts} (belt engagement)
        \item Encoder confirms position reached
        \item Servo actuator retracts to \textbf{2,048 encoder counts} (home position)
        \item Encoder confirms return to home
    \end{itemize}
    \item When cycle completes successfully:
    \begin{itemize}
        \item OSD displays: \osd{BELT} (green)
        \item \textbf{Ammo Loaded} indicator light illuminates (yellow)
        \item System ready for firing (when armed)
    \end{itemize}
    \item \textbf{Total cycle time}: 2-5 seconds (normal operation)
\end{enumerate}

\textbf{Expected Result}:
\begin{itemize}
    \item Belt firmly seated in weapon receiver
    \item Servo actuator at home position
    \item OSD shows \osd{BELT} status
    \item Ready for engagement
\end{itemize}
\end{procedurebox}

\begin{warningbox}[CYCLE IN PROGRESS]
\textbf{Do NOT interrupt the ammunition feed cycle}. Once started:
\begin{itemize}
    \item Cycle runs to completion automatically (MIL-STD compliant)
    \item Button release is IGNORED during cycle
    \item Servo must complete extend → retract sequence
    \item Interruption can cause jam or fault condition
\end{itemize}
\end{warningbox}

\subsection{Fault Recovery}

\textbf{If FAULT Occurs}:

\begin{procedurebox}[AMMUNITION FEED FAULT RECOVERY]
\textbf{Fault Indication}:
\begin{itemize}
    \item OSD displays: \osd{FAULT} (red flashing)
    \item Servo did not reach expected position within 6 seconds (watchdog timeout)
    \item System safes weapon automatically
\end{itemize}

\textbf{Possible Causes}:
\begin{itemize}
    \item Belt jam or misalignment
    \item Servo actuator mechanical obstruction
    \item Servo drive fault or power loss
    \item Encoder feedback error
\end{itemize}

\textbf{Recovery Procedure}:
\begin{enumerate}
    \item Press \button{RESET FEED FAULT} button on DCU
    \item System attempts safe retraction to home position
    \item If retraction successful:
    \begin{itemize}
        \item OSD returns to \osd{IDLE} (gray)
        \item Inspect weapon receiver for jams
        \item Clear any obstructions
        \item Retry ammunition loading procedure
    \end{itemize}
    \item If retraction fails or fault persists:
    \begin{itemize}
        \item Press \button{EMERGENCY STOP}
        \item Set \button{Station Enable} to OFF
        \item Manually inspect servo actuator and weapon receiver
        \item Report to maintenance
    \end{itemize}
\end{enumerate}
\end{procedurebox}

\begin{notebox}[TECHNICAL NOTE - ENCODER DETECTION]
The ammunition feed system uses \textbf{servo encoder position feedback} to confirm belt loading:
\begin{itemize}
    \item \textbf{Extended position (63,000 counts)}: Belt engagement verified
    \item \textbf{Home position (2,048 counts)}: Ready state confirmed
    \item \textbf{Position tolerance}: ±200 counts
    \item \textbf{No belt sensor}: System does NOT use a physical "belt seated" sensor
\end{itemize}

This method provides reliable detection without additional sensors that can fail or give false positives.
\end{notebox}

% ============================================================================
\section{Display and Control Unit (DCU) Operations}
% ============================================================================

\subsection{DCU Button and Switch Functions}

\subsubsection{Emergency Stop Button (RED)}

\textbf{Location}: Top left of DCU panel, large RED button

\textbf{Function}: Immediate system shutdown for safety emergencies

\textbf{Operation}:
\begin{enumerate}
    \item Press button (no confirmation required)
    \item System immediately:
    \begin{itemize}
        \item Stops all gimbal movement
        \item Safes weapon (trigger disabled)
        \item Locks servos in place
        \item Aborts any active operations (homing, tracking, ammunition feed)
        \item Displays "EMERGENCY STOP ACTIVE" on OSD
    \end{itemize}
\end{enumerate}

\textbf{To Reset}:
\begin{enumerate}
    \item Twist/pull button to release (mechanical latch)
    \item Verify emergency condition is resolved
    \item Press \button{Station Enable} OFF then ON to restart
    \item System will re-execute homing sequence
\end{enumerate}

\begin{warningbox}[CRITICAL]
Do NOT hesitate to use Emergency Stop. Better safe than sorry. Emergency Stop aborts ALL operations including homing and ammunition feed cycles.
\end{warningbox}

\subsubsection{Station Enable Switch}

\textbf{Positions}: OFF / ON

\textbf{OFF Position}:
\begin{itemize}
    \item Gimbal motors disabled (turret cannot move)
    \item Video still displays (cameras remain powered)
    \item Weapon is safed
    \item Safe to approach turret for inspection
\end{itemize}

\textbf{ON Position}:
\begin{itemize}
    \item Gimbal motors enabled
    \item All subsystems operational
    \item Turret can move if joystick input received
    \item Stay clear of turret
\end{itemize}

\subsubsection{Gun Arm/Safe Switch}

\textbf{Positions}: SAFE / ARM

\textbf{SAFE Position} (Default):
\begin{itemize}
    \item Weapon trigger disabled
    \item Gun Armed light is OFF
    \item Trigger pull has no effect
    \item Safe for non-combat operations
\end{itemize}

\textbf{ARM Position}:
\begin{itemize}
    \item Weapon trigger enabled
    \item Gun Armed light illuminates RED
    \item Trigger pull will fire weapon (if other conditions met)
    \item Only use during combat or live fire training
\end{itemize}

\begin{warningbox}
When Gun Armed light is RED, treat weapon as HOT. One trigger pull away from firing.
\end{warningbox}

\subsubsection{Fire Mode Selector}

\textbf{Positions}: SINGLE / SHORT BURST / LONG BURST

\begin{twocol}
\textbf{SINGLE}:
\begin{itemize}
    \item One round per trigger pull
    \item Most accurate mode
    \item Use for precision engagement
\end{itemize}

\textbf{SHORT BURST}:
\begin{itemize}
    \item 3-5 rounds per trigger pull
    \item Good balance of accuracy and firepower
    \item Use for moving targets
\end{itemize}

\textbf{LONG BURST}:
\begin{itemize}
    \item Continuous fire while trigger held
    \item Less accurate due to recoil
    \item Use for area suppression
\end{itemize}
\end{twocol}

\subsubsection{Speed Select Switch}

\textbf{Positions}: LOW / MEDIUM / HIGH

\small
\begin{longtable}{|L{0.20\textwidth}|L{0.35\textwidth}|L{0.38\textwidth}|}
\hline
\rowcolor{milblue!20}
\textbf{Speed} & \textbf{Use Case} & \textbf{Max Slew Rate} \\
\hline
LOW & Zeroing, fine adjustments, precision & ~5\degree/second \\
\hline
MEDIUM & Normal surveillance, target acquisition & ~20\degree/second \\
\hline
HIGH & Close-range threats, emergencies & ~60\degree/second \\
\hline
\end{longtable}

\subsection{DCU Indicator Lights}

\begin{twocol}
\begin{itemize}
    \item \textbf{Power} (Green): Vehicle power supplied
    \item \textbf{System Ready} (Green): All subsystems operational
    \item \textbf{Gun Armed} (Red): Weapon is armed
    \item \textbf{Ammo Loaded} (Yellow): Belt loaded (encoder confirmed)
    \item \textbf{Authorized} (Green): Operator authorized
    \item \textbf{Fault/Alarm} (Red): System error detected
\end{itemize}
\end{twocol}

% ============================================================================
\section{Joystick Controller Operations}
% ============================================================================

\subsection{Joystick Control Techniques}

\subsubsection{Proper Grip}

\textbf{Right Hand Position}:
\begin{enumerate}
    \item Wrap fingers around joystick grip
    \item Index finger rests on \button{Button 5} trigger (outside trigger guard when not firing)
    \item Thumb on top, near top buttons
    \item \button{Button 3} Dead Man Switch on rear of grip - squeeze with palm/fingers to engage
\end{enumerate}

\subsubsection{Gimbal Slew Technique}

\textbf{Small Movements} (Precision):
\begin{itemize}
    \item Deflect stick slightly from center (10-20\%)
    \item Gimbal moves slowly
    \item Good for: Tracking, zeroing, fine adjustments
\end{itemize}

\textbf{Large Movements} (Rapid Slew):
\begin{itemize}
    \item Deflect stick fully (80-100\%)
    \item Gimbal moves at maximum speed (depends on Speed Select switch)
    \item Good for: Searching, responding to threats, sector scans
\end{itemize}

\subsection{Complete Joystick Button Reference (19 Buttons)}

The joystick has **19 programmable buttons** (Button 0 through Button 18). This section provides the complete button mapping.

\subsubsection{Primary Combat Controls}

\small
\begin{longtable}{|C{0.12\textwidth}|L{0.28\textwidth}|L{0.53\textwidth}|}
\hline
\rowcolor{milblue!20}
\textbf{Button} & \textbf{Function} & \textbf{Operation} \\
\hline
\textbf{Button 0} & Engagement Mode & START/STOP engagement sequence (lesson 5) \\
\hline
\textbf{Button 1} & Laser Rangefinder (LRF) &
\textbf{Single-click}: Single measurement \\
\textbf{Double-click} (<500ms): Continuous ranging mode \\
\hline
\textbf{Button 2} & Lead Angle Compensation (LAC) &
Toggle LAC ON/OFF for moving targets (Lesson 6) \\
OSD displays LAC status \\
\hline
\textbf{Button 3} & Dead Man Switch &
\textbf{MUST HOLD} for weapon operation \\
Safety interlock - release to safe weapon \\
\hline
\textbf{Button 4} & Tracking Control &
\textbf{Single-click}: Enter acquisition / size gate \\
\textbf{Double-click} (<500ms): Start/stop tracking \\
\hline
\textbf{Button 5} & Fire Weapon &
Weapon trigger (when armed) \\
Respects fire mode selector setting \\
\hline
\end{longtable}

\begin{warningbox}[MEMORIZE THESE BUTTONS]
Buttons 0-5 are CRITICAL for combat operations. Operators must be able to activate these buttons instinctively without looking at the joystick.
\end{warningbox}

\subsubsection{Camera \& Display Controls}

\small
\begin{longtable}{|C{0.12\textwidth}|L{0.28\textwidth}|L{0.53\textwidth}|}
\hline
\rowcolor{milblue!20}
\textbf{Button} & \textbf{Function} & \textbf{Operation} \\
\hline
\textbf{Button 6} & Zoom In &
Increase camera magnification \\
FOV decreases, zoom factor increases \\
\hline
\textbf{Button 7} & Thermal LUT + &
Next video Look-Up Table (thermal camera only) \\
Cycles through: White Hot, Black Hot, Ironbow, etc. \\
\hline
\textbf{Button 8} & Zoom Out &
Decrease camera magnification \\
FOV increases, zoom factor decreases \\
\hline
\textbf{Button 9} & Thermal LUT - &
Previous video Look-Up Table (thermal camera only) \\
Reverse cycle through thermal palettes \\
\hline
\textbf{Button 10} & LRF Clear &
Clear displayed range reading from OSD \\
Stops continuous ranging if active \\
\hline
\end{longtable}

\subsubsection{Motion Mode \& Surveillance Controls}

\small
\begin{longtable}{|C{0.12\textwidth}|L{0.28\textwidth}|L{0.53\textwidth}|}
\hline
\rowcolor{milblue!20}
\textbf{Button} & \textbf{Function} & \textbf{Operation} \\
\hline
\textbf{Button 11} & Mode Cycle &
Cycle motion modes: Manual → AutoSectorScan → TRP Scan → Manual \\
OSD displays current mode \\
\hline
\textbf{Button 12} & \textit{Available} &
Reserved for future use \\
Currently unassigned \\
\hline
\textbf{Button 13} & Mode Cycle &
Duplicate of Button 11 (ergonomic placement) \\
Same function, different location on joystick \\
\hline
\textbf{Button 14} & Select Next TRP/Scan &
During TRP Scan or AutoSectorScan modes: \\
Advance to next TRP page or scan zone \\
\hline
\textbf{Button 15} & \textit{Reserved} &
Reserved for future use \\
\hline
\textbf{Button 16} & Select Previous TRP/Scan &
During TRP Scan or AutoSectorScan modes: \\
Return to previous TRP page or scan zone \\
\hline
\textbf{Button 17} & \textit{Reserved} &
Reserved for future use \\
\hline
\textbf{Button 18} & \textit{Reserved} &
Reserved for future use \\
\hline
\end{longtable}

\subsection{Detailed Button Operations}

\subsubsection{Button 1: Laser Rangefinder (LRF)}

\textbf{Location}: Front grip, easily accessible with index/middle finger

\textbf{Two Operating Modes}:

\textbf{1. Single Measurement Mode (Single-Click)}:
\begin{procedurebox}[SINGLE LRF MEASUREMENT]
\begin{enumerate}
    \item Aim reticle at target
    \item \textbf{Single-click} Button 1
    \item Laser fires (invisible infrared beam)
    \item OSD displays range: \osd{RNG: XXXm}
    \item Range value remains displayed for reference
    \item LRF automatically times out after 5 seconds
\end{enumerate}
\end{procedurebox}

\textbf{2. Continuous Ranging Mode (Double-Click)}:
\begin{procedurebox}[CONTINUOUS LRF MODE]
\begin{enumerate}
    \item Aim at target area or moving target
    \item \textbf{Double-click} Button 1 (within 500ms)
    \item LRF enters continuous mode
    \item Laser fires repeatedly (2-5 Hz update rate)
    \item OSD displays: \osd{RNG: XXXm (CONT)} - note "CONT" indicator
    \item Range updates continuously as you track target
    \item \textbf{To exit}: Single-click Button 1 OR press Button 10 (LRF Clear)
\end{enumerate}
\end{procedurebox}

\textbf{When to Use Each Mode}:
\begin{itemize}
    \item \textbf{Single Mode}: Static targets, pre-engagement checks, verification shots
    \item \textbf{Continuous Mode}: Moving targets, scanning area, tracking, ballistics updates
\end{itemize}

\textbf{Button 10 - LRF Clear}:
\begin{itemize}
    \item Press Button 10 to clear range display from OSD
    \item Stops continuous ranging if active
    \item Range indicator (\osd{RNG: XXXm}) disappears
    \item Use when range data is no longer relevant
\end{itemize}

\textbf{Range Performance}:
\begin{itemize}
    \item \textbf{Minimum range}: 50 meters
    \item \textbf{Maximum range}: 4000 meters
    \item \textbf{Accuracy}: ±5 meters
    \item \textbf{Update rate (continuous)}: 2-5 Hz
\end{itemize}

\begin{warningbox}[LASER SAFETY]
Class 3B laser device. Do not aim at people, animals, or reflective surfaces at close range. Eye damage can occur. Laser beam is invisible (infrared 1550nm wavelength).
\end{warningbox}

\subsubsection{Button 2: Lead Angle Compensation (LAC)}

\textbf{Location}: Front grip, near Button 1

\textbf{Function}: Toggle Lead Angle Compensation ON/OFF for engaging moving targets

\textbf{Operation}:
\begin{procedurebox}[LAC ACTIVATION/DEACTIVATION]
\begin{enumerate}
    \item Press \button{Button 2} (single-click)
    \item LAC toggles between OFF and ON
    \item OSD displays current LAC status (see OSD section below)
    \item System calculates motion lead when ON and tracking active
\end{enumerate}
\end{procedurebox}

\textbf{LAC Status Display on OSD}:

The OSD displays LAC status in the bottom-right area:

\begin{itemize}
    \item \osd{LAC: OFF} (gray) - Lead angle compensation disabled
    \item \osd{LAC: ON} (green) - Lead angle active and functioning normally
    \item \osd{LAC: LAG} (yellow) - Lead angle at maximum limit
    \item \osd{LAC: ZOOM OUT} (red) - Lead angle too large for current FOV, zoom out required
\end{itemize}

\textbf{When to Use LAC}:
\begin{itemize}
    \item \textbf{Enable} when engaging moving targets (vehicles, personnel)
    \item \textbf{Disable} for static targets (to avoid unnecessary corrections)
    \item System automatically calculates lead based on:
    \begin{itemize}
        \item Target velocity (from tracking system)
        \item Current range (from LRF)
        \item Muzzle velocity (configured)
    \end{itemize}
\end{itemize}

\begin{notebox}[IMPORTANT - JOYSTICK ONLY]
LAC can \textbf{ONLY} be toggled via \button{Button 2} on the joystick. There is NO menu option for LAC activation. This allows rapid LAC toggling during combat without accessing menus.
\end{notebox}

\textbf{Detailed LAC operation covered in Lesson 6 (Ballistics \& Fire Control).}

\subsubsection{Button 3: Dead Man Switch}

\textbf{Location}: Rear of joystick grip (palm/finger squeeze activation)

\textbf{Purpose}:
\begin{itemize}
    \item Critical safety interlock
    \item Prevents accidental discharge if operator is incapacitated
    \item Automatic safety if operator loses grip on joystick
    \item Required safety condition for weapon firing
\end{itemize}

\textbf{Operation}:
\begin{itemize}
    \item Squeeze grip to engage Dead Man Switch
    \item Switch must be held continuously when weapon is armed
    \item Release switch → weapon immediately safes (trigger disabled)
    \item Spring-loaded return ensures automatic release
\end{itemize}

\begin{warningbox}[CRITICAL SAFETY RULE]
\textbf{Release Dead Man Switch immediately} when:
\begin{itemize}
    \item Not actively engaging a target
    \item Threat eliminated or abort engagement
    \item Any unsafe condition develops
    \item Friendly forces enter area
    \item Communication lost with command
\end{itemize}
Dead Man Switch release is your \textbf{primary immediate weapon safe action}.
\end{warningbox}

\subsubsection{Button 4: Tracking Control}

\textbf{Two Functions (Single vs. Double-Click)}:

\textbf{Single-Click}: Enter acquisition mode / size gate
\begin{itemize}
    \item Enters tracking acquisition mode (yellow gate appears)
    \item During acquisition: Use D-Pad to size gate around target
\end{itemize}

\textbf{Double-Click (<500ms)}: Start tracking / Emergency abort
\begin{itemize}
    \item In acquisition: Initiates tracker lock-on attempt
    \item While tracking: \textbf{Emergency abort} (stops tracking immediately)
\end{itemize}

\textbf{Detailed tracking procedures covered in Lesson 5.}

\subsubsection{Buttons 6/8: Zoom Control}

\textbf{Button 6}: Zoom In (increase magnification)
\begin{itemize}
    \item FOV decreases (narrower field of view)
    \item Image magnification increases
    \item OSD displays updated zoom factor (e.g., \osd{ZOOM: 15×})
\end{itemize}

\textbf{Button 8}: Zoom Out (decrease magnification)
\begin{itemize}
    \item FOV increases (wider field of view)
    \item Image magnification decreases
    \item OSD displays updated zoom factor
\end{itemize}

\textbf{Zoom Ranges}:
\begin{itemize}
    \item \textbf{Day camera}: 2° to 60° FOV (continuous optical zoom)
    \item \textbf{Thermal camera}: Wide (10.4° HFOV) or Narrow (5.2° HFOV) + digital zoom
\end{itemize}

\subsubsection{Buttons 7/9: Thermal Video LUT}

\textbf{LUT} = Look-Up Table (color palette for thermal imaging)

\textbf{Button 7}: Next LUT (cycle forward through palettes)
\textbf{Button 9}: Previous LUT (cycle backward)

\textbf{Available Thermal LUTs}:
\begin{enumerate}
    \item \textbf{White Hot} - Hot objects appear white (default)
    \item \textbf{Black Hot} - Hot objects appear black
    \item \textbf{Ironbow} - Rainbow color mapping (red=hot, blue=cold)
    \item \textbf{Lava} - Red/orange gradient
    \item \textbf{Rainbow} - Full spectrum color mapping
\end{enumerate}

\textbf{When to Change LUT}:
\begin{itemize}
    \item Improve target contrast in different environments
    \item Operator preference
    \item Specific mission requirements (e.g., White Hot for urban, Ironbow for vehicle detection)
\end{itemize}

\begin{notebox}
LUT buttons only affect thermal camera. No effect when day camera is active. Current LUT is saved and restored when switching cameras.
\end{notebox}

\subsubsection{Buttons 11/13: Motion Mode Cycle}

\textbf{Function}: Cycle through available motion modes

\textbf{Mode Sequence}:
\begin{lstlisting}
Manual → AutoSectorScan → TRP Scan → Manual (repeats)
\end{lstlisting}

\textbf{Current mode displayed on OSD}: \osd{MODE: Manual}, \osd{MODE: Sector Scan}, \osd{MODE: TRP}

\textbf{Why Two Buttons (11 and 13)?}:
\begin{itemize}
    \item Ergonomic placement on different parts of joystick
    \item Allows mode change regardless of hand position
    \item Redundancy for critical function
\end{itemize}

\textbf{Restriction}: Cannot change modes during tracking acquisition. Must abort tracking first (Button 4 double-click).

\textbf{Detailed motion mode operations covered in Lesson 4.}

\subsubsection{Buttons 14/16: TRP/Scan Zone Selection}

\textbf{Active Modes}: AutoSectorScan, TRP Scan

\textbf{Button 14}: Select Next TRP page or scan zone
\textbf{Button 16}: Select Previous TRP page or scan zone

\textbf{During AutoSectorScan Mode}:
\begin{itemize}
    \item If multiple sector scan zones are defined
    \item Press Button 14 → System slews to next sector zone
    \item Press Button 16 → System slews to previous sector zone
    \item OSD displays current zone name (e.g., \osd{SECTOR 2: East Perimeter})
\end{itemize}

\textbf{During TRP Scan Mode}:
\begin{itemize}
    \item If multiple TRP pages are defined
    \item Press Button 14 → Load next TRP page
    \item Press Button 16 → Load previous TRP page
    \item OSD displays current page (e.g., \osd{TRP PAGE 2})
    \item System begins visiting TRPs from new page
\end{itemize}

\textbf{Benefits}:
\begin{itemize}
    \item Rapid zone/page switching without menu access
    \item Hands stay on joystick, eyes on video
    \item Faster reaction to threats in different sectors
\end{itemize}

\textbf{If only one zone/page defined}: Buttons have no effect, OSD shows current zone name.

% ============================================================================
\section{On-Screen Display (OSD) Interpretation}
% ============================================================================

\subsection{Complete OSD Layout}

\begin{lstlisting}
┌──────────────────────────────────────────────────────────┐
│ AZ: 045°  EL: +12°  | DAY  FOV: 9.0° ZOOM: 15×      │ ← Top
│                                                           │
│                    [Tracking Box]                        │
│                        ┌──────┐                          │
│                        │      │                          │
│                        └──────┘                          │
│                          +      ← Reticle                │
│                         (*)     ← CCIP Pipper            │
│                                                           │
│ RNG: 850m         MODE: Manual       BELT               │ ← Bottom
│ LAC: ON           TRACK: Off         STAB: ON           │
└──────────────────────────────────────────────────────────┘
\end{lstlisting}

\subsection{Top Bar Elements}

\begin{itemize}
    \item \textbf{AZ: XXX\degree}: Current azimuth position (000\degree\ to 359\degree, relative to vehicle forward)
    \item \textbf{EL: ±XX\degree}: Current elevation position (-20\degree\ to +60\degree, relative to horizon)
    \item \textbf{DAY / THERMAL}: Active camera type
    \item \textbf{FOV: X.X\degree}: Current horizontal field of view in degrees
    \item \textbf{ZOOM: XX×}: Zoom magnification factor (calculated from FOV)
\end{itemize}

\subsection{Center Area Elements}

\textbf{Reticle (+)}:
\begin{itemize}
    \item Main aiming point with zeroing applied
    \item Where the gun bore axis is aimed
    \item \textbf{Zeroing} correction shifts reticle from optical center
    \item Fire weapon with reticle on target (static targets)
\end{itemize}

\textbf{CCIP Pipper ((*))}:
\begin{itemize}
    \item \textbf{Continuously Computed Impact Point}
    \item Shows where bullet will actually hit with \textbf{ALL ballistic corrections}:
    \begin{itemize}
        \item Zeroing offset
        \item Ballistic drop (gravity + wind deflection)
        \item Motion lead (if LAC active and tracking)
    \end{itemize}
    \item May be offset from reticle if ballistics are applied
    \item \textbf{Always aim with CCIP pipper for accurate hits} (moving or long-range targets)
\end{itemize}

\textbf{Tracking Box}:
\begin{itemize}
    \item Appears during tracking acquisition and active tracking
    \item Color indicates tracking state (yellow, cyan, green, red - see Lesson 5)
    \item Size adjustable with D-Pad during acquisition
\end{itemize}

\subsection{Bottom Bar Elements}

\subsubsection{Left Side}

\begin{itemize}
    \item \textbf{RNG: XXXm}: Distance to target from laser rangefinder
    \begin{itemize}
        \item Single measurement: \osd{RNG: 850m}
        \item Continuous mode: \osd{RNG: 850m (CONT)}
        \item No reading: \osd{RNG: ---}
    \end{itemize}

    \item \textbf{LAC Status}: Lead Angle Compensation status (Button 2)
    \begin{itemize}
        \item \osd{LAC: OFF} (gray) - Disabled
        \item \osd{LAC: ON} (green) - Active and functioning
        \item \osd{LAC: LAG} (yellow) - Lead at maximum limit
        \item \osd{LAC: ZOOM OUT} (red) - Lead too large, zoom out required
    \end{itemize}
\end{itemize}

\subsubsection{Center}

\begin{itemize}
    \item \textbf{MODE}: Current motion mode
    \begin{itemize}
        \item \osd{MODE: Manual} - Direct joystick control
        \item \osd{MODE: Sector Scan} - Automatic sector scanning
        \item \osd{MODE: TRP} - Target Reference Point scanning
    \end{itemize}

    \item \textbf{TRACK}: Tracking system status
    \begin{itemize}
        \item \osd{TRACK: Off} - Not tracking
        \item \osd{TRACK: Acquisition} - Sizing gate
        \item \osd{TRACK: Lock Pending} - Attempting lock
        \item \osd{TRACK: Active} - Locked on target
        \item \osd{TRACK: Coast} - Temporary target loss
    \end{itemize}
\end{itemize}

\subsubsection{Right Side}

\begin{itemize}
    \item \textbf{Ammunition Status}: Belt loading state
    \begin{itemize}
        \item \osd{IDLE} (gray) - No belt loaded
        \item \osd{FEED...} (orange) - Feed cycle in progress
        \item \osd{BELT} (green) - Belt loaded and ready
        \item \osd{FAULT} (red flash) - Feed fault, reset required
    \end{itemize}

    \item \textbf{STAB}: Stabilization status
    \begin{itemize}
        \item \osd{STAB: OFF} - Stabilization disabled
        \item \osd{STAB: ON} - AHRS stabilization active
        \item \osd{VEHICLE STATIONARY} - Enhanced stabilization (vehicle not moving)
    \end{itemize}
\end{itemize}

\subsection{Warning Messages}

\textbf{Critical warnings appear in center screen (overlay on video)}:

\begin{itemize}
    \item \textbf{NO-FIRE ZONE WARNING} (Red, flashing) - Reticle in no-fire zone, trigger disabled
    \item \textbf{NO-TRAVERSE WARNING} (Yellow) - Gimbal approaching no-traverse zone
    \item \textbf{EMERGENCY STOP ACTIVE} (Red, solid) - System safed, all motion stopped
    \item \textbf{SYSTEM FAULT} (Red/yellow) - Check system status menu
    \item \textbf{TARGET LOST} (Yellow) - Tracking lost, coasting mode active
    \item \textbf{HOMING TIMEOUT} (Red) - Homing failed, fault condition
    \item \textbf{AMMO FEED FAULT} (Red flash) - Ammunition feed cycle timeout
\end{itemize}

% ============================================================================
\section{System Shutdown Procedure}
% ============================================================================

\subsection{Shutdown Sequence (7 Steps)}

\begin{procedurebox}[STEP 1: SAFE THE WEAPON]
\begin{itemize}
    \item Move \button{Gun Arm/Safe} switch to SAFE
    \item Verify Gun Armed light is OFF
    \item Release Dead Man Switch on joystick (Button 3)
\end{itemize}
\end{procedurebox}

\begin{procedurebox}[STEP 2: RETURN TO HOME POSITION]
\begin{itemize}
    \item Press \button{HOME} button on DCU
    \item System executes homing sequence (same as startup)
    \item Wait for gimbal to complete homing (5-15 seconds)
    \item Verify OSD displays: \osd{AZ: 000° EL: 00°}
\end{itemize}
\end{procedurebox}

\begin{procedurebox}[STEP 3: DISABLE STABILIZATION]
\begin{itemize}
    \item Move \button{Stabilization} switch to OFF
    \item OSD displays: \osd{STAB: OFF}
\end{itemize}
\end{procedurebox}

\begin{procedurebox}[STEP 4: ACCESS SHUTDOWN MENU]
\begin{itemize}
    \item Press \button{MENU ✓} button on DCU
    \item Navigate to \textbf{SYSTEM} → \textbf{Shutdown} (use ▲/▼ buttons)
    \item Select "Shutdown System"
    \item Confirmation prompt appears: "SHUTDOWN SYSTEM?"
    \item Select "YES, Shutdown"
    \item Press \button{MENU ✓} to confirm
\end{itemize}

\textbf{System Shutdown Sequence}:
\begin{itemize}
    \item OSD displays: \osd{SHUTTING DOWN...}
    \item Progress indicator shows shutdown steps
    \item Cameras power off (video feed stops)
    \item Motors de-energize
    \item Configuration saved to non-volatile memory
    \item Log files closed properly
    \item OSD displays: \osd{SHUTDOWN COMPLETE - Safe to power off}
\end{itemize}
\end{procedurebox}

\begin{procedurebox}[STEP 5: DISABLE STATION]
\begin{itemize}
    \item Move \button{Station Enable} switch to OFF
    \item System Ready light turns OFF
    \item Gimbal motors de-energize (if not already)
\end{itemize}
\end{procedurebox}

\begin{procedurebox}[STEP 6: REMOVE VEHICLE POWER]
\textbf{If end of mission/shift}:
\begin{itemize}
    \item Turn off circuit breaker, OR
    \item Disconnect power cable (if external)
    \item Power light turns OFF
    \item DCU screen goes black
\end{itemize}
\end{procedurebox}

\begin{procedurebox}[STEP 7: SECURE WEAPON AND EQUIPMENT]
\begin{itemize}
    \item Clear weapon per Appendix A (if required)
    \item Remove ammunition (if required by SOP)
    \item Install protective covers on cameras
    \item Lock operator station (if applicable)
    \item Complete operator log entry
\end{itemize}
\end{procedurebox}

\begin{warningbox}[IMPORTANT - WAIT FOR SHUTDOWN COMPLETE]
Always wait for \osd{SHUTDOWN COMPLETE} message before cutting power. Interrupting shutdown can:
\begin{itemize}
    \item Corrupt configuration files (lose zeroing, zone data)
    \item Prevent proper log file closure
    \item Require configuration re-initialization on next startup
\end{itemize}
\end{warningbox}

\subsection{Post-Shutdown Checks}

\begin{checklist}
    \item Gun Armed light is OFF
    \item Gimbal is at home position (0\degree\ AZ, 0\degree\ EL)
    \item Station Enable is OFF
    \item Weapon is cleared (if required)
    \item Camera protective covers installed (if required)
    \item Operator log entry complete with:
    \begin{itemize}
        \item Mission start/end time
        \item Rounds fired (if any)
        \item Any faults or anomalies encountered
        \item Operator name and signature
    \end{itemize}
\end{checklist}

% ============================================================================
\section{Student Review Questions}
% ============================================================================

\begin{enumerate}
    \item What is the first step in the system startup procedure?
    \item What should you do if a device shows [FAIL] during self-test?
    \item Describe the automatic homing sequence including timing and HOME-END signals.
    \item What are the 4 OSD ammunition status indicators and their meanings?
    \item How does the system confirm ammunition loading without a physical belt sensor?
    \item Name the joystick button numbers for: LRF, LAC, Dead Man Switch, Fire, Tracking
    \item How do you activate continuous LRF mode?
    \item What are the 4 LAC status display states on the OSD?
    \item What is the difference between the reticle and CCIP pipper?
    \item What joystick buttons (by number) control zoom in/out?
    \item What is the purpose of Buttons 14 and 16?
    \item What does \osd{STAB: ON} indicate and what technology does it use?
    \item What happens if homing times out after 30 seconds?
    \item Why must you wait for "SHUTDOWN COMPLETE" before cutting power?
    \item What is the recovery procedure for ammunition feed FAULT?
\end{enumerate}
