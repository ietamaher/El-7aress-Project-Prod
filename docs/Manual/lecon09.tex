% ============================================================================
% LESSON 9 - OPERATOR MAINTENANCE & TROUBLESHOOTING
% ============================================================================
\lesson{9 - OPERATOR MAINTENANCE \& TROUBLESHOOTING}

\begin{tabular}{@{}ll@{}}
\textbf{Duration:} & 3 hours (Classroom 1h + Practical 2h) \\
\textbf{Type:} & Classroom + Practical \\
\textbf{References:} & Maintenance manual, troubleshooting guide \\
\end{tabular}

\section{LEARNING OBJECTIVES}

Upon completion of this lesson, operators will be able to:
\begin{itemize}
    \item Perform daily operator checks
    \item Troubleshoot common issues
    \item Apply systematic troubleshooting methodology
    \item Determine when to escalate to maintenance
\end{itemize}

% ============================================================================
\section{DAILY OPERATOR CHECKS}

\subsection{PRE-OPERATION CHECKLIST}

\textbf{Perform before each mission / daily:}

\begin{longtable}{|L{0.15\textwidth}|L{0.25\textwidth}|L{0.25\textwidth}|L{0.25\textwidth}|}
\hline
\rowcolor{milblue!20}
\textbf{System} & \textbf{Check} & \textbf{GO Criteria} & \textbf{NO-GO Criteria} \\
\hline
\endfirsthead

\hline
\rowcolor{milblue!20}
\textbf{System} & \textbf{Check} & \textbf{GO Criteria} & \textbf{NO-GO Criteria} \\
\hline
\endhead

\textbf{Power} & Main power, Station power & All ON, no alarms & Power OFF, alarms present \\
\hline
\textbf{Control Panel} & Buttons, switches, lights responsive & All functional & Non-responsive, lights out \\
\hline
\textbf{Joystick} & Axes, buttons, Dead Man Switch & Smooth motion, all buttons work & Sticky, unresponsive \\
\hline
\textbf{Gimbal} & Azimuth/Elevation motion & Smooth, quiet, full ROM & Grinding, jerky, limits hit \\
\hline
\textbf{Cameras} & Day \& Thermal video & Clear video, both cameras & No video, errors \\
\hline
\textbf{LRF} & Ranging test (known target) & Range within ±5m & No reading, error \\
\hline
\textbf{Tracking} & Acquisition test & Gate appears, tracking functional & No gate, tracking fails \\
\hline
\textbf{Weapon} & Arming, Safe & Arm/Safe switches functional & Switches stuck, no response \\
\hline
\textbf{System Status} & All devices connected & All ✓ Connected & Devices disconnected/faulted \\
\hline
\end{longtable}

\begin{warningbox}[NO-GO CRITERIA]
If NO-GO: Do NOT proceed with mission. Notify maintenance.
\end{warningbox}

\subsection{Pre-Operation Check Procedure}

\begin{procedurebox}[DAILY PRE-OPERATION CHECK]
\textbf{1. Visual Inspection}
\begin{itemize}
    \item Inspect gimbal for physical damage
    \item Check cable connections (secure, not frayed)
    \item Verify E-Stop button not engaged
    \item Check for fluid leaks, corrosion, loose parts
\end{itemize}

\textbf{2. Power-Up Sequence}
\begin{itemize}
    \item Turn main power ON
    \item Turn Station Enable ON
    \item Observe startup sequence (all lights illuminate)
    \item Wait for system initialization (~30 seconds)
\end{itemize}

\textbf{3. System Status Check}
\begin{itemize}
    \item Access System Status display
    \item Verify all devices "Connected"
    \item Check for fault flags (should be none)
    \item Note baseline temperatures
\end{itemize}

\textbf{4. Functional Tests}
\begin{itemize}
    \item Joystick: Test all axes and buttons
    \item Gimbal: Test azimuth/elevation motion (smooth, full range)
    \item Cameras: Switch between day/thermal, verify clear video
    \item LRF: Range test on known target (verify ±5m accuracy)
    \item Tracking: Quick acquisition test (gate appears, lock achieved)
    \item Weapon: Test arm/safe switches (DO NOT load weapon)
\end{itemize}

\textbf{5. Documentation}
\begin{itemize}
    \item Record check completion in logbook
    \item Note any discrepancies or warnings
    \item Sign off as "GO" or "NO-GO"
\end{itemize}
\end{procedurebox}

% ============================================================================
\section{TROUBLESHOOTING METHODOLOGY}

\subsection{STOP-LOOK-ASSESS-FIX (SLAF) METHOD}

\subsubsection{S - STOP}
\textbf{Don't rush to "fix" without understanding}
\begin{itemize}
    \item Take E-Stop if safety concern
    \item Document symptoms precisely
    \item Don't make random changes hoping for improvement
    \item Pause and think before acting
\end{itemize}

\subsubsection{L - LOOK}
\textbf{Observe all symptoms carefully}
\begin{itemize}
    \item Check System Status display
    \item Review recent actions (what changed?)
    \item Note all error messages and alarms
    \item Observe physical indicators (lights, sounds, motion)
\end{itemize}

\subsubsection{A - ASSESS}
\textbf{Narrow down to specific subsystem}
\begin{itemize}
    \item Check simplest causes first (power, connections, switches)
    \item Consult troubleshooting charts
    \item Eliminate possibilities systematically
    \item Determine if problem is operator-level or maintenance-level
\end{itemize}

\subsubsection{F - FIX}
\textbf{Apply appropriate solution}
\begin{itemize}
    \item Apply operator-level fix if authorized
    \item Escalate to maintenance if needed
    \item Verify fix resolved issue
    \item Document actions taken
\end{itemize}

\subsection{Troubleshooting Best Practices}

\begin{itemize}
    \item \textbf{Start simple:} Check power, cables, switches before assuming complex failure
    \item \textbf{Change one thing at a time:} If you change multiple things, you won't know what fixed it
    \item \textbf{Verify the fix:} Test that the problem is actually resolved
    \item \textbf{Document everything:} What was the symptom? What did you try? What worked?
\end{itemize}

% ============================================================================
\section{COMMON ISSUES \& FIXES}

\subsection{JOYSTICK NOT RESPONDING}

\subsubsection{Symptoms}
\begin{itemize}
    \item No gimbal motion when joystick moved
    \item Buttons not working
    \item Joystick LED off
\end{itemize}

\subsubsection{Diagnosis}

\begin{procedurebox}[JOYSTICK DIAGNOSIS]
\begin{enumerate}
    \item Check USB cable (both ends firmly connected)
    \item Check joystick power LED (should be illuminated)
    \item Check System Status → Joystick Connected
    \item Try moving joystick (any response?)
\end{enumerate}
\end{procedurebox}

\subsubsection{Operator-Level Fixes}

\begin{checklist}
    \item Reconnect USB cable firmly (both ends)
    \item Power cycle system (turn Station OFF, wait 10 seconds, turn ON)
    \item Try different USB port (if available)
    \item Check for physical damage to cable or joystick
\end{checklist}

\textbf{Escalate if:} Still not detected after power cycle.

% ----------------------------------------------------------------------------
\subsection{NO VIDEO FROM CAMERA}

\subsubsection{Symptoms}
\begin{itemize}
    \item Black screen
    \item "No Signal" message
    \item Frozen image
    \item Severe pixelation or artifacts
\end{itemize}

\subsubsection{Diagnosis}

\begin{procedurebox}[CAMERA DIAGNOSIS]
\begin{enumerate}
    \item Check camera selection (Day vs. Night - Button 10/12)
    \item Check System Status → Camera Connected
    \item Verify lens cap removed (if applicable)
    \item Check for obstructions in front of lens
\end{enumerate}
\end{procedurebox}

\subsubsection{Operator-Level Fixes}

\begin{checklist}
    \item Switch cameras (\button{Button 10/12})
    \item Power cycle system
    \item Check video cable connections (if accessible)
    \item Clean lens (if dirty/foggy)
    \item Adjust focus (if autofocus failed)
\end{checklist}

\textbf{Escalate if:} Both cameras offline.

% ----------------------------------------------------------------------------
\subsection{TRACKING FAILS TO LOCK}

\subsubsection{Symptoms}
\begin{itemize}
    \item Stays in LOCK PENDING indefinitely
    \item Returns to ACQUISITION after lock attempt
    \item Gate flickers or disappears
    \item Lock achieved but immediately lost
\end{itemize}

\subsubsection{Diagnosis}

\begin{procedurebox}[TRACKING DIAGNOSIS]
\begin{enumerate}
    \item Check target contrast (clearly visible on screen?)
    \item Check acquisition gate sizing (10-30\% margin?)
    \item Check gimbal stability (holding steady during lock?)
    \item Check target motion (stationary or slow-moving?)
\end{enumerate}
\end{procedurebox}

\subsubsection{Operator-Level Fixes}

\begin{checklist}
    \item Improve target contrast (switch camera, adjust thermal LUT)
    \item Re-size acquisition gate (ensure 10-30\% margin around target)
    \item Hold gimbal steady during lock attempt (no joystick input)
    \item Select better target (higher contrast, larger, clearer edges)
    \item Switch to manual engagement (no tracking)
\end{checklist}

\textbf{Escalate if:} Tracking fails on all targets (tracker malfunction).

% ----------------------------------------------------------------------------
\subsection{GIMBAL NOT MOVING}

\subsubsection{Symptoms}
\begin{itemize}
    \item Joystick input has no effect
    \item Gimbal completely frozen
    \item Only one axis moving (azimuth OR elevation)
    \item Very slow, sluggish motion
\end{itemize}

\subsubsection{Diagnosis}

\begin{procedurebox}[GIMBAL DIAGNOSIS]
\begin{enumerate}
    \item Check E-Stop (engaged/latched?)
    \item Check Station Enable switch (ON?)
    \item Check motion mode (Manual mode selected?)
    \item Check System Status → Servos Connected
    \item Check for gimbal limit switches (at mechanical limit?)
\end{enumerate}
\end{procedurebox}

\subsubsection{Operator-Level Fixes}

\begin{checklist}
    \item Reset E-Stop (if engaged)
    \item Turn Station Enable ON
    \item Cycle to Manual mode (\button{Button 11/13})
    \item Move gimbal away from limits (if at limit switch)
    \item Power cycle system
\end{checklist}

\textbf{Escalate if:} Servos disconnected or faulted.

% ----------------------------------------------------------------------------
\subsection{LRF NOT RANGING}

\subsubsection{Symptoms}
\begin{itemize}
    \item No range reading displayed
    \item "NO ECHO" message (persistent)
    \item Range reading frozen/stale
    \item Wildly inaccurate range (e.g., 50,000m)
\end{itemize}

\subsubsection{Diagnosis}

\begin{procedurebox}[LRF DIAGNOSIS]
\begin{enumerate}
    \item Check target distance (within max range ~5km for hard targets?)
    \item Check target reflectivity (solid surface, not transparent/absorbent?)
    \item Check LRF temperature (System Status - overheated?)
    \item Check atmospheric conditions (fog, smoke, heavy rain?)
    \item Check for obstructions between LRF and target
\end{enumerate}
\end{procedurebox}

\subsubsection{Operator-Level Fixes}

\begin{checklist}
    \item Re-fire LRF (multiple attempts - 3-5 tries)
    \item Aim at different target (more reflective surface)
    \item Allow LRF to cool (if overheated - wait 5-10 minutes)
    \item Switch to manual range estimation (map, known landmarks)
    \item Wait for atmospheric conditions to improve
\end{checklist}

\textbf{Escalate if:} LRF disconnected or constant fault flag.

% ----------------------------------------------------------------------------
\subsection{LRF OPERATING MODES}

The LRF supports two operating modes:

\subsubsection{Single-Shot Mode (Default)}

\begin{itemize}
    \item Press \button{Button 1} once to trigger a single range measurement
    \item Range displayed on OSD: \osd{RNG: xxxx m}
    \item Best for stationary targets or occasional ranging
\end{itemize}

\subsubsection{Continuous Mode (5Hz Automatic Ranging)}

\begin{procedurebox}[ENABLING CONTINUOUS LRF MODE]
\textbf{To enable:}
\begin{enumerate}
    \item \textbf{Double-press} \button{Button 1} within 1 second
    \item System enters continuous ranging mode (5Hz)
    \item Range updates automatically ~5 times per second
\end{enumerate}

\textbf{To disable:}
\begin{enumerate}
    \item \textbf{Double-press} \button{Button 1} again within 1 second
    \item System returns to single-shot mode
\end{enumerate}
\end{procedurebox}

\begin{notebox}
When continuous mode is active, single presses of Button 1 are \textbf{ignored}. You must double-press again to disable continuous mode and return to single-shot operation.
\end{notebox}

\textbf{When to use Continuous Mode:}
\begin{itemize}
    \item Tracking moving targets where range changes rapidly
    \item When you need constant range updates for ballistic calculations
    \item During active engagements with maneuvering targets
\end{itemize}

\textbf{Caution:}
\begin{itemize}
    \item Continuous mode increases LRF component wear
    \item May cause faster temperature rise
    \item Disable when not needed
\end{itemize}

% ----------------------------------------------------------------------------
\subsection{LRF CLEAR VS ZEROING CLEAR}

These are \textbf{separate functions}:

\begin{itemize}
    \item \textbf{LRF Clear (Button 10):} Clears the current LRF range measurement only. Sets stored range to 0. Does \textbf{NOT} affect weapon zeroing.
    \item \textbf{Clear Active Zero (Menu):} Clears weapon zeroing offsets. Does \textbf{NOT} affect LRF measurements.
\end{itemize}

\begin{notebox}
If you want to clear both LRF range and zeroing, you must perform both actions separately.
\end{notebox}

% ----------------------------------------------------------------------------
\subsection{HIGH TEMPERATURE WARNING}

\subsubsection{Symptoms}
\begin{itemize}
    \item "HIGH TEMP" alarm (yellow or red)
    \item Device temperature >60\degree C in System Status
    \item Thermal shutdown (device stops functioning)
\end{itemize}

\subsubsection{Diagnosis}

\begin{procedurebox}[HIGH TEMPERATURE DIAGNOSIS]
\begin{enumerate}
    \item Check System Status → which device is overheating?
    \item Check ambient temperature (extreme heat environment?)
    \item Check operation tempo (continuous motion/firing?)
    \item Check ventilation (blocked air intakes?)
\end{enumerate}
\end{procedurebox}

\subsubsection{Operator-Level Fixes}

\begin{checklist}
    \item Reduce operations (slower movements, less frequent firing)
    \item Allow cooling period (5-15 minutes, depending on device)
    \item Limit continuous motion/firing (use intermittent operations)
    \item Increase ventilation (if possible - open vents, improve airflow)
    \item Monitor temperature trend (should decrease during cooldown)
\end{checklist}

\textbf{Escalate if:} Temperature remains >80\degree C after cooldown period.

\subsubsection{Temperature Management}

\textbf{Prevention:}
\begin{itemize}
    \item Avoid sustained high-tempo operations in hot environments
    \item Allow periodic cooling breaks (5 minutes every 30 minutes in extreme heat)
    \item Monitor temperature trends proactively
    \item Plan operations during cooler times of day (if possible)
\end{itemize}

% ============================================================================
\section{TROUBLESHOOTING QUICK REFERENCE}

\begin{longtable}{|L{0.18\textwidth}|L{0.22\textwidth}|L{0.25\textwidth}|L{0.25\textwidth}|}
\hline
\rowcolor{milblue!20}
\textbf{Symptom} & \textbf{Likely Cause} & \textbf{Operator Fix} & \textbf{Escalate If} \\
\hline
\endfirsthead

\hline
\rowcolor{milblue!20}
\textbf{Symptom} & \textbf{Likely Cause} & \textbf{Operator Fix} & \textbf{Escalate If} \\
\hline
\endhead

\textbf{No joystick response} & USB disconnected & Reconnect USB, reboot & Still offline after reboot \\
\hline
\textbf{Black video} & Camera fault & Switch cameras & Both cameras offline \\
\hline
\textbf{Tracking won't lock} & Poor contrast, bad gate & Improve contrast, resize gate & Fails on all targets \\
\hline
\textbf{Gimbal frozen} & E-Stop, Station OFF & Reset E-Stop, Station ON & Servo fault flag \\
\hline
\textbf{LRF no reading} & Target too far/obscured & Re-fire, different target & LRF disconnected \\
\hline
\textbf{High temp} & Overuse & Reduce ops, cool down & Temp stays >80\degree C \\
\hline
\textbf{Weapon won't arm} & Safety interlocks & Check switches, authorization & Switches non-functional \\
\hline
\textbf{Lost zero accuracy} & Drift, maintenance & Re-zero weapon & Persistent after re-zero \\
\hline
\end{longtable}

% ============================================================================
\section{ADVANCED TROUBLESHOOTING}

\subsection{INTERMITTENT PROBLEMS}

\subsubsection{Characteristics}
\begin{itemize}
    \item Problem appears and disappears randomly
    \item Works sometimes, fails other times
    \item Difficult to reproduce consistently
\end{itemize}

\subsubsection{Troubleshooting Approach}

\textbf{Document the Pattern:}
\begin{itemize}
    \item When does it occur? (time of day, temperature, specific operation)
    \item What makes it better/worse?
    \item How often does it happen?
    \item Any common factors when it occurs?
\end{itemize}

\textbf{Common Causes:}
\begin{itemize}
    \item Loose connections (intermittent contact)
    \item Temperature-related (works cold, fails hot)
    \item Software timing issues (race conditions)
    \item Electrical interference (when other systems operating)
\end{itemize}

\begin{cautionbox}
Intermittent problems often indicate impending failure. Report immediately even if problem self-clears.
\end{cautionbox}

% ----------------------------------------------------------------------------
\subsection{MULTIPLE SIMULTANEOUS FAULTS}

\subsubsection{Diagnosis Strategy}

When multiple systems fail simultaneously:

\begin{enumerate}
    \item \textbf{Look for common cause:}
    \begin{itemize}
        \item Power supply issue (affects all systems)
        \item Main computer failure (affects all subsystems)
        \item Communication bus failure (multiple devices offline)
    \end{itemize}
    
    \item \textbf{Prioritize troubleshooting:}
    \begin{itemize}
        \item Start with power (most fundamental)
        \item Then communication (affects device connectivity)
        \item Then individual devices
    \end{itemize}
    
    \item \textbf{Consider cascade failures:}
    \begin{itemize}
        \item One device failure may cause others
        \item Example: IMU failure → tracking fails → LAC fails
    \end{itemize}
\end{enumerate}

\textbf{Escalate immediately if:}
\begin{itemize}
    \item >3 devices offline simultaneously
    \item Power supply issues suspected
    \item Unknown common cause
\end{itemize}

% ----------------------------------------------------------------------------
\subsection{DEGRADED PERFORMANCE}

\subsubsection{Symptoms}
\begin{itemize}
    \item System works, but not as well as before
    \item Slower response times
    \item Reduced accuracy
    \item Increased errors/retries needed
\end{itemize}

\subsubsection{Investigation}

\textbf{Compare to Baseline:}
\begin{itemize}
    \item Check tracking confidence (was 90\%, now 70\%?)
    \item Check ranging accuracy (was ±2m, now ±10m?)
    \item Check gimbal smoothness (jerky vs. smooth)
    \item Check response time (lag vs. immediate)
\end{itemize}

\textbf{Possible Causes:}
\begin{itemize}
    \item Mechanical wear (servos, bearings)
    \item Sensor degradation (camera, LRF, IMU)
    \item Software configuration drift
    \item Environmental factors (temperature, contamination)
\end{itemize}

\textbf{Operator Actions:}
\begin{itemize}
    \item Document specific performance degradation
    \item Attempt re-calibration (re-zero, re-configure)
    \item Report to maintenance (even if still operational)
\end{itemize}

% ============================================================================
\section{PREVENTIVE MAINTENANCE}

\subsection{OPERATOR-LEVEL MAINTENANCE}

\subsubsection{Daily Tasks}
\begin{itemize}
    \item Clean camera lenses (soft cloth, approved cleaner)
    \item Inspect cables and connectors (secure, no damage)
    \item Check fluid levels (if applicable)
    \item Wipe down control panels (remove dirt, moisture)
    \item Visual inspection for damage or corrosion
\end{itemize}

\subsubsection{Weekly Tasks}
\begin{itemize}
    \item Verify zero accuracy (test shot at known range)
    \item Exercise full range of motion (azimuth/elevation)
    \item Test all emergency procedures (E-Stop, tracking abort)
    \item Check backup systems (backup radio, backup camera)
    \item Review system logs (check for recurring errors)
\end{itemize}

\subsubsection{Monthly Tasks (or as directed)}
\begin{itemize}
    \item Re-zero weapon
    \item Full system health check (all devices, all functions)
    \item Update system software (if authorized and available)
    \item Lubricate moving parts (if authorized and trained)
\end{itemize}

\begin{notebox}
Preventive maintenance prevents failures. Regular checks catch problems before they become critical.
\end{notebox}

% ----------------------------------------------------------------------------
\subsection{WHAT OPERATORS SHOULD NOT DO}

\textbf{Operators are NOT authorized to:}
\begin{itemize}
    \item ❌ Open sealed enclosures
    \item ❌ Repair circuit boards or electronics
    \item ❌ Replace major components (servos, cameras, computers)
    \item ❌ Modify software or configuration files
    \item ❌ Perform calibrations beyond re-zeroing
    \item ❌ Disable safety interlocks
\end{itemize}

\textbf{Operators ARE authorized to:}
\begin{itemize}
    \item ✅ Clean external surfaces and lenses
    \item ✅ Reconnect loose cables (external only)
    \item ✅ Re-zero weapon
    \item ✅ Adjust user settings (zoom, brightness, environmental parameters)
    \item ✅ Power cycle system
    \item ✅ Replace batteries (if trained and authorized)
\end{itemize}

% ============================================================================
\section{MAINTENANCE ESCALATION}

\subsection{WHEN TO CALL MAINTENANCE}

\textbf{Call maintenance immediately for:}
\begin{itemize}
    \item Device disconnected (doesn't reconnect after reboot)
    \item Persistent fault flags
    \item Servo malfunction (erratic motion, no motion, unusual noises)
    \item Physical damage observed
    \item Thermal runaway (temperature >80\degree C, not decreasing)
    \item Electrical issues (burning smell, smoke, arcing)
    \item Weapon malfunction (jam, accidental discharge, failure to fire)
    \item Safety system failure (E-Stop, interlocks not working)
\end{itemize}

\subsection{INFORMATION TO PROVIDE}

When calling maintenance, provide:

\begin{checklist}
    \item System serial number / unit identifier
    \item Specific symptoms (be detailed)
    \item When problem first occurred
    \item Frequency (constant, intermittent, getting worse?)
    \item Recent events (maintenance, configuration changes, incidents)
    \item System Status information (which devices faulted?)
    \item Error messages or alarm codes
    \item Troubleshooting already attempted
    \item Current system state (powered on, E-Stopped, etc.)
\end{checklist}

\subsection{MAINTENANCE RESPONSE PRIORITIES}

\textbf{Priority 1 - Immediate Response:}
\begin{itemize}
    \item Safety system failure
    \item Fire/smoke/electrical hazard
    \item Weapon malfunction
    \item Multiple critical systems offline
\end{itemize}

\textbf{Priority 2 - Same Day Response:}
\begin{itemize}
    \item Single critical system offline (LRF, camera, servo)
    \item Degraded but operational (reduced capability)
\end{itemize}

\textbf{Priority 3 - Scheduled Maintenance:}
\begin{itemize}
    \item Minor issues (cosmetic damage, non-critical warnings)
    \item Preventive maintenance tasks
    \item Performance optimization
\end{itemize}

% ============================================================================
\section{TROUBLESHOOTING CASE STUDIES}

\subsection{CASE STUDY 1: Tracking Works Initially, Then Fails}

\textbf{Symptoms:}
\begin{itemize}
    \item Tracking locks successfully
    \item After 5-10 minutes, tracking becomes erratic
    \item Eventually loses lock completely
    \item Problem recurs each mission
\end{itemize}

\textbf{SLAF Analysis:}
\begin{itemize}
    \item \textbf{Stop:} Problem is intermittent, temperature-related?
    \item \textbf{Look:} Check System Status → Camera temperature rising
    \item \textbf{Assess:} Camera overheating, thermal FFC becoming more frequent
    \item \textbf{Fix:} Report to maintenance - camera cooling system fault
\end{itemize}

\textbf{Lesson:} Intermittent problems that worsen over time indicate component failure. Don't ignore!

% ----------------------------------------------------------------------------
\subsection{CASE STUDY 2: Gimbal Moves Slowly in One Direction}

\textbf{Symptoms:}
\begin{itemize}
    \item Azimuth movement slow when turning left
    \item Normal speed turning right
    \item Elevation normal
    \item No fault flags in System Status
\end{itemize}

\textbf{SLAF Analysis:}
\begin{itemize}
    \item \textbf{Stop:} Asymmetric problem suggests mechanical issue
    \item \textbf{Look:} Check torque readings → high torque on left turns
    \item \textbf{Assess:} Mechanical resistance (bearing wear, obstruction)
    \item \textbf{Fix:} Escalate to maintenance - mechanical inspection needed
\end{itemize}

\textbf{Lesson:} Directional or asymmetric problems often indicate mechanical issues, not software/electrical.

% ----------------------------------------------------------------------------
\subsection{CASE STUDY 3: System Won't Power Up}

\textbf{Symptoms:}
\begin{itemize}
    \item Main power switch ON, but no lights
    \item No response from any system
    \item Complete dead system
\end{itemize}

\textbf{SLAF Analysis:}
\begin{itemize}
    \item \textbf{Stop:} Total power failure - electrical issue
    \item \textbf{Look:} Check external power source (generator, battery)
    \item \textbf{Assess:} External power available? Cables connected? Circuit breaker tripped?
    \item \textbf{Fix:} 
    \begin{itemize}
        \item If external power issue: Restore external power
        \item If external power OK: Escalate - internal power supply fault
    \end{itemize}
\end{itemize}

\textbf{Lesson:} Start troubleshooting from the outside in. Check simplest causes (external power) before assuming complex internal failures.

% ============================================================================
% END OF LESSON 9
% ============================================================================