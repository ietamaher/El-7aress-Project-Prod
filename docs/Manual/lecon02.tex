% ============================================================================
% LESSON 2 - BASIC OPERATION
% ============================================================================
\lesson{2: BASIC OPERATION}

\noindent
\begin{tabular}{@{}ll@{}}
\textbf{Duration:} & 4 hours \\
\textbf{Type:} & Classroom + Practical \\
\textbf{References:} & Operator manual, system startup checklist \\
\end{tabular}

\section{Introduction}

\textbf{Lesson Purpose}: This lesson teaches complete system startup, DCU and joystick operations, OSD interpretation, and proper shutdown procedures.

\textbf{Learning Objectives}:
\begin{itemize}
    \item Perform complete system startup procedure
    \item Operate all DCU buttons, switches, and controls
    \item Control gimbal movement using joystick
    \item Switch between day and thermal cameras
    \item Operate camera zoom controls
    \item Interpret all OSD elements correctly
    \item Perform normal system shutdown
\end{itemize}

\section{System Startup Procedure}

\subsection{Pre-Startup Checklist}

Before powering on the system, verify:

\begin{checklist}
    \item Walk-around inspection complete (Lesson 1) - all items GO
    \item Weapon cleared per Appendix A
    \item Ammunition removed or accounted for
    \item All personnel clear of turret (minimum 2 meters)
    \item Vehicle power available (check voltage: 20-30V DC nominal)
    \item Operator qualified and authorized
    \item Mission briefing received (zones, ROE, threats)
    \item Communication established with command
\end{checklist}

\begin{warningbox}
Do not start system if any checklist item is not complete.
\end{warningbox}

\subsection{Startup Sequence (10 Steps)}

Perform steps in order. Do not skip steps.

\begin{procedurebox}[STEP 1: INITIAL POWER-UP]
\textbf{Action}:
\begin{itemize}
    \item Ensure \button{Station Enable} switch is in OFF position
    \item Apply vehicle power to RCWS (circuit breaker ON or power cable connected)
\end{itemize}

\textbf{Expected Result}:
\begin{itemize}
    \item \textbf{Power} indicator light illuminates (Green)
    \item DCU screen displays boot logo
    \item System begins self-test (approximately 30 seconds)
\end{itemize}

\textbf{If NO Power Light}:
\begin{itemize}
    \item Check vehicle power supply (voltage 20-30V DC)
    \item Check circuit breaker
    \item Check cable connections
    \item Report to maintenance if power available but no light
\end{itemize}
\end{procedurebox}

\begin{procedurebox}[STEP 2: BOOT SELF-TEST]
\textbf{Action}:
\begin{itemize}
    \item Observe DCU screen during boot
    \item Wait for self-test to complete (DO NOT interrupt)
\end{itemize}

\textbf{Expected Display Sequence}:
\begin{lstlisting}
Checking Devices...
[ OK ] Day Camera
[ OK ] Thermal Camera
[ OK ] Laser Rangefinder
[ OK ] Azimuth Motor
[ OK ] Elevation Motor
[ OK ] Joystick Controller
[ -- ] Weapon Actuator (if not installed)

System Ready
Press STATION ENABLE to continue
\end{lstlisting}

\textbf{If Any Device Shows [FAIL]}:
\begin{itemize}
    \item DO NOT PROCEED with startup
    \item Note which device failed
    \item Report to maintenance immediately
    \item System may operate in degraded mode but requires supervisor approval
\end{itemize}
\end{procedurebox}

\begin{procedurebox}[STEP 3: ENABLE STATION]
\textbf{Action}:
\begin{itemize}
    \item Move \button{Station Enable} switch from OFF to ON
\end{itemize}

\textbf{Expected Result}:
\begin{itemize}
    \item \textbf{System Ready} light illuminates (Green)
    \item Gimbal motors energize (you may hear a soft hum)
    \item Video feed appears on DCU screen
    \item OSD overlay displays system information
    \item Gimbal automatically moves to Home Position (0\degree\ AZ, 0\degree\ EL)
\end{itemize}

\begin{cautionbox}
Gimbal will move during this step. Ensure area is clear.
\end{cautionbox}
\end{procedurebox}

\begin{procedurebox}[STEP 4: VERIFY HOME POSITION]
\textbf{Action}:
\begin{itemize}
    \item Observe OSD azimuth and elevation readings
    \item If not at home (AZ: 000\degree, EL: 00\degree), press \button{Home} button
\end{itemize}

\textbf{Expected Result}:
\begin{itemize}
    \item OSD displays: \osd{AZ: 000° EL: 00°} (±2\degree\ tolerance)
    \item Reticle is centered on screen
    \item Gimbal points directly forward relative to vehicle
\end{itemize}
\end{procedurebox}

\subsubsection{Detailed Homing Sequence (50ms Control Loop)}

The homing process follows a precise timing sequence:

\begin{lstlisting}
Cycle 0:   Operator presses HOME button
           → State = Requested
           → OSD displays: "homing init"

Cycle 1:   (50ms later)
           → GimbalController sends HOME command to PLC42
           → State = InProgress
           → OSD displays: "HOMING..."
           → 30-second timeout timer starts

Wait:      Oriental Motor servos execute homing sequence
           → Azimuth motor moves to home position
           → Elevation motor moves to home position
           → Motors send HOME-END signals when complete
           → Azimuth HOME-END (DI6), Elevation HOME-END (DI7)

Complete:  Both HOME-END signals received
           → State = Completed
           → OSD displays: "HOME COMPLETE" (5 seconds)
\end{lstlisting}

\textbf{Homing Timeout Handling}:
\begin{itemize}
    \item \textbf{30-second timeout} if HOME-END signals not received
    \item OSD displays: \osd{HOMING TIMEOUT - FAULT} (red flashing)
    \item \textbf{Recovery procedure}:
    \begin{enumerate}
        \item Press Emergency Stop
        \item Clear any gimbal obstruction
        \item Release Emergency Stop
        \item Retry homing procedure
    \end{enumerate}
\end{itemize}

\begin{warningbox}
If homing fails repeatedly, report to maintenance. Do not operate gimbal without successful homing.
\end{warningbox}

\begin{procedurebox}[STEP 5: SELECT CAMERA]
\textbf{Action}:
\begin{itemize}
    \item Press \button{CAM} button on joystick to select Day or Thermal camera
    \item Default camera is DAY (visible spectrum)
\end{itemize}

\textbf{Expected Result}:
\begin{itemize}
    \item OSD displays camera type: \osd{DAY} or \osd{THERMAL}
    \item Video image switches between color (day) and grayscale/colorized (thermal)
    \item FOV (Field of View) value updates on OSD
\end{itemize}
\end{procedurebox}

\begin{procedurebox}[STEP 6: TEST GIMBAL MOVEMENT]
\textbf{Action}:
\begin{itemize}
    \item Gently move joystick in all directions (left/right/up/down)
    \item Verify gimbal responds smoothly
    \item Return joystick to center (gimbal stops)
\end{itemize}

\textbf{Expected Result}:
\begin{itemize}
    \item Gimbal slews in direction of joystick movement
    \item OSD azimuth/elevation values update in real-time
    \item No grinding, binding, or unusual noises
    \item Gimbal stops when joystick returns to center
\end{itemize}
\end{procedurebox}

\begin{procedurebox}[STEP 7: TEST CAMERA ZOOM]
\textbf{Action}:
\begin{itemize}
    \item Press Zoom Rocker UP (zoom in) and DOWN (zoom out)
    \item Observe video image magnification change
    \item Observe FOV value on OSD
\end{itemize}

\textbf{Expected Result}:
\begin{itemize}
    \item Image magnifies when zooming in (FOV decreases)
    \item Image wide-angle when zooming out (FOV increases)
    \item Zoom is smooth with no jerking
    \item OSD FOV updates continuously
\end{itemize}
\end{procedurebox}

\begin{procedurebox}[STEP 8: TEST LASER RANGEFINDER (LRF)]
\textbf{Action}:
\begin{itemize}
    \item Aim reticle at a known object 100m+ away
    \item Press and hold \button{LRF} button on joystick
\end{itemize}

\textbf{Expected Result}:
\begin{itemize}
    \item Laser fires (you will NOT see visible beam - infrared)
    \item OSD displays range reading: \osd{RNG: XXXm}
    \item Range updates within 1 second
    \item LRF automatically times out after 5 seconds
\end{itemize}

\begin{warningbox}
Do not aim LRF at people, animals, or reflective surfaces at close range. Eye damage can occur.
\end{warningbox}
\end{procedurebox}

\begin{procedurebox}[STEP 9: ENABLE STABILIZATION]
\textbf{Action}:
\begin{itemize}
    \item Set \button{Stabilization} switch to ON
\end{itemize}

\textbf{Expected Result}:
\begin{itemize}
    \item OSD displays \osd{STAB: ON}
    \item Gimbal compensates for vehicle movement
    \item Reticle remains steady on target even if vehicle rocks
\end{itemize}
\end{procedurebox}

\begin{procedurebox}[STEP 10: SYSTEM READY - FINAL CHECK]
\textbf{Verify all indicator lights}:
\begin{itemize}
    \item \textbf{Power}: Green (ON)
    \item \textbf{System Ready}: Green (ON)
    \item \textbf{Gun Armed}: OFF (system is SAFE)
    \item \textbf{Fault/Alarm}: OFF (no errors)
\end{itemize}

\textbf{Verify OSD displays}:
\begin{itemize}
    \item Live video feed (day or thermal)
    \item Azimuth and elevation values
    \item Current FOV
    \item System mode (Manual)
    \item No warning messages
\end{itemize}

\textbf{If All Checks Pass}: System is ready for operation
\end{procedurebox}

\subsection{Startup Troubleshooting}

\small
\begin{longtable}{|L{0.25\textwidth}|L{0.30\textwidth}|L{0.38\textwidth}|}
\hline
\rowcolor{milblue!20}
\textbf{Problem} & \textbf{Possible Cause} & \textbf{Action} \\
\hline
No power light & No vehicle power & Check circuit breaker, voltage \\
\hline
Self-test fails & Device malfunction & Note failed device, report to maintenance \\
\hline
No video feed & Camera error & Check camera connections, restart system \\
\hline
Gimbal won't move & Motors disabled or fault & Check Station Enable, check for faults \\
\hline
Erratic gimbal & Joystick calibration & Recalibrate joystick (maintenance task) \\
\hline
LRF no reading & Out of range or bad target & Aim at closer/better reflective target \\
\hline
Thermal frozen & FFC in progress & Wait 5 seconds for FFC to complete \\
\hline
\end{longtable}

\section{Display and Control Unit (DCU) Operations}

\subsection{DCU Button and Switch Functions}

\subsubsection{Emergency Stop Button (RED)}

\textbf{Location}: Top left of DCU panel, large RED button

\textbf{Function}: Immediate system shutdown for safety emergencies

\textbf{Operation}:
\begin{enumerate}
    \item Press button (no confirmation required)
    \item System immediately:
    \begin{itemize}
        \item Stops all gimbal movement
        \item Safes weapon (trigger disabled)
        \item Locks servos in place
        \item Displays "EMERGENCY STOP ACTIVE" on OSD
    \end{itemize}
\end{enumerate}

\textbf{To Reset}:
\begin{enumerate}
    \item Twist/pull button to release
    \item Verify emergency condition is resolved
    \item Press \button{Station Enable} OFF then ON to restart
\end{enumerate}

\begin{warningbox}[CRITICAL]
Do NOT hesitate to use Emergency Stop. Better safe than sorry.
\end{warningbox}

\subsubsection{Station Enable Switch}

\textbf{Positions}: OFF / ON

\textbf{OFF Position}:
\begin{itemize}
    \item Gimbal motors disabled (turret cannot move)
    \item Video still displays (cameras remain powered)
    \item Weapon is safed
    \item Safe to approach turret for inspection
\end{itemize}

\textbf{ON Position}:
\begin{itemize}
    \item Gimbal motors enabled
    \item All subsystems operational
    \item Turret can move if joystick input received
    \item Stay clear of turret
\end{itemize}

\subsubsection{Gun Arm/Safe Switch}

\textbf{Positions}: SAFE / ARM

\textbf{SAFE Position} (Default):
\begin{itemize}
    \item Weapon trigger disabled
    \item Gun Armed light is OFF
    \item Trigger pull has no effect
    \item Safe for non-combat operations
\end{itemize}

\textbf{ARM Position}:
\begin{itemize}
    \item Weapon trigger enabled
    \item Gun Armed light illuminates RED
    \item Trigger pull will fire weapon (if other conditions met)
    \item Only use during combat or live fire training
\end{itemize}

\begin{warningbox}
When Gun Armed light is RED, treat weapon as HOT. One trigger pull away from firing.
\end{warningbox}

\subsubsection{Fire Mode Selector}

\textbf{Positions}: SINGLE / SHORT BURST / LONG BURST

\begin{twocol}
\textbf{SINGLE}:
\begin{itemize}
    \item One round per trigger pull
    \item Most accurate mode
    \item Use for precision engagement
\end{itemize}

\textbf{SHORT BURST}:
\begin{itemize}
    \item 3-5 rounds per trigger pull
    \item Good balance of accuracy and firepower
    \item Use for moving targets
\end{itemize}

\textbf{LONG BURST}:
\begin{itemize}
    \item Continuous fire while trigger held
    \item Less accurate due to recoil
    \item Use for area suppression
\end{itemize}
\end{twocol}

\subsubsection{Speed Select Switch}

\textbf{Positions}: LOW / MEDIUM / HIGH

\small
\begin{longtable}{|L{0.20\textwidth}|L{0.35\textwidth}|L{0.38\textwidth}|}
\hline
\rowcolor{milblue!20}
\textbf{Speed} & \textbf{Use Case} & \textbf{Max Slew Rate} \\
\hline
LOW & Zeroing, fine adjustments, precision & ~5\degree/second \\
\hline
MEDIUM & Normal surveillance, target acquisition & ~20\degree/second \\
\hline
HIGH & Close-range threats, emergencies & ~60\degree/second \\
\hline
\end{longtable}

\subsection{DCU Indicator Lights}

\begin{twocol}
\begin{itemize}
    \item \textbf{Power} (Green): Vehicle power supplied
    \item \textbf{System Ready} (Green): All subsystems operational
    \item \textbf{Gun Armed} (Red): Weapon is armed
    \item \textbf{Ammo Loaded} (Yellow): Ammunition detected
    \item \textbf{Authorized} (Green): Operator authorized
    \item \textbf{Fault/Alarm} (Red): System error detected
\end{itemize}
\end{twocol}

\section{Joystick Controller Operations}

\subsection{Joystick Control Techniques}

\subsubsection{Proper Grip}

\textbf{Right Hand Position}:
\begin{enumerate}
    \item Wrap fingers around joystick grip
    \item Index finger rests on trigger (outside trigger guard when not firing)
    \item Thumb on top, near CAM and TRK buttons
    \item Dead Man Switch on rear of grip - squeeze with palm/fingers to engage
\end{enumerate}

\subsubsection{Gimbal Slew Technique}

\textbf{Small Movements} (Precision):
\begin{itemize}
    \item Deflect stick slightly from center (10-20\%)
    \item Gimbal moves slowly
    \item Good for: Tracking, zeroing, fine adjustments
\end{itemize}

\textbf{Large Movements} (Rapid Slew):
\begin{itemize}
    \item Deflect stick fully (80-100\%)
    \item Gimbal moves at maximum speed
    \item Good for: Searching, responding to threats, sector scans
\end{itemize}

\subsection{Joystick Button Functions}

The joystick provides 19 programmable buttons (Button 0-18) plus analog axes and a hat switch.

\subsubsection{Complete Joystick Button Reference (19 Buttons)}

\small
\begin{longtable}{|C{0.08\textwidth}|L{0.30\textwidth}|L{0.52\textwidth}|}
\hline
\rowcolor{milblue!20}
\textbf{Button} & \textbf{Function} & \textbf{Operation} \\
\hline
\textbf{0} & Engagement Mode / Master Arm & Press = Enter engagement, Release = Exit engagement \\
\hline
\textbf{1} & LRF (Laser Range Finder) & Single press = Single measurement, \textbf{Double-click = Toggle continuous LRF (5Hz)} \\
\hline
\textbf{2} & Lead Angle Compensation (LAC) & \textbf{Toggle LAC ON/OFF (joystick ONLY - no menu access!)} \\
\hline
\textbf{3} & Dead Man Switch (Palm Switch) & \textbf{MUST HOLD} for weapon operation and LAC toggle \\
\hline
\textbf{4} & Tracking Control & Single = Cycle tracking phase, \textbf{Double-click (<1 sec) = ABORT tracking} \\
\hline
\textbf{5} & Fire Weapon & Trigger - fires weapon when armed \\
\hline
\textbf{6} & Zoom + & Increase camera magnification (hold for continuous) \\
\hline
\textbf{7} & Thermal LUT + & Next thermal video look-up table (thermal camera only) \\
\hline
\textbf{8} & Zoom - & Decrease camera magnification (hold for continuous) \\
\hline
\textbf{9} & Thermal LUT - & Previous thermal look-up table (thermal camera only) \\
\hline
\textbf{10} & LRF Clear & Clear range reading, stop continuous LRF mode \\
\hline
\textbf{11} & Mode Cycle & Cycle: Manual → Sector Scan → TRP → (Radar) → Manual \\
\hline
\textbf{12} & Reserved & Available for future use \\
\hline
\textbf{13} & Mode Cycle & Duplicate of Button 11 (ergonomic placement) \\
\hline
\textbf{14} & Select Next Zone/TRP & Next TRP page or sector scan zone (during surveillance) \\
\hline
\textbf{15} & Reserved & Future use \\
\hline
\textbf{16} & Select Previous Zone/TRP & Previous TRP page or sector scan zone (during surveillance) \\
\hline
\textbf{17} & Reserved & Future use \\
\hline
\textbf{18} & Reserved & Future use \\
\hline
\end{longtable}

\subsubsection{Analog Controls}

\small
\begin{longtable}{|L{0.28\textwidth}|L{0.65\textwidth}|}
\hline
\rowcolor{milblue!20}
\textbf{Control} & \textbf{Function} \\
\hline
\textbf{Main Stick X-Axis} & Gimbal azimuth slew (left/right) \\
\hline
\textbf{Main Stick Y-Axis} & Gimbal elevation slew (up/down) \\
\hline
\textbf{Hat Switch (D-Pad)} & Tracking gate resize during Acquisition phase / Menu navigation \\
\hline
\end{longtable}

\begin{warningbox}[CRITICAL]
\textbf{Button 2 (LAC)} is the \textbf{ONLY} method to activate Lead Angle Compensation. There is NO menu access for LAC. Dead Man Switch (Button 3) must be held when toggling LAC.
\end{warningbox}

\subsubsection{Camera Switch (CAM Button)}

\textbf{When to Use Day Camera}:
\begin{itemize}
    \item Good lighting conditions
    \item Need color information
    \item Need maximum zoom range (2\degree\ to 60\degree\ FOV)
\end{itemize}

\textbf{When to Use Thermal Camera}:
\begin{itemize}
    \item Darkness, dawn, dusk
    \item Smoke, fog, dust
    \item Detecting hidden personnel (heat signatures)
    \item Identifying recently fired weapons (barrel heat)
\end{itemize}

\subsubsection{Laser Rangefinder (Button 1)}

\textbf{Single Measurement Mode} (default):
\begin{enumerate}
    \item Aim reticle at target
    \item Press Button 1 (single press)
    \item Laser fires (invisible infrared beam)
    \item OSD displays range: \osd{RNG: XXXm}
    \item Range used for ballistic calculations
\end{enumerate}

\textbf{Continuous LRF Mode} (5Hz automatic ranging):
\begin{enumerate}
    \item \textbf{Double-click Button 1} (within 1 second)
    \item System announces: "CONTINUOUS LRF ENABLED"
    \item LRF fires automatically at 5Hz (5 times per second)
    \item Range continuously updates on OSD
    \item \textbf{Double-click Button 1 again} to disable
    \item Or press \textbf{Button 10} (LRF Clear) to stop
\end{enumerate}

\begin{notebox}
Continuous LRF mode is useful for tracking moving targets where range is constantly changing. Disable when not needed to preserve laser life.
\end{notebox}

\textbf{Specifications}:\\
\textbf{Range}: 50m to 4000m\\
\textbf{Accuracy}: ±5 meters\\
\textbf{Continuous Rate}: 5Hz (when enabled)

\begin{warningbox}
Class 3B laser. Do not aim at people or reflective surfaces at close range.
\end{warningbox}

\subsubsection{Dead Man Switch}

\textbf{Location}: Rear of joystick grip

\textbf{Purpose}:
\begin{itemize}
    \item Prevents accidental discharge if operator is incapacitated
    \item Automatic safety if operator loses grip
    \item Required safety interlock for firing
\end{itemize}

\begin{warningbox}[CRITICAL SAFETY RULE]
Release Dead Man Switch immediately when not actively engaging a target.
\end{warningbox}

\section{On-Screen Display (OSD) Interpretation}

\subsection{Complete OSD Layout}

\begin{lstlisting}
┌──────────────────────────────────────────────────────┐
│ AZ: 045°  EL: +12°  | DAY  FOV: 9.0° ZOOM: 15× │ ← Top
│                                                       │
│                    [Target Box]                      │
│                        ┌──┐                          │
│                        └──┘                          │
│                          + ← Reticle                 │
│                         (*) ← CCIP Pipper            │
│                                                       │
│ RNG: 850m         MODE: Manual       STATUS: ARMED   │ ← Bottom
│ TEMP: 35°C        TRACK: Off         AMMO: 450      │
└──────────────────────────────────────────────────────┘
\end{lstlisting}

\subsection{Top Bar Elements}

\begin{itemize}
    \item \textbf{AZ: XXX\degree}: Current azimuth position (000\degree\ to 359\degree)
    \item \textbf{EL: ±XX\degree}: Current elevation position (-20\degree\ to +60\degree)
    \item \textbf{DAY / THERMAL}: Active camera type
    \item \textbf{FOV: X.X\degree}: Current field of view
    \item \textbf{ZOOM: XX×}: Zoom magnification factor
\end{itemize}

\subsection{Center Area Elements}

\textbf{Reticle (+)}:
\begin{itemize}
    \item Main aiming point
    \item Where the gun is currently aimed (with zeroing applied)
    \item Fire weapon with reticle on target
\end{itemize}

\textbf{CCIP Pipper ((*))}:
\begin{itemize}
    \item Continuously Computed Impact Point
    \item Shows where bullet will actually hit with all ballistic corrections
    \item May be offset from reticle if ballistics are applied
    \item Always aim with CCIP for accurate hits
\end{itemize}

\subsection{Bottom Bar Elements}

\begin{twocol}
\textbf{Left Side}:
\begin{itemize}
    \item \textbf{RNG: XXXm}: Distance to target
    \item \textbf{TEMP: XX\degree C}: System temperature
\end{itemize}

\textbf{Center}:
\begin{itemize}
    \item \textbf{MODE}: Current motion mode
    \item \textbf{TRACK}: Tracking status
\end{itemize}

\textbf{Right Side}:
\begin{itemize}
    \item \textbf{STATUS}: Weapon system status
    \item \textbf{AMMO}: Remaining rounds
    \item \textbf{STAB}: Stabilization status
\end{itemize}
\end{twocol}

\subsection{Warning Messages}

\textbf{Critical warnings appear in center screen}:

\begin{itemize}
    \item \textbf{NO-FIRE ZONE WARNING}: Red, flashing - Do NOT fire
    \item \textbf{NO-TRAVERSE WARNING}: Yellow - Movement restricted
    \item \textbf{EMERGENCY STOP ACTIVE}: Red, solid - System safed
    \item \textbf{SYSTEM FAULT}: Red/yellow - Check status
    \item \textbf{TARGET LOST}: Yellow - Tracking lost, coast mode
\end{itemize}

\section{System Shutdown Procedure}

\subsection{Shutdown Sequence (7 Steps)}

\begin{procedurebox}[STEP 1: SAFE THE WEAPON]
\begin{itemize}
    \item Move \button{Gun Arm/Safe} switch to SAFE
    \item Verify Gun Armed light is OFF
    \item Release Dead Man Switch on joystick
\end{itemize}
\end{procedurebox}

\begin{procedurebox}[STEP 2: RETURN TO HOME POSITION]
\begin{itemize}
    \item Press \button{Home} button on DCU
    \item Wait for gimbal to slew to 0\degree\ AZ, 0\degree\ EL
\end{itemize}
\end{procedurebox}

\begin{procedurebox}[STEP 3: DISABLE STABILIZATION]
\begin{itemize}
    \item Move \button{Stabilization} switch to OFF
\end{itemize}
\end{procedurebox}

\begin{procedurebox}[STEP 4: ACCESS SHUTDOWN MENU]
\begin{itemize}
    \item Press \button{MENU ✓} button
    \item Navigate to SYSTEM → Shutdown
    \item Select "Shutdown System"
    \item Confirm shutdown
    \item System performs orderly shutdown
    \item Wait for "SHUTDOWN COMPLETE" message
\end{itemize}
\end{procedurebox}

\begin{procedurebox}[STEP 5: DISABLE STATION]
\begin{itemize}
    \item Move \button{Station Enable} switch to OFF
    \item System Ready light turns OFF
    \item Gimbal motors de-energize
\end{itemize}
\end{procedurebox}

\begin{procedurebox}[STEP 6: REMOVE VEHICLE POWER]
\textbf{If end of mission/shift}:
\begin{itemize}
    \item Turn off circuit breaker, OR
    \item Disconnect power cable (if external)
    \item Power light turns OFF
    \item DCU screen goes black
\end{itemize}
\end{procedurebox}

\begin{procedurebox}[STEP 7: SECURE WEAPON AND EQUIPMENT]
\begin{itemize}
    \item Clear weapon per Appendix A (if required)
    \item Remove ammunition (if required by SOP)
    \item Install protective covers on cameras
    \item Lock operator station (if applicable)
    \item Complete operator log entry
\end{itemize}
\end{procedurebox}

\subsection{Post-Shutdown Checks}

\begin{checklist}
    \item Gun Armed light is OFF
    \item Gimbal is at home position (0\degree\ AZ, 0\degree\ EL)
    \item Station Enable is OFF
    \item Weapon is cleared (if required)
    \item Covers installed (if required)
    \item Operator log entry complete
\end{checklist}

\section{Student Review Questions}

\begin{enumerate}
    \item What is the first step in the system startup procedure?
    \item What should you do if a device shows [FAIL] during self-test?
    \item Where is the Emergency Stop button located?
    \item What are the three fire mode settings?
    \item What is the difference between the reticle and CCIP pipper?
    \item When should you use the thermal camera instead of the day camera?
    \item What is the purpose of the Dead Man Switch?
    \item What does \osd{STAB: ON} indicate on the OSD?
    \item What is the proper sequence for system shutdown?
    \item What must be done before removing vehicle power?
\end{enumerate}