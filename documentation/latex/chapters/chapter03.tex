% MENU SYSTEM & SETTINGS

\noindent
\textbf{Lesson Duration:} 2 hours (Practical)

\vspace{0.3cm}

\noindent
\textbf{Learning Objectives:}

Upon completion of this lesson, you will be able to:

\begin{itemize}[leftmargin=2em,itemsep=3pt]
\item Navigate all menu structures using DCU controls
\item Configure reticle types and display colors
\item Access system status information
\item Modify operational settings safely
\end{itemize}

\vspace{0.5cm}


\section{3.1 MENU NAVIGATION BASICS}

\subsection{ACCESSING THE MAIN MENU}

\textbf{Method}: Press \textbf{MENU $\checkmark$} button on DCU

\textbf{Result}:
\begin{itemize}
\item Video feed dims (still visible in background)
\item Menu window appears in center of screen
\item Current selection is highlighted
\item Menu title displayed at top
\end{itemize}

\subsection{MENU CONTROLS}

% TABLE ROW: | Button | Function |
% TABLE ROW: |--------|----------|
% TABLE ROW: | **MENU ✓** | Open menu / Confirm selection / Exit menu |
% TABLE ROW: | **MENU ▲** | Move selection up / Increase value |
% TABLE ROW: | **MENU ▼** | Move selection down / Decrease value |

\textbf{Alternative}: Joystick HAT switch can also be used for menu navigation (UP/DOWN/LEFT/RIGHT)

\subsection{NAVIGATION WORKFLOW}

\textbf{5-Step Process}:

\begin{enumerate}
\item \textbf{Open Menu}
\end{enumerate}
\begin{itemize}
\item Press \textbf{MENU $\checkmark$}
\item Main menu appears
\end{itemize}

\begin{enumerate}
\item \textbf{Navigate}
\end{enumerate}
\begin{itemize}
\item Press \textbf{▲} or \textbf{▼} to highlight desired option
\item Section headers (e.g., "--- BALLISTICS ---") are skipped automatically
\end{itemize}

\begin{enumerate}
\item \textbf{Select}
\end{enumerate}
\begin{itemize}
\item Press \textbf{MENU $\checkmark$} to enter submenu or activate option
\item Submenu opens (if applicable)
\end{itemize}

\begin{enumerate}
\item \textbf{Return}
\end{enumerate}
\begin{itemize}
\item Navigate to "Return..." option at bottom of menu
\item Press \textbf{MENU $\checkmark$}
\item OR: Press \textbf{MENU $\checkmark$} repeatedly on section headers to go back
\end{itemize}

\begin{enumerate}
\item \textbf{Exit to Operational Screen}
\end{enumerate}
\begin{itemize}
\item Continue pressing "Return..." until back at live video
\item System resumes normal operation
\end{itemize}

\textbf{Tip}: If you get lost in menus, keep selecting "Return..." until you exit completely.

\section{3.2 MAIN MENU STRUCTURE}

\subsection{COMPLETE MENU TREE}

\begin{verbatim}
MAIN MENU
├── --- RETICLE & DISPLAY ---
│   ├── Personalize Reticle
│   └── Personalize Colors
│
├── --- BALLISTICS ---
│   ├── Zeroing (Lesson 6)
│   ├── Environmental Parameters (Lesson 6)
│   └── Lead Angle Compensation (Lesson 6)
│
├── --- SYSTEM ---
│   ├── Zone Definitions (Lesson 4)
│   ├── System Status (Lesson 7)
│   └── Shutdown System
│
├── --- INFO ---
│   └── About
│
└── Return ...
\end{verbatim}

\textbf{Note}: Ballistics and Zone Management are covered in detail in later lessons. This lesson focuses on display settings and basic navigation.

\section{3.3 RETICLE \& DISPLAY SETTINGS}

\subsection{3.3.1 PERSONALIZE RETICLE}

\textbf{Access}: Main Menu $\rightarrow$ "Personalize Reticle"

\textbf{Purpose}: Select reticle type (crosshair style) for aiming

\subsubsection{AVAILABLE RETICLE TYPES}

\textbf{1. Box Crosshair} (Default)
\begin{verbatim}
        │
    ┌───┼───┐
────┤   +   ├────
    └───┼───┘
        │
\end{verbatim}
\begin{itemize}
\item Cross with surrounding box
\item Good for general purpose and tracking
\item Recommended for most operations
\end{itemize}

\textbf{2. Brackets Reticle}
\begin{verbatim}
    ┌─     ─┐
        +
    └─     ─┘
\end{verbatim}
\begin{itemize}
\item Corner brackets with center crosshair
\item Enhanced visibility
\item Good for low-contrast targets
\end{itemize}

\textbf{3. Duplex Crosshair}
\begin{verbatim}
  ████│████
──────+──────
  ████│████
\end{verbatim}
\begin{itemize}
\item Thick outer lines, thin center
\item Sniper/precision style
\item Good for long-range
\end{itemize}

\textbf{4. Fine Crosshair}
\begin{verbatim}
      │
    ──+──
      │
\end{verbatim}
\begin{itemize}
\item Thin precision crosshair with range ticks
\item Minimal obstruction
\item Best for extreme precision
\end{itemize}

\textbf{5. Chevron Reticle}
\begin{verbatim}
      ˅
    ──+──
\end{verbatim}
\begin{itemize}
\item Downward pointing chevron
\item CQB (Close Quarters Battle) style
\item Good for rapid engagement
\end{itemize}

\subsubsection{HOW TO CHANGE RETICLE}

\textbf{Procedure}:

\begin{enumerate}
\item Press \textbf{MENU $\checkmark$}
\item Navigate to "Personalize Reticle" (▼)
\item Press \textbf{MENU $\checkmark$} to enter
\item Use ▲/▼ to highlight desired reticle
\item Press \textbf{MENU $\checkmark$} to select
\item \textbf{Reticle changes immediately} on screen
\item Navigate to "Return..." (▼)
\item Press \textbf{MENU $\checkmark$} to return to Main Menu
\end{enumerate}

\textbf{Current Selection}: Active reticle is marked with "$\checkmark$"

\textbf{Recommendation}: Use \textbf{Box Crosshair} for combat, \textbf{Fine Crosshair} for long-range surveillance.

\subsection{3.3.2 PERSONALIZE COLORS}

\textbf{Access}: Main Menu $\rightarrow$ "Personalize Colors"

\textbf{Purpose}: Change OSD color scheme for visibility

\subsubsection{AVAILABLE COLOR THEMES}

% TABLE ROW: | Theme | Primary Color | Use Case |
% TABLE ROW: |-------|---------------|----------|
% TABLE ROW: | **Green** | Bright green (70, 226, 165) | Default - good for day and night |
% TABLE ROW: | **Red** | Red | Night vision compatible |
% TABLE ROW: | **Yellow** | Yellow | High contrast, bright conditions |
% TABLE ROW: | **Cyan** | Cyan | Alternative for user preference |
% TABLE ROW: | **White** | White | Maximum contrast |

\textbf{What Changes}:
\begin{itemize}
\item Reticle color
\item OSD text color
\item Menu text color
\item Tracking box color (when not status-coded)
\end{itemize}

\textbf{What Does NOT Change}:
\begin{itemize}
\item Warning messages (always red or yellow)
\item Tracking status colors (yellow/red/green based on state)
\item CCIP pipper color
\end{itemize}

\subsubsection{HOW TO CHANGE COLOR}

\textbf{Procedure}:

\begin{enumerate}
\item Press \textbf{MENU $\checkmark$}
\item Navigate to "Personalize Colors" (▼)
\item Press \textbf{MENU $\checkmark$} to enter
\item Use ▲/▼ to highlight desired color
\item Press \textbf{MENU $\checkmark$} to select
\item \textbf{OSD changes immediately}
\item Evaluate visibility
\item Change again if needed, or select "Return..."
\end{enumerate}

\textbf{Best Practices}:
\begin{itemize}
\item \textbf{Green}: Good all-around choice (default)
\item \textbf{Red}: Use with night vision equipment
\item \textbf{Yellow}: Use in bright sunlight or snow
\item \textbf{White}: Maximum contrast on dark backgrounds
\end{itemize}

\NOTE{Color selection is personal preference. Choose what you can see best in your operating environment.}

\section{3.4 SYSTEM MENU OPTIONS}

\subsection{3.4.1 ZONE DEFINITIONS}

\textbf{Access}: Main Menu $\rightarrow$ "Zone Definitions"

\textbf{Purpose}: Manage no-fire zones, no-traverse zones, sector scans, and TRPs

\textbf{Detailed Coverage}: See Lesson 4 (Motion Modes \& Surveillance)

\textbf{Quick Access Functions}:
\begin{itemize}
\item View active zones
\item Enable/disable zones
\item Navigate to zone editor (supervisor/commander function)
\end{itemize}

\textbf{Operator Note}: Zone modification usually requires supervisor authorization. Operators can VIEW zones but typically cannot CHANGE them.

\subsection{3.4.2 SYSTEM STATUS}

\textbf{Access}: Main Menu $\rightarrow$ "System Status"

\textbf{Purpose}: View detailed system health and diagnostics

\textbf{Detailed Coverage}: See Lesson 7 (System Status \& Monitoring)

\textbf{Quick Preview}:

Displays status of all subsystems:
\begin{itemize}
\item Cameras (Day/Night) - Connected/Disconnected
\item Servos (Azimuth/Elevation) - Position, temperature, faults
\item Laser Rangefinder - Status, temperature
\item Joystick - Connected/Calibration
\item Stabilization System - Active/Inactive
\item Tracking System - State
\item Weapon Actuator - Position, status
\end{itemize}

\textbf{When to Check}:
\begin{itemize}
\item At startup (verify all green)
\item When Fault light illuminates
\item Before critical operations
\item During troubleshooting
\end{itemize}

\subsection{3.4.3 SHUTDOWN SYSTEM}

\textbf{Access}: Main Menu $\rightarrow$ "Shutdown System"

\textbf{Purpose}: Perform orderly software shutdown before powering off

\textbf{Why Use Menu Shutdown}:
\begin{itemize}
\item Saves configuration settings
\item Saves zone data
\item Closes log files properly
\item Prevents data corruption
\item Powers down subsystems in correct sequence
\end{itemize}

\textbf{Procedure}:

\begin{enumerate}
\item Press \textbf{MENU $\checkmark$}
\item Navigate to "Shutdown System" (▼ multiple times)
\item Press \textbf{MENU $\checkmark$}
\item \textbf{Confirmation prompt appears}:
\begin{verbatim}
   ┌────────────────────────────────┐
   │      SHUTDOWN SYSTEM?          │
   │                                │
   │   This will power down the     │
   │   RCWS safely.                 │
   │                                │
   │   > YES, Shutdown              │
   │     NO, Cancel                 │
   └────────────────────────────────┘
\end{verbatim}
\item Select "YES, Shutdown" (should be highlighted)
\item Press \textbf{MENU $\checkmark$} to confirm
\item \textbf{System shuts down}:
\end{enumerate}
\begin{itemize}
\item "SHUTTING DOWN..." message appears
\item Progress indicator shows shutdown steps
\item Cameras power off
\item Motors de-energize
\item "SHUTDOWN COMPLETE - Safe to power off" message
\end{itemize}
\begin{enumerate}
\item \textbf{Disable Station Enable} switch on DCU
\item \textbf{Cut vehicle power} (if end of shift)
\end{enumerate}

\textbf{ IMPORTANT}: Wait for \osd{SHUTDOWN COMPLETE} message before cutting power. Interrupting shutdown can corrupt configuration files.

\subsection{3.4.4 ABOUT / INFO}

\textbf{Access}: Main Menu $\rightarrow$ "About"

\textbf{Purpose}: Display system information for troubleshooting and support

\textbf{Information Displayed}:
\begin{itemize}
\item System name: "El 7arress RCWS"
\item Software version (e.g., "v4.5.2")
\item Build date
\item Serial number (if configured)
\item Uptime (hours since power-on)
\item Operator name (if logged in)
\end{itemize}

\textbf{Use Case}: Provide this information when reporting issues to maintenance.

\section{3.5 BALLISTICS MENU (OVERVIEW)}

\textbf{Access}: Main Menu $\rightarrow$ "--- BALLISTICS ---" section

The ballistics menu provides access to fire control settings. These are covered in detail in \textbf{Lesson 6} but are introduced here for awareness.

\subsection{BALLISTICS SUBMENU OPTIONS}

\subsubsection{1. Zeroing}

\begin{itemize}
\item Align weapon point of impact with camera crosshair
\item Adjust azimuth and elevation offsets
\item Save/load zeroing profiles
\item \textbf{Detailed in Lesson 6.1}
\end{itemize}

\subsubsection{2. Environmental Parameters}

\begin{itemize}
\item Set temperature (\degree{}C)
\item Set altitude (meters above sea level)
\item Set crosswind speed and direction
\item Apply environmental corrections to ballistics
\item \textbf{Detailed in Lesson 6.2}
\end{itemize}

\subsubsection{3. Lead Angle Compensation}

\begin{itemize}
\item Enable/disable lead angle for moving targets
\item View lead angle status (Off/On/Lag/ZoomOut)
\item Automatically calculates lead based on target velocity
\item \textbf{Detailed in Lesson 6.3}
\end{itemize}

\textbf{ OPERATOR NOTE}: Do not modify ballistics settings unless trained. Incorrect settings can cause missed shots or dangerous ricochets. Zeroing and environmental settings are usually performed by designated personnel (e.g., gunner, vehicle commander).

\section{3.6 MENU QUICK REFERENCE}

\subsection{COMMON MENU TASKS}

\subsubsection{Task 1: Change Reticle}

\begin{verbatim}
MENU ✓ → "Personalize Reticle" → MENU ✓
→ Select reticle (▲/▼) → MENU ✓
→ "Return..." → MENU ✓
\end{verbatim}
\textbf{Time}: \textasciitilde{}10 seconds

\subsubsection{Task 2: Change Color Scheme}

\begin{verbatim}
MENU ✓ → "Personalize Colors" → MENU ✓
→ Select color (▲/▼) → MENU ✓
→ "Return..." → MENU ✓
\end{verbatim}
\textbf{Time}: \textasciitilde{}10 seconds

\subsubsection{Task 3: Check System Status}

\begin{verbatim}
MENU ✓ → "System Status" → MENU ✓
→ Review status → "Return..." → MENU ✓
\end{verbatim}
\textbf{Time}: \textasciitilde{}15 seconds (plus review time)

\subsubsection{Task 4: Shutdown via Menu}

\begin{verbatim}
MENU ✓ → "Shutdown System" → MENU ✓
→ "YES, Shutdown" → MENU ✓
→ Wait for "SHUTDOWN COMPLETE"
→ Disable Station Enable → Cut power
\end{verbatim}
\textbf{Time}: \textasciitilde{}45 seconds

\subsection{MENU NAVIGATION TIPS}

\begin{enumerate}
\item \textbf{Muscle Memory}: Practice menu navigation until you can do it without looking at button labels
\end{enumerate}

\begin{enumerate}
\item \textbf{HAT Switch Alternative}: Use joystick HAT switch if your hands are already on the joystick
\end{enumerate}

\begin{enumerate}
\item \textbf{Quick Exit}: If lost in menus, repeatedly press \textbf{MENU $\checkmark$} on section headers to back out quickly
\end{enumerate}

\begin{enumerate}
\item \textbf{Video Still Visible}: Menu is semi-transparent or overlaid - you can still monitor situation while in menu
\end{enumerate}

\begin{enumerate}
\item \textbf{Menu Timeout}: Some menus auto-exit after 60 seconds of inactivity (returns to operational screen)
\end{enumerate}

\begin{enumerate}
\item \textbf{Combat Discipline}: Minimize menu time during operations. Configure settings during planning/prep, not during engagement.
\end{enumerate}

\subsection{MENU TROUBLESHOOTING}

% TABLE ROW: | Problem | Possible Cause | Solution |
% TABLE ROW: |---------|---------------|----------|
% TABLE ROW: | Menu won't open | Button stuck or system fault | Try joystick HAT switch, or restart system |
% TABLE ROW: | Can't select option | On section header | Use ▲/▼ to move to selectable item |
% TABLE ROW: | Menu frozen | Software hang | Press Emergency Stop, restart system |
% TABLE ROW: | Settings don't save | Shutdown without menu | Always use "Shutdown System" menu before power-off |
% TABLE ROW: | Menu text unreadable | Color scheme issue | Change to White or Yellow color theme |

\section{3.7 MENU BEST PRACTICES}

\subsection{WHEN TO USE MENUS}

$\checkmark$ \textbf{DO use menus for}:
\begin{itemize}
\item Changing display preferences (reticle, color)
\item Checking system status
\item Reviewing zone definitions
\item Configuring ballistics (when trained)
\item Orderly system shutdown
\end{itemize}

$\times$ \textbf{DO NOT use menus during}:
\begin{itemize}
\item Active engagement
\item Emergency situations
\item When gimbal must be controlled continuously
\item Under time pressure
\end{itemize}

\textbf{Rule of Thumb}: Menus are for setup and configuration, not combat operations.

\subsection{SETTINGS THAT PERSIST}

\textbf{Saved Between Power Cycles} (stored in configuration):
\begin{itemize}
\item Reticle type selection
\item Color scheme
\item Zeroing offsets (if saved)
\item Environmental parameters (if saved)
\item Zone definitions
\end{itemize}

\textbf{NOT Saved} (reset on power-up):
\begin{itemize}
\item Gimbal position (returns to home)
\item Active tracking (aborted)
\item Temporary warnings
\item Menu navigation position
\end{itemize}

\subsection{OPERATOR VS. SUPERVISOR FUNCTIONS}

\textbf{Operator Can}:
\begin{itemize}
\item Change reticle and colors
\item View system status
\item View zones
\item Access ballistics menus (view)
\item Shutdown system
\end{itemize}

\textbf{Operator Usually CANNOT} (requires authorization):
\begin{itemize}
\item Modify zone boundaries
\item Override no-fire zones
\item Change ballistics profiles (depends on unit SOP)
\item Access maintenance menus
\item Modify system configuration files
\end{itemize}

\textbf{Consult your unit SOP for specific authorization levels.}
