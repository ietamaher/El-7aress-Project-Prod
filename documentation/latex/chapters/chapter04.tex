\# LESSON 4: MOTION MODES \& SURVEILLANCE

\textbf{Lesson Duration:} 3 hours (Classroom + Practical)

\textbf{Learning Objectives:}
\begin{itemize}
\item Explain the purpose of each motion mode
\item Switch between motion modes safely
\item Operate automatic sector scan mode
\item Utilize Target Reference Point (TRP) scan mode
\item Define and manage no-fire zones
\item Define and manage no-traverse zones
\item Save and load zone configurations
\end{itemize}



\section{4.1 MOTION MODES OVERVIEW}

\subsection{WHAT ARE MOTION MODES?}

Motion modes control \textbf{how the gimbal moves} during operations:

\begin{itemize}
\item \textbf{Manual Mode}: You control gimbal directly with joystick
\item \textbf{Auto Sector Scan}: System automatically scans between two points
\item \textbf{TRP Scan}: System sequentially visits pre-defined Target Reference Points
\item \textbf{Radar Slew} (if equipped): Gimbal follows radar detections
\end{itemize}

\textbf{Purpose}: Different missions require different surveillance patterns. Motion modes let you switch between direct control and automated surveillance.



\subsection{MODE SELECTION}

\textbf{How to Change Modes}:
\begin{itemize}
\item Press joystick \textbf{\button{Button 11}} or \textbf{\button{Button 13}} (either button cycles modes)
\item Modes cycle in sequence:
\begin{verbatim}
  Manual → AutoSectorScan → TRP Scan → Radar Slew → Manual (loops)
\end{verbatim}
\end{itemize}

\textbf{Current Mode Display}:
\begin{itemize}
\item OSD bottom center shows: \textbf{MODE: Manual}, \textbf{MODE: Sector Scan}, \textbf{MODE: TRP}, or \textbf{MODE: Radar}
\end{itemize}

\textbf{ RESTRICTION}: Cannot change modes during tracking acquisition. Must abort tracking first (press TRK button).



\section{4.2 MANUAL MODE}

\textbf{Description}: Direct operator control via joystick (default mode)

\textbf{When to Use}:
\begin{itemize}
\item Direct target engagement
\item Precise aiming
\item Immediate threat response
\item Search operations requiring operator judgment
\end{itemize}

\textbf{Operation}:
\begin{itemize}
\item Joystick LEFT/RIGHT $\rightarrow$ Azimuth control
\item Joystick UP/DOWN $\rightarrow$ Elevation control
\item Stick deflection = gimbal speed
\item Center stick = gimbal stops
\end{itemize}

\textbf{Already Covered}: See Lesson 2, Section 2.3 for detailed joystick control

\textbf{Transitioning Out of Manual}:
\begin{itemize}
\item Press \button{Button 11} or 13
\item Mode advances to next (AutoSectorScan)
\item Gimbal behavior changes immediately
\end{itemize}



\section{4.3 AUTO SECTOR SCAN MODE}

\subsection{WHAT IS SECTOR SCAN?}

\textbf{Definition}: Automated gimbal movement that continuously scans between two pre-defined points (left and right limits)

\textbf{Purpose}:
\begin{itemize}
\item Perimeter surveillance
\item Monitoring a defined sector without operator input
\item Frees operator to monitor other systems or threats
\end{itemize}

\textbf{Visual}:
\begin{verbatim}
    Left Limit              Right Limit
        ↓                       ↓
        ●←←←←←←←scan←←←←←←←●
        ●→→→→→→→scan→→→→→→→●
             (repeats)
\end{verbatim}

Gimbal continuously scans left-to-right, then right-to-left, repeat.



\subsection{ACTIVATING SECTOR SCAN}

\textbf{Prerequisites}:
\begin{enumerate}
\item Sector scan zone must be defined (see Section 4.7)
\item At least one sector scan zone exists
\item System is in AutoSectorScan mode
\end{enumerate}

\textbf{Procedure}:

\begin{enumerate}
\item \textbf{Cycle to Sector Scan Mode}:
\end{enumerate}
\begin{itemize}
\item Press \button{Button 11} or 13 until OSD shows: \textbf{MODE: Sector Scan}
\end{itemize}

\begin{enumerate}
\item \textbf{System Behavior}:
\end{enumerate}
\begin{itemize}
\item Gimbal automatically slews to \textbf{left limit} of active sector
\item Scans to \textbf{right limit} (slow, smooth movement)
\item Reverses, scans back to \textbf{left limit}
\item Repeats continuously
\end{itemize}

\begin{enumerate}
\item \textbf{Scan Speed}:
\end{enumerate}
\begin{itemize}
\item Default: \textasciitilde{}5\degree{}/second (configurable in zone definition)
\item Allows visual inspection as sector scans
\end{itemize}

\begin{enumerate}
\item \textbf{Elevation}:
\end{enumerate}
\begin{itemize}
\item Scans at elevation defined in zone (usually 0\degree{} horizon)
\item Elevation can be fixed or variable (depending on zone setup)
\end{itemize}



\subsection{WHILE SECTOR SCANNING}

\textbf{Operator Actions}:

\textbf{$\checkmark$ CAN DO}:
\begin{itemize}
\item Monitor video feed as sector scans
\item Switch cameras (CAM button)
\item Zoom in/out (Zoom rocker)
\item Fire LRF (LRF button) at targets of interest
\item Initiate tracking (TRK button) - \textit{aborts sector scan, switches to Manual}
\end{itemize}

\textbf{$\times$ CANNOT DO}:
\begin{itemize}
\item Joystick axes do NOT control gimbal (ignored)
\item Cannot manually slew gimbal (must exit mode first)
\end{itemize}

\textbf{To Override}:
\begin{itemize}
\item Press \button{Button 11}/13 to return to Manual mode
\item OR: Press TRK to start tracking (auto-switches to Manual)
\end{itemize}



\subsection{PAUSING / RESUMING SCAN}

\textbf{Not Applicable}: Sector scan runs continuously. To stop:
\begin{itemize}
\item Cycle back to Manual mode (\button{Button 11}/13)
\end{itemize}

\textbf{Tip}: If you need to temporarily examine something in the sector, use \textbf{Manual} mode to stop, then return to \textbf{AutoSectorScan} when ready.



\subsection{MULTIPLE SECTOR ZONES}

\textbf{If Multiple Sectors Defined}:
\begin{itemize}
\item System scans \textbf{active sector} (last selected via menu)
\item To change active sector:
\end{itemize}
\begin{enumerate}
\item Cycle to Manual mode
\item Access menu: Zone Definitions $\rightarrow$ Sector Scans
\item Select desired sector
\item Return to AutoSectorScan mode
\end{enumerate}



\section{4.4 TRP SCAN MODE}

\subsection{WHAT IS TRP SCAN?}

\textbf{TRP} = \textbf{Target Reference Point} (pre-defined location of interest)

\textbf{Definition}: System sequentially slews to each TRP, dwells (pauses) for observation, then moves to next TRP in list.

\textbf{Purpose}:
\begin{itemize}
\item Checkpoint verification (e.g., gate 1, gate 2, gate 3)
\item Known threat areas (e.g., sniper hide sites)
\item Periodic scans of fixed locations
\item More efficient than manual searching
\end{itemize}

\textbf{Visual}:
\begin{verbatim}
TRP 1 (30s dwell) → TRP 2 (30s dwell) → TRP 3 (30s dwell) → TRP 1 (repeat)
\end{verbatim}



\subsection{ACTIVATING TRP SCAN}

\textbf{Prerequisites}:
\begin{enumerate}
\item At least one TRP must be defined (see Section 4.8)
\item System is in TRP Scan mode
\end{enumerate}

\textbf{Procedure}:

\begin{enumerate}
\item \textbf{Cycle to TRP Scan Mode}:
\end{enumerate}
\begin{itemize}
\item Press \button{Button 11} or 13 until OSD shows: \textbf{MODE: TRP}
\end{itemize}

\begin{enumerate}
\item \textbf{System Behavior}:
\end{enumerate}
\begin{itemize}
\item Gimbal slews to \textbf{first TRP} in list
\item \textbf{Dwells} for configured time (default: 30 seconds)
\item Slews to \textbf{next TRP}
\item Dwells again
\item Repeats through entire TRP list, then loops back to first TRP
\end{itemize}

\begin{enumerate}
\item \textbf{Dwell Time}:
\end{enumerate}
\begin{itemize}
\item Configurable per TRP (5-120 seconds)
\item Allows operator to observe location before moving on
\end{itemize}



\subsection{WHILE TRP SCANNING}

\textbf{Operator Actions}:

\textbf{During Dwell} (gimbal stationary at TRP):
\begin{itemize}
\item $\checkmark$ Observe video
\item $\checkmark$ Switch cameras
\item $\checkmark$ Zoom
\item $\checkmark$ Fire LRF
\item $\checkmark$ Initiate tracking (if threat detected)
\end{itemize}

\textbf{During Slew} (gimbal moving between TRPs):
\begin{itemize}
\item Gimbal is in motion
\item Video may be blurred (fast slew)
\item Wait for next dwell to observe
\end{itemize}

\textbf{To Skip to Next TRP}:
\begin{itemize}
\item \textbf{Not directly available} - must wait for dwell to complete, OR
\item Cycle to Manual, reposition, cycle back to TRP mode (restarts from TRP 1)
\end{itemize}



\subsection{TRP SCAN BEST PRACTICES}

\begin{enumerate}
\item \textbf{Define TRPs During Planning}:
\end{enumerate}
\begin{itemize}
\item Pre-mission: Identify key locations
\item Enter TRP coordinates via menu (Section 4.8)
\item Test TRP scan before mission start
\end{itemize}

\begin{enumerate}
\item \textbf{Prioritize TRPs}:
\end{enumerate}
\begin{itemize}
\item Place highest-threat areas first in list
\item System visits TRPs in order defined
\end{itemize}

\begin{enumerate}
\item \textbf{Appropriate Dwell Times}:
\end{enumerate}
\begin{itemize}
\item Short dwell (10-15s) for quick checks
\item Long dwell (60s+) for detailed observation or known threat areas
\end{itemize}

\begin{enumerate}
\item \textbf{Combine with Manual}:
\end{enumerate}
\begin{itemize}
\item Use TRP scan for routine surveillance
\item Switch to Manual when threat detected
\end{itemize}



\section{4.5 RADAR SLEW MODE (OPTIONAL)}

\textbf{Availability}: Only if radar system is integrated

\textbf{Description}: Gimbal automatically slews to radar-detected targets

\textbf{Purpose}:
\begin{itemize}
\item Rapid threat response
\item Automated cueing from radar
\item Reduces operator workload
\end{itemize}

\textbf{Operation}:

\begin{enumerate}
\item \textbf{Cycle to Radar Slew Mode}:
\end{enumerate}
\begin{itemize}
\item Press \button{Button 11}/13 until OSD shows: \textbf{MODE: Radar}
\end{itemize}

\begin{enumerate}
\item \textbf{System Behavior}:
\end{enumerate}
\begin{itemize}
\item Waits for radar detection
\item When radar detects target $\rightarrow$ Gimbal slews to radar coordinates
\item OSD displays: \textbf{RADAR CUE} or \textbf{SLEWING TO RADAR}
\item Operator confirms threat visually on video
\end{itemize}

\begin{enumerate}
\item \textbf{Operator Decision}:
\end{enumerate}
\begin{itemize}
\item If threat confirmed $\rightarrow$ Initiate tracking (TRK button) or engage
\item If false alarm $\rightarrow$ Wait for next radar cue, OR cycle to Manual
\end{itemize}

\NOTE{Most El 7arress RCWS systems do NOT have radar. This mode will show "RADAR NOT AVAILABLE" if no radar connected.}



\section{4.6 MOTION MODE QUICK REFERENCE}

% TABLE ROW: | Mode | Use Case | Gimbal Control | Exits to Manual When... |
% TABLE ROW: |------|----------|----------------|-------------------------|
% TABLE ROW: | **Manual** | Direct engagement, search | Joystick | N/A (already manual) |
% TABLE ROW: | **AutoSectorScan** | Perimeter surveillance | Automatic (between limits) | Cycle mode or press TRK |
% TABLE ROW: | **TRP Scan** | Checkpoint monitoring | Automatic (TRP sequence) | Cycle mode or press TRK |
% TABLE ROW: | **Radar Slew** | Radar integration | Automatic (radar cues) | Cycle mode or press TRK |

\textbf{Emergency Return to Manual}: Press \button{Button 11}/13 repeatedly until \textbf{MODE: Manual} displays



\section{4.7 ZONE MANAGEMENT - SECTOR SCANS}

\subsection{DEFINING SECTOR SCAN ZONES}

Sector scan zones set the \textbf{left and right limits} for AutoSectorScan mode.

\textbf{Access}: Main Menu $\rightarrow$ Zone Definitions $\rightarrow$ Sector Scans



\subsubsection{SECTOR SCAN ZONE PARAMETERS}

Each sector scan zone has:

% TABLE ROW: | Parameter | Description | Typical Value |
% TABLE ROW: |-----------|-------------|---------------|
% TABLE ROW: | **Name** | Zone identifier | "Front Gate", "Perimeter East" |
% TABLE ROW: | **Left Limit (Az)** | Starting azimuth | 045° |
% TABLE ROW: | **Right Limit (Az)** | Ending azimuth | 135° |
% TABLE ROW: | **Elevation** | Scan elevation angle | 0° (horizon) |
% TABLE ROW: | **Scan Speed** | Degrees per second | 5°/sec |
% TABLE ROW: | **Active** | Enable/disable zone | ON / OFF |



\subsubsection{CREATING A SECTOR SCAN ZONE (STEP-BY-STEP)}

\textbf{Method 1: Manual Entry} (via menu)

\begin{enumerate}
\item \textbf{Access Menu}:
\end{enumerate}
\begin{itemize}
\item MENU $\checkmark$ $\rightarrow$ Zone Definitions $\rightarrow$ Sector Scans $\rightarrow$ Add New Sector
\end{itemize}

\begin{enumerate}
\item \textbf{Enter Name}:
\end{enumerate}
\begin{itemize}
\item Use ▲/▼ to enter name characters
\item MENU $\checkmark$ to confirm
\end{itemize}

\begin{enumerate}
\item \textbf{Set Left Limit}:
\end{enumerate}
\begin{itemize}
\item Method A: Slew gimbal to desired left position, press MENU $\checkmark$ to "Capture Current Position"
\item Method B: Manually enter azimuth value using ▲/▼
\end{itemize}

\begin{enumerate}
\item \textbf{Set Right Limit}:
\end{enumerate}
\begin{itemize}
\item Same as left limit (capture or manual entry)
\end{itemize}

\begin{enumerate}
\item \textbf{Set Elevation}:
\end{enumerate}
\begin{itemize}
\item Enter elevation angle (typically 0\degree{} for horizon scan)
\end{itemize}

\begin{enumerate}
\item \textbf{Set Scan Speed}:
\end{enumerate}
\begin{itemize}
\item Enter degrees/second (5\degree{}/sec recommended)
\end{itemize}

\begin{enumerate}
\item \textbf{Enable Zone}:
\end{enumerate}
\begin{itemize}
\item Set Active = ON
\end{itemize}

\begin{enumerate}
\item \textbf{Save}:
\end{enumerate}
\begin{itemize}
\item MENU $\checkmark$ $\rightarrow$ "Save Sector Scan"
\item Confirmation: \osd{SECTOR SCAN SAVED}
\end{itemize}



\textbf{Method 2: Quick Capture} (using joystick)

\begin{enumerate}
\item \textbf{Position Gimbal}:
\end{enumerate}
\begin{itemize}
\item Use Manual mode to slew to desired \textbf{left limit}
\item Press \textbf{FN Button} + Hold for 2 seconds
\item OSD displays: \osd{LEFT LIMIT CAPTURED}
\end{itemize}

\begin{enumerate}
\item \textbf{Position Right Limit}:
\end{enumerate}
\begin{itemize}
\item Slew to desired \textbf{right limit}
\item Press \textbf{FN Button} + Hold for 2 seconds
\item OSD displays: "RIGHT LIMIT CAPTURED, SECTOR SCAN ZONE CREATED"
\end{itemize}

\begin{enumerate}
\item \textbf{System Auto-Creates Zone}:
\end{enumerate}
\begin{itemize}
\item Default name: "Sector X" (where X = next available number)
\item Default elevation: Current elevation
\item Default speed: 5\degree{}/sec
\end{itemize}

\begin{enumerate}
\item \textbf{Edit if Needed}:
\end{enumerate}
\begin{itemize}
\item Access menu to rename or adjust parameters
\end{itemize}



\subsection{ACTIVATING / DEACTIVATING SECTOR ZONES}

\textbf{Multiple Zones}:
\begin{itemize}
\item You can define multiple sector scan zones
\item Only ONE can be active at a time
\end{itemize}

\textbf{To Activate a Zone}:
\begin{enumerate}
\item MENU $\checkmark$ $\rightarrow$ Zone Definitions $\rightarrow$ Sector Scans
\item Use ▲/▼ to select desired zone
\item MENU $\checkmark$ $\rightarrow$ "Set as Active"
\item OSD displays: "SECTOR ZONE [Name] ACTIVE"
\end{enumerate}

\textbf{To Deactivate All Sectors}:
\begin{enumerate}
\item MENU $\checkmark$ $\rightarrow$ Zone Definitions $\rightarrow$ Sector Scans
\item Select active zone
\item MENU $\checkmark$ $\rightarrow$ "Deactivate"
\end{enumerate}



\section{4.8 ZONE MANAGEMENT - TARGET REFERENCE POINTS (TRPs)}

\subsection{DEFINING TRPs}

TRPs are \textbf{fixed locations} the system can automatically slew to.

\textbf{Access}: Main Menu $\rightarrow$ Zone Definitions $\rightarrow$ TRPs



\subsubsection{TRP PARAMETERS}

Each TRP has:

% TABLE ROW: | Parameter | Description | Typical Value |
% TABLE ROW: |-----------|-------------|---------------|
% TABLE ROW: | **Name** | TRP identifier | "Gate 1", "Bunker", "Hill 203" |
% TABLE ROW: | **Azimuth** | Direction to TRP | 090° |
% TABLE ROW: | **Elevation** | Angle to TRP | +5° |
% TABLE ROW: | **Dwell Time** | Observation time | 30 seconds |
% TABLE ROW: | **Active** | Enable/disable | ON / OFF |



\subsubsection{CREATING A TRP (STEP-BY-STEP)}

\textbf{Method 1: Capture Current Position}

\begin{enumerate}
\item \textbf{Manual Mode}:
\end{enumerate}
\begin{itemize}
\item Slew gimbal to desired TRP location using joystick
\item Zoom/focus on exact point
\end{itemize}

\begin{enumerate}
\item \textbf{Access Menu}:
\end{enumerate}
\begin{itemize}
\item MENU $\checkmark$ $\rightarrow$ Zone Definitions $\rightarrow$ TRPs $\rightarrow$ Add TRP
\end{itemize}

\begin{enumerate}
\item \textbf{Capture Position}:
\end{enumerate}
\begin{itemize}
\item Select "Capture Current Position"
\item MENU $\checkmark$ to confirm
\item System records current Az/El
\end{itemize}

\begin{enumerate}
\item \textbf{Enter Name}:
\end{enumerate}
\begin{itemize}
\item Use ▲/▼ to enter TRP name
\item MENU $\checkmark$ to confirm
\end{itemize}

\begin{enumerate}
\item \textbf{Set Dwell Time}:
\end{enumerate}
\begin{itemize}
\item Enter seconds (5-120)
\item Default: 30 seconds
\end{itemize}

\begin{enumerate}
\item \textbf{Enable}:
\end{enumerate}
\begin{itemize}
\item Set Active = ON
\end{itemize}

\begin{enumerate}
\item \textbf{Save}:
\end{enumerate}
\begin{itemize}
\item MENU $\checkmark$ $\rightarrow$ "Save TRP"
\item Confirmation: "TRP [Name] SAVED"
\end{itemize}



\textbf{Method 2: Manual Coordinate Entry}

\begin{enumerate}
\item \textbf{Access Menu}:
\end{enumerate}
\begin{itemize}
\item MENU $\checkmark$ $\rightarrow$ Zone Definitions $\rightarrow$ TRPs $\rightarrow$ Add TRP
\end{itemize}

\begin{enumerate}
\item \textbf{Enter Azimuth}:
\end{enumerate}
\begin{itemize}
\item Use ▲/▼ to enter degrees (000-359)
\end{itemize}

\begin{enumerate}
\item \textbf{Enter Elevation}:
\end{enumerate}
\begin{itemize}
\item Use ▲/▼ to enter degrees (-20 to +60)
\end{itemize}

\begin{enumerate}
\item \textbf{Continue} with name, dwell time, save (same as Method 1)
\end{enumerate}



\subsection{MANAGING TRP LIST}

\textbf{TRP Sequence}:
\begin{itemize}
\item TRPs are visited in the order they appear in list
\item To reorder:
\end{itemize}
\begin{enumerate}
\item MENU $\checkmark$ $\rightarrow$ Zone Definitions $\rightarrow$ TRPs
\item Select TRP
\item "Move Up" or "Move Down"
\end{enumerate}

\textbf{Editing TRPs}:
\begin{itemize}
\item Select TRP from list
\item MENU $\checkmark$ $\rightarrow$ "Edit TRP"
\item Modify parameters
\item Save changes
\end{itemize}

\textbf{Deleting TRPs}:
\begin{itemize}
\item Select TRP from list
\item MENU $\checkmark$ $\rightarrow$ "Delete TRP"
\item Confirm deletion
\end{itemize}



\section{4.9 ZONE MANAGEMENT - NO-FIRE \& NO-TRAVERSE ZONES}

(Covered conceptually in Lesson 1, Section 1.5 - this section covers \textbf{viewing} zones via menu)

\subsection{VIEWING NO-FIRE ZONES}

\textbf{Access}: Main Menu $\rightarrow$ Zone Definitions $\rightarrow$ No-Fire Zones

\textbf{Display}:
\begin{itemize}
\item List of all defined no-fire zones
\item Each zone shows:
\item Name (e.g., "Friendly FOB", "Civilian Area 1")
\item Boundary type (Polygon, Circle, Arc)
\item Active status (ON/OFF)
\end{itemize}

\textbf{Operator Permission}:
\begin{itemize}
\item \textbf{CAN}: View zones, see boundaries on map overlay (if available)
\item \textbf{CANNOT}: Modify boundaries, delete zones, override zones
\end{itemize}

\textbf{Modification}: Usually requires commander/supervisor authorization



\subsection{VIEWING NO-TRAVERSE ZONES}

\textbf{Access}: Main Menu $\rightarrow$ Zone Definitions $\rightarrow$ No-Traverse Zones

\textbf{Display}:
\begin{itemize}
\item List of all defined no-traverse zones
\item Each zone shows:
\item Name (e.g., "Rear 90\degree{}", "Antenna Area")
\item Azimuth limits
\item Active status
\end{itemize}

\textbf{Purpose Reminder}:
\begin{itemize}
\item No-traverse zones prevent gimbal movement into restricted areas
\item Protects vehicle structure, equipment, personnel
\end{itemize}



\section{4.10 SAVING \& LOADING ZONE CONFIGURATIONS}

\subsection{SAVING ZONE CONFIGURATION}

\textbf{Purpose}: Save all zones (sectors, TRPs, no-fire, no-traverse) to file for later use

\textbf{Procedure}:

\begin{enumerate}
\item \textbf{Access Menu}:
\end{enumerate}
\begin{itemize}
\item MENU $\checkmark$ $\rightarrow$ Zone Definitions $\rightarrow$ Save/Load $\rightarrow$ Save Configuration
\end{itemize}

\begin{enumerate}
\item \textbf{Enter Filename}:
\end{enumerate}
\begin{itemize}
\item Use ▲/▼ to enter filename
\item Example: "MISSION\_20250115", "PERIMETER\_CONFIG"
\end{itemize}

\begin{enumerate}
\item \textbf{Confirm Save}:
\end{enumerate}
\begin{itemize}
\item MENU $\checkmark$ $\rightarrow$ "Save"
\item OSD displays: \osd{ZONE CONFIG SAVED}
\end{itemize}

\textbf{File Location}: Saved to internal storage (typically /configs/zones/)



\subsection{LOADING ZONE CONFIGURATION}

\textbf{Purpose}: Load previously saved zone configuration

\textbf{Procedure}:

\begin{enumerate}
\item \textbf{Access Menu}:
\end{enumerate}
\begin{itemize}
\item MENU $\checkmark$ $\rightarrow$ Zone Definitions $\rightarrow$ Save/Load $\rightarrow$ Load Configuration
\end{itemize}

\begin{enumerate}
\item \textbf{Select File}:
\end{enumerate}
\begin{itemize}
\item Use ▲/▼ to browse saved configurations
\item Displays: Filename, date saved, number of zones
\end{itemize}

\begin{enumerate}
\item \textbf{Confirm Load}:
\end{enumerate}
\begin{itemize}
\item MENU $\checkmark$ $\rightarrow$ "Load"
\item OSD displays: \osd{ZONE CONFIG LOADED}
\end{itemize}

\WARNING{Loading a configuration **overwrites** current zones. Save current zones first if needed.}



\subsection{DEFAULT ZONE CONFIGURATION}

\textbf{Default Zones}:
\begin{itemize}
\item System ships with default no-traverse zones (vehicle-specific)
\item Default no-fire zones may be empty (mission-dependent)
\end{itemize}

\textbf{Restoring Defaults}:
\begin{enumerate}
\item MENU $\checkmark$ $\rightarrow$ Zone Definitions $\rightarrow$ Save/Load $\rightarrow$ Restore Defaults
\item Confirm: "RESTORE DEFAULT ZONES?"
\item MENU $\checkmark$ $\rightarrow$ \osd{YES}
\end{enumerate}

\textbf{Use Case}: Reset after training or if zones become corrupted



\section{4.11 SURVEILLANCE BEST PRACTICES}

\subsection{CHOOSING THE RIGHT MODE}

% TABLE ROW: | Situation | Recommended Mode | Rationale |
% TABLE ROW: |-----------|------------------|-----------|
% TABLE ROW: | Direct threat engagement | Manual | Full operator control, immediate response |
% TABLE ROW: | Perimeter watch (quiet) | AutoSectorScan | Automated, frees attention, consistent coverage |
% TABLE ROW: | Checkpoint routine | TRP Scan | Efficient for fixed locations, repeatable |
% TABLE ROW: | High-threat area scan | Manual | Requires operator judgment, unpredictable threats |
% TABLE ROW: | Radar-integrated ops | Radar Slew (if available) | Rapid response to radar cues |



\subsection{COMBINING MODES WITH TRACKING}

\textbf{Workflow Example}:

\begin{enumerate}
\item \textbf{Start in AutoSectorScan} (perimeter surveillance)
\item \textbf{Threat detected} during scan
\item \textbf{Press TRK} $\rightarrow$ System switches to Manual, starts tracking acquisition
\item \textbf{Lock onto threat} (second TRK press)
\item \textbf{Engage or monitor} as threat tracked
\item \textbf{Abort tracking} (third TRK press)
\item \textbf{Resume surveillance}: Press \button{Button 11}/13 to return to AutoSectorScan
\end{enumerate}



\subsection{ZONE DISCIPLINE}

\textbf{Before Mission}:
\begin{itemize}
\item $\checkmark$ Load appropriate zone configuration
\item $\checkmark$ Verify no-fire zones match current ROE
\item $\checkmark$ Test sector scans and TRPs
\item $\checkmark$ Brief all operators on zones
\end{itemize}

\textbf{During Mission}:
\begin{itemize}
\item $\checkmark$ Respect all zone warnings
\item $\checkmark$ Never attempt to override no-fire zones without authorization
\item $\checkmark$ Report zone boundary errors to command
\item $\checkmark$ Update TRPs as mission evolves (if authorized)
\end{itemize}

\textbf{After Mission}:
\begin{itemize}
\item $\checkmark$ Save zone configuration if modified
\item $\checkmark$ Debrief on zone effectiveness
\item $\checkmark$ Recommend adjustments for future missions
\end{itemize}



\textbf{END OF LESSON 4}



