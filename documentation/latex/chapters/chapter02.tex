% BASIC OPERATION

\noindent
\textbf{Lesson Duration:} 4 hours (Classroom + Practical)

\vspace{0.3cm}

\noindent
\textbf{Learning Objectives:}

Upon completion of this lesson, you will be able to:

\begin{itemize}[leftmargin=2em,itemsep=3pt]
\item Perform complete system startup procedure
\item Operate all DCU buttons, switches, and controls
\item Control gimbal movement using joystick
\item Switch between day and thermal cameras
\item Operate camera zoom controls
\item Interpret all OSD elements correctly
\item Perform normal system shutdown
\end{itemize}

\vspace{0.5cm}


\section{2.1 SYSTEM STARTUP PROCEDURE}

\subsection{PRE-STARTUP CHECKLIST}

Before powering on the system, verify:

\begin{enumerate}
\item $\checkmark$ \textbf{Walk-around inspection complete} (Lesson 1) - all items GO
\item $\checkmark$ \textbf{Weapon cleared} per Appendix A
\item $\checkmark$ \textbf{Ammunition removed} or accounted for
\item $\checkmark$ \textbf{All personnel clear} of turret (minimum 2 meters)
\item $\checkmark$ \textbf{Vehicle power available} (check voltage: 20-30V DC nominal)
\item $\checkmark$ \textbf{Operator qualified} and authorized
\item $\checkmark$ \textbf{Mission briefing received} (zones, ROE, threats)
\item $\checkmark$ \textbf{Communication established} with command
\end{enumerate}

\WARNING{Do not start system if any checklist item is not complete.}

\subsection{STARTUP SEQUENCE (10 STEPS)}

Perform steps in order. Do not skip steps.

\subsubsection{STEP 1: INITIAL POWER-UP}

\textbf{Action}:
\begin{itemize}
\item Ensure \textbf{Station Enable} switch is in \textbf{OFF} position
\item Apply vehicle power to RCWS (circuit breaker ON or power cable connected)
\end{itemize}

\textbf{Expected Result}:
\begin{itemize}
\item \textbf{Power} indicator light illuminates (Green)
\item DCU screen displays boot logo
\item System begins self-test (approximately 30 seconds)
\end{itemize}

\textbf{If NO Power Light}:
\begin{itemize}
\item Check vehicle power supply (voltage 20-30V DC)
\item Check circuit breaker
\item Check cable connections
\item Report to maintenance if power available but no light
\end{itemize}

\subsubsection{STEP 2: BOOT SELF-TEST}

\textbf{Action}:
\begin{itemize}
\item Observe DCU screen during boot
\item Wait for self-test to complete (DO NOT interrupt)
\end{itemize}

\textbf{Expected Display Sequence}:
\begin{enumerate}
\item El 7arress logo appears
\item "SYSTEM INITIALIZING..." message
\item Progress bar advances
\item Device status list appears:
\begin{verbatim}
   Checking Devices...
   [ OK ] Day Camera
   [ OK ] Thermal Camera
   [ OK ] Laser Rangefinder
   [ OK ] Azimuth Motor
   [ OK ] Elevation Motor
   [ OK ] Joystick Controller
   [ -- ] Weapon Actuator (if not installed)

   System Ready
   Press STATION ENABLE to continue
\end{verbatim}
\end{enumerate}

\textbf{If Any Device Shows [FAIL]}:
\begin{itemize}
\item \textbf{DO NOT PROCEED} with startup
\item Note which device failed
\item Report to maintenance immediately
\item System may operate in degraded mode (e.g., without LRF) but requires supervisor approval
\end{itemize}

\subsubsection{STEP 3: ENABLE STATION}

\textbf{Action}:
\begin{itemize}
\item Move \textbf{Station Enable} switch from \textbf{OFF} to \textbf{ON}
\end{itemize}

\textbf{Expected Result}:
\begin{itemize}
\item \textbf{System Ready} light illuminates (Green)
\item Gimbal motors energize (you may hear a soft hum)
\item Video feed appears on DCU screen
\item OSD overlay displays system information
\item Gimbal automatically moves to \textbf{Home Position} (0\degree{} AZ, 0\degree{} EL)
\end{itemize}

\CAUTION{Gimbal will move during this step. Ensure area is clear.}

\textbf{If Gimbal Does Not Move to Home}:
\begin{itemize}
\item Check for mechanical obstructions
\item Verify no-traverse zones are not blocking home position
\item Press \textbf{Home} button manually (see Step 4)
\item If still no movement, report to maintenance
\end{itemize}

\subsubsection{STEP 4: VERIFY HOME POSITION}

\textbf{Action}:
\begin{itemize}
\item Observe OSD azimuth and elevation readings
\item If not at home (AZ: 000\degree{}, EL: 00\degree{}), press \textbf{Home} button
\end{itemize}

\textbf{Expected Result}:
\begin{itemize}
\item OSD displays: \textbf{AZ: 000\degree{} EL: 00\degree{}} (±2\degree{} tolerance)
\item Reticle is centered on screen
\item Gimbal points directly forward relative to vehicle
\end{itemize}

\textbf{Why Home Position Matters}:
\begin{itemize}
\item Provides known reference point
\item Ensures gimbal is mechanically centered
\item Verifies servo position feedback is accurate
\item Required before calibration or zeroing
\end{itemize}

\subsubsection{STEP 5: SELECT CAMERA}

\textbf{Action}:
\begin{itemize}
\item Press \textbf{CAM} button on joystick to select Day or Thermal camera
\item Default camera is \textbf{Day} (visible spectrum)
\end{itemize}

\textbf{Expected Result}:
\begin{itemize}
\item OSD displays camera type: \textbf{DAY} or \textbf{THERMAL}
\item Video image switches between color (day) and grayscale/colorized (thermal)
\item FOV (Field of View) value updates on OSD
\end{itemize}

\textbf{Day Camera}:
\begin{itemize}
\item Color image
\item Better detail in good lighting
\item FOV: 2\degree{} to 60\degree{} (depending on zoom)
\end{itemize}

\textbf{Thermal Camera}:
\begin{itemize}
\item Grayscale or false-color image
\item Detects heat signatures
\item Works in darkness, smoke, fog
\item FOV: 10.4\degree{} (wide) or 5.2\degree{} (narrow - digital zoom)
\end{itemize}

\subsubsection{STEP 6: TEST GIMBAL MOVEMENT}

\textbf{Action}:
\begin{itemize}
\item Gently move joystick in all directions (left/right/up/down)
\item Verify gimbal responds smoothly
\item Return joystick to center (gimbal stops)
\end{itemize}

\textbf{Expected Result}:
\begin{itemize}
\item Gimbal slews in direction of joystick movement
\item OSD azimuth/elevation values update in real-time
\item No grinding, binding, or unusual noises
\item Gimbal stops when joystick returns to center
\end{itemize}

\textbf{Movement Directions}:
\begin{itemize}
\item Joystick LEFT $\rightarrow$ Gimbal slews LEFT (azimuth decreases)
\item Joystick RIGHT $\rightarrow$ Gimbal slews RIGHT (azimuth increases)
\item Joystick UP $\rightarrow$ Gimbal tilts UP (elevation increases)
\item Joystick DOWN $\rightarrow$ Gimbal tilts DOWN (elevation decreases)
\end{itemize}

\textbf{If Gimbal Moves Erratically or Not at All}:
\begin{itemize}
\item Press \textbf{Emergency Stop} immediately
\item Check for loose cables or obstructions
\item Report to maintenance
\end{itemize}

\subsubsection{STEP 7: TEST CAMERA ZOOM}

\textbf{Action}:
\begin{itemize}
\item Press \textbf{Zoom Rocker} UP (zoom in) and DOWN (zoom out)
\item Observe video image magnification change
\item Observe FOV value on OSD
\end{itemize}

\textbf{Expected Result}:
\begin{itemize}
\item Image magnifies when zooming in (FOV decreases)
\item Image wide-angle when zooming out (FOV increases)
\item Zoom is smooth with no jerking
\item OSD FOV updates continuously
\end{itemize}

\textbf{Day Camera Zoom}:
\begin{itemize}
\item Wide: 60\degree{} FOV
\item Narrow: 2\degree{} FOV
\item Continuous motorized zoom
\end{itemize}

\textbf{Thermal Camera Zoom}:
\begin{itemize}
\item Wide: 10.4\degree{} HFOV
\item Narrow: 5.2\degree{} HFOV (2× digital zoom)
\item OSD indicates "ZOOM ×1" or "ZOOM ×2"
\end{itemize}

\subsubsection{STEP 8: TEST LASER RANGEFINDER (LRF)}

\textbf{Action}:
\begin{itemize}
\item Aim reticle at a known object 100m+ away
\item Press and hold \textbf{LRF} button on joystick
\end{itemize}

\textbf{Expected Result}:
\begin{itemize}
\item Laser fires (you will NOT see visible beam - infrared)
\item OSD displays range reading: \textbf{RNG: XXXm}
\item Range updates within 1 second
\item LRF automatically times out after 5 seconds
\end{itemize}

\WARNING{Do not aim LRF at people, animals, or reflective surfaces at close range. Eye damage can occur.}

\textbf{If No Range Reading}:
\begin{itemize}
\item Target may be out of range (50m to 4000m valid)
\item Target may be non-reflective (water, glass)
\item LRF may be faulty (check System Status in Lesson 7)
\end{itemize}

\subsubsection{STEP 9: ENABLE STABILIZATION}

\textbf{Action}:
\begin{itemize}
\item Set \textbf{Stabilization} switch to \textbf{ON}
\end{itemize}

\textbf{Expected Result}:
\begin{itemize}
\item OSD displays \textbf{STAB: ON}
\item Gimbal compensates for vehicle movement
\item Reticle remains steady on target even if vehicle rocks
\end{itemize}

\textbf{Stabilization Modes}:
\begin{itemize}
\item \textbf{ON}: Gimbal actively compensates for vehicle pitch/roll/yaw
\item \textbf{OFF}: Gimbal is fixed relative to vehicle body
\end{itemize}

\textbf{When to Use Stabilization}:
\begin{itemize}
\item $\checkmark$ Always ON during normal operations
\item $\checkmark$ Essential for accurate fire on moving vehicle
\item $\checkmark$ Required for tracking moving targets
\item $\times$ OFF only for maintenance or troubleshooting
\end{itemize}

\subsubsection{STEP 10: SYSTEM READY - FINAL CHECK}

\textbf{Action}:
\begin{itemize}
\item Verify all indicator lights show correct status:
\item \textbf{Power}: Green (ON)
\item \textbf{System Ready}: Green (ON)
\item \textbf{Gun Armed}: OFF (system is SAFE)
\item \textbf{Ammo Loaded}: Yellow (if ammo present) or OFF
\item \textbf{Authorized}: Green (if logged in) or OFF
\item \textbf{Fault/Alarm}: OFF (no errors)
\end{itemize}

\begin{itemize}
\item Verify OSD displays:
\item Live video feed (day or thermal)
\item Azimuth and elevation values
\item Current FOV
\item System mode (Manual)
\item No warning messages
\end{itemize}

\textbf{If All Checks Pass}:
\begin{itemize}
\item $\checkmark$ System is ready for operation
\item $\checkmark$ Proceed with mission tasks
\item $\checkmark$ Log startup time in operator log
\end{itemize}

\textbf{If Any Check Fails}:
\begin{itemize}
\item $\times$ Do not proceed to weapons operation
\item $\times$ Troubleshoot using Lesson 9
\item $\times$ Report to supervisor/maintenance
\end{itemize}

\subsection{STARTUP TROUBLESHOOTING (QUICK REFERENCE)}

% TABLE ROW: | Problem | Possible Cause | Action |
% TABLE ROW: |---------|---------------|--------|
% TABLE ROW: | No power light | No vehicle power | Check circuit breaker, voltage |
% TABLE ROW: | Self-test fails | Device malfunction | Note failed device, report to maintenance |
% TABLE ROW: | No video feed | Camera error | Check camera connections, restart system |
% TABLE ROW: | Gimbal won't move | Motors disabled or fault | Check Station Enable, check for faults |
% TABLE ROW: | Erratic gimbal movement | Joystick calibration | Recalibrate joystick (maintenance task) |
% TABLE ROW: | LRF no reading | Out of range or bad target | Aim at closer/better reflective target |
% TABLE ROW: | Thermal image frozen | FFC in progress | Wait 5 seconds for FFC to complete |

\section{2.2 DISPLAY AND CONTROL UNIT (DCU) OPERATIONS}

\subsection{DCU BUTTON AND SWITCH FUNCTIONS (DETAILED)}

\subsubsection{EMERGENCY STOP BUTTON (RED)}

\textbf{Location}: Top left of DCU panel, large RED button

\textbf{Function}: Immediate system shutdown for safety emergencies

\textbf{Operation}:
\begin{enumerate}
\item Press button (no confirmation required)
\item System immediately:
\end{enumerate}
\begin{itemize}
\item Stops all gimbal movement
\item Safes weapon (trigger disabled)
\item Locks servos in place
\item Displays \osd{EMERGENCY STOP ACTIVE} on OSD
\end{itemize}

\textbf{To Reset}:
\begin{enumerate}
\item Twist/pull button to release (depending on button type)
\item Verify emergency condition is resolved
\item Press \textbf{Station Enable} OFF then ON to restart
\item System will re-initialize (30 seconds)
\end{enumerate}

\textbf{When to Use}:
\begin{itemize}
\item Personnel enter turret hazard zone
\item Runaway gun movement
\item Fire or smoke
\item Any unsafe condition
\item Loss of control
\end{itemize}

\textbf{ CRITICAL}: Do NOT hesitate to use Emergency Stop. Better safe than sorry.

\subsubsection{STATION ENABLE SWITCH}

\textbf{Location}: DCU panel, toggle switch

\textbf{Positions}: OFF / ON

\textbf{Function}: Master enable/disable for entire RCWS

\textbf{OFF Position}:
\begin{itemize}
\item Gimbal motors disabled (turret cannot move)
\item Video still displays (cameras remain powered)
\item Weapon is safed
\item Safe to approach turret for inspection
\end{itemize}

\textbf{ON Position}:
\begin{itemize}
\item Gimbal motors enabled
\item All subsystems operational
\item Turret can move if joystick input received
\item Stay clear of turret
\end{itemize}

\textbf{When to Use OFF}:
\begin{itemize}
\item Before approaching turret
\item During maintenance
\item When leaving operator station
\item End of shift/mission
\end{itemize}

\subsubsection{HOME POSITION BUTTON}

\textbf{Location}: DCU panel

\textbf{Function}: Returns gimbal to forward-facing position (0\degree{} AZ, 0\degree{} EL)

\textbf{Operation}:
\begin{enumerate}
\item Press button once
\item Gimbal automatically slews to home position (takes 3-10 seconds depending on current position)
\item Gimbal stops at AZ: 000\degree{}, EL: 00\degree{}
\end{enumerate}

\textbf{When to Use}:
\begin{itemize}
\item At startup (if gimbal doesn't auto-home)
\item Before shutdown
\item To establish known reference
\item When disoriented in azimuth
\end{itemize}

\CAUTION{Gimbal will move when button is pressed. Ensure area is clear.}

\subsubsection{GUN ARM/SAFE SWITCH}

\textbf{Location}: DCU panel, guarded toggle switch (flip up guard to access)

\textbf{Positions}: SAFE / ARM

\textbf{Function}: Arms or safes the weapon system

\textbf{SAFE Position} (Default):
\begin{itemize}
\item Weapon trigger disabled
\item \textbf{Gun Armed} light is OFF
\item Trigger pull has no effect
\item Safe for non-combat operations
\end{itemize}

\textbf{ARM Position}:
\begin{itemize}
\item Weapon trigger enabled
\item \textbf{Gun Armed} light illuminates RED
\item Trigger pull will fire weapon (if other conditions met)
\item Only use during combat or live fire training
\end{itemize}

\textbf{Safety Interlocks} (all must be true to fire):
\begin{enumerate}
\item Gun Arm switch in ARM position
\item Dead Man Switch held on joystick
\item Authorized (if authorization system enabled)
\item Not in a no-fire zone
\item Trigger pulled
\end{enumerate}

\WARNING{When Gun Armed light is RED, treat weapon as HOT. One trigger pull away from firing.}

\subsubsection{FIRE MODE SELECTOR}

\textbf{Location}: DCU panel, rotary selector switch

\textbf{Positions}: SINGLE / SHORT BURST / LONG BURST

\textbf{Function}: Selects weapon fire mode

\textbf{SINGLE}:
\begin{itemize}
\item One round per trigger pull
\item Trigger must be released and pulled again for next round
\item Most accurate mode
\item Use for precision engagement
\end{itemize}

\textbf{SHORT BURST}:
\begin{itemize}
\item 3-5 rounds per trigger pull (depends on weapon)
\item Automatic burst, then stops
\item Good balance of accuracy and firepower
\item Use for moving targets or suppression
\end{itemize}

\textbf{LONG BURST}:
\begin{itemize}
\item Continuous fire while trigger held
\item Full automatic until trigger released or ammo exhausted
\item Less accurate due to recoil
\item Use for area suppression or close-range threats
\end{itemize}

\CAUTION{Fire mode selector position should be verified before each engagement. Wrong mode can waste ammo or fail to stop threat.}

\subsubsection{SPEED SELECT SWITCH}

\textbf{Location}: DCU panel, 3-position switch

\textbf{Positions}: LOW / MEDIUM / HIGH

\textbf{Function}: Sets gimbal slew speed (how fast turret moves with joystick)

\textbf{LOW Speed}:
\begin{itemize}
\item Slow, precise movements
\item Good for: Zeroing, fine adjustments, long-range precision
\item Maximum slew rate: \textasciitilde{}5\degree{}/second
\end{itemize}

\textbf{MEDIUM Speed} (Default):
\begin{itemize}
\item Moderate speed for general use
\item Good for: Normal surveillance, target acquisition
\item Maximum slew rate: \textasciitilde{}20\degree{}/second
\end{itemize}

\textbf{HIGH Speed}:
\begin{itemize}
\item Fast movements for rapid target engagement
\item Good for: Close-range threats, multiple targets, emergency response
\item Maximum slew rate: \textasciitilde{}60\degree{}/second
\end{itemize}

\textbf{Tip}: Start with MEDIUM for most operations. Switch to LOW for precision, HIGH for emergencies.

\subsubsection{STABILIZATION ON/OFF SWITCH}

\textbf{Location}: DCU panel, toggle switch

\textbf{Positions}: OFF / ON

\textbf{Function}: Enables or disables platform stabilization

\textbf{OFF}:
\begin{itemize}
\item Gimbal is fixed relative to vehicle body
\item If vehicle tilts, reticle tilts with it
\item Use only for troubleshooting or maintenance
\end{itemize}

\textbf{ON} (Default):
\begin{itemize}
\item Gimbal compensates for vehicle movement
\item Reticle stays on target even if vehicle rocks
\item Essential for accurate fire from moving platform
\end{itemize}

\textbf{How It Works}:
\begin{itemize}
\item System reads vehicle pitch, roll, yaw
\item Gimbal motors counter-rotate to cancel movement
\item Reticle stays stabilized on target
\end{itemize}

\textbf{ IMPORTANT}: Always use ON during operations. OFF mode degrades accuracy significantly.

\subsubsection{DETECTION ON/OFF SWITCH}

\textbf{Location}: DCU panel, toggle switch

\textbf{Positions}: OFF / ON

\textbf{Function}: Enables or disables automatic target detection (if equipped)

\textbf{OFF} (Default):
\begin{itemize}
\item No automatic detection
\item Operator manually searches for targets
\end{itemize}

\textbf{ON}:
\begin{itemize}
\item System highlights potential targets on screen
\item Bounding boxes appear around detected objects (people, vehicles)
\item Operator still makes final decision to engage
\end{itemize}

\NOTE{Detection system is AI-assisted. It is NOT perfect. Operator is responsible for target identification. Never fire without positive ID.}

\subsubsection{MENU BUTTONS (▲ / ▼ / $\checkmark$)}

\textbf{Location}: DCU panel, three buttons

\textbf{Function}: Navigate system menus and adjust settings

\textbf{MENU ▲ (UP)}:
\begin{itemize}
\item Move selection up in menu
\item Increase parameter values
\end{itemize}

\textbf{MENU ▼ (DOWN)}:
\begin{itemize}
\item Move selection down in menu
\item Decrease parameter values
\end{itemize}

\textbf{MENU $\checkmark$ (VALIDATE)}:
\begin{itemize}
\item Open main menu (first press)
\item Confirm selection
\item Enter submenu
\item Exit menu (when at top level)
\end{itemize}

\textbf{Usage}: Covered in detail in Lesson 3.

\subsection{DCU INDICATOR LIGHTS (DETAILED)}

\subsubsection{POWER (Green)}

\begin{itemize}
\item \textbf{ON}: Vehicle power is supplied to RCWS
\item \textbf{OFF}: No power (check circuit breaker)
\end{itemize}

\subsubsection{SYSTEM READY (Green)}

\begin{itemize}
\item \textbf{ON}: All subsystems initialized and operational
\item \textbf{OFF}: System booting, in error state, or disabled
\item \textbf{Flashing}: System is degraded (some devices failed but operation possible)
\end{itemize}

\subsubsection{GUN ARMED (Red)}

\begin{itemize}
\item \textbf{ON}: Weapon is armed and ready to fire
\item \textbf{OFF}: Weapon is safed (trigger disabled)
\WARNING{When lit, one trigger pull from firing}
\end{itemize}

\subsubsection{AMMO LOADED (Yellow)}

\begin{itemize}
\item \textbf{ON}: Ammunition belt/magazine detected
\item \textbf{OFF}: No ammunition or feed system disconnected
\end{itemize}

\subsubsection{AUTHORIZED (Green)}

\begin{itemize}
\item \textbf{ON}: Operator has entered valid authorization code
\item \textbf{OFF}: Operator not authorized (weapon will not fire even if armed)
\NOTE{Authorization system may be disabled depending on configuration}
\end{itemize}

\subsubsection{FAULT/ALARM (Red)}

\begin{itemize}
\item \textbf{OFF}: No faults
\item \textbf{Flashing Slow}: Warning condition (non-critical, check System Status)
\item \textbf{Flashing Fast}: Alarm condition (critical fault, system may shut down)
\item \textbf{Solid ON}: Emergency stop active or critical failure
\end{itemize}

\textbf{When Fault Light Activates}:
\begin{enumerate}
\item Note the flash pattern
\item Access System Status menu (Lesson 7)
\item Read fault description
\item Take corrective action or report to maintenance
\end{enumerate}

\section{2.3 JOYSTICK CONTROLLER OPERATIONS}

\subsection{JOYSTICK CONTROL TECHNIQUES}

\subsubsection{PROPER GRIP}

\textbf{Right Hand Position}:
\begin{enumerate}
\item Wrap fingers around joystick grip
\item Index finger rests on trigger (outside trigger guard when not firing)
\item Thumb on top, near CAM and TRK buttons
\item \textbf{Dead Man Switch is on rear of grip} - squeeze with palm/fingers to engage
\end{enumerate}

\textbf{Left Hand} (optional):
\begin{itemize}
\item Can assist with zoom rocker if needed
\item Generally free for other tasks (radio, notes, etc.)
\end{itemize}

\textbf{Posture}:
\begin{itemize}
\item Sit upright with elbow supported
\item Wrist relaxed (not bent)
\item Don't "death grip" the joystick - light touch is better
\end{itemize}

\subsubsection{GIMBAL SLEW TECHNIQUE}

\textbf{Main Stick (Azimuth and Elevation Control)}:

\textbf{Small Movements} (Precision):
\begin{itemize}
\item Deflect stick slightly from center (10-20\%)
\item Gimbal moves slowly
\item Good for: Tracking, zeroing, fine adjustments
\end{itemize}

\textbf{Large Movements} (Rapid Slew):
\begin{itemize}
\item Deflect stick fully (80-100\%)
\item Gimbal moves at maximum speed (based on Speed Select switch)
\item Good for: Searching, responding to threats, sector scans
\end{itemize}

\textbf{Smooth Tracking}:
\begin{itemize}
\item Apply constant, smooth stick pressure
\item Anticipate target movement
\item Lead the target slightly
\item Practice makes perfect
\end{itemize}

\textbf{Joystick Center Detent}:
\begin{itemize}
\item Joystick has small dead-zone at center
\item Helps prevent accidental movement
\item If gimbal drifts when stick is released, joystick needs calibration (maintenance task)
\end{itemize}

\subsubsection{CAMERA SWITCH (CAM BUTTON)}

\textbf{Location}: Top of joystick, left button

\textbf{Function}: Toggle between Day and Thermal cameras

\textbf{Operation}:
\begin{itemize}
\item Press once $\rightarrow$ Switches camera
\item OSD updates to show \textbf{DAY} or \textbf{THERMAL}
\item Video image changes immediately
\end{itemize}

\textbf{When to Use Day Camera}:
\begin{itemize}
\item $\checkmark$ Good lighting conditions
\item $\checkmark$ Need color information (e.g., identifying vehicle markings)
\item $\checkmark$ Need maximum zoom range (2\degree{} to 60\degree{} FOV)
\end{itemize}

\textbf{When to Use Thermal Camera}:
\begin{itemize}
\item $\checkmark$ Darkness, dawn, dusk
\item $\checkmark$ Smoke, fog, dust
\item $\checkmark$ Detecting hidden personnel (heat signatures)
\item $\checkmark$ Identifying recently fired weapons (barrel heat)
\end{itemize}

\textbf{Tip}: Experienced operators frequently switch between cameras to get best situational awareness.

\subsubsection{TRACK SELECT (TRK BUTTON)}

\textbf{Location}: Top of joystick, right button

\textbf{Function}: Initiate or abort automatic target tracking

\textbf{Operation}:
\begin{enumerate}
\item \textbf{First Press} $\rightarrow$ Enters Acquisition mode (yellow box appears)
\item \textbf{Second Press} $\rightarrow$ Locks onto target in box (tracking starts)
\item \textbf{Third Press} $\rightarrow$ Aborts tracking (returns to manual mode)
\end{enumerate}

\textbf{Detailed Procedure}: Covered in Lesson 5 (Target Engagement Process)

\subsubsection{LASER RANGEFINDER (LRF BUTTON)}

\textbf{Location}: Middle of joystick, left button

\textbf{Function}: Fires laser to measure range to target

\textbf{Operation}:
\begin{enumerate}
\item Aim reticle at target
\item Press and hold LRF button
\item Laser fires (invisible infrared beam)
\item OSD displays range: \textbf{RNG: XXXm}
\item Release button when done (or auto-timeout after 5 seconds)
\end{enumerate}

\textbf{Range}: 50m to 4000m (depending on target reflectivity)

\textbf{Accuracy}: ±5 meters

\WARNING{Class 3B laser. Do not aim at people or reflective surfaces at close range.}

\textbf{Tip}: Always range your target before engaging. Ballistics compensation requires accurate range.

\subsubsection{FUNCTION (FN BUTTON)}

\textbf{Location}: Middle of joystick, right button

\textbf{Function}: Context-sensitive - function changes based on current mode

\textbf{Typical Functions}:
\begin{itemize}
\item \textbf{In Manual Mode}: No function (reserved for future use)
\item \textbf{In Tracking Mode}: Adjust acquisition box size
\item \textbf{In Menu Mode}: Quick-select common functions
\item \textbf{In Zeroing Mode}: Capture impact point
\end{itemize}

\textbf{Consult OSD for current FN button assignment} (may display hint at bottom of screen)

\subsubsection{HAT SWITCH (8-WAY DIRECTIONAL)}

\textbf{Location}: Center of joystick, top surface

\textbf{Function}: Multi-purpose control depending on mode

\textbf{Positions}:
\begin{itemize}
\item Center (neutral)
\item Up, Down, Left, Right
\item Up-Left, Up-Right, Down-Left, Down-Right (diagonals)
\end{itemize}

\textbf{Functions by Mode}:

% TABLE ROW: | Mode | Hat Switch Function |
% TABLE ROW: |------|---------------------|
% TABLE ROW: | **Manual Mode** | Quick slew gimbal (alternative to main stick) |
% TABLE ROW: | **Tracking Mode** | Move acquisition box position |
% TABLE ROW: | **Menu Mode** | Navigate menus (alternative to DCU buttons) |
% TABLE ROW: | **Zeroing Mode** | Adjust offset values |

\textbf{Most Common Use}: Moving acquisition box during target tracking setup.

\subsubsection{ZOOM ROCKER (▲ / ▼)}

\textbf{Location}: Very top of joystick

\textbf{Function}: Camera zoom in/out

\textbf{Operation}:
\begin{itemize}
\item Press \textbf{UP (▲)} $\rightarrow$ Zoom in (magnify image, FOV decreases)
\item Press \textbf{DOWN (▼)} $\rightarrow$ Zoom out (wide view, FOV increases)
\item Hold to zoom continuously
\item Release to stop zooming
\end{itemize}

\textbf{Zoom Ranges}:
\begin{itemize}
\item \textbf{Day Camera}: 2\degree{} to 60\degree{} FOV (30× optical zoom)
\item \textbf{Thermal Camera}: 5.2\degree{} to 10.4\degree{} HFOV (2× digital zoom)
\end{itemize}

\textbf{Zoom Strategy}:
\begin{itemize}
\item \textbf{Wide zoom} for searching/surveillance
\item \textbf{Narrow zoom} for identification and precision engagement
\item \textbf{Medium zoom} for tracking and general operations
\end{itemize}

\textbf{Auto-Focus} (Day Camera Only):
\begin{itemize}
\item Camera automatically focuses as you zoom
\item Can be disabled for manual focus (advanced)
\end{itemize}

\subsubsection{WEAPON TRIGGER}

\textbf{Location}: Front of joystick grip, inside trigger guard

\textbf{Function}: Fires weapon (when all safety interlocks satisfied)

\textbf{Trigger Pull Technique}:
\begin{enumerate}
\item Place index finger on trigger
\item Take up slack (first stage - no resistance)
\item Breathe out slowly
\item Squeeze smoothly (second stage - resistance increases)
\item \textbf{BANG} - weapon fires
\item Hold trigger for burst mode, release for single shot
\end{enumerate}

\textbf{Safety Interlocks} (all must be met):
\begin{itemize}
\item $\checkmark$ Gun Armed switch in ARM position
\item $\checkmark$ Dead Man Switch held
\item $\checkmark$ Authorized (if required)
\item $\checkmark$ Not in no-fire zone
\item $\checkmark$ Ammo loaded
\end{itemize}

\textbf{If Trigger Pulls But No Fire}:
\begin{itemize}
\item Check Gun Armed light (RED = armed)
\item Check Dead Man Switch is engaged
\item Check Authorized light (if applicable)
\item Check OSD for "NO-FIRE ZONE" warning
\item Check weapon is loaded
\item If all OK, possible weapon malfunction (report to maintenance)
\end{itemize}

\subsubsection{DEAD MAN SWITCH}

\textbf{Location}: Rear of joystick grip (squeeze with palm/fingers)

\textbf{Function}: Safety device - must be held for weapon operation

\textbf{Operation}:
\begin{itemize}
\item \textbf{Squeezed} $\rightarrow$ Dead Man Switch engaged (weapon can fire if other conditions met)
\item \textbf{Released} $\rightarrow$ Dead Man Switch disengaged (weapon safed immediately)
\end{itemize}

\textbf{Purpose}:
\begin{itemize}
\item Prevents accidental discharge if operator is incapacitated
\item Automatic safety if operator loses grip
\item Required safety interlock for firing
\end{itemize}

\textbf{Proper Technique}:
\begin{itemize}
\item Hold naturally as part of grip
\item Don't squeeze too hard (causes fatigue)
\item Practice rapid release (for emergencies)
\end{itemize}

\textbf{ CRITICAL SAFETY RULE}: Release Dead Man Switch immediately when not actively engaging a target.

\textbf{Failure Mode}:
\begin{itemize}
\item If Dead Man Switch fails to spring back $\rightarrow$ \textbf{NO-GO} condition
\item Report to maintenance immediately
\item Do not operate system
\end{itemize}

\section{2.4 ON-SCREEN DISPLAY (OSD) INTERPRETATION}

\subsection{OSD LAYOUT AND ELEMENTS}

The OSD provides real-time system information overlaid on the video feed. Learn to interpret all elements quickly.

\subsection{COMPLETE OSD LAYOUT}

\begin{verbatim}
┌──────────────────────────────────────────────────────────────┐
│ AZ: 045°  EL: +12°   |  DAY  FOV: 9.0°  ZOOM: 15×   [●REC]  │ ← Top Bar
│                                                                │
│                                                                │
│                          [Target Box]                         │ ← Tracking Box (if active)
│                              ┌──┐                             │
│                              └──┘                             │
│                                                                │
│                                + ←Reticle (Aimpoint)          │
│                               (*)←CCIP Pipper (Impact Point)  │
│                                                                │
│  NO-FIRE ZONE WARNING                    STAB: ON            │
│                                                                │
│                                                                │
│ RNG: 850m          MODE: Manual          STATUS: ARMED        │ ← Bottom Bar
│ TEMP: 35°C         TRACK: Off            AMMO: 450           │
└──────────────────────────────────────────────────────────────┘
\end{verbatim}

\subsection{TOP BAR ELEMENTS (LEFT TO RIGHT)}

\subsubsection{AZ: XXX\degree{} (Azimuth)}

\begin{itemize}
\item Current gimbal azimuth position
\item Range: 000\degree{} to 359\degree{}
\item 000\degree{} = North (or vehicle forward, depending on configuration)
\item Clockwise: 090\degree{} = East, 180\degree{} = South, 270\degree{} = West
\end{itemize}

\subsubsection{EL: ±XX\degree{} (Elevation)}

\begin{itemize}
\item Current gimbal elevation position
\item Range: -20\degree{} to +60\degree{}
\item 00\degree{} = Horizon
\item Positive = Above horizon (tilted up)
\item Negative = Below horizon (tilted down)
\end{itemize}

\subsubsection{DAY / THERMAL (Camera Type)}

\begin{itemize}
\item Shows which camera is active
\item \textbf{DAY} = Visible spectrum camera (color)
\item \textbf{THERMAL} = Infrared camera (heat signatures)
\end{itemize}

\subsubsection{FOV: X.X\degree{} (Field of View)}

\begin{itemize}
\item Current camera field of view (horizontal angle)
\item Smaller value = more zoomed in
\item Larger value = wider view
\item Day camera: 2\degree{} to 60\degree{}
\item Thermal camera: 5.2\degree{} to 10.4\degree{}
\end{itemize}

\subsubsection{ZOOM: XX× (Zoom Factor)}

\begin{itemize}
\item Shows zoom magnification relative to wide view
\item Day camera: 1× to 30× optical
\item Thermal camera: 1× or 2× digital
\end{itemize}

\subsubsection{[●REC] (Recording Indicator - if equipped)}

\begin{itemize}
\item Shows if video is being recorded
\item Red dot = recording active
\item May not be present on all systems
\end{itemize}

\subsection{CENTER AREA ELEMENTS}

\subsubsection{Reticle (+)}

\begin{itemize}
\item \textbf{Main aiming point} (usually center of screen)
\item This is where the \textbf{gun is currently aimed} (with zeroing applied)
\item Crosshair design depends on reticle type selected (see Lesson 3)
\item \textbf{Fire weapon with reticle on target}
\end{itemize}

\subsubsection{CCIP Pipper ((*))}

\begin{itemize}
\item \textbf{Continuously Computed Impact Point}
\item Shows where bullet will actually hit with \textbf{ALL ballistic corrections}:
\item Zeroing offset
\item Environmental corrections (temp, altitude, wind)
\item Lead angle (if tracking moving target)
\item May be offset from reticle if ballistics are applied
\item \textbf{For moving targets, put CCIP pipper on target, not reticle}
\end{itemize}

\textbf{Reticle vs. CCIP}:
\begin{itemize}
\item \textbf{No ballistics applied} $\rightarrow$ Reticle and CCIP are same location
\item \textbf{Ballistics applied} $\rightarrow$ CCIP offset from reticle
\item \textbf{Always aim with CCIP for accurate hits}
\end{itemize}

\subsubsection{Tracking Box (Yellow/Red/Green)}

\begin{itemize}
\item Appears only when tracking is active
\item Color indicates tracking state:
\item \textbf{Yellow Solid} = Acquisition mode (user positioning box)
\item \textbf{Yellow Dashed} = Lock pending or coast mode
\item \textbf{Red Dashed} = Active lock (tracking target)
\item \textbf{Green Dashed} = Firing mode (holding position)
\item Box size adapts to target
\item See Lesson 5 for detailed tracking states
\end{itemize}

\subsection{BOTTOM BAR ELEMENTS (LEFT SIDE)}

\subsubsection{RNG: XXXm (Range)}

\begin{itemize}
\item Distance to target from last LRF measurement
\item Updates when you press LRF button
\item Range: 50m to 4000m
\item \textbf{---m} = No range data (LRF not fired or failed)
\end{itemize}

\subsubsection{TEMP: XX\degree{}C (Temperature)}

\begin{itemize}
\item System temperature (ambient or component)
\item Used for: Monitoring, environmental ballistics
\item Typical range: -20\degree{}C to +50\degree{}C
\WARNING{If temp exceeds 60\degree{}C, system may shut down}
\end{itemize}

\subsection{BOTTOM BAR ELEMENTS (CENTER)}

\subsubsection{MODE: XXXXX (Motion Mode)}

\begin{itemize}
\item Shows current gimbal control mode:
\item \textbf{Manual} = Joystick control
\item \textbf{Auto Track} = Automatic target tracking
\item \textbf{Sector Scan} = Automatic sector scanning
\item \textbf{TRP Scan} = Target Reference Point scan
\item \textbf{Idle} = System idle (startup or fault)
\item See Lesson 4 for all motion modes
\end{itemize}

\subsubsection{TRACK: XXXXX (Tracking Status)}

\begin{itemize}
\item Shows tracking system state:
\item \textbf{Off} = Not tracking
\item \textbf{Acquiring} = Positioning acquisition box
\item \textbf{Lock Pending} = Attempting to lock onto target
\item \textbf{Active} = Locked and following target
\item \textbf{Coast} = Temporarily lost target, predicting position
\item \textbf{Firing} = Holding position during weapon fire
\item See Lesson 5 for tracking details
\end{itemize}

\subsection{BOTTOM BAR ELEMENTS (RIGHT SIDE)}

\subsubsection{STATUS: XXXXX (System Status)}

\begin{itemize}
\item Overall weapon system status:
\item \textbf{SAFE} = Gun is safed (cannot fire)
\item \textbf{ARMED} = Gun is armed (ready to fire)
\item \textbf{READY} = All conditions met for firing
\item \textbf{FAULT} = System error (check Fault light)
\end{itemize}

\subsubsection{AMMO: XXX (Ammunition Count - if equipped)}

\begin{itemize}
\item Remaining rounds in belt/magazine
\item Counts down as rounds are fired
\WARNING{When count reaches 0, weapon is empty}
\item May show \textbf{---} if ammo counter not installed
\end{itemize}

\subsubsection{STAB: ON/OFF (Stabilization Status)}

\begin{itemize}
\item Shows if platform stabilization is enabled
\item \textbf{ON} = Gimbal compensating for vehicle movement (normal)
\item \textbf{OFF} = Gimbal fixed to vehicle (troubleshooting only)
\end{itemize}

\subsection{WARNING MESSAGES (CENTER SCREEN)}

Warnings appear in large text when critical conditions are detected:

\subsubsection{NO-FIRE ZONE WARNING}

\begin{itemize}
\item \textbf{Message}: "NO-FIRE ZONE - WEAPON SAFED"
\item \textbf{Color}: Red, flashing
\item \textbf{Meaning}: Reticle is inside a no-fire zone
\item \textbf{Action}: Do NOT fire. Slew gimbal out of zone.
\end{itemize}

\subsubsection{NO-TRAVERSE WARNING}

\begin{itemize}
\item \textbf{Message}: "NO-TRAVERSE ZONE - MOVEMENT RESTRICTED"
\item \textbf{Color}: Yellow
\item \textbf{Meaning}: Gimbal approaching or in no-traverse zone
\item \textbf{Action}: Slew in opposite direction
\end{itemize}

\subsubsection{EMERGENCY STOP ACTIVE}

\begin{itemize}
\item \textbf{Message}: "EMERGENCY STOP - SYSTEM SAFED"
\item \textbf{Color}: Red, solid
\item \textbf{Meaning}: Emergency stop button has been pressed
\item \textbf{Action}: Resolve emergency, reset button, restart system
\end{itemize}

\subsubsection{SYSTEM FAULT}

\begin{itemize}
\item \textbf{Message}: "FAULT: [description]"
\item \textbf{Color}: Red or yellow depending on severity
\item \textbf{Meaning}: Hardware or software fault detected
\item \textbf{Action}: Access System Status (Lesson 7) for details
\end{itemize}

\subsubsection{TARGET LOST}

\begin{itemize}
\item \textbf{Message}: "TRACKING LOST - COAST MODE"
\item \textbf{Color}: Yellow
\item \textbf{Meaning}: Target tracking lost (occlusion, out of frame)
\item \textbf{Action}: System predicting, manual reacquisition may be needed
\end{itemize}

\subsubsection{AUTHORIZATION REQUIRED}

\begin{itemize}
\item \textbf{Message}: "NOT AUTHORIZED - ENTER CODE"
\item \textbf{Color}: Yellow
\item \textbf{Meaning}: Operator authorization required before firing
\item \textbf{Action}: Enter authorization code via menu
\end{itemize}

\subsection{OSD QUICK INTERPRETATION DRILL}

Practice reading the OSD in under 3 seconds:

\textbf{Scan Pattern (Top to Bottom, Left to Right)}:
\begin{enumerate}
\item \textbf{Azimuth/Elevation} $\rightarrow$ Where am I pointed?
\item \textbf{Camera \& FOV} $\rightarrow$ What am I seeing?
\item \textbf{Reticle/CCIP} $\rightarrow$ Where will I hit?
\item \textbf{Warnings} $\rightarrow$ Any safety issues?
\item \textbf{Status} $\rightarrow$ Ready to fire?
\end{enumerate}

\textbf{Example OSD Read}:
\begin{verbatim}
AZ: 045°  EL: +12°  |  DAY  FOV: 9.0°
RNG: 850m  MODE: Manual  STATUS: ARMED
\end{verbatim}

\textbf{Interpretation}:
\begin{itemize}
\item Pointed northeast (45\degree{}), tilted up 12\degree{}
\item Using day camera, medium zoom (9\degree{} FOV)
\item Range to target: 850 meters
\item Manual control mode
\item System is armed and ready to fire
\end{itemize}

\textbf{ CHECK}: No warnings displayed $\rightarrow$ Good to engage if target identified

\section{2.5 SYSTEM SHUTDOWN PROCEDURE}

Always perform proper shutdown. Do NOT just cut power.

\subsection{SHUTDOWN SEQUENCE (7 STEPS)}

\subsubsection{STEP 1: SAFE THE WEAPON}

\textbf{Action}:
\begin{itemize}
\item Move \textbf{Gun Arm/Safe} switch to \textbf{SAFE}
\item Verify \textbf{Gun Armed} light is OFF
\item Release Dead Man Switch on joystick
\end{itemize}

\textbf{Why}: Ensures weapon cannot fire during shutdown

\subsubsection{STEP 2: RETURN TO HOME POSITION}

\textbf{Action}:
\begin{itemize}
\item Press \textbf{Home} button on DCU
\item Wait for gimbal to slew to 0\degree{} AZ, 0\degree{} EL
\end{itemize}

\textbf{Why}: Parks gimbal in known position for next startup

\subsubsection{STEP 3: DISABLE STABILIZATION}

\textbf{Action}:
\begin{itemize}
\item Move \textbf{Stabilization} switch to \textbf{OFF}
\end{itemize}

\textbf{Why}: Reduces power draw and motor wear

\subsubsection{STEP 4: ACCESS SHUTDOWN MENU (Optional but Recommended)}

\textbf{Action}:
\begin{itemize}
\item Press \textbf{MENU $\checkmark$} button
\item Navigate to \textbf{SYSTEM} $\rightarrow$ \textbf{Shutdown}
\item Select \textbf{"Shutdown System"}
\item Confirm shutdown
\end{itemize}

\textbf{Result}:
\begin{itemize}
\item System performs orderly shutdown
\item Saves settings and logs
\item Powers down cameras and motors
\item Displays \osd{SHUTDOWN COMPLETE} message
\end{itemize}

\textbf{Why}: Proper software shutdown prevents data corruption

\subsubsection{STEP 5: DISABLE STATION}

\textbf{Action}:
\begin{itemize}
\item Move \textbf{Station Enable} switch to \textbf{OFF}
\end{itemize}

\textbf{Result}:
\begin{itemize}
\item \textbf{System Ready} light turns OFF
\item Gimbal motors de-energize
\item System enters standby mode
\item Video feed may remain (depending on configuration)
\end{itemize}

\subsubsection{STEP 6: REMOVE VEHICLE POWER (End of Shift)}

\textbf{Action}:
\begin{itemize}
\item If end of mission/shift, disconnect vehicle power:
\item Turn off circuit breaker, OR
\item Disconnect power cable (if external)
\end{itemize}

\textbf{Result}:
\begin{itemize}
\item \textbf{Power} light turns OFF
\item DCU screen goes black
\item All systems powered down
\end{itemize}

\textbf{Skip this step} if:
\begin{itemize}
\item Operator change only (next operator will power up)
\item Short break (< 1 hour)
\end{itemize}

\subsubsection{STEP 7: SECURE WEAPON AND EQUIPMENT}

\textbf{Action}:
\begin{itemize}
\item Clear weapon per Appendix A (if required)
\item Remove ammunition (if required by SOP)
\item Install protective covers on cameras (if environmental exposure)
\item Lock operator station (if applicable)
\item Complete operator log entry
\end{itemize}

\subsection{POST-SHUTDOWN CHECKS}

\begin{itemize}
\item $\checkmark$ Gun Armed light is OFF
\item $\checkmark$ Gimbal is at home position (0\degree{} AZ, 0\degree{} EL)
\item $\checkmark$ Station Enable is OFF
\item $\checkmark$ Weapon is cleared (if required)
\item $\checkmark$ Covers installed (if required)
\item $\checkmark$ Operator log entry complete
\end{itemize}

\subsection{SHUTDOWN TROUBLESHOOTING}

% TABLE ROW: | Problem | Possible Cause | Action |
% TABLE ROW: |---------|---------------|--------|
% TABLE ROW: | Gimbal won't go to home | Obstruction or fault | Manually slew to approximate home, report fault |
% TABLE ROW: | Shutdown menu not responding | Software hang | Press Emergency Stop, wait 10 sec, power off |
% TABLE ROW: | Can't disable Station Enable | Switch stuck | Press Emergency Stop, cut vehicle power, report maintenance |
% TABLE ROW: | System won't power off | Software error | Press Emergency Stop, cut vehicle power at source |

\chapter{PART II: OPERATIONAL PROCEDURES}
