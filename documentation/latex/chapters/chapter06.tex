% BALLISTICS & FIRE CONTROL

\noindent
\textbf{Lesson Duration:} 5 hours (Classroom 2h + Practical 3h)

\vspace{0.3cm}

\noindent
\textbf{Learning Objectives:}

Upon completion of this lesson, you will be able to:

\begin{itemize}[leftmargin=2em,itemsep=3pt]
\item Perform weapon zeroing (boresight alignment)
\item Configure environmental parameters for ballistic corrections
\item Activate and employ Lead Angle Compensation (LAC) for moving targets
\item Interpret fire control status indicators
\item Combine zeroing, environmental settings, and LAC for accurate engagements
\end{itemize}

\vspace{0.5cm}


\section{6.1 WEAPON ZEROING (BORESIGHT ALIGNMENT)}

\subsection{THE BORESIGHT OFFSET PROBLEM}

\textbf{Physical Reality}:
\begin{itemize}
\item Camera and weapon barrel are physically separated (typically 15-30cm)
\item Camera points at reticle center
\item Weapon points at different location
\item \textbf{Without correction}: Weapon impacts below/beside reticle aim point
\end{itemize}

\textbf{Diagram}:
\begin{verbatim}
Camera → ⊙ (reticle center)
         ↓ offset
Weapon → • (actual impact point)
\end{verbatim}

\textbf{Zeroing Solution}:
\begin{itemize}
\item Applies angular offsets to compensate for physical separation
\item \textbf{Azimuth Offset}: Horizontal correction (left/right)
\item \textbf{Elevation Offset}: Vertical correction (up/down)
\item \textbf{After Zeroing}: Reticle shows where weapon will actually hit
\end{itemize}

\subsection{ZEROING PROCEDURE}

\subsubsection{Pre-Zeroing Requirements}

\textbf{Environmental Conditions}:
\begin{itemize}
\item $\checkmark$ Calm wind (<5 knots)
\item $\checkmark$ Good visibility (daylight preferred)
\item $\checkmark$ Stable platform (vehicle stationary)
\item $\checkmark$ Temperature moderate
\end{itemize}

\textbf{Range Setup}:
\begin{itemize}
\item $\checkmark$ Known range to target (recommended 100-300m)
\item $\checkmark$ Fixed target (large, visible, safe backstop)
\item $\checkmark$ No obstructions
\end{itemize}

\textbf{System Status}:
\begin{itemize}
\item $\checkmark$ Station powered and initialized
\item $\checkmark$ Camera operational (day camera for initial zero)
\item $\checkmark$ Weapon loaded and ready
\item $\checkmark$ Manual mode (no active tracking or motion modes)
\end{itemize}

\subsubsection{COMPLETE ZEROING PROCEDURE (STEP-BY-STEP)}

\textbf{STEP 1: Access Zeroing Menu}
\begin{enumerate}
\item Press \textbf{MENU $\checkmark$} on Control Panel
\item Navigate to \textbf{"Zeroing"} option (▲/▼)
\item Press \textbf{VAL} to enter
\item Zeroing screen appears with instructions
\end{enumerate}

\textbf{STEP 2: Aim at Target}
\begin{enumerate}
\item Use joystick to center reticle on target center
\item Ensure stable aim
\item Target should be large (minimum 30cm × 30cm)
\item Target at known range (100-300m recommended)
\end{enumerate}

\textbf{STEP 3: Fire Test Shot(s)}
\begin{enumerate}
\item Hold \textbf{\button{Button 0}} (Master Arm)
\item Press \textbf{\button{Button 5}} (Fire) - single shot or short burst
\item Observe impact point on target
\item Note offset from reticle center (direction and distance)
\end{enumerate}

\textbf{Example}:
\begin{itemize}
\item Reticle centered on target bullseye
\item Impact observed 20cm low and 10cm right
\item This offset will be corrected in next steps
\end{itemize}

\textbf{STEP 4: Adjust Reticle to Impact}

\textbf{OSD displays}: "Use JOYSTICK to move main RETICLE to ACTUAL IMPACT POINT"

\begin{enumerate}
\item Use joystick to move reticle from target center to \textbf{actual impact location}
\item Position reticle exactly where weapon hit
\item This movement is captured as zeroing offset
\item Do NOT fire again during this step
\end{enumerate}

\textbf{Example Continued}:
\begin{itemize}
\item Move joystick down and right
\item Reticle now positioned at impact point (20cm low, 10cm right of center)
\item System records this offset
\end{itemize}

\textbf{STEP 5: Apply Zero}
\begin{enumerate}
\item Press \textbf{MENU $\checkmark$ / VAL} button to apply
\item System calculates offsets (Azimuth, Elevation)
\item Completion screen shows: \textbf{"Zeroing Adjustment Applied!"}
\item Displays final offsets (e.g., "Az: 0.85\degree{}, El: -1.23\degree{}")
\item \textbf{"Z"} indicator appears on OSD (confirms zero active)
\end{enumerate}

\textbf{STEP 6: Verify Zero}
\begin{enumerate}
\item \textbf{Return reticle to target center} (joystick)
\item Fire verification shot
\item Impact should now match reticle center
\item \textbf{If offset remains}: Repeat procedure
\item \textbf{If accurate}: Zero complete
\end{enumerate}

\NOTE{Zero values saved to configuration file, loaded automatically on startup.}

\subsection{MULTI-RANGE ZERO VALIDATION}

\textbf{Why Multiple Ranges Matter}:
\begin{itemize}
\item Ballistic arc is curved, not straight
\item Zero at 100m may not be accurate at 500m
\item Single-range zero is a compromise
\end{itemize}

\textbf{Recommended Zero Range}: \textbf{200-250m} (best for general-purpose zero, acceptable accuracy 50m-600m)

\subsubsection{MULTI-RANGE VALIDATION PROCEDURE}

After initial zero at 200m, test at multiple ranges:

% TABLE ROW: | Range | Procedure | Acceptable Error |
% TABLE ROW: |-------|-----------|------------------|
% TABLE ROW: | **100m** | Aim at center, fire test shot | ±5cm |
% TABLE ROW: | **300m** | Aim at center, fire test shot | ±10cm |
% TABLE ROW: | **500m** | Aim at center, fire test shot | ±20cm (may be low due to drop) |

\textbf{If Errors Exceed Acceptable}:
\begin{itemize}
\item Re-zero at primary engagement range (e.g., 250m)
\item Compromise between short and long range accuracy
\end{itemize}

\subsection{CLEARING ACTIVE ZERO}

\textbf{When to Clear Zero}:
\begin{itemize}
\item Weapon or camera repositioned (maintenance, replacement)
\item Zero no longer accurate (tested and failed verification)
\item Different weapon or ammunition type installed
\end{itemize}

\textbf{Procedure}:
\begin{enumerate}
\item MENU $\checkmark$ $\rightarrow$ Zeroing $\rightarrow$ \textbf{Clear Zero}
\item Confirm: \textbf{"CLEAR ACTIVE ZERO?"}
\item MENU $\checkmark$ $\rightarrow$ \textbf{\osd{YES}}
\item OSD \textbf{"Z"} indicator disappears
\item System returns to factory boresight (no offset)
\item Must perform new zeroing procedure before live fire
\end{enumerate}

\section{6.2 ENVIRONMENTAL PARAMETERS}

\subsection{ENVIRONMENTAL EFFECTS ON BALLISTICS}

Ballistic trajectory is affected by environmental conditions:

% TABLE ROW: | Parameter | Effect on Trajectory | System Correction |
% TABLE ROW: |-----------|----------------------|-------------------|
% TABLE ROW: | **Temperature** | Air density changes → drag changes | Ballistic Look-Up Table (LUT) adjustment |
% TABLE ROW: | **Altitude** | Lower air pressure → less drag | LUT adjustment for altitude |
% TABLE ROW: | **Crosswind** | Lateral projectile drift | Azimuth offset correction |

\textbf{System Fields} (SystemStateData):
\begin{itemize}
\item \texttt{environmentalTemperatureCelsius} (default: 15\degree{}C)
\item \texttt{environmentalAltitudeMeters} (default: 0m / sea level)
\item \texttt{environmentalCrosswindMS} (meters per second)
\item \texttt{environmentalAppliedToBallistics} (boolean flag)
\end{itemize}

\subsection{ENVIRONMENTAL PARAMETERS SETUP}

\subsubsection{ACCESSING ENVIRONMENTAL MENU}

\begin{enumerate}
\item Press \textbf{MENU $\checkmark$}
\item Navigate to \textbf{"Environmental Parameters"} (▲/▼)
\item Press \textbf{VAL} to enter
\item Environmental configuration screen appears
\end{enumerate}

\subsubsection{SETTING TEMPERATURE}

\textbf{Purpose}: Correct for air density changes due to temperature

\textbf{Procedure}:
\begin{enumerate}
\item Select \textbf{"Temperature"} option
\item Use \textbf{▲/▼} to adjust value (range: -40\degree{}C to +60\degree{}C)
\item Set to \textbf{current ambient temperature}
\item Press \textbf{VAL} to confirm
\end{enumerate}

\textbf{Temperature Guidelines}:

% TABLE ROW: | Condition | Temperature | Air Density | Trajectory Effect |
% TABLE ROW: |-----------|-------------|-------------|-------------------|
% TABLE ROW: | **Cold** | <0°C | High (dense air) | More drag, shorter range |
% TABLE ROW: | **Standard** | 15°C | Standard (ISO) | No correction |
% TABLE ROW: | **Hot** | >30°C | Low (thin air) | Less drag, longer range |

\NOTE{Temperature measured at **shooter location**, not target location.}

\subsubsection{SETTING ALTITUDE}

\textbf{Purpose}: Correct for air pressure changes with elevation

\textbf{Procedure}:
\begin{enumerate}
\item Select \textbf{"Altitude"} option
\item Use \textbf{▲/▼} to adjust value (range: -500m to +4000m)
\item Set to \textbf{current altitude above sea level} (meters)
\item Press \textbf{VAL} to confirm
\end{enumerate}

\textbf{Altitude Guidelines}:

% TABLE ROW: | Altitude | Air Pressure | Trajectory Effect |
% TABLE ROW: |----------|--------------|-------------------|
% TABLE ROW: | **Sea Level (0m)** | Standard (101.3 kPa) | No correction |
% TABLE ROW: | **1000m** | Lower (~90 kPa) | Less drag, longer range, less drop |
% TABLE ROW: | **2000m+** | Much lower | Significantly less drag, flatter trajectory |

\textbf{ TIP}: Use GPS, map, or barometric altimeter to determine altitude.

\subsubsection{SETTING CROSSWIND}

\textbf{Purpose}: Correct for lateral projectile drift due to wind

\textbf{Procedure}:
\begin{enumerate}
\item Select \textbf{"Crosswind"} option
\item Use \textbf{▲/▼} to adjust value (range: 0-25 m/s)
\item Set to \textbf{estimated crosswind speed} (perpendicular to fire direction)
\item Press \textbf{VAL} to confirm
\end{enumerate}

\textbf{Crosswind Assessment}:

% TABLE ROW: | Wind Speed (m/s) | Wind Speed (knots) | Visual Cues |
% TABLE ROW: |------------------|-------------------|-------------|
% TABLE ROW: | **0-2** | 0-4 | Calm, smoke rises vertically |
% TABLE ROW: | **3-5** | 6-10 | Light breeze, leaves rustle, flags extended |
% TABLE ROW: | **6-10** | 12-20 | Moderate wind, small branches move, dust raised |
% TABLE ROW: | **11-15** | 21-30 | Fresh wind, small trees sway |
% TABLE ROW: | **16+** | 31+ | Strong wind, large branches move |

\textbf{Crosswind Direction}:
\begin{itemize}
\item \textbf{Full crosswind}: Wind perpendicular to fire direction (maximum effect)
\item \textbf{Partial crosswind}: Wind at angle to fire direction (reduced effect)
\item \textbf{Headwind/Tailwind}: Wind along fire direction (minimal lateral drift, affects range only)
\end{itemize}

\textbf{System Assumption}: Crosswind value entered represents \textbf{full crosswind component} (perpendicular). Operator should estimate effective crosswind speed accounting for wind angle.

\subsubsection{APPLYING ENVIRONMENTAL SETTINGS}

\textbf{STEP 1: Configure All Parameters}
\begin{itemize}
\item Temperature set
\item Altitude set
\item Crosswind set
\end{itemize}

\textbf{STEP 2: Apply to Ballistics}
\begin{enumerate}
\item Select \textbf{"Apply Environmental Settings"}
\item Confirm: \textbf{"APPLY TO BALLISTICS?"}
\item Press \textbf{VAL} $\rightarrow$ \textbf{\osd{YES}}
\item OSD displays: \textbf{\osd{ENV}} indicator (confirms environmental corrections active)
\end{enumerate}

\textbf{STEP 3: Verify Active}
\begin{itemize}
\item Check OSD for \textbf{\osd{ENV}} indicator
\item Environmental corrections now applied to CCIP reticle
\item All subsequent shots use environmental-corrected ballistics
\end{itemize}

\subsubsection{CLEARING ENVIRONMENTAL SETTINGS}

\textbf{When to Clear}:
\begin{itemize}
\item Environmental conditions changed significantly
\item Moving to different location (altitude, temperature change)
\item Wind conditions changed
\item Returning to standard conditions
\end{itemize}

\textbf{Procedure}:
\begin{enumerate}
\item MENU $\checkmark$ $\rightarrow$ Environmental Parameters $\rightarrow$ \textbf{Clear Settings}
\item Confirm: \textbf{"CLEAR ENVIRONMENTAL SETTINGS?"}
\item MENU $\checkmark$ $\rightarrow$ \textbf{\osd{YES}}
\item OSD \textbf{\osd{ENV}} indicator disappears
\item System returns to standard conditions (15\degree{}C, 0m altitude, 0 m/s wind)
\end{enumerate}

\subsection{ENVIRONMENTAL PARAMETERS QUICK REFERENCE}

% TABLE ROW: | Situation | Temperature | Altitude | Crosswind | Apply? |
% TABLE ROW: |-----------|-------------|----------|-----------|--------|
% TABLE ROW: | **Sea level, standard day, calm** | 15°C | 0m | 0 m/s | No (factory default) |
% TABLE ROW: | **Desert, hot, calm** | 40°C | 0m | 0 m/s | Yes (temperature correction) |
% TABLE ROW: | **Mountain, cold, windy** | -10°C | 2000m | 8 m/s | Yes (all corrections) |
% TABLE ROW: | **Temperate, mild, light breeze** | 20°C | 300m | 3 m/s | Optional (small corrections) |

\section{6.3 LEAD ANGLE COMPENSATION (LAC)}

\subsection{THE MOVING TARGET PROBLEM}

\textbf{Without Lead Compensation}:
\begin{verbatim}
Time T=0 (Fire)          Time T=TOF (Impact)

Target: [X]              Target: -----> [X]
         ↓
Bullet:  •               Bullet:         •

Result: MISS (bullet hits where target WAS)
\end{verbatim}

\textbf{With Lead Compensation}:
\begin{verbatim}
Time T=0 (Fire)          Time T=TOF (Impact)

Target: [X]              Target: -----> [X]
         ↓ (aim ahead)                   ↓
Bullet:  →→→ •           Bullet:         •

Result: HIT (bullet meets target at predicted position)
\end{verbatim}

\subsection{LEAD ANGLE FUNDAMENTALS}

\textbf{Lead Angle}: Angular offset between target's current position and predicted intercept point

\textbf{Factors Affecting Lead Angle}:
\begin{enumerate}
\item \textbf{Target Velocity}: Faster target = more lead
\item \textbf{Target Direction}: Crossing target = maximum lead, approaching/receding = minimal
\item \textbf{Range to Target}: Greater range = longer TOF = more lead
\item \textbf{Projectile Velocity}: Slower projectile = longer TOF = more lead
\item \textbf{Target Angular Rate}: How fast target crosses FOV
\end{enumerate}

\textbf{System Calculation Process} (30 Hz update rate):
\begin{enumerate}
\item Measure target motion (tracking system provides angular rates)
\item Determine range (LRF)
\item Calculate Time-of-Flight (TOF)
\item Predict target position (current position + angular rate × TOF)
\item Calculate lead angle
\item Apply offset to CCIP reticle
\end{enumerate}

\subsection{LAC ACTIVATION REQUIREMENTS}

\textbf{Prerequisites}:
\begin{enumerate}
\item $\checkmark$ Active target track established (Tracking Phase = Active Lock)
\item $\checkmark$ Valid range data from LRF
\item $\checkmark$ Target exhibiting motion (angular rate > threshold)
\item $\checkmark$ Sufficient camera FOV (not zoomed in excessively)
\item $\checkmark$ System initialized and operational
\end{enumerate}

\textbf{Safety Interlock}:
\begin{itemize}
\item \textbf{Dead Man Switch (\button{Button 3}) MUST be held} during activation
\end{itemize}

\subsection{LAC ACTIVATION PROCEDURE}

\textbf{PRE-CONDITIONS}:
\begin{itemize}
\item $\square$ Target tracked (Active Lock achieved)
\item $\square$ Target moving at measurable velocity
\item $\square$ Range data valid (LRF fired successfully)
\item $\square$ Camera FOV adequate (avoid max zoom)
\end{itemize}

\textbf{ACTIVATION}:
\begin{enumerate}
\item Hold \textbf{\button{Button 3}} (Dead Man Switch)
\item Press \textbf{\button{Button 2}} (LAC Toggle)
\item Release \button{Button 2}
\item Release \button{Button 3}
\end{enumerate}

\textbf{VERIFICATION}:
\begin{itemize}
\item $\square$ \textbf{\osd{LEAD ANGLE ON}} indicator appears (GREEN)
\item $\square$ \textbf{\osd{LAC}} bracket appears on CCIP reticle
\item $\square$ Reticle pipper shifts to lead position (ahead of target)
\item $\square$ Track confidence remains >70\%
\end{itemize}

\subsection{LAC STATUS INDICATORS}

\subsubsection{STATUS: "LEAD ANGLE ON" (GREEN)}

\textbf{Meaning}: LAC active and functioning correctly

\textbf{Display}:
\begin{itemize}
\item CCIP reticle offset ahead of target
\item \osd{LAC} bracket visible on reticle
\item Lead offset calculated and applied
\end{itemize}

\textbf{Operator Action}:
\begin{itemize}
\item Aim at CCIP reticle (not at target directly)
\item CCIP shows where to aim for predicted intercept
\item Fire when CCIP on target (system handles lead)
\end{itemize}

\subsubsection{STATUS: "LEAD ANGLE LAG" (YELLOW)}

\textbf{Meaning}: Insufficient tracking data for accurate lead calculation

\textbf{Causes}:
\begin{itemize}
\item Track recently established (<2 seconds)
\item Target velocity not yet stable
\item Tracking confidence fluctuating
\end{itemize}

\textbf{Display}:
\begin{itemize}
\item \osd{LEAD ANGLE LAG} in YELLOW
\item Reticle may show partial lead offset (unreliable)
\end{itemize}

\textbf{Operator Action}:
\begin{itemize}
\item \textbf{WAIT} (2-5 seconds) for tracking to stabilize
\item Do NOT fire until status changes to \osd{LEAD ANGLE ON} (GREEN)
\item If LAG persists >10 seconds: Check tracking quality, verify target moving
\end{itemize}

\subsubsection{STATUS: "ZOOM OUT" (RED)}

\textbf{Meaning}: Camera FOV too narrow for accurate lead calculation

\textbf{Causes}:
\begin{itemize}
\item Zoomed in too far (max zoom or near-max zoom)
\item Target angular rate too high for current FOV
\item System cannot measure target velocity accurately
\end{itemize}

\textbf{Display}:
\begin{itemize}
\item \osd{ZOOM OUT} in RED
\item LAC non-functional (lead calculation disabled)
\end{itemize}

\textbf{Operator Action}:
\begin{enumerate}
\item \textbf{Zoom out} (\button{Button 8}) gradually
\item Wait for status to change to \osd{LEAD ANGLE ON} (GREEN)
\item If problem persists: Zoom out more
\item Acceptable FOV: Typically Wide or Mid zoom (not Tele/Max)
\end{enumerate}

\NOTE{Extremely fast-moving targets may require wide FOV even at close range.}

\subsection{LAC DEACTIVATION PROCEDURE}

\textbf{When to Deactivate}:
\begin{itemize}
\item Target stops moving
\item Lost track of target
\item \osd{ZOOM OUT} warning persists
\item \osd{LEAD ANGLE LAG} persists
\item Engagement complete
\item Switching to stationary target
\end{itemize}

\textbf{DEACTIVATION}:
\begin{enumerate}
\item Hold \textbf{\button{Button 3}} (Dead Man Switch)
\item Press \textbf{\button{Button 2}} (LAC Toggle)
\item Release \button{Button 2}
\item Release \button{Button 3}
\end{enumerate}

\textbf{VERIFICATION}:
\begin{itemize}
\item $\square$ \osd{LEAD ANGLE ON} indicator disappears
\item $\square$ \osd{LAC} bracket removed from CCIP reticle
\item $\square$ Reticle pipper returns to boresight alignment
\item $\square$ System ready for stationary target engagement
\end{itemize}

\subsection{USING LAC IN ENGAGEMENT}

\subsubsection{MOVING TARGET ENGAGEMENT (WITH LAC)}

\textbf{Complete Procedure}:

\textbf{1. Acquire and Track Target} (Lesson 5)
\begin{itemize}
\item Enter acquisition mode (\button{Button 4})
\item Size gate, request lock-on
\item Achieve Active Lock (green gate, tracking)
\end{itemize}

\textbf{2. Range Target}
\begin{itemize}
\item Fire LRF (or automatic ranging during tracking)
\item Verify range displayed (OSD: RNG: xxxx m)
\end{itemize}

\textbf{3. Enable LAC}
\begin{itemize}
\item Hold \button{Button 3} + Press \button{Button 2}
\item Verify \textbf{\osd{LEAD ANGLE ON}} (GREEN)
\item Wait if \textbf{\osd{LEAD ANGLE LAG}} (YELLOW)
\item Zoom out if \textbf{\osd{ZOOM OUT}} (RED)
\end{itemize}

\textbf{4. Observe Lead Offset}
\begin{itemize}
\item CCIP reticle offset ahead of target in direction of motion
\item Lead offset adjusts continuously as target moves
\item \textbf{Aim at CCIP reticle} (not at target center)
\end{itemize}

\textbf{5. Fire}
\begin{itemize}
\item Master Arm (\button{Button 0})
\item Fire (\button{Button 5})
\item Rounds impact at predicted intercept point
\end{itemize}

\textbf{6. Observe Effect}
\begin{itemize}
\item Rounds should hit target (not behind target)
\item If missing: Check tracking quality, verify LAC ON, verify range accurate
\end{itemize}

\subsection{LAC LIMITATIONS}

\textbf{When LAC Works Best}:
\begin{itemize}
\item $\checkmark$ Target moving laterally (crossing FOV)
\item $\checkmark$ Target speed >5 m/s (\textasciitilde{}10 mph)
\item $\checkmark$ Range >100 meters
\item $\checkmark$ Constant target velocity
\end{itemize}

\textbf{When LAC is NOT Needed}:
\begin{itemize}
\item $\times$ Stationary targets (LAC shows no offset)
\item $\times$ Targets moving radially (toward/away) - minimal lateral lead
\item $\times$ Very close range (<50m) - TOF too short for significant lead
\end{itemize}

\textbf{LAC Limitations}:
\begin{itemize}
\item Requires active tracking (cannot use LAC without track)
\item Requires sufficient FOV (may need to zoom out)
\item Assumes constant target velocity (less accurate if target maneuvering)
\item Not effective for erratic or unpredictable motion
\end{itemize}

\section{6.4 COMBINING FIRE CONTROL SYSTEMS}

\subsection{FIRE CONTROL SOLUTION HIERARCHY}

The complete fire control solution combines multiple corrections:

\begin{verbatim}
FINAL AIM POINT = Gun Boresight + Zeroing Offset + Environmental Corrections + Lead Angle Offset
\end{verbatim}

\textbf{System Integration}:
\begin{enumerate}
\item \textbf{Gun Boresight}: Factory default (camera-to-weapon offset)
\item \textbf{+ Zeroing Offset}: Operator-configured (corrects boresight error)
\item \textbf{+ Environmental Corrections}: Ballistic LUT adjustments (temperature, altitude, wind)
\item \textbf{+ Lead Angle Offset}: Real-time moving target compensation
\end{enumerate}

\textbf{OSD Indicators}:
\begin{itemize}
\item \textbf{"Z"}: Zeroing active
\item \textbf{\osd{ENV}}: Environmental parameters active
\item \textbf{\osd{LEAD ANGLE ON}}: LAC active
\end{itemize}

\textbf{All Active Example}:
\begin{itemize}
\item OSD displays: \textbf{Z ENV LEAD ANGLE ON}
\item CCIP reticle shows: Zeroing + Environmental + Lead corrections
\item Most accurate engagement solution
\end{itemize}

\subsection{ENGAGEMENT SCENARIOS}

\subsubsection{SCENARIO 1: Stationary Target, Standard Conditions}

\textbf{Configuration}:
\begin{itemize}
\item Zeroing: Active (Z)
\item Environmental: Not needed (standard conditions)
\item LAC: Not needed (stationary target)
\end{itemize}

\textbf{OSD}: \textbf{Z}

\textbf{Fire Control}: Zeroing offset only

\subsubsection{SCENARIO 2: Stationary Target, Hot Desert, High Altitude}

\textbf{Configuration}:
\begin{itemize}
\item Zeroing: Active (Z)
\item Environmental: Active (Temp: 45\degree{}C, Alt: 1500m, Wind: 0 m/s)
\item LAC: Not needed (stationary target)
\end{itemize}

\textbf{OSD}: \textbf{Z ENV}

\textbf{Fire Control}: Zeroing + Environmental (temperature, altitude)

\subsubsection{SCENARIO 3: Moving Target, Windy Conditions, Mountain}

\textbf{Configuration}:
\begin{itemize}
\item Zeroing: Active (Z)
\item Environmental: Active (Temp: -5\degree{}C, Alt: 2500m, Wind: 10 m/s)
\item LAC: Active (target moving 15 m/s lateral)
\end{itemize}

\textbf{OSD}: \textbf{Z ENV LEAD ANGLE ON}

\textbf{Fire Control}: Zeroing + Environmental + Lead Angle (full solution)

\subsubsection{SCENARIO 4: Close Range, Moving Target, Standard Conditions}

\textbf{Configuration}:
\begin{itemize}
\item Zeroing: Active (Z)
\item Environmental: Not needed
\item LAC: Active (target moving, but close range <100m may not require much lead)
\end{itemize}

\textbf{OSD}: \textbf{Z LEAD ANGLE ON}

\textbf{Fire Control}: Zeroing + Lead Angle

\subsection{FIRE CONTROL QUICK REFERENCE}

% TABLE ROW: | Target Type | Range | Conditions | Zeroing | Environmental | LAC | OSD |
% TABLE ROW: |-------------|-------|------------|---------|---------------|-----|-----|
% TABLE ROW: | Stationary | Any | Standard | ✅ | ❌ | ❌ | Z |
% TABLE ROW: | Stationary | Any | Extreme temp/alt | ✅ | ✅ | ❌ | Z ENV |
% TABLE ROW: | Stationary | Any | Windy | ✅ | ✅ (wind) | ❌ | Z ENV |
% TABLE ROW: | Moving (slow) | <100m | Standard | ✅ | ❌ | Optional | Z (LEAD ANGLE ON) |
% TABLE ROW: | Moving (fast) | >100m | Standard | ✅ | ❌ | ✅ | Z LEAD ANGLE ON |
% TABLE ROW: | Moving (fast) | >100m | Extreme | ✅ | ✅ | ✅ | Z ENV LEAD ANGLE ON |

\section{6.5 FIRE CONTROL BEST PRACTICES}

\subsection{ZEROING}

\textbf{Best Practices}:
\begin{itemize}
\item $\checkmark$ Zero weapon before first operational use
\item $\checkmark$ Verify zero periodically (weekly or after transport)
\item $\checkmark$ Re-zero if weapon or camera serviced/replaced
\item $\checkmark$ Use consistent ammunition type for zeroing and operations
\item $\checkmark$ Zero at primary engagement range (200-250m recommended)
\end{itemize}

\textbf{Common Errors}:
\begin{itemize}
\item $\times$ Skipping multi-range validation (zero may be inaccurate at long range)
\item $\times$ Moving gimbal between firing test shot and adjusting reticle (invalidates offset measurement)
\item $\times$ Assuming zero is permanent (mechanical drift can occur over time)
\end{itemize}

\subsection{ENVIRONMENTAL PARAMETERS}

\textbf{Best Practices}:
\begin{itemize}
\item $\checkmark$ Update environmental settings at mission start
\item $\checkmark$ Re-assess if conditions change significantly during mission
\item $\checkmark$ Use handheld weather instruments (thermometer, anemometer, altimeter) for accuracy
\item $\checkmark$ Estimate conservatively (err toward standard conditions if unsure)
\end{itemize}

\textbf{Common Errors}:
\begin{itemize}
\item $\times$ Using old/stale environmental data from previous mission
\item $\times$ Ignoring significant temperature or wind changes
\item $\times$ Over-correcting for minor environmental variations (<5\degree{}C temp change, <2 m/s wind)
\end{itemize}

\subsection{LEAD ANGLE COMPENSATION}

\textbf{Best Practices}:
\begin{itemize}
\item $\checkmark$ Only use LAC for targets moving >5 m/s at ranges >100m
\item $\checkmark$ Wait for \osd{LEAD ANGLE ON} (GREEN) before firing (ignore \osd{LAG} yellow)
\item $\checkmark$ Zoom out if \osd{ZOOM OUT} (RED) warning appears
\item $\checkmark$ Trust the system - aim at CCIP reticle, not at target
\end{itemize}

\textbf{Common Errors}:
\begin{itemize}
\item $\times$ Activating LAC for stationary targets (adds unnecessary complexity)
\item $\times$ Firing while \osd{LEAD ANGLE LAG} (YELLOW) - lead calculation incomplete
\item $\times$ Ignoring \osd{ZOOM OUT} (RED) warning - LAC non-functional
\item $\times$ "Kentucky windage" (manually aiming off-target) while LAC active - double-correcting
\end{itemize}

\subsection{SAFETY REMINDERS}

\textbf{ALWAYS}:
\begin{itemize}
\item $\checkmark$ Verify zero before live fire operations
\item $\checkmark$ Update environmental parameters for current conditions
\item $\checkmark$ Check OSD indicators before firing (Z, ENV, LEAD ANGLE ON)
\item $\checkmark$ Use Dead Man Switch when activating/deactivating LAC (safety interlock)
\end{itemize}

\textbf{NEVER}:
\begin{itemize}
\item $\times$ Fire without valid zero (impacts will be off-target)
\item $\times$ Assume environmental corrections are active (verify \osd{ENV} indicator)
\item $\times$ Fire with \osd{ZOOM OUT} (RED) warning (LAC non-functional)
\item $\times$ Bypass Dead Man Switch safety interlock
\end{itemize}
